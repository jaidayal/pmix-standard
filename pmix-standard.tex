% Welcome to pmix-standard.tex.
% This is the master LaTex file for the PMIx Standard document.
%
% The files in this set include:
%
%    pmix-standard.tex                - this file, the master file
%    Makefile                         - makes the document
%    pmix.sty                         - the main style file
%    Title_Page.tex                   - the title page
%    Chap_Introduction.tex            - unnumbered introductory chapter
%    figs/*.png                       - Figures
%    sources/*.c, *.f                 - C/C++/Fortran example source files
%
% When editing this file:
%
%    1. To change formatting, appearance, or style, please edit pmix.sty.
%
%    2. Custom commands and macros are defined in pmix.sty.
%
%    3. Be kind to other editors -- keep a consistent style by copying-and-pasting to
%       create new content.
%
%    4. We use semantic markup, e.g. (see pmix.sty for a full list):
%         \code{}     % for bold monospace keywords, code, operators, etc.
%
%    5. Other recommendations:
%         Use the convenience macros defined in pmix.sty for the minor headers
%         such as Comments, Syntax, etc.
%
%         To keep items together on the same page, prefer the use of
%         \begin{samepage}.... Avoid \parbox for text blocks as it interrupts line numbering.
%         When possible, avoid \filbreak, \pagebreak, \newpage, \clearpage unless that's
%         what you mean. Use \needspace{} cautiously for troublesome paragraphs.
%
%         Avoid absolute lengths and measures in this file; use relative units when possible.
%         Vertical space can be relative to \baselineskip or ex units. Horizontal space
%         can be relative to \linewidth or em units.
%
%         Prefer \emph{} to italicize terminology, e.g.:
%             This is a \emph{definition}, not a placeholder.
%             This is a \plc{var-name}.
%

% The following says letter size, but the style sheet may change the size
\documentclass[10pt,letterpaper,twoside,makeidx,hidelinks]{scrreprt}

%%%%%%%%%%%%%%%%%%%
% Release Managers:
% - Set is_unofficial_draft = false to remove the watermark
% - Set VER to the correct version number
% - Set VERDATE to the Month+Year of the release
%%%%%%%%%%%%%%%%%%%
\usepackage[us,24hr]{datetime}
\usepackage{ifthen}
\newboolean{is_unofficial_draft}
\setboolean{is_unofficial_draft}{true}

% Text to appear in the footer on even-numbered pages:
\ifthenelse{\boolean{is_unofficial_draft}}
  {\newcommand{\VER}{3.2 (Draft)}
   \newcommand{\VERDATE}{\emph{Created on \today}}
  }
  {\newcommand{\VER}{3.2}
   \newcommand{\VERDATE}{Month Year}
  }
\newcommand{\footerText}{PMIx Standard -- Version \VER{} -- \VERDATE}

% Unified style sheet for PMIx documents:
% This is pmix.sty, the preamble and style definitions for the PMIx specification.
%
% This specification file, and latex structure was derived from/inspired by the OpenMP specification. So some similarity between the two latex files is expected.
%
%%%%%%%%%%%%%%%%%%%%%%%%%%%%%%%%%%%%%%%%%%%%%%%%%%%%%%%%%%%%%%%%%%%%%%%%%%%%%%%%%%%%%%%%%%%%%
% Quick list of the environments, commands and macros supported.
% Search below for more details.
%
% Formatting Text:
%   -----------------------
%   \notestart            - "Note:" Callout section
%   \noteheader           - \noteheader is optional "Note:" prefix for text
%     ...
%   \noteend
%   -----------------------
%   \rationalestart       - "Rationale" Callout section
%     ...
%   \rationaleend
%   -----------------------
%   \adviceuserstart      - "Advice to users" Callout section
%     ...
%   \adviceuserend
%   -----------------------
%   \adviceimplstart      - "Advice to PMIx library implementers" Callout section
%     ...
%   \adviceimplend
%   -----------------------
%   \advicermstart      - "Advice to PMIx server hosts" Callout section
%     ...
%   \advicermend
%   -----------------------
%
% Formatting Code:
%   \code{}               - Code text
%   \var{}                - Variable
%   -----------------------
%   \begin{codepar}       - Section of generic code
%     ...                 - use language specific macro if language specific code
%   \end[codepar}
%   -----------------------
%   \cspecificstart       - C specific code block
%     ...
%   \cspecificend
%   -----------------------
%
% Attributes:
%   \refAttributeItem{}   - Cross reference
%   \refattr{}            - Same as above
%   \pasteAttributeItem{} - Paste full description
%
% Structures:
%   \refstruct{}          - Reference a structure
%   \structref{}          - Same as above
%   \specrefstruct{}      - Reference a structure by section number and page
%
% APIs:
%   \refapi{}             - Reference an API function
%   \refconst{}           - Constant reference
%   \refarg{} / \argref{} - Reference an argument to an API function
%
% Cross referencing:
%   \chapterref{}         - Reference a Chapter by number and page
%   \specref{}            - Reference a Section by number and page
%
%%%%%%%%%%%%%%%%%%%%%%%%%%%%%%%%%%%%%%%%%%%%%%%%%%%%%%%%%%%%%%%%%%%%%%%%%%%%%%%%%%%%%%%%%%%%%
\usepackage{comment}            % allow use of \begin{comment}
\usepackage{ifpdf,ifthen}       % allow conditional tests in LaTeX definitions
\usepackage{makecell}           % Allows common formatting in cells with \thread & \makecell

\usepackage[T1]{fontenc}        % Allow us to use underscore freely in the document
\catcode`\_=12                  % Use \sb for subscripts
\usepackage{verbatim}


%%%%%%%%%%%%%%%%%%%%%%%%%%%%%%%%%%%%%%%%%%%%%%%%%%%%%%%%%%%%%%%%%%%%%%%%%%%%%%%%%%%%%%%%%%%%%
% Document data
%
\author{}


%%%%%%%%%%%%%%%%%%%%%%%%%%%%%%%%%%%%%%%%%%%%%%%%%%%%%%%%%%%%%%%%%%%%%%%%%%%%%%%%%%%%%%%%%%%%%
% Fonts

\usepackage{amsmath}
\usepackage{amsfonts}
\usepackage{amssymb}
\usepackage{courier}
\usepackage{helvet}
\usepackage[utf8]{inputenc}

% Main body serif font:
\usepackage{tgtermes}
\usepackage[T1]{fontenc}


%%%%%%%%%%%%%%%%%%%%%%%%%%%%%%%%%%%%%%%%%%%%%%%%%%%%%%%%%%%%%%%%%%%%%%%%%%%%%%%%%%%%%%%%%%%%%
% Graphic elements

\usepackage{graphicx}
\usepackage{framed}    % for making boxes with \begin{framed}
\usepackage{tikz}      % for flow charts, diagrams, arrows


%%%%%%%%%%%%%%%%%%%%%%%%%%%%%%%%%%%%%%%%%%%%%%%%%%%%%%%%%%%%%%%%%%%%%%%%%%%%%%%%%%%%%%%%%%%%%
% Page formatting

\usepackage[paperwidth=7.5in, paperheight=9in,
            top=0.75in, bottom=1.0in, left=1.4in, right=0.6in]{geometry}

\usepackage{changepage}   % allows left/right-page margin readjustments

\setlength{\oddsidemargin}{0.185in}
\setlength{\evensidemargin}{0.185in}
\raggedbottom


%%%%%%%%%%%%%%%%%%%%%%%%%%%%%%%%%%%%%%%%%%%%%%%%%%%%%%%%%%%%%%%%%%%%%%%%%%%%%%%%%%%%%%%%%%%%%
% Paragraph formatting

\usepackage{setspace}     % allows use of \singlespacing, \onehalfspacing
\usepackage{needspace}    % allows use of \needspace to keep lines together
\usepackage{parskip}      % removes paragraph indenting

\raggedright
\usepackage[raggedrightboxes]{ragged2e}  % is this needed?

\lefthyphenmin=60         % only hyphenate if the left part is >= this many chars
\righthyphenmin=60        % only hyphenate if the right part is >= this many chars


%%%%%%%%%%%%%%%%%%%%%%%%%%%%%%%%%%%%%%%%%%%%%%%%%%%%%%%%%%%%%%%%%%%%%%%%%%%%%%%%%%%%%%%%%%%%%%
% Bulleted (itemized) lists
%    Align bullets with section header
%    Align text left
%    Small bullets
%    \compactitem for single-spaced lists (used in the Examples doc)

\usepackage{enumitem}     % for setting margins on lists
\setlist{leftmargin=*}    % don't indent bullet items
\renewcommand{\labelitemi}{{\normalsize$\bullet$}} % bullet size

% There is a \compactitem defined in package parlist (and perhaps others), however,
% we'll define our own version of compactitem in terms of package enumitem that
% we already use:
\newenvironment{compactitem}
{\begin{itemize}[itemsep=-1.2ex]}
{\end{itemize}}

%%%%%%%%%%%%%%%%%%%%%%%%%%%%%%%%%%%%%%%%%%%%%%%%%%%%%%%%%%%%%%%%%%%%%%%%%%%%%%%%%%%%%%%%%%%%%
% Floating version
%\usepackage[showboxes]{textpos}
\usepackage{textpos}

\setlength{\TPHorizModule}{1pt}%
\setlength{\TPVertModule}{\TPHorizModule}%
\TPMargin{1pt}%

\newcommand{\versionMarker}[1]{%
 % y is 8 = \parskip
 \begin{textblock}{50}(-55,8)%
   \textit{PMIx v#1}%
   \raggedright
 \end{textblock}%
}
% Alternative is to make a box inline, but that gets tricky when positioning close
% to codepar's
% \makebox[-7pt][r]{\textit{PMIx #4}\raggedright}

%%%%%%%%%%%%%%%%%%%%%%%%%%%%%%%%%%%%%%%%%%%%%%%%%%%%%%%%%%%%%%%%%%%%%%%%%%%%%%%%%%%%%%%%%%%%%%
% Enumerated list with lowercase alphabet lettering
%    \alphaenum for default-spaced lists
%    \compactalphaenum for single-spaced lists

% There is a \compactitem defined in package parlist (and perhaps others), however,
% we'll define our own version of compactitem in terms of package enumitem that
% we already use:
\newenvironment{alphaenum}
{\begin{enumerate}[label=\alph*)]}
{\end{enumerate}}

\newenvironment{compactalphaenum}
{\begin{enumerate}[label=\alph*),itemsep=-1.2ex]}
{\end{enumerate}}

% Argument list for an interface, for use in a \begin{arglist} section
% \argin      Input argument
% \argout     Output argument
% \arginout   Input/Output argument
% \argreturn  Value returned
%%% Old Method using tables.... line numbers didn't work if a cell wrapped...
%\newlength\argdesclen
%\setlength\argdesclen{\dimexpr \linewidth -13em -4\tabcolsep}
%\newenvironment{arglist}{%
%    \begin{edtable}{tabular}{p{3em}p{10em}p{\argdesclen}}}
%    {\end{edtable}\vspace{.25em}}
%
%\newcommand{\argin}[2]{\textbf{IN} & \code{#1} & #2\\}
%\newcommand{\argout}[2]{\textbf{OUT} & \code{#1} & #2\\}
%\newcommand{\arginout}[2]{\textbf{INOUT} & \code{#1} & #2\\}

\newenvironment{arglist}
{\begin{description}[style=nextline,labelindent=\parindent,leftmargin=*,itemindent=\dimexpr-17pt-\labelsep\relax,itemsep=-1.3ex]}
{\end{description}}

\newcommand{\argin}[2]{\item[IN ~~~~\code{#1}] #2}
\newcommand{\argout}[2]{\item[OUT ~~~\code{#1}] #2}
\newcommand{\arginout}[2]{\item[INOUT ~\code{#1}] #2}

% Constant list
%   \declareconstitem  Declare constant with description
\newenvironment{constantdesc}
{\begin{description}[itemsep=-1.3ex,itemindent=\dimexpr-17pt-\labelsep\relax]}
{\end{description}}

\newcommand{\declareconstitem}[1]{\item[\code{#1}] \index{#1} \label{const:#1} \hspace{1em}}
\newcommand{\declareconstitemvalue}[2]{\item[\code{#1}] \index{#1} \hspace{0.25em} \code{#2}  \hspace{1em}}
\newcommand{\declareconstitemDEP}[2]{\item[\code{#1} (Deprecated in PMIx #2)] \index{#1} \label{const:#1} \hspace{1em}}
\newcommand{\declareconstitemNEW}[1]{\item[\color{magenta}\code{#1}] \index{#1} \label{const:#1} \hspace{1em}}


%%%%%%%%%%%%%%%%%%%%%%%%%%%%%%%%%%%%%%%%%%%%%%%%%%%%%%%%%%%%%%%%%%%%%%%%%%%%%%%%%%%%%%%%%%%%%%
% Tables

% This allows tables to flow across page breaks, headers on each new page, etc.
\usepackage{supertabular}
\usepackage{caption}


%%%%%%%%%%%%%%%%%%%%%%%%%%%%%%%%%%%%%%%%%%%%%%%%%%%%%%%%%%%%%%%%%%%%%%%%%%%%%%%%%%%%%%%%%%%%%
% Line numbering

\usepackage[pagewise,edtable]{lineno}       % for line numbers on left side of the page
\pagewiselinenumbers
\setlength\linenumbersep{6em}
\renewcommand\linenumberfont{\normalfont\small\sffamily}
\nolinenumbers            % start with line numbers off


%%%%%%%%%%%%%%%%%%%%%%%%%%%%%%%%%%%%%%%%%%%%%%%%%%%%%%%%%%%%%%%%%%%%%%%%%%%%%%%%%%%%%%%%%%%%%
% Footers

\usepackage{fancyhdr}     % makes right/left footers
\pagestyle{fancy}
\fancyhead{} % clear all header fields
\cfoot{}
\renewcommand{\headrulewidth}{0pt}

% Left side on even pages:
% This requires that \footerText be defined in the master document:
\fancyfoot[LE]{\bfseries \thepage \mdseries \hspace{2em} \footerText}
\fancyhfoffset[E]{4em}

% Right side on odd pages:
\fancyfoot[RO]{\mdseries  \leftmark \hspace{2em} \bfseries \thepage}


%%%%%%%%%%%%%%%%%%%%%%%%%%%%%%%%%%%%%%%%%%%%%%%%%%%%%%%%%%%%%%%%%%%%%%%%%%%%%%%%%%%%%%%%%%%%%
% Section header format - we use five levels: \chapter \section \subsection \subsubsection

\usepackage{titlesec}     % format headers with \titleformat{}

% Format and spacing for chapter, section, subsection, and subsubsection headers:

\setcounter{secnumdepth}{5}          % show numbers down to subsubsection level

\titleformat{\chapter}[display]%
{\normalfont\sffamily\upshape\Huge\bfseries\nolinenumbers\fontsize{20}{20}\selectfont}%
{\normalfont\sffamily\scshape\large\bfseries\nolinenumbers \hspace{-0.7in} \MakeUppercase%
    {\chaptertitlename} \thechapter}%
{0em}{}[\vspace{1.0em}\hrule]
% {<left>}{<before-sep>}{<after-sep>}
\titlespacing{\chapter}{0ex}{0em plus 1em minus 1em}{1em plus 1em minus 1em}[10em]

\titleformat{\section}[hang]{\huge\bfseries\sffamily\fontsize{16}{16}\selectfont}{\thesection}{1.0em}{}
% {<left>}{<before-sep>}{<after-sep>}
\titlespacing{\section}{-5em}{2em plus 1em minus 1em}{1em plus 0.5em minus 0em}[10em]

\titleformat{\subsection}[hang]{\LARGE\bfseries\sffamily\fontsize{14}{14}\selectfont}{\thesubsection}{1.0em}{}
\titlespacing{\subsection}{-5em}{2em plus 1em minus 2.0em}{0.75em plus 0.5em minus 0em}[10em]

\titleformat{\subsubsection}[hang]{\needspace{1\baselineskip}%
\Large\bfseries\sffamily\fontsize{12}{12}\selectfont}{\thesubsubsection}{1.0em}{}
\titlespacing{\subsubsection}{-5em}{1em plus 0.8em minus 0.8em}{0.5em plus 0.5em minus 0em}[10em]


%%%%%%%%%%%%%%%%%%%%%%%%%%%%%%%%%%%%%%%%%%%%%%%%%%%%%%%%%%%%%%%%%%%%%%%%%%%%%%%%%%%%%%%%%%%%%%
% Macros for minor headers: Summary, Syntax, Description, etc.
% These headers are defined in terms of \paragraph

\titleformat{\paragraph}[block]{\large\bfseries\sffamily\fontsize{11}{11}\selectfont}{}{}{}
\titlespacing{\paragraph}{0em}{1.0em plus 0.55em minus 0.5em}{0.0em plus 0.55em minus 0.0em}

% Use one of the convenience macros below, or \littleheader{} for an arbitrary header
\newcommand{\littleheader}[1] {\paragraph*{#1}}

\newcommand{\comments} {\littleheader{Comments}}
\newcommand{\descr} {\littleheader{Description}}
\newcommand{\format} {\littleheader{Format}}
\newcommand{\summary} {\littleheader{Summary}}
\newcommand{\history} {\littleheader{History}}
\newcommand{\priattr} {\littleheader{PRI Attributes}}
\newcommand{\reqattr} {\littleheader{\ac{RM} Required Attributes}}
\newcommand{\optattr} {\littleheader{\ac{RM} Optional Attributes}}

%%%%%%%%%%%%%%%%%%%%%%%%%%%%%%%%%%%%%%%%%%%%%%%%%%%%%%%%%%%%%%%%%%%%%%%%%%%%%%%%%%%%%%%%%%%%%
% Clipboard
%
% \StdCopy{TAG}{BODY}
% \StdPaste{TAG}
%
% Inspired by this thread:
%   https://tex.stackexchange.com/questions/150790/how-to-make-text-be-copied-to-another-part-of-a-document
\makeatletter
\newcommand\StdCopy              [2] {
  \immediate\write\@auxout{\unexpanded{\global\long\@namedef{clipbrd@#1}{#2}}}
}
\newcommand\StdCopyEcho          [2] {
  \StdCopy{#1}{#2}%
  #2
}
\newcommand\StdPaste             [1] {%
  \ifcsname clipbrd@#1\endcsname
    \@nameuse{clipbrd@#1}%
  \else
    ??unknown??
  \fi
}
\makeatother


% Attributes
%   \declareAttribute       Declare an attribute with a description
%   \pasteAttributeItem     Paste the attribute description here
%   \refAttributeItem       Reference the original definition of the attribute
%
\newcommand{\declareAttribute}[4]{%
    \code{#1} ~~\code{#2}~~(\code{#3})%
    \index{#1!Definition|textbf} \label{attr:#1}%
    \StdCopy{str:#1}{\code{#2}}%
    \StdCopy{attr:#1}{\code{#3}}%
    \vspace{-1.3ex}%
      \expandafter\begin{adjustwidth}{.95cm}{}%
      \StdCopyEcho{#1}{#4}%
    \end{adjustwidth}%
  \vspace{-1.3ex}%
}

\newcommand{\declareNewAttribute}[4]{%
   {\color{magenta}\code{#1}} ~~\code{#2}~~(\code{#3})%
    \index{#1!Definition|textbf} \label{attr:#1}%
    \StdCopy{str:#1}{\code{#2}}%
    \StdCopy{attr:#1}{\code{#3}}%
    \vspace{-1.3ex}%
      \expandafter\begin{adjustwidth}{.95cm}{}%
      \StdCopyEcho{#1}{#4}%
    \end{adjustwidth}%
  \vspace{-1.3ex}%
}

\newcommand{\declareDepAttribute}[4]{%
   {\color{green!80!black}\code{#1}} ~~\code{#2}~~(\code{#3})%
    \index{#1!Definition|textbf} \label{attr:#1}%
    \StdCopy{str:#1}{\code{#2}}%
    \StdCopy{attr:#1}{\code{#3}}%
    \vspace{-1.3ex}%
      \expandafter\begin{adjustwidth}{.95cm}{}%
      \StdCopyEcho{#1}{#4}%
    \end{adjustwidth}%
  \vspace{-1.3ex}%
}

\newcommand{\pasteAttributeItemBegin}[1]{
  \refAttributeItem{#1} ~~\StdPaste{str:#1}~~(\StdPaste{attr:#1})
  \vspace{-1.3ex}
   \expandafter
   \begin{adjustwidth}{.95cm}{}
    \StdPaste{#1}
}
\newcommand{\pasteAttributeItemEnd}{
   \end{adjustwidth}
}
\newcommand{\pasteAttributeItem}[1]{
	\pasteAttributeItemBegin{#1}
	\pasteAttributeItemEnd{}
}
\newcommand{\refAttributeItem}[1]{\index{#1}\hyperref[attr:#1]{\code{#1}}}
\newcommand{\refattr}[1]{\refAttributeItem{#1}}

\newcommand{\refPRIAttributeItem}[1]{\index{#1}\hyperref[attr:#1]{\color{red}\code{#1}}}

\newcommand{\pastePRIAttributeItemBegin}[1]{
  \refPRIAttributeItem{#1} ~~\StdPaste{str:#1}~~(\StdPaste{attr:#1})
  \vspace{-1.3ex}
   \expandafter
   \begin{adjustwidth}{.95cm}{}
    \StdPaste{#1}
}
\newcommand{\pastePRIAttributeItemEnd}{
   \end{adjustwidth}
}

\newcommand{\pastePRIAttributeItem}[1]{
    \pastePRIAttributeItemBegin{#1}
    \pastePRIAttributeItemEnd{}
}

\newcommand{\refPRRTEAttributeItem}[1]{\index{#1}\hyperref[attr:#1]{\color{green!60!black}\code{#1}}}

\newcommand{\pastePRRTEAttributeItemBegin}[1]{
  \refPRRTEAttributeItem{#1} ~~\StdPaste{str:#1}~~(\StdPaste{attr:#1})
  \vspace{-1.3ex}
   \expandafter
   \begin{adjustwidth}{.95cm}{}
    \StdPaste{#1}
}
\newcommand{\pastePRRTEAttributeItemEnd}{
   \end{adjustwidth}
}

\newcommand{\pastePRRTEAttributeItem}[1]{
    \pastePRRTEAttributeItemBegin{#1}
    \pastePRRTEAttributeItemEnd{}
}

%%%%%%%%%%%%%%%%%%%%%%%%%%%%%%%%%%%%%%%%%%%%%%%%%%%%%%%%%%%%%%%%%%%%%%%%%%%%%%%%%%%%%%%%%%%%%%
% Code and placeholder semantic tagging.
%
% When possible, prefer semantic tags instead of typographic tags. The
% following semantics tags are defined here:
%
%     \code{}     % for bold monospace keywords, code, operators, etc.
%     \plc{}      % for italic placeholder names, grammar, etc.
%
% For function prototypes or other code snippets, you can use \code{} as
% the outer wrapper, and use \plc{{} inside. Example:
%
%     \code{\#pragma omp directive ( \plc{some-placeholder-identifier} :}
%
% To format text in italics for emphasis (rather than text as a placeholder),
% use the generic \emph{} command. Example:
%
%     This sentence \emph{emphasizes some non-placeholder words}.

% Enable \alltt{} for formatting blocks of code:
\usepackage{alltt}

% This sets the default \code{} font to tt (monospace) and bold:
\newcommand{\code}[1]{{\texttt{\textbf{#1}}}}
\newcommand{\var}[1] {{\textrm{\textmd{\itshape{#1}}}}}


% Environment for a paragraph of literal code, single-spaced, no outline, no indenting:
\newenvironment{codepar}[1]
{\begin{alltt}\bfseries #1}
{\end{alltt}}

\usepackage{setspace}

%%%%%%%%%%%%%%%%%%%%%%%%%%%%%%%%%%%%%%%%%%%%%%%%%%%%%%%%%%%%%%%%%%%%%%%%%%%%%%%%%%%%%%%%%%%%%%
% Macros for the black and blue lines and arrows delineating language-specific
% and notes sections. Example:
%
%   \fortranspecificstart
%   This is text that applies to Fortran.
%   \fortranspecificend

% local parameters for use \linewitharrows and \notelinewitharrows:
\newlength{\sbsz}\setlength{\sbsz}{0.05in}  % size of arrows
\newlength{\sblw}\setlength{\sblw}{1.35pt}  % line width (thickness)
\newlength{\sbtw}                           % text width
\newlength{\sblen}                          % total width of horizontal rule
\newlength{\sbht}                           % height of arrows
\newlength{\sbhadj}                         % vertical adjustment for aligning arrows with the line
\newlength{\sbns}\setlength{\sbns}{7\baselineskip}       % arg for \needspace for downward arrows

% \notelinewitharrows is a helper command that makes a black Note marker:
%     arg 1 = 1 or -1 for up or down arrows
%     arg 2 = solid or dashed or loosely dashed, etc.
\newcommand{\notelinewitharrows}[2]{%
    \needspace{0.1\baselineskip}%
    \vbox{\begin{tikzpicture}%
        \setlength{\sblen}{\linewidth}%
        \setlength{\sbht}{#1\sbsz}\setlength{\sbht}{1.4\sbht}%
        \setlength{\sbhadj}{#1\sblw}\setlength{\sbhadj}{0.25\sbhadj}%
        \filldraw (\sblen, 0) -- (\sblen - \sbsz, \sbht) -- (\sblen - 2\sbsz, 0) -- (\sblen, 0);
        \draw[line width=\sblw, #2] (2\sbsz - \sblw, \sbhadj) -- (\sblen - 2\sbsz + \sblw, \sbhadj);
        \filldraw (0, 0) -- (\sbsz, \sbht) -- (0 + 2\sbsz, 0) -- (0, 0);
    \end{tikzpicture}}}

% \adviceuserline is a helper command that makes a red horizontal line, up or down arrows, and some text:
% arg 1 = 1 or -1 for up or down arrows
% arg 2 = solid or dashed or loosely dashed, etc.
% arg 3 = text
% arg 4 = text width
\newcommand{\adviceuserline}[4]{%
    \needspace{0.1\baselineskip}%
    \vbox to 1\baselineskip {\begin{tikzpicture}%
        \setlength{\sbtw}{#4}%
        \setlength{\sblen}{\linewidth}%
        \setlength{\sbht}{#1\sbsz}\setlength{\sbht}{1.4\sbht}%
        \setlength{\sbhadj}{#1\sblw}\setlength{\sbhadj}{0.25\sbhadj}%
        \filldraw[color=red!80!black] (\sblen, 0) -- (\sblen - \sbsz, \sbht) -- (\sblen - 2\sbsz, 0) -- (\sblen, 0);
        \draw[line width=\sblw, color=red!80!black, #2] (2\sbsz - \sblw, \sbhadj) -- (0.5\sblen - 0.5\sbtw, \sbhadj);
        \draw[line width=\sblw, color=red!80!black, #2] (0.5\sblen + 0.5\sbtw, \sbhadj) -- (\sblen - 2\sbsz + \sblw, \sbhadj);
        \filldraw[color=red!80!black] (0, 0) -- (\sbsz, \sbht) -- (0 + 2\sbsz, 0) -- (0, 0);
        \node[color=red!80!black] at (0.5\sblen, 0) {\large  \textsf{\textup{#3}}};
    \end{tikzpicture}}}

% \adviceimpline is a helper command that makes a green horizontal line, up or down arrows, and some text:
% arg 1 = 1 or -1 for up or down arrows
% arg 2 = solid or dashed or loosely dashed, etc.
% arg 3 = text
% arg 4 = text width
\newcommand{\adviceimpline}[4]{%
    \needspace{0.1\baselineskip}%
    \vbox to 1\baselineskip {\begin{tikzpicture}%
        \setlength{\sbtw}{#4}%
        \setlength{\sblen}{\linewidth}%
        \setlength{\sbht}{#1\sbsz}\setlength{\sbht}{1.4\sbht}%
        \setlength{\sbhadj}{#1\sblw}\setlength{\sbhadj}{0.25\sbhadj}%
        \filldraw[color=green!60!black] (\sblen, 0) -- (\sblen - \sbsz, \sbht) -- (\sblen - 2\sbsz, 0) -- (\sblen, 0);
        \draw[line width=\sblw, color=green!60!black, #2] (2\sbsz - \sblw, \sbhadj) -- (0.5\sblen - 0.5\sbtw, \sbhadj);
        \draw[line width=\sblw, color=green!60!black, #2] (0.5\sblen + 0.5\sbtw, \sbhadj) -- (\sblen - 2\sbsz + \sblw, \sbhadj);
        \filldraw[color=green!60!black] (0, 0) -- (\sbsz, \sbht) -- (0 + 2\sbsz, 0) -- (0, 0);
        \node[color=green!60!black] at (0.5\sblen, 0) {\large  \textsf{\textup{#3}}};
    \end{tikzpicture}}}

% \advicermline is a helper command that makes an orange horizontal line, up or down arrows, and some text:
% arg 1 = 1 or -1 for up or down arrows
% arg 2 = solid or dashed or loosely dashed, etc.
% arg 3 = text
% arg 4 = text width
\newcommand{\advicermline}[4]{%
    \needspace{0.1\baselineskip}%
    \vbox to 1\baselineskip {\begin{tikzpicture}%
        \setlength{\sbtw}{#4}%
        \setlength{\sblen}{\linewidth}%
        \setlength{\sbht}{#1\sbsz}\setlength{\sbht}{1.4\sbht}%
        \setlength{\sbhadj}{#1\sblw}\setlength{\sbhadj}{0.25\sbhadj}%
        \filldraw[color=orange!60!black] (\sblen, 0) -- (\sblen - \sbsz, \sbht) -- (\sblen - 2\sbsz, 0) -- (\sblen, 0);
        \draw[line width=\sblw, color=orange!60!black, #2] (2\sbsz - \sblw, \sbhadj) -- (0.5\sblen - 0.5\sbtw, \sbhadj);
        \draw[line width=\sblw, color=orange!60!black, #2] (0.5\sblen + 0.5\sbtw, \sbhadj) -- (\sblen - 2\sbsz + \sblw, \sbhadj);
        \filldraw[color=orange!60!black] (0, 0) -- (\sbsz, \sbht) -- (0 + 2\sbsz, 0) -- (0, 0);
        \node[color=orange!60!black] at (0.5\sblen, 0) {\large  \textsf{\textup{#3}}};
    \end{tikzpicture}}}

% \ratline is a helper command that makes a purple horizontal line, up or down arrows, and some text:
% arg 1 = 1 or -1 for up or down arrows
% arg 2 = solid or dashed or loosely dashed, etc.
% arg 3 = text
% arg 4 = text width
\newcommand{\ratline}[4]{%
    \needspace{0.1\baselineskip}%
    \vbox to 1\baselineskip {\begin{tikzpicture}%
        \setlength{\sbtw}{#4}%
        \setlength{\sblen}{\linewidth}%
        \setlength{\sbht}{#1\sbsz}\setlength{\sbht}{1.4\sbht}%
        \setlength{\sbhadj}{#1\sblw}\setlength{\sbhadj}{0.25\sbhadj}%
        \filldraw[color=purple!40] (\sblen, 0) -- (\sblen - \sbsz, \sbht) -- (\sblen - 2\sbsz, 0) -- (\sblen, 0);
        \draw[line width=\sblw, color=purple!40, #2] (2\sbsz - \sblw, \sbhadj) -- (0.5\sblen - 0.5\sbtw, \sbhadj);
        \draw[line width=\sblw, color=purple!40, #2] (0.5\sblen + 0.5\sbtw, \sbhadj) -- (\sblen - 2\sbsz + \sblw, \sbhadj);
        \filldraw[color=purple!40] (0, 0) -- (\sbsz, \sbht) -- (0 + 2\sbsz, 0) -- (0, 0);
        \node[color=purple!90] at (0.5\sblen, 0) {\large  \textsf{\textup{#3}}};
    \end{tikzpicture}}}

% \linewitharrows is a helper command that makes a blue horizontal line, up or down arrows, and some text:
% arg 1 = 1 or -1 for up or down arrows
% arg 2 = solid or dashed or loosely dashed, etc.
% arg 3 = text
% arg 4 = text width
\newcommand{\linewitharrows}[4]{%
    \needspace{0.1\baselineskip}%
    \vbox to 1\baselineskip {\begin{tikzpicture}%
        \setlength{\sbtw}{#4}%
        \setlength{\sblen}{\linewidth}%
        \setlength{\sbht}{#1\sbsz}\setlength{\sbht}{1.4\sbht}%
        \setlength{\sbhadj}{#1\sblw}\setlength{\sbhadj}{0.25\sbhadj}%
        \filldraw[color=blue!40] (\sblen, 0) -- (\sblen - \sbsz, \sbht) -- (\sblen - 2\sbsz, 0) -- (\sblen, 0);
        \draw[line width=\sblw, color=blue!40, #2] (2\sbsz - \sblw, \sbhadj) -- (0.5\sblen - 0.5\sbtw, \sbhadj);
        \draw[line width=\sblw, color=blue!40, #2] (0.5\sblen + 0.5\sbtw, \sbhadj) -- (\sblen - 2\sbsz + \sblw, \sbhadj);
        \filldraw[color=blue!40] (0, 0) -- (\sbsz, \sbht) -- (0 + 2\sbsz, 0) -- (0, 0);
        \node[color=blue!90] at (0.5\sblen, 0) {\large  \textsf{\textup{#3}}};
    \end{tikzpicture}}}

\newcommand{\VSPb}{\vspace{0.5ex plus 5ex minus 0.25ex}}
\newcommand{\VSPa}{\vspace{0.25ex plus 5ex minus 0.25ex}}

% C
\newcommand{\cspecificstart}{\needspace{\sbns}\linewitharrows{-1}{solid}{C}{3em}}
\newcommand{\cspecificend}{\linewitharrows{1}{solid}{C}{3em}\VSPa}

% Fortran
\newcommand{\fortranspecificstart}{\VSPb\linewitharrows{-1}{solid}{Fortran}{6em}\VSPa}
\newcommand{\fortranspecificend}{\VSPb\linewitharrows{1}{solid}{Fortran}{6em}\VSPa}

% Note
\newcommand{\notestart}{\VSPb\notelinewitharrows{-1}{solid}\VSPa}
\newcommand{\noteend}{\VSPb\notelinewitharrows{1}{solid}\VSPa}
% convenience macro for formatting the word "Note:" at the beginning of note blocks:
\newcommand{\noteheader}{{\textrm{\textsf{\textbf\textup\normalsize{{{{Note: }}}}}}}}

% Rationale
\newcommand{\rationalestart}{\VSPb\ratline{-1}{dashed}{Rationale}{7em}\VSPa}
\newcommand{\rationaleend}{\VSPb\ratline{1}{dashed}{}{0em}\VSPa}

% Advice to users
\newcommand{\adviceuserstart}{\VSPb\adviceuserline{-1}{solid}{Advice to users}{10em}\VSPa}
\newcommand{\adviceuserend}{\VSPb\adviceuserline{1}{solid}{}{0em}\VSPa}

% Advice to implementers
\newcommand{\adviceimplstart}{\VSPb\adviceimpline{-1}{solid}{Advice to PMIx library implementers}{20em}\VSPa}
\newcommand{\adviceimplend}{\VSPb\adviceimpline{1}{solid}{}{0em}\VSPa}

% Advice to hosts
\newcommand{\advicermstart}{\VSPb\advicermline{-1}{solid}{Advice to PMIx server hosts}{16em}\VSPa}
\newcommand{\advicermend}{\VSPb\advicermline{1}{solid}{}{0em}\VSPa}

% Required attributes
\newcommand{\reqattrstart}{\VSPb\adviceuserline{-1}{dashed}{Required Attributes}{16em}\VSPa}
\newcommand{\reqattrend}{\VSPb\adviceuserline{1}{dashed}{}{0em}\VSPa}

% Optional attributes
\newcommand{\optattrstart}{\VSPb\adviceimpline{-1}{dashed}{Optional Attributes}{16em}\VSPa}
\newcommand{\optattrend}{\VSPb\adviceimpline{1}{dashed}{}{0em}\VSPa}


%%%%%%%%%%%%%%%%%%%%%%%%%%%%%%%%%%%%%%%%%%%%%%%%%%%%%%%%%%%%%%%%%%%%%%%%%%%%%%%%%%%%%%%%%%%%%%
% Glossary formatting

\newcommand{\glossaryterm}[1]{\needspace{1ex}
\begin{adjustwidth}{-0.75in}{0.0in}
\nolinenumbers\parbox[b][-0.95\baselineskip][t]{1.4in}{\flushright \textbf{#1}}
\end{adjustwidth}\linenumbers}

\newcommand{\glossarydefstart}{
\begin{adjustwidth}{0.79in}{0.0in}}

\newcommand{\glossarydefend}{
\end{adjustwidth}\vspace{-1.5\baselineskip}}


%%%%%%%%%%%%%%%%%%%%%%%%%%%%%%%%%%%%%%%%%%%%%%%%%%%%%%%%%%%%%%%%%%%%%%%%%%%%%%%%%%%%%%%%%%%%%
% Indexing and Table of Contents

\usepackage{imakeidx}
\usepackage[nodotinlabels]{titletoc}   % required for its [nodotinlabels] option

% Clickable links in TOC and index:
\usepackage[hyperindex=true,linktocpage=true]{hyperref}
\hypersetup{
  bookmarksnumbered = true,
  bookmarksopen     = false,
  colorlinks  = true, % Colors links instead of red boxes
  urlcolor    = blue, % Color for external links
  linkcolor   = blue  % Color for internal links
}

% \url styled in Roman font.
\urlstyle{rm}

%%%%%%%%%%%%%%%%%%%%%%%%%%%%%%%%%%%%%%%%%%%%%%%%%%%%%%%%%%%%%%%%%%%%%%%%%%%%%%%%%%%%%%%%%%%%%
% Cross reference macros
% This defines:
%     \specref          cross reference label as "Section X on page Y"
%     \refsection       Link this label to a specific section label in the document
%
%     \declarstruct     Mark the declaration of a structure
%     \refstruct        Reference the structure declaration
%
%     \declareapi       Mark the declaration of an API function
%     \refapi           Reference the API declaration
%
%     \declaremacro     Mark the declaration of a user-level macro
%     \refmacro         Reference the macro declaration
%

\newcommand{\chapterref}[1]{Chapter~\ref{#1} on page~\pageref{#1}}
\newcommand{\specref}[1]{Section~\ref{#1} on page~\pageref{#1}}

\newcommand{\refsection}[2]{\hyperref[#1]{#2}}

\newcommand{\declarestruct}[1]{\index{#1!Definition|textbf} \label{struct:#1}}
\newcommand{\refstruct}[1]{\index{#1}\hyperref[struct:#1]{\code{#1}}}
\newcommand{\structref}[1] {\refstruct{#1}}
\newcommand{\specrefstruct}[1]{Section~\ref{struct:#1} on page~\pageref{struct:#1}}

\newcommand{\declareapi}[1]{\index{#1!Definition|textbf} \label{api:#1}}
\newcommand{\refapi}[1]{\index{#1}\hyperref[api:#1]{\code{#1}}}
\newcommand{\argapi}[1] {\refapi{#1}}

\newcommand{\refconst}[1]{\hyperref[const:#1]{\code{#1}}}

\newcommand{\declareattr}[1]{\index{#1!Definition|textbf} \label{attr:#1}}

\newcommand{\refarg}[1] {{\textrm{\textmd{\itshape{#1}}}}}
\newcommand{\argref}[1] {\refarg{#1}}

\newcommand{\declaremacro}[1]{\index{#1!Definition|textbf} \label{macro:#1}}
\newcommand{\refmacro}[1]{\index{#1}\hyperref[macro:#1]{\code{#1}}}

\newcommand{\declareterm}[1]{\index{#1!Definition|textbf} \label{macro:#1}}
\newcommand{\refterm}[1]{\index{#1}\hyperref[macro:#1]{\code{#1}}}

% Place in text for in-text questions during review
\newcommand{\rcomment}[1]{(REVIEW COMMENT: \textbf{#1})}

%%%%%%%%%%%%%%%%%%%%%%%%%%%%%%%%%%%%%%%%%%%%%%%%%%%%%%%%%%%%%%%%%%%%%%%%%%%%%%%%%%%%%%%%%%%%%
% Set default fonts:
\rmfamily\mdseries\upshape\normalsize

%%%%%%%%%%%%%%%%%%%%%%%%%%%%%%%%%%%%%%%%%%%%%%%%%
% Define a divider for splitting implementer vs host attribute requirements/options
\newcommand{\divider}{\noindent\makebox[\linewidth]{\rule{\linewidth}{0.8pt}}}

\makeindex[intoc,columns=2]

% Watermark for the Unofficial Drafts
% Note: "allpages","evenpages","oddpages" do not work. Likely because of the index.
\ifthenelse{\boolean{is_unofficial_draft}}
  {\usepackage[printwatermark]{xwatermark}
   \newwatermark[firstpage,color=gray!30,angle=45,scale=3,xpos=-5,ypos=0]{Unofficial Draft}
  }{}

%%%%%%%%%%%%%%%%%%%
\usepackage{acronym}
\acrodef{PMI}[PMI]{Process Management Interface}
\acrodef{PMIx}[PMIx]{Process Management Interface - Exascale}
\acrodef{HPC}[HPC]{High Performance Computing}
\acrodef{MPI}[MPI]{Message Passing Interface}
\acrodef{MPE}[MPE]{Message Passing Environment}

\acrodef{RM}[RM]{resource manager}
\acrodef{RTE}[RTE]{RunTime Environment}
\acrodef{SMS}[SMS]{system management software stack}
\acrodef{ASC}[ASC]{Administrative Steering Committee}
\acrodef{WLM}[WLM]{workload manager}
\acrodef{GDS}[GDS]{global data storage}
\acrodef{BCX}[BCX]{business card exchange}

\acrodef{PID}[PID]{process identifier}
\acrodef{URI}[URI]{uniform resource identifier}
\acrodef{CIDR}[CIDR]{Classless Inter-Domain Routing}
\acrodef{XML}[XML]{eXtensible Markup Language}

\acrodef{RAS}[RAS]{Reliability and Survivability}
\acrodef{API}[API]{Application Programming Interface}
\acrodef{PRRTE}[PRRTE]{PMIx-based Reference RunTime Environment}
\acrodef{PRI}[PRI]{PMIx Reference Implementation}
\acrodef{ECC}[ECC]{Error Check and Correction}
\acrodef{FM}[FM]{Fabric Manager}
\acrodef{IO}[IO]{Input/Output}
\acrodef{MPMD}{Multiple Program Multiple Data}
\acrodef{PU}{Processing Unit}
\acrodef{HWLOC}{Hardware Locality}
\acrodef{OS}{Operating System}

%%%%%%%%%%%%%%%%%%%


\begin{document}
%
% Title page
%
    \pagenumbering{roman}
    %%%%%%%%%%%%%%%%%%%%%%%%%%%%%%%%%%%%%%%%%%%%%%%%%
% Title page
%%%%%%%%%%%%%%%%%%%%%%%%%%%%%%%%%%%%%%%%%%%%%%%%%

  \begin{titlepage}
    \begin{flushleft}
     \hspace{-6em} \includegraphics[width=0.4\textwidth]{figs/pmix-logo.png}
    \end{flushleft}

    \begin{adjustwidth}{-0.75in}{0in}
    \begin{center}
      \Huge
      \textsf{Process Management Interface\\for Exascale (PMIx) Standard}

      \vspace{1.0in}
	  \huge
      \textbf{Version \VER{}}

      \vspace{0.15in}
	  \Large
      \textbf{\VERDATE}

    \end{center}
    \end{adjustwidth}

    \vspace{1.2in}

\par
This document describes the Process Management Interface for Exascale (PMIx) Standard, version \VER{}.

\par
\textbf{Comments:}
Please provide comments on the PMIx Standard by filing issues on the document repository \url{https://github.com/pmix/pmix-standard/issues} or by sending them to the PMIx Community mailing list at \url{https://groups.google.com/forum/#!forum/pmix}.
Comments should include the version of the PMIx standard you are commenting about, and the page, section, and line numbers that you are referencing.
Please note that messages sent to the mailing list from an unsubscribed e-mail address will be ignored.

\vfill

\begin{adjustwidth}{0pt}{1em}\setlength{\parskip}{0.25\baselineskip}%
Copyright \textsuperscript{\textcopyright} 2018-2020 PMIx \acf{ASC}.\\
Permission to copy without fee all or part of this material is granted,
provided the PMIx \ac{ASC} copyright notice and
the title of this document appear, and notice is given that copying is by
permission of PMIx \ac{ASC}.
\end{adjustwidth}

  \end{titlepage}

%%%%%%%%%%%%%%%%%%%%%%%%%%%%%%%%%%%%%%%%%%%%%%%%%
% Blank page
%%%%%%%%%%%%%%%%%%%%%%%%%%%%%%%%%%%%%%%%%%%%%%%%%
\clearpage
\thispagestyle{empty}
\phantom{a}
\begin{center}
\emph{This page intentionally left blank}
\end{center}

\vfill



%
% Table of contents
%
    \setcounter{page}{0}
    \setcounter{tocdepth}{2}

    \begin{spacing}{1.3}
        \RedeclareSectionCommand[tocnumwidth=2.6em]{section}
        \RedeclareSectionCommand[tocnumwidth=3.7em,tocindent=4.1em]{subsection}
        \tableofcontents
    \end{spacing}

%
% Introductory materials
%
    % Uncomment the next line to enable line numbering on the main body text:
    \linenumbers\pagewiselinenumbers
    \newpage\pagenumbering{arabic}
    \setcounter{chapter}{0}  % start chapter numbering here

%
% Chapters
%
    % Introduction to PMIx
    %  - Overview, Goals, Arch.
    %%%%%%%%%%%%%%%%%%%%%%%%%%%%%%%%%%%%%%%%%%%%%%%%%
% Chapter: Introduction
%%%%%%%%%%%%%%%%%%%%%%%%%%%%%%%%%%%%%%%%%%%%%%%%%
\chapter{Introduction}
\label{chap:intro}

The \ac{PMI} has been used for quite some time as a means of exchanging wireup information needed for inter-process communication.
Two versions (PMI-1 and PMI-2) have been released as part of the MPICH effort, with PMI-2 demonstrating better scaling properties than its PMI-1 predecessor. However, two significant challenges face the \ac{HPC} community as it continues to move towards machines capable of exaflop and higher performance levels:

\begin{itemize}
\item the physical scale of the machines, and the corresponding number of total processes they support, is expected to reach levels approaching  1 million processes executing across 100 thousand nodes. Prior methods for initiating applications relied on exchanging communication endpoint information between the processes, either directly or in some form of hierarchical collective operation. Regardless of the specific mechanism employed, the exchange across such large applications would consume considerable time, with estimates running in excess of 5-10 minutes; and
\item whether it be hybrid applications that combine OpenMP threading operations with MPI, or application-steered workflow computations, the HPC community is experiencing an unprecedented wave of new approaches for computing at exascale levels. One common thread across the proposed methods is an increasing need for orchestration between the application and the \ac{SMS} comprising the scheduler (a.k.a. the \ac{WLM}), the \ac{RM}, global file system, fabric, and other subsystems. The lack of available support for application-to-SMS integration has forced researchers to develop "virtual" environments that hide the SMS behind a customized abstraction layer, but this results in considerable duplication of effort and a lack of portability.
\end{itemize}

\ac{PMIx} represents an attempt to resolve these questions by providing an extended version of the \ac{PMI} definitions specifically designed to support clusters up to exascale and larger sizes.
The overall objective of the project is not to branch the existing definitions -- in fact, \ac{PMIx} fully supports both of the existing PMI-1 and PMI-2 \acp{API} -- but rather to:

\begin{compactalphaenum}
\item add flexibility to the existing \acp{API} by adding an array of key-value ``attribute'' pairs to each \ac{API} signature that allows implementers to customize the behavior of the \ac{API} as future needs emerge without having to alter or create new variants of it;

\item add new APIs that provide extended capabilities such as asynchronous event notification plus dynamic resource allocation and management;

\item establish a collaboration between \ac{SMS} subsystem providers including resource manager, fabric, file system, and programming library developers to define integration points between the various subsystems as well as agreed upon definitions for associated \acp{API}, attribute names, and data types;

\item form a standards-like body for the definitions; and

\item provide a reference implementation of the \ac{PMIx} standard.
\end{compactalphaenum}

Complete information about the \ac{PMIx} standard and affiliated projects can be found at the \ac{PMIx} web site: \url{https://pmix.org}


%%%%%%%%%%%%%%%%%%%%%%%%%%%%%%%%%%%%%%%%%%%%%%%%%
%%%%%%%%%%%%%%%%%%%%%%%%%%%%%%%%%%%%%%%%%%%%%%%%%
\section{Charter}
\label{chap:intro:charter}

The charter of the PMIx community is to:
\begin{itemize}
\item Define a set of agnostic APIs (not affiliated with any specific programming model or code base) to support interactions between application processes and the \ac{SMS}.
\item Develop an open source (non-copy-left licensed) standalone ``reference'' library implementation to facilitate adoption of the \ac{PMIx} standard.
\item Retain transparent backward compatibility with the existing PMI-1 and PMI-2 definitions, any future \ac{PMI} releases, and across all \ac{PMIx} versions.
\item Support the ``Instant On'' initiative for rapid startup of applications at exascale and beyond.
\item Work with the \ac{HPC} community to define and implement new \acp{API} that support evolving programming model requirements for application interactions with the \ac{SMS}.
\end{itemize}

Participation in the \ac{PMIx} community is open to anyone, and not restricted to only code contributors to the reference implementation.


%%%%%%%%%%%%%%%%%%%%%%%%%%%%%%%%%%%%%%%%%%%%%%%%%
%%%%%%%%%%%%%%%%%%%%%%%%%%%%%%%%%%%%%%%%%%%%%%%%%
\section{PMIx Standard Overview}
\label{chap:intro:std_overview}

The \ac{PMIx} Standard defines and describes the interface developed by the \acf{PRI}.
Much of this document is specific to the \acf{PRI}'s design and implementation.
Specifically the standard describes the functionality provided by the \ac{PRI}, and what the \ac{PRI} requires of the clients and \acfp{RM} that use it's interface.

%%%%%%%%%%%
\subsection{Who should use the standard?}

The \ac{PMIx} Standard informs PMIx clients and \acp{RM} of the syntax and semantics of the \ac{PMIx} APIs.

\ac{PMIx} clients (e.g., tools, \ac{MPE} libraries) can use this standard to understand the set of attributes provided by various APIs of the \ac{PRI} and their intended behavior.
Additional information about the rationale for the selection of specific interfaces and attributes is also provided.

\ac{PMIx}-enabled \acp{RM} can use this standard to understand the expected behavior required of them when they support various interfaces/attributes.
In addition, optional features and suggestions on behavior are also included in the discussion to help guide \ac{RM} design and implementation.

%%%%%%%%%%%
\subsection{What is defined in the standard?}

The \ac{PMIx} Standard defines and describes the interface developed by the \acf{PRI}.
It defines the set of attributes that the \ac{PRI} supports; the set of attributes that are required of a \ac{RM} to support, for a given interface; and the set of optional attributes that an \ac{RM} may choose to support, for a given interface.

%%%%%%%%%%%
\subsection{What is \emph{not} defined in the standard?}

No standards body can require an implementer to support something in their standard, and \ac{PMIx} is no different in that regard. While an implementer of the \ac{PMIx} library itself must at least include the standard \ac{PMIx} headers and instantiate each function, they are free to return ``not supported'' for any function they choose not to implement.

This also applies to the host environments. Resource managers and other system management stack components retain the right to decide on support of a particular function. The \ac{PMIx} community continues to look at ways to assist \ac{SMS} implementers in their decisions by highlighting functions that are critical to basic application execution (e.g., \refapi{PMIx_Get}), while leaving flexibility for tailoring a vendor's software for their target market segment.

One area where this can become more complicated is regarding the attributes that provide information to the client process and/or control the behavior of a \ac{PMIx} standard \ac{API}. For example, the \refattr{PMIX_TIMEOUT} attribute can be used to specify the time (in seconds) before the requested operation should time out. The intent of this attribute is to allow the client to avoid ``hanging'' in a request that takes longer than the client wishes to wait, or may never return (e.g., a \refapi{PMIx_Fence} that a blocked participant never enters).

If an application (for example) truly relies on the \refattr{PMIX_TIMEOUT} attribute in a call to \refapi{PMIx_Fence}, it should set the required flag in the \refstruct{pmix_info_t} for that attribute. This informs the library and its \ac{SMS} host that it must return an immediate error if this attribute is not supported. By not setting the flag, the library and \ac{SMS} host are allowed to treat the attribute as optional, ignoring it if support is not available.

It is therefore critical that users and application implementers:

\begin{compactalphaenum}
\item consider whether or not a given attribute is required, marking it accordingly; and

\item check the return status on all \ac{PMIx} function calls to ensure support was present and that the request was accepted. Note that for non-blocking \acp{API}, a return of \refconst{PMIX_SUCCESS} only indicates that the request had no obvious errors and is being processed – the eventual callback will return the status of the requested operation itself.
\end{compactalphaenum}

While a \ac{PMIx} library implementer, or an \ac{SMS} component server, may choose to support a particular \ac{PMIx} \ac{API}, they are not required to support every attribute that might apply to it. This would pose a significant barrier to entry for an implementer as there can be a broad range of applicable attributes to a given \ac{API}, at least some of which may rarely be used. The \ac{PMIx} community is attempting to help differentiate the attributes by indicating those that are generally used (and therefore, of higher importance to support) vs those that a ``complete implementation'' would support.

Note that an environment that does not include support for a particular attribute/\ac{API} pair is not ``incomplete'' or of lower quality than one that does include that support. Vendors must decide where to invest their time based on the needs of their target markets, and it is perfectly reasonable for them to perform cost/benefit decisions when considering what functions and attributes to support.

The flip side of that statement is also true: Users who find that their current vendor does not support a function or attribute they require may raise that concern with their vendor and request that the implementation be expanded. Alternatively, users may wish to utilize the \ac{PRRTE} as a ``shim'' between their application and the host environment as it might provide the desired support until the vendor can respond. Finally, in the extreme, one can exploit the portability of PMIx-based applications to change vendors.

%%%%%%%%%%%
\subsection{General Guidance for PMIx Users and Implementors}

The \ac{PMIx} Standard defines the behavior of the \acf{PRI}.
A complete system harnessing the \ac{PMIx} interface requires an agreement between the \ac{PMIx} client, be it a tool or library, and the \ac{PMIx}-enabled \ac{RM}.
The \ac{PRI} acts as an intermediary between these two entities by providing a standard API for the exchange of requests and responses.
The degree to which the \ac{PMIx} client and the \ac{PMIx}-enabled \ac{RM} may interact needs to be defined by those developer communities.
The \ac{PMIx} standard can be used to define the specifics of this interaction.

\ac{PMIx} clients (e.g., tools, \ac{MPE} libraries) may find that they depend only on a small subset of interfaces and attributes to work correctly.
\ac{PMIx} clients are strongly advised to define a document itemizing the \ac{PMIx} interfaces and associated attributes that are required for correct operation, and are optional but recommended for full functionality.
The \ac{PMIx} standard cannot define this list for all given \ac{PMIx} clients, but such a list is valuable to \acp{RM} desiring to support these clients.

\ac{PMIx}-enabled \acp{RM} may choose to implement a subset of the \ac{PMIx} standard and/or define attributes beyond those defined herein.
\ac{PMIx}-enabled \acp{RM} are strongly advised to define a document itemizing the \ac{PMIx} interfaces and associated attributes they support, with any annotations about behavior limitations.
The \ac{PMIx} standard cannot define this list for all given \ac{PMIx}-enabled \acp{RM}, but such a list is valuable to \ac{PMIx} clients desiring to support a broad range of \ac{PMIx}-enabled \acp{RM}.



%%%%%%%%%%%%%%%%%%%%%%%%%%%%%%%%%%%%%%%%%%%%%%%%%
%%%%%%%%%%%%%%%%%%%%%%%%%%%%%%%%%%%%%%%%%%%%%%%%%
\section{PMIx Architecture Overview}
\label{chap:intro:arch_overview}

This section presents a brief overview of the \ac{PMIx} Architecture~\cite{2017-Castain-EuroMPI}.
Note that this is a conceptual model solely used to help guide the standards process --- it does not represent
a design requirement on any \ac{PMIx} implementation. Instead, the model is used by the
\ac{PMIx} community as a sounding board for evaluating proposed interfaces and avoid unintentionally imposing
constraints on implementers. Built into the model are two guiding principles also reflected in the standard. First,
\ac{PMIx} operates in the mode of a \textit{messenger}, and not a \textit{doer} --- i.e., the role
of \ac{PMIx} is to provide communication between the various participants, relaying requests and returning
responses. The intent of the standard is not to suggest that \ac{PMIx} itself actually perform any of
the defined operations --- this is left to the various \ac{SMS} elements and/or the application. Any exceptions to that intent are left to the discretion of the particular implementation.

\begingroup
\begin{figure*}[ht!]
  \begin{center}
    \includegraphics[clip,width=0.8\textwidth]{figs/PMIxRoles.pdf}
  \end{center}
  \caption{PMIx-SMS Interactions}
  \label{fig:roles}
\end{figure*}
\endgroup


Thus, as the diagram in Fig.~\ref{fig:roles} shows, the application is built against a \ac{PMIx} client library that contains the client-side \acp{API},
attribute definitions, and communication support for interacting with the local \ac{PMIx} server. Intra-process cross-library interactions
are supported at the client level to avoid unnecessary burdens on the server. Orchestration requests are sent to the
local \ac{PMIx} server, which subsequently passes them to the host \ac{SMS} (here represented by an \ac{RM} daemon) using the \ac{PMIx} server callback functions the host \ac{SMS} registered during PMIx\_server\_init. The host \ac{SMS} can indicate its lack of support for any operation by simply providing a \textit{NULL} for the associated callback function, or can create a function entry that returns \textit{not supported} when called.

The conceptual model places the burden of fulfilling the request on the host \ac{SMS}. This includes performing any
inter-node communications, or interacting with other \ac{SMS} elements. Thus, a client request for a network traffic report
does not go directly from the client to the \ac{FM}, but instead is relayed to the \ac{PMIx} server, and then passed to the host \ac{SMS}
for execution. This architecture reflects the second principle underlying the standard --- namely, that connectivity is to be minimized by channeling all application interactions with the \ac{SMS} through the local \ac{PMIx} server.

Recognizing the burden this places on SMS vendors, the PMIx community has included interfaces by
which the host can request support from local SMS elements. Once the SMS has transferred the request to
an appropriate location, a PMIx server interface can be used to pass the request between SMS subsystems.
For example, a request for network traffic statistics can utilize the
PMIx networking abstractions to retrieve the information from the FM. This reduces the portability and
interoperability issues between the individual subsystems by transferring the burden of defining the
interoperable interfaces from the SMS subsystems to the PMIx community, which continues
to work with those providers to develop the necessary support.

Tools, whether standalone or embedded in job scripts, are an exception to the communication rule and can connect to
any PMIx server providing they are given adequate rendezvous information. The PMIx conceptual model views the
collection of PMIx servers as a cloud-like conglomerate --- i.e., orchestration and information requests can be
given to any server regardless of location. However, tools frequently execute on locations that may not house an
operating PMIx server --- e.g., a users notebook computer. Thus, tools need the ability to remotely connect to
the PMIx server ``cloud''.

The scope of the PMIx standard therefore spans the range of these interactions, between client-and-SMS and between SMS
subsystems. Note again that this does not impose a requirement on any given PMIx implementation to cover the entire
range --- implementers are free to return \textit{not supported} from any PMIx function.


%%%%%%%%%%%
\subsection{The \acf{PRI}}

The \ac{PMIx} community has committed to providing a complete, reference implementation of each version of the standard. Note that the definition of the \ac{PMIx} Standard is not contingent upon use of the \acf{PRI} --- any implementation that supports the defined \acp{API} is a \ac{PMIx} Standard compliant implementation.
The \ac{PRI} is provided solely for the following purposes:
\begin{itemize}
\item Validation of the standard.\\
No proposed change and/or extension to the \ac{PMIx} standard is accepted without an accompanying prototype implementation in the \ac{PRI}.
This ensures that the proposal has undergone at least some minimal level of scrutiny and testing before being considered.
\item Ease of adoption.\\
The \ac{PRI} is designed to be particularly easy for resource managers (and the \ac{SMS} in general) to adopt, thus facilitating a rapid uptake into that community for application portability.
Both client and server \ac{PMIx} libraries are included, along with examples of client usage and server-side integration.
A list of supported environments and versions is maintained on the \ac{PMIx} web site \url{https://pmix.org/support/faq/what-apis-are-supported-on-my-rm/}
\end{itemize}

The \ac{PRI} does provide some internal implementations that lie outside the scope of the \ac{PMIx} standard. This includes several convenience macros as well as support for consolidating collectives for optimization purposes (e.g., the \ac{PMIx} server aggregates all local \refapi{PMIx_Fence} calls before
passing them to the \ac{SMS} for global execution). In a few additional cases, the \ac{PMIx} community (in partnership with the \ac{SMS} subsystem providers) have determined that a base level of support for a given operation can best be portably provided by including it in the \ac{PRI}.

Instructions for downloading, and installing the \ac{PRI} are available
on the community's web site \url{https://pmix.org/code/getting-the-reference-implementation/}.The \ac{PRI} targets support for the Linux operating system.
A reasonable effort is made to support all major, modern Linux distributions; however, validation is limited to the most recent 2-3 releases of RedHat Enterprise Linux (RHEL), Fedora, CentOS, and SUSE Linux Enterprise Server (SLES).
In addition, development support is maintained for Mac OSX.
Production support for vendor-specific operating systems is included as provided by the vendor.


%%%%%%%%%%%
\subsection{The PMIx Reference RunTime Environment (PRRTE)}

The \ac{PMIx} community has also released \ac{PRRTE} --- i.e., a runtime environment
containing the reference implementation and capable of operating within a host \ac{SMS}. \ac{PRRTE}
provides an easy way of exploring \ac{PMIx} capabilities and testing PMIx-based
applications outside of a PMIx-enabled environment by providing a ``shim'' between the application and the host environment that includes full support for the \ac{PRI}. The intent of \ac{PRRTE} is not to replace any existing production environment, but rather to enable developers to work on systems that do not yet feature a PMIx-enabled host \ac{SMS} or one that lacks a \ac{PMIx} feature of interest. Instructions for downloading,
installing, and using \ac{PRRTE} are available
on the community's web site \url{https://pmix.org/code/getting-the-pmix-reference-server/}

%%%%%%%%%%%%%%%%%%%%%%%%%%%%%%%%%%%%%%%%%%%%%%%%%
\section{Organization of this document}

The remainder of this document is structured as follows:

\begin{itemize}
\item Introduction and Overview in \chapterref{chap:intro}
\item Terms and Conventions in \chapterref{chap:terms}
\item Data Structures and Types in \chapterref{chap:struct}
\item \ac{PMIx} Initialization and Finalization in \chapterref{chap:api_init}
\item Key/Value Management in \chapterref{chap:api_kv_mgmt}
\item Process Management in \chapterref{chap:api_proc_mgmt}
\item Job Management in \chapterref{chap:api_job_mgmt}
\item Event Notification in \chapterref{chap:api_event}
\item Data Packing and Unpacking in \chapterref{chap:api_data_mgmt}
\item \ac{PMIx} Server Specific Interfaces in \chapterref{chap:api_server}
\end{itemize}

%%%%%%%%%%%%%%%%%%%%%%%%%%%%%%%%%%%%%%%%%%%%%%%%%
%%%%%%%%%% History: Version 1.0
\section{Version 1.0: June 12, 2015}

\par
The \ac{PMIx} version 1.0 \textit{ad hoc} standard was defined in the \acf{PRI} header files as part of the \ac{PRI} v1.0.0 release prior to the creation of the formal \ac{PMIx} 2.0 standard.
Below are a summary listing of the interfaces defined in the 1.0 headers.

\begin{itemize}
\item Client APIs
\begin{itemize}
\item PMIx\_Init, \refapi{PMIx_Initialized}, \refapi{PMIx_Abort}, \refapi{PMIx_Finalize}
\item \refapi{PMIx_Put}, \refapi{PMIx_Commit},
\item \refapi{PMIx_Fence}, \refapi{PMIx_Fence_nb}
\item \refapi{PMIx_Get}, \refapi{PMIx_Get_nb}
\item \refapi{PMIx_Publish}, \refapi{PMIx_Publish_nb}
\item \refapi{PMIx_Lookup}, \refapi{PMIx_Lookup}
\item \refapi{PMIx_Unpublish}, \refapi{PMIx_Unpublish_nb}
\item \refapi{PMIx_Spawn}, \refapi{PMIx_Spawn_nb}
\item \refapi{PMIx_Connect}, \refapi{PMIx_Connect_nb}
\item \refapi{PMIx_Disconnect}, \refapi{PMIx_Disconnect_nb}
\item \refapi{PMIx_Resolve_nodes}, \refapi{PMIx_Resolve_peers}
\end{itemize}
\item Server APIs
\begin{itemize}
\item \refapi{PMIx_server_init}, \refapi{PMIx_server_finalize}
\item \refapi{PMIx_generate_regex}, \refapi{PMIx_generate_ppn}
\item \refapi{PMIx_server_register_nspace}, \refapi{PMIx_server_deregister_nspace}
\item \refapi{PMIx_server_register_client}, \refapi{PMIx_server_deregister_client}
\item \refapi{PMIx_server_setup_fork}, \refapi{PMIx_server_dmodex_request}
\end{itemize}
\item Common APIs
\begin{itemize}
\item \refapi{PMIx_Get_version}, \refapi{PMIx_Store_internal}, \refapi{PMIx_Error_string}
\item PMIx_Register_errhandler, PMIx_Deregister_errhandler, PMIx_Notify_error
\end{itemize}
\end{itemize}

The \code{PMIx_Init} \ac{API} was subsequently modified in the \ac{PRI} release v1.1.0.

%%%%%%%%%%%%%%%%%%%%%%%%%%%%%%%%%%%%%%%%%%%%%%%%%
%%%%%%%%%% History: Version 2.0
\section{Version 2.0: Sept. 2018}

The following \acp{API} were introduced in v2.0 of the PMIx Standard:

\begin{itemize}
\item Client APIs
\begin{itemize}
\item \refapi{PMIx_Query_info_nb}, \refapi{PMIx_Log_nb}
\item \refapi{PMIx_Allocation_request_nb}, \refapi{PMIx_Job_control_nb}, \refapi{PMIx_Process_monitor_nb}, \refmacro{PMIx_Heartbeat}
\end{itemize}
\item Server APIs
\begin{itemize}
\item \refapi{PMIx_server_setup_application}, \refapi{PMIx_server_setup_local_support}
\end{itemize}
\item Tool APIs
\begin{itemize}
\item \refapi{PMIx_tool_init}, \refapi{PMIx_tool_finalize}
\end{itemize}
\item Common APIs
\begin{itemize}
\item \refapi{PMIx_Register_event_handler}, \refapi{PMIx_Deregister_event_handler}
\item \refapi{PMIx_Notify_event}
\item \refapi{PMIx_Proc_state_string}, \refapi{PMIx_Scope_string}
\item \refapi{PMIx_Persistence_string}, \refapi{PMIx_Data_range_string}
\item \refapi{PMIx_Info_directives_string}, \refapi{PMIx_Data_type_string}
\item \refapi{PMIx_Alloc_directive_string}
\item \refapi{PMIx_Data_pack}, \refapi{PMIx_Data_unpack}, \refapi{PMIx_Data_copy}
\item \refapi{PMIx_Data_print}, \refapi{PMIx_Data_copy_payload}
\end{itemize}
\end{itemize}

The \refapi{PMIx_Init} \ac{API} was modified in v2.0 of the standard from its \textit{ad hoc} v1.0 signature to include passing of a \refstruct{pmix_info_t} array for flexibility and ``future-proofing'' of the \ac{API}.
In addition, the PMIx_Notify_error, PMIx_Register_errhandler, and PMIx_Deregister_errhandler \acp{API} were replaced.

%%%%%%%%%%%%%%%%%%%%%%%%%%%%%%%%%%%%%%%%%%%%%%%%%
%%%%%%%%%% History: Version 2.1
\section{Version 2.1: Dec. 2018}

The v2.1 update includes clarifications and corrections from the v2.0 document, plus addition of examples:

\begin{itemize}
    \item Clarify description of \refapi{PMIx_Connect} and \refapi{PMIx_Disconnect} \acp{API}.
    \item Explain that values for the \refattr{PMIX_COLLECTIVE_ALGO} are environment-dependent
    \item Identify the namespace/rank values required for retrieving attribute-associated information using the \refapi{PMIx_Get} \ac{API}
    \item Provide definitions for \refterm{session}, \refterm{job}, \refterm{application}, and other terms used throughout the document
    \item Clarify definitions of \refattr{PMIX_UNIV_SIZE} versus \refattr{PMIX_JOB_SIZE}
    \item Clarify server module function return values
    \item Provide examples of the use of \refapi{PMIx_Get} for retrieval of information
    \item Clarify the use of \refapi{PMIx_Get} versus \refapi{PMIx_Query_info_nb}
    \item Clarify return values for non-blocking \acp{API} and emphasize that callback functions must not be invoked prior to return from the \ac{API}
    \item Provide detailed example for construction of the \refapi{PMIx_server_register_nspace} input information array
    \item Define information levels (e.g., \refterm{session} vs \refterm{job}) and associated attributes for both storing and retrieving values
    \item Clarify roles of \ac{PMIx} server library and host environment for collective operations
    \item Clarify definition of \refattr{PMIX_UNIV_SIZE}
\end{itemize}

%%%%%%%%%%%%%%%%%%%%%%%%%%%%%%%%%%%%%%%%%%%%%%%%%
%%%%%%%%%% History: Version 2.2
\section{Version 2.2: Jan 2019}

The v2.2 update includes the following clarifications and corrections from the v2.1 document:

\begin{itemize}
    \item Direct modex upcall function (\refapi{pmix_server_dmodex_req_fn_t}) cannot complete atomically as the \ac{API} cannot return the requested information except via the provided callback function
    \item Add missing \refstruct{pmix_data_array_t} definition and support macros
    \item Add a rule divider between implementer and host environment required attributes for clarity
    \item Add \refmacro{PMIX_QUERY_QUALIFIERS_CREATE} macro to simplify creation of \refstruct{pmix_query_t} qualifiers
    \item Add \refmacro{PMIX_APP_INFO_CREATE} macro to simplify creation of \refstruct{pmix_app_t} directives
    \item Add flag and \refmacro{PMIX_INFO_IS_END} macro for marking and detecting the end of a \refstruct{pmix_info_t} array
    \item Clarify the allowed hierarchical nesting of the \refattr{PMIX_SESSION_INFO_ARRAY}, \refattr{PMIX_JOB_INFO_ARRAY}, and associated attributes
\end{itemize}

%%%%%%%%%%%%%%%%%%%%%%%%%%%%%%%%%%%%%%%%%%%%%%%%%
%%%%%%%%%% History: Version 3.0
\section{Version 3.0: Dec. 2018}

The following \acp{API} were introduced in v3.0 of the PMIx Standard:

\begin{itemize}
\item Client APIs
\begin{itemize}
\item \refapi{PMIx_Log}, \refapi{PMIx_Job_control}
\item \refapi{PMIx_Allocation_request}, \refapi{PMIx_Process_monitor}
\item \refapi{PMIx_Get_credential}, \refapi{PMIx_Validate_credential}
\end{itemize}
\item Server APIs
\begin{itemize}
\item \refapi{PMIx_server_IOF_deliver}
\item \refapi{PMIx_server_collect_inventory}, \refapi{PMIx_server_deliver_inventory}
\end{itemize}
\item Tool APIs
\begin{itemize}
\item \refapi{PMIx_IOF_pull}, \refapi{PMIx_IOF_push}, \refapi{PMIx_IOF_deregister}
\item \refapi{PMIx_tool_connect_to_server}
\end{itemize}
\item Common APIs
\begin{itemize}
\item \refapi{PMIx_IOF_channel_string}
\end{itemize}
\end{itemize}

The document added a chapter on security credentials, a new section for \ac{IO} forwarding to the Process Management chapter, and a few blocking forms of previously-existing non-blocking \acp{API}. Attributes supporting the new \acp{API} were introduced, as well as additional attributes for a few existing functions.

%%%%%%%%%%%%%%%%%%%%%%%%%%%%%%%%%%%%%%%%%%%%%%%%%
%%%%%%%%%% History: Version 3.1
\section{Version 3.1: Jan. 2019}

The v3.1 update includes clarifications and corrections from the v3.0 document:

\begin{itemize}
    \item Direct modex upcall function (\refapi{pmix_server_dmodex_req_fn_t}) cannot complete atomically as the \ac{API} cannot return the requested information except via the provided callback function
    \item Fix typo in name of \refattr{PMIX_FWD_STDDIAG} attribute
    \item Correctly identify the information retrieval and storage attributes as ``new'' to v3 of the standard
    \item Add missing \refstruct{pmix_data_array_t} definition and support macros
    \item Add a rule divider between implementer and host environment required attributes for clarity
    \item Add \refmacro{PMIX_QUERY_QUALIFIERS_CREATE} macro to simplify creation of \refstruct{pmix_query_t} qualifiers
    \item Add \refmacro{PMIX_APP_INFO_CREATE} macro to simplify creation of \refstruct{pmix_app_t} directives
    \item Add new attributes to specify the level of information being requested where ambiguity may exist (see \ref{api:struct:attributes:retrieval})
    \item Add new attributes to assemble information by its level for storage where ambiguity may exist (see \ref{api:struct:attributes:storage})
    \item Add flag and \refmacro{PMIX_INFO_IS_END} macro for marking and detecting the end of a \refstruct{pmix_info_t} array
    \item Clarify that \refattr{PMIX_NUM_SLOTS} is duplicative of (a) \refattr{PMIX_UNIV_SIZE} when used at the \refterm{session} level and (b) \refattr{PMIX_MAX_PROCS} when used at the \refterm{job} and \refterm{application} levels, but leave it in for backward compatibility.
    \item Clarify difference between \refattr{PMIX_JOB_SIZE} and \refattr{PMIX_MAX_PROCS}
    \item Clarify that \refapi{PMIx_server_setup_application} must be called per-\refterm{job} instead of per-\refterm{application} as the name implies. Unfortunately, this is a historical artifact. Note that both \refattr{PMIX_NODE_MAP} and \refattr{PMIX_PROC_MAP} must be included as input in the \refarg{info} array provided to that function. Further descriptive explanation of the ``instant on'' procedure will be provided in the next version of the \ac{PMIx} Standard.
    \item Clarify how the \ac{PMIx} server expects data passed to the host by \refapi{pmix_server_fencenb_fn_t} should be aggregated across nodes, and provide a code snippet example
\end{itemize}

%%%%%%%%%%%%%%%%%%%%%%%%%%%%%%%%%%%%%%%%%%%%%%%%%
%%%%%%%%%% History: Version 3.2
\section{Version 3.2: Sept. 2020}

The v3.2 update includes clarifications and corrections from the v3.1 document:

\begin{itemize}
    \item Correct an error in the \refapi{PMIx_Allocation_request} function signature, and clarify the allocation ID attributes
    \item Rename the \refattr{PMIX_ALLOC_ID} attribute to \refattr{PMIX_ALLOC_REQ_ID} to clarify that this is a string the user provides as a means to identify their request to query status
    \item Add a new \refattr{PMIX_ALLOC_ID} attribute that contains the identifier (provided by the host environment) for the resulting allocation which can later be used to reference the allocated resources in, for example, a call to \refapi{PMIx_Spawn}
    \item Update the \refapi{PMIx_generate_regex} and \refapi{PMIx_generate_ppn} descriptions to clarify that the output from these generator functions may not be a NULL-terminated string, but instead could be a byte array of arbitrary binary content.
    \item Add a new \refconst{PMIX_REGEX} constant that represents a regular expression data type.
\end{itemize}


    % PMIx Terms and Conventions
    %%%%%%%%%%%%%%%%%%%%%%%%%%%%%%%%%%%%%%%%%%%%%%%%%
% Chapter: Terms and Conventions
%%%%%%%%%%%%%%%%%%%%%%%%%%%%%%%%%%%%%%%%%%%%%%%%%
\chapter{PMIx Terms and Conventions}
\label{chap:terms}

In this chapter we describe some common terms and conventions used throughout
this document. The \ac{PMIx} Standard has adopted the widespread use of
key-value \textit{attributes} to add flexibility to the functionality expressed
in the \acp{API}. Accordingly, the \ac{ASC} has chosen to require that
the definition of each standard \ac{API} include the passing of an array of
attributes. These provide a means of customizing the behavior of the \ac{API}
as future needs emerge without having to alter or create new variants of it. In
addition, attributes provide a mechanism by which researchers can easily
explore new approaches to a given operation without having to modify the
\ac{API} itself.

In an effort to maintain long-term backward compatibility, \ac{PMIx} does not include large numbers of \acp{API} that each focus on a narrow scope of functionality, but instead relies on the definition of fewer generic \acp{API} that include arrays of key-value attributes for ``tuning'' the function's behavior. Thus, modifications to the \ac{PMIx} standard primarily consist of the definition of new attributes along with a description of the \acp{API} to which they relate and the expected behavior when used with those \acp{API}.

The following terminology is used throughout this document:

\begin{itemize}

\item \declareterm{session} 
refers to a set of resources assigned by the \ac{WLM} that has been 
reserved for one or more users. 
A session is identified by a \emph{session ID} that is  
unique within the scope of the governing \acp{WLM}.
Historically, \ac{HPC} sessions have consisted of a static allocation of resources - i.e., a block of resources assigned to a user in response to a specific request and managed as a unified collection. However, this is changing in response to the growing use of dynamic programming models that require on-the-fly allocation and release of system resources. Accordingly, the term \emph{session} in this document refers to a potentially dynamic entity, perhaps comprised of resources accumulated as a result of multiple allocation requests that are managed as a single unit by the \ac{WLM}.

\item \declareterm{job} refers to a set of one or more \emph{applications} executed as a single invocation by the user within a session with a unique identifier, the \emph{job ID}, assigned by the \ac{RM} or launcher. For example, the command line ``\textit{mpiexec -n 1 app1 : -n 2 app2}'' generates a single \ac{MPMD} job containing two applications. A user may execute multiple \emph{jobs} within a given session, either sequentially or concurrently.

\item \declareterm{namespace} refers to a character string value assigned by the \ac{RM} to a \textit{job}.  The requester of the \textit{job} (e.g., \code{mpiexec} or a \ac{PMIx} client) may request a specific \emph{namespace} value.  All \textit{applications} executed as part of that \textit{job} share the same \emph{namespace}. The \emph{namespace} assigned to each \emph{job} must be unique within the scope of the governing \ac{RM} and often is implemented as a string representation of the numerical emph{Job ID}. The \emph{namespace} and \emph{job} terms will be used interchangeably throughout the document.

\item \declareterm{application} represents a set of identical, but not necessarily unique,
execution contexts within a \emph{job}.

\item \declareterm{process} refers to an operating system process, also commonly referred to as a \emph{heavyweight} process. A process is often comprised of multiple \emph{lightweight threads}, commonly known as simply \declaretermAlt{threads}{thread}.

\item \declaretermAlt{client}{clients} refers to a process that was registered with the \ac{PMIx} server prior to being started, and connects to that \ac{PMIx} server via \refapi{PMIx_Init} using its assigned namespace and rank with the information required to connect to that server being provided to the process at time of start of execution.

\item \declaretermAlt{clone}{clones} refers to a process that was directly started by a \ac{PMIx} client (e.g., using \emph{fork/exec}) and calls \refapi{PMIx_Init}, thus connecting to its local \ac{PMIx} server using the same namespace and rank as its parent process.

\item \declaretermAlt{slot}{slots} refers to an allocated set of resources assigned to a process.  \acp{WLM} frequently allocate entire nodes to a \emph{session}, but can also be configured to define the maximum number of processes that can simultaneously be executed on each node. This often corresponds to the number of hardware \acp{PU} (typically cores, but can also be defined as hardware threads) on the node. However, the correlation between hardware \acp{PU} and slot allocations strictly depends upon system configuration.

\item \declareterm{rank} refers to the numerical location (starting from zero) of a process within the defined scope. Thus, \emph{job rank} is the rank of a process within its \emph{job} and is synonymous with its unqualified \emph{rank}, while \emph{application rank} is the rank of that process within its \emph{application}.

\item \declaretermAlt{peer}{peers} refers to another process within the same \refterm{job}.

\item \declaretermAlt{workflow}{workflows} refers to an orchestrated execution plan typically involving multiple \emph{jobs} carried out under the control of a \emph{workflow manager}. An example workflow might first execute a computational job to generate the flow of liquid through a complex cavity, followed by a visualization job that takes the output of the first job as its input to produce an image output.

\item \declareterm{scheduler} refers to the component of the \ac{SMS} responsible for scheduling of resource allocations. This is also generally referred to as the \emph{system workflow manager} - for the purposes of this document, the \emph{WLM} acronym will be used interchangeably to refer to the scheduler.

\item \declaretermAlt{resource manager}{RM} is used in a generic sense to represent the subsystem that will host the \ac{PMIx} server library. This could be a vendor-supplied resource manager or a third-party agent such as a programming model's runtime library.

\item \declareterm{host environment} is used interchangeably with \emph{resource manager} to refer to the process hosting the \ac{PMIx} server library.

\item \declareterm{node} refers to a single operating system instance. Note that this may encompass one or more physical objects.

\item \declareterm{package} refers to a single object that is either soldered or connected to a printed circuit board via a mechanical socket. Packages may contain multiple chips that include (but are not limited to) processing units, memory, and peripheral interfaces.

\item \declareterm{processing unit}, or \emph{PU}, is the electronic circuitry within a computer that executes instructions. Depending upon architecture and configuration settings, it may consist of a single hardware thread or multiple hardware threads collectively organized as a \emph{core}.

\item \declaretermAlt{fabric}{fabrics} is used in a generic sense to refer to the networks within the system regardless of speed or protocol. Any use of the term \emph{network} in the document should be considered interchangeable with \emph{fabric}.

\item \declaretermAlt{fabric device}{device} (or \declaretermAlt{fabric devices}{devices}) refers to an operating system fabric interface, which may be physical or virtual. Any use of the term \ac{NIC} in the document should be considered interchangeable with \emph{fabric device}.

\item \declaretermAlt{fabric plane}{fabric planes} refers to a collection of fabric devices in a common logical or physical configuration. Fabric planes are often implemented in \ac{HPC} clusters as separate overlay or physical networks controlled by a dedicated fabric manager.

\item \declareterm{attribute} refers to a key-value pair comprised of a string key (represented by a \refstruct{pmix_key_t} structure) and an associated value containing a \ac{PMIx} data type (e.g., boolean, integer, or a more complex \ac{PMIx} structure). Attributes are used both as directives when passed as qualifiers to \acp{API} (e.g., in a \refstruct{pmix_info_t} array), and to identify the contents of information (e.g., to specify that the contents of the corresponding \refstruct{pmix_value_t} in a \refstruct{pmix_info_t} represent the \refattr{PMIX_UNIV_SIZE}).

\item \declareterm{key} refers to the string component of a defined \emph{attribute}. The \ac{PMIx} Standard will often refer to passing of a \emph{key} to an \ac{API} (e.g., to the \refapi{PMIx_Query_info} or \refapi{PMIx_Get} \acp{API}) as a means of identifying requested information. In this context, the \emph{data type} specified in the \emph{attribute's} definition indicates the data type the caller should expect to receive in return. Note that not all \emph{attributes} can be used as \emph{keys} as some have specific uses solely as \ac{API} qualifiers.

\item \declareterm{instant on} refers to a \ac{PMIx} concept defined as: "All information required for setup and communication (including the address vector of endpoints for every process) is available to each process at start of execution"

\end{itemize}

The following sections provide an overview of the conventions used throughout the \ac{PMIx} Standard document.

%%%%%%%%%%%
\section{Notational Conventions}

Some sections of this document describe programming language specific examples or \acp{API}.
Text that applies only to programs for which the base language is C is shown as follows:

\cspecificstart
C specific text...
\begin{codepar}
int foo = 42;
\end{codepar}
\cspecificend

Some text is for information only, and is not part of the normative specification.
These take several forms, described in their examples below:

\notestart
\noteheader
General text...
\noteend

\rationalestart
Throughout this document, the rationale for the design choices made in the interface specification is set off in this section.
Some readers may wish to skip these sections, while readers interested in interface design may want to read them carefully.
\rationaleend

\adviceuserstart
Throughout this document, material aimed at users and that illustrates usage is set off in this section.
Some readers may wish to skip these sections, while readers interested in programming with the \ac{PMIx} \ac{API} may want to read them carefully.
\adviceuserend

\adviceimplstart
Throughout this document, material that is primarily commentary to \ac{PMIx} library implementers is set off in this section.
Some readers may wish to skip these sections, while readers interested in \ac{PMIx} implementations may want to read them carefully.
\adviceimplend

\advicermstart
Throughout this document, material that is primarily commentary aimed at host environments (e.g., \acp{RM} and \acp{RTE}) providing support for the \ac{PMIx} server library is set off in this section.
Some readers may wish to skip these sections, while readers interested in integrating \ac{PMIx} servers into their environment may want to read them carefully.
\advicermend

Attributes added in this version of the standard are shown in \textit{\textbf{\color{magenta}magenta}} to distinguish them from those defined in prior versions, which are shown in \textit{\textbf{black}}. Deprecated attributes are shown in \textit{\textbf{\color{green!80!black}green}} and may be removed in a future version of the standard.

%%%%%%%%%%%
\section{Semantics}

The following terms will be taken to mean:

\begin{itemize}
\item \emph{shall}, \emph{must} and \emph{will} indicate that the specified behavior is \emph{required} of all conforming implementations
\item \emph{should} and \emph{may} indicate behaviors that a complete implementation would include, but are not required of all conforming implementations
\end{itemize}

%%%%%%%%%%%
\section{Naming Conventions}

The \ac{PMIx} standard has adopted the following conventions:

\begin{itemize}
\item \ac{PMIx} constants and attributes are prefixed with \textbf{\code{PMIX_}}.
\item Structures and type definitions are prefixed with \code{pmix_}.
\item Underscores are used to separate words in a function or variable name.
\item Lowercase letters are used in \ac{PMIx} client \acp{API} except for the \ac{PMIx} prefix (noted below) and the first letter of the word following it.
For example, \refapi{PMIx_Get_version}.
\item \ac{PMIx} server and tool \acp{API} are all lower case letters following the prefix - e.g., \refapi{PMIx_server_register_nspace}.
\item The \code{PMIx_} prefix is used to denote functions.
\item The \code{pmix_} prefix is used to denote function pointer and type definitions.
\end{itemize}

Users shall not use the \textbf{\code{"PMIX"}}, \textbf{\code{"PMIx"}}, or \textbf{\code{"pmix"}} prefixes in their applications or libraries so as to avoid symbol conflicts with \ac{PMIx} implementations.

\section{Procedure Conventions}

While the current \acp{API} are based on the C programming language, it is not the intent of the \ac{PMIx} Standard to preclude the use of other languages.
Accordingly, the procedure specifications in the \ac{PMIx} Standard are written in a language-independent syntax with the arguments marked as IN, OUT, or INOUT.
The meanings of these are:
\begin{itemize}
\item IN:
The call may use the input value but does not update the argument from the perspective of the caller at any time during the calls execution,
\item OUT:
The call may update the argument but does not use its input value
\item INOUT:
The call may both use and update the argument.
\end{itemize}

Many \ac{PMIx} interfaces, particularly nonblocking interfaces, use a \code{(void*)} callback data object passed to the function that is then passed to the associated callback. On the client side, the callback data object is an opaque, client-provided context that the client can pass to a non-blocking call. When the nonblocking call completes, the callback data object is passed back to the client without modification by the \ac{PMIx} library, thus allowing the client to associate a context with that callback. This is useful if there are many outstanding nonblocking calls.

A similar model is used for the server module functions (see \ref{server:module_fns}). In this case, the \ac{PMIx} library is making an upcall into its host via the \ac{PMIx} server module callback function and passing a specific callback function pointer and callback data object. The \ac{PMIx} library expects the host to call the cbfunc with the necessary arguments and pass back the original callback data obect upon completing the operation. This gives the server-side \ac{PMIx} library the ability to associate a context with the call back (since multiple operations may be outstanding). The host has no visibility into the contents of the callback data object object, nor is permitted to alter it in any way.


    % Data Structures, Types, Constants
    %  - Includes: Reserved attributes, Keys
    %%%%%%%%%%%%%%%%%%%%%%%%%%%%%%%%%%%%%%%%%%%%%%%%%
% Chapter: Data Structures
%%%%%%%%%%%%%%%%%%%%%%%%%%%%%%%%%%%%%%%%%%%%%%%%%
\chapter{Data Structures and Types}
\label{chap:struct}

This chapter defines PMIx standard data structures (along with macros for convenient use), types, and constants.
These apply to all consumers of the PMIx interface.
Where necessary for clarification, the description of, for example, an attribute may be copied from this chapter into a section where it is used.

A PMIx implementation may define additional attributes beyond those specified in this document.

\adviceimplstart
Structures, types, and macros in the \ac{PMIx} Standard are defined in terms of the C-programming language. Implementers wishing to support other languages should provide the equivalent definitions in a language-appropriate manner.

If a PMIx implementation chooses to define additional attributes they should avoid using the \code{PMIX} prefix in their name or starting the attribute string with a \textit{pmix} prefix.
This helps the end user distinguish between what is defined by the PMIx standard and what is specific to that PMIx implementation, and avoids potential conflicts with attributes defined by the standard.
\adviceimplend

\adviceuserstart
Use of increment/decrement operations on indices inside \ac{PMIx} macros is discouraged due to unpredictable behavior. For example, the following sequence:

\begin{codepar}
PMIX_INFO_LOAD(&array[n++], "mykey", &mystring, PMIX_STRING);
PMIX_INFO_LOAD(&array[n++], "mykey2", &myint, PMIX_INT);
\end{codepar}

will load the given key-values into incorrect locations if the macro is implemented as:

\begin{codepar}
define PMIX_INFO_LOAD(m, k, v, t)                      \textbackslash
  do \{                                                 \textbackslash
    if (NULL != (k)) \{                                 \textbackslash
      pmix_strncpy((m)->key, (k), PMIX_MAX_KEYLEN);    \textbackslash
    \}                                                  \textbackslash
    (m)->flags = 0;                                    \textbackslash
    pmix_value_load(&((m)->value), (v), (t));          \textbackslash
  \} while (0)
\end{codepar}

since the index is cited more than once in the macro. The \ac{PMIx} standard only governs the existence and syntax of macros - it does not specify their implementation. Given the freedom of implementation, a safer call sequence might be as follows:

\begin{codepar}
PMIX_INFO_LOAD(&array[n], "mykey", &mystring, PMIX_STRING);
++n;
PMIX_INFO_LOAD(&array[n], "mykey2", &myint, PMIX_INT);
++n;
\end{codepar}

\adviceuserend

%%%%%%%%%%%
\section{Constants}

\ac{PMIx} defines a few values that are used throughout the standard to set the size of fixed arrays or as a means of identifying values with special meaning.
The community makes every attempt to minimize the number of such definitions.
The constants defined in this section may be used before calling any \ac{PMIx} library initialization routine.
Additional constants associated with specific data structures or types are defined in the section describing that data structure or type.

\begin{constantdesc}
%
\declareconstitem{PMIX_MAX_NSLEN}
Maximum namespace string length as an integer.
\end{constantdesc}

\adviceimplstart
\refconst{PMIX_MAX_NSLEN} should have a minimum value of 63 characters. Namespace arrays in \ac{PMIx} defined structures must reserve
a space of size \refconst{PMIX_MAX_NSLEN}+1 to allow room for the \code{NULL} terminator
\adviceimplend

\begin{constantdesc}
%
\declareconstitem{PMIX_MAX_KEYLEN}
Maximum key string length as an integer.
%
\end{constantdesc}

\adviceimplstart
\refconst{PMIX_MAX_KEYLEN} should have a minimum value of 63 characters. Key arrays in \ac{PMIx} defined structures must reserve
a space of size \refconst{PMIX_MAX_KEYLEN}+1 to allow room for the \code{NULL} terminator
\adviceimplend


%%%%%%%%%%%
\subsection{PMIx Error Constants}
\label{api:struct:errors}
\declarestruct{pmix_status_t}

The \refstruct{pmix_status_t} structure is an \code{int} type for return status.

The tables shown in this section define the possible values for \refstruct{pmix_status_t}.
PMIx errors are required to always be negative, with 0 reserved for \refconst{PMIX_SUCCESS}. Values in the list that were deprecated in later standards are denoted as such. Values added to the list in this version of the standard are shown in \textbf{\color{magenta}magenta}.

\adviceimplstart
A PMIx implementation must define all of the constants defined in this section, even if they will never return the specific value to the caller.
\adviceimplend

\adviceuserstart
Other than \refconst{PMIX_SUCCESS} (which is required to be zero), the actual value of any \ac{PMIx} error constant is left to the \ac{PMIx} library implementer. Thus, users are advised to always refer to constant by name, and not a specific implementation's value, for portability between implementations and compatibility across library versions.
\adviceuserend

\subsubsection{General Error Constants}

These are general constants originally defined in versions 1 and 2 of the PMIx Standard.

\begin{constantdesc}
%
\declareconstitem{PMIX_SUCCESS}
Success
%
\declareconstitem{PMIX_ERROR}
General Error
%
\declareconstitem{PMIX_ERR_SILENT}
Silent error
%
\declareconstitem{PMIX_ERR_DEBUGGER_RELEASE}
Error in debugger release
%
\declareconstitem{PMIX_ERR_PROC_RESTART}
Fault tolerance: Error in process restart
%
\declareconstitem{PMIX_ERR_PROC_CHECKPOINT}
Fault tolerance: Error in process checkpoint
%
\declareconstitem{PMIX_ERR_PROC_MIGRATE}
Fault tolerance: Error in process migration
%
\declareconstitem{PMIX_ERR_PROC_ABORTED}
Process was aborted
%
\declareconstitem{PMIX_ERR_PROC_REQUESTED_ABORT}
Process is already requested to abort
%
\declareconstitem{PMIX_ERR_PROC_ABORTING}
Process is being aborted
%
\declareconstitem{PMIX_ERR_SERVER_FAILED_REQUEST}
Failed to connect to the server
%
\declareconstitem{PMIX_EXISTS}
Requested operation would overwrite an existing value
%
\declareconstitem{PMIX_ERR_INVALID_CRED}
Invalid security credentials
%
\declareconstitem{PMIX_ERR_HANDSHAKE_FAILED}
Connection handshake failed
%
\declareconstitem{PMIX_ERR_READY_FOR_HANDSHAKE}
Ready for handshake
%
\declareconstitem{PMIX_ERR_WOULD_BLOCK}
Operation would block
%
\declareconstitem{PMIX_ERR_UNKNOWN_DATA_TYPE}
Unknown data type
%
\declareconstitem{PMIX_ERR_PROC_ENTRY_NOT_FOUND}
Process not found
%
\declareconstitem{PMIX_ERR_TYPE_MISMATCH}
Invalid type
%
\declareconstitem{PMIX_ERR_UNPACK_INADEQUATE_SPACE}
Inadequate space to unpack data
%
\declareconstitem{PMIX_ERR_UNPACK_FAILURE}
Unpack failed
%
\declareconstitem{PMIX_ERR_PACK_FAILURE}
Pack failed
%
\declareconstitem{PMIX_ERR_PACK_MISMATCH}
Pack mismatch
%
\declareconstitem{PMIX_ERR_NO_PERMISSIONS}
No permissions
%
\declareconstitem{PMIX_ERR_TIMEOUT}
Timeout expired
%
\declareconstitem{PMIX_ERR_UNREACH}
Unreachable
%
\declareconstitem{PMIX_ERR_IN_ERRNO}
Error defined in \code{errno}
%
\declareconstitem{PMIX_ERR_BAD_PARAM}
Bad parameter
%
\declareconstitem{PMIX_ERR_RESOURCE_BUSY}
Resource busy
%
\declareconstitem{PMIX_ERR_OUT_OF_RESOURCE}
Resource exhausted
%
\declareconstitem{PMIX_ERR_DATA_VALUE_NOT_FOUND}
Data value not found
%
\declareconstitem{PMIX_ERR_INIT}
Error during initialization
%
\declareconstitem{PMIX_ERR_NOMEM}
Out of memory
%
\declareconstitem{PMIX_ERR_INVALID_ARG}
Invalid argument
%
\declareconstitem{PMIX_ERR_INVALID_KEY}
Invalid key
%
\declareconstitem{PMIX_ERR_INVALID_KEY_LENGTH}
Invalid key length
%
\declareconstitem{PMIX_ERR_INVALID_VAL}
Invalid value
%
\declareconstitem{PMIX_ERR_INVALID_VAL_LENGTH}
Invalid value length
%
\declareconstitem{PMIX_ERR_INVALID_LENGTH}
Invalid argument length
%
\declareconstitem{PMIX_ERR_INVALID_NUM_ARGS}
Invalid number of arguments
%
\declareconstitem{PMIX_ERR_INVALID_ARGS}
Invalid arguments
%
\declareconstitem{PMIX_ERR_INVALID_NUM_PARSED}
Invalid number parsed
%
\declareconstitem{PMIX_ERR_INVALID_KEYVALP}
Invalid key/value pair
%
\declareconstitem{PMIX_ERR_INVALID_SIZE}
Invalid size
%
\declareconstitem{PMIX_ERR_INVALID_NAMESPACE}
Invalid namespace
%
\declareconstitem{PMIX_ERR_SERVER_NOT_AVAIL}
Server is not available
%
\declareconstitem{PMIX_ERR_NOT_FOUND}
Not found
%
\declareconstitem{PMIX_ERR_NOT_SUPPORTED}
Not supported
%
\declareconstitem{PMIX_ERR_NOT_IMPLEMENTED}
Not implemented
%
\declareconstitem{PMIX_ERR_COMM_FAILURE}
Communication failure
%
\declareconstitem{PMIX_ERR_UNPACK_READ_PAST_END_OF_BUFFER}
Unpacking past the end of the buffer provided
%
\declareconstitem{PMIX_ERR_LOST_CONNECTION_TO_SERVER}
Lost connection to server
%
\declareconstitem{PMIX_ERR_LOST_PEER_CONNECTION}
Lost connection to peer
%
\declareconstitem{PMIX_ERR_LOST_CONNECTION_TO_CLIENT}
Lost connection to client
%
\declareconstitem{PMIX_QUERY_PARTIAL_SUCCESS}
Query partial success (used by query system)
%
\declareconstitem{PMIX_NOTIFY_ALLOC_COMPLETE}
Notify that allocation is complete
%
\declareconstitem{PMIX_JCTRL_CHECKPOINT}
Job control: Monitored by PMIx client to trigger checkpoint operation
%
\declareconstitem{PMIX_JCTRL_CHECKPOINT_COMPLETE}
Job control: Sent by PMIx client and monitored by PMIx server to notify that requested checkpoint operation has completed.
%
\declareconstitem{PMIX_JCTRL_PREEMPT_ALERT}
Job control: Monitored by PMIx client to detect an \ac{RM} intending to preempt the job.
%
\declareconstitem{PMIX_MONITOR_HEARTBEAT_ALERT}
Job monitoring: Heartbeat alert
%
\declareconstitem{PMIX_MONITOR_FILE_ALERT}
Job monitoring: File alert
%
\declareconstitem{PMIX_PROC_TERMINATED}
Process terminated - can be either normal or abnormal termination
%
\declareconstitem{PMIX_ERR_INVALID_TERMINATION}
Process terminated without calling \refapi{PMIx_Finalize}, or was a member of an assemblage formed via \refapi{PMIx_Connect} and terminated or called \refapi{PMIx_Finalize} without first calling \refapi{PMIx_Disconnect} (or its non-blocking form) from that assemblage.
%
\end{constantdesc}

\subsubsection{Operational Error Constants}

\begin{constantdesc}
%
\declareconstitem{PMIX_ERR_EVENT_REGISTRATION}
Error in event registration
%
\declareconstitem{PMIX_ERR_JOB_TERMINATED}
Error job terminated
%
\declareconstitem{PMIX_ERR_UPDATE_ENDPOINTS}
Error updating endpoints
%
\declareconstitem{PMIX_MODEL_DECLARED}
Model declared
%
\declareconstitem{PMIX_GDS_ACTION_COMPLETE}
The \ac{GDS} action has completed
%
\declareconstitem{PMIX_ERR_INVALID_OPERATION}
The requested operation is supported by the implementation and host environment, but fails to meet a requirement (e.g., requesting to \textit{disconnect} from processes without first \textit{connecting} to them).

\declareconstitemNEW{PMIX_PROC_HAS_CONNECTED}
A tool or client has connected to the \ac{PMIx} server
%
\declareconstitemNEW{PMIX_CONNECT_REQUESTED}
Connection has been requested by a PMIx-based tool
%
\declareconstitemNEW{PMIX_MODEL_RESOURCES}
Resource usage by a programming model has changed
%
\declareconstitemNEW{PMIX_OPENMP_PARALLEL_ENTERED}
An OpenMP parallel code region has been entered
%
\declareconstitemNEW{PMIX_OPENMP_PARALLEL_EXITED}
An OpenMP parallel code region has completed
%
\declareconstitemNEW{PMIX_LAUNCH_DIRECTIVE}
Launcher directives have been received from a PMIx-enabled tool
%
\declareconstitemNEW{PMIX_LAUNCHER_READY}
Application launcher (e.g., mpiexec) is ready to receive directives from a PMIx-enabled tool
%
\declareconstitemNEW{PMIX_OPERATION_IN_PROGRESS}
A requested operation is already in proigress
%
\declareconstitem{PMIX_OPERATION_SUCCEEDED}
The requested operation was performed atomically - no callback function will be executed

\end{constantdesc}

\subsubsection{System error constants}

\begin{constantdesc}
%
\declareconstitem{PMIX_ERR_NODE_DOWN}
Node down
%
\declareconstitem{PMIX_ERR_NODE_OFFLINE}
Node is marked as offline
%
\declareconstitem{PMIX_ERR_SYS_OTHER}
Mark the beginning of a dedicated range of constants for system event reporting.
\end{constantdesc}
%

\subsubsection{Event handler error constants}

\begin{constantdesc}
%
\declareconstitem{PMIX_EVENT_NO_ACTION_TAKEN}
Event handler: No action taken
%
\declareconstitem{PMIX_EVENT_PARTIAL_ACTION_TAKEN}
Event handler: Partial action taken
%
\declareconstitem{PMIX_EVENT_ACTION_DEFERRED}
Event handler: Action deferred
%
\declareconstitem{PMIX_EVENT_ACTION_COMPLETE}
Event handler: Action complete
%
\end{constantdesc}

%%%%%%%%%%%
\subsubsection{User-Defined Error Constants}

PMIx establishes an error code boundary for constants defined in the PMIx standard. Negative values larger than this (and any positive values greater than zero) are guaranteed not to conflict with PMIx values.

\begin{constantdesc}
%
\declareconstitem{PMIX_EXTERNAL_ERR_BASE}
A starting point for user-level defined error constants.
Negative values lower than this are guaranteed not to conflict with PMIx values.
Definitions should always be based on the \refconst{PMIX_EXTERNAL_ERR_BASE} constant and not a specific value as the value of the constant may change.
%
\end{constantdesc}

%%%%%%%%%%%
\subsection{Macros for use with PMIx constants}

\subsubsection{Detect system event constant}
\declaremacro{pmix_system_event}

Test a given error constant to see if it falls within the dedicated range of constants for system event reporting.

\versionMarker{2.2}
\cspecificstart
\begin{codepar}
PMIX_SYSTEM_EVENT(a)
\end{codepar}
\cspecificend

\begin{arglist}
\argin{a}{Error constant to be checked (\refstruct{pmix_status_t})}
\end{arglist}

Returns \code{true} if the provided values falls within the dedicated range of constants for system event reporting


%%%%%%%%%%%
\section{Data Types}

This section defines various data types used by the PMIx APIs. The version of the standard in which a particular data type was introduced is shown in the margin.

%%%%%%%%%%%
\subsection{Key Structure}
\declarestruct{pmix_key_t}

The \refstruct{pmix_key_t} structure is a statically defined character array of length \refconst{PMIX_MAX_KEYLEN}+1, thus supporting keys of maximum length \refconst{PMIX_MAX_KEYLEN} while preserving space for a mandatory \code{NULL} terminator.

\versionMarker{2.0}
\cspecificstart
\begin{codepar}
typedef char pmix_key_t[PMIX_MAX_KEYLEN+1];
\end{codepar}
\cspecificend

Characters in the key must be standard alphanumeric values supported by common utilities such as \textit{strcmp}.

\adviceuserstart
References to keys in \ac{PMIx} v1 rwere defined simply as an array of characters of size \code{PMIX_MAX_KEYLEN+1}. The \refstruct{pmix_key_t} type definition was introduced in version 2 of the standard. The two definitions are code-compatible and thus do not represent a break in backward compatibility.

Passing a \refstruct{pmix_key_t} value to the standard \textit{sizeof} utility can result in compiler warnings of incorrect returned value. Users are advised to avoid using \textit{sizeof(pmix_key_t)} and instead rely on the \refconst{PMIX_MAX_KEYLEN} constant.
\adviceuserend

\subsubsection{Key support macro}
\declaremacro{pmix_check_key}

Compare the key in a \refstruct{pmix_info_t} to a given value

\versionMarker{3.0}
\cspecificstart
\begin{codepar}
PMIX_CHECK_KEY(a, b)
\end{codepar}
\cspecificend

\begin{arglist}
\argin{a}{Pointer to the structure whose key is to be checked (pointer to \refstruct{pmix_info_t})}
\argin{b}{String value to be compared against (\code{char*})}
\end{arglist}

Returns \code{true} if the key matches the given value

%%%%%%%%%%%
\subsection{Namespace Structure}
\declarestruct{pmix_nspace_t}

The \refstruct{pmix_nspace_t} structure is a statically defined character array of length \refconst{PMIX_MAX_NSLEN}+1, thus supporting namespaces of maximum length \refconst{PMIX_MAX_NSLEN} while preserving space for a mandatory \code{NULL} terminator.

\versionMarker{2.0}
\cspecificstart
\begin{codepar}
typedef char pmix_nspace_t[PMIX_MAX_NSLEN+1];
\end{codepar}
\cspecificend

Characters in the namespace must be standard alphanumeric values supported by common utilities such as \textit{strcmp}.

\adviceuserstart
References to namespace values in \ac{PMIx} v1 rwere defined simply as an array of characters of size \code{PMIX_MAX_NSLEN+1}. The \refstruct{pmix_nspace_t} type definition was introduced in version 2 of the standard. The two definitions are code-compatible and thus do not represent a break in backward compatibility.

Passing a \refstruct{pmix_nspace_t} value to the standard \textit{sizeof} utility can result in compiler warnings of incorrect returned value. Users are advised to avoid using \textit{sizeof(pmix_nspace_t)} and instead rely on the \refconst{PMIX_MAX_NSLEN} constant.
\adviceuserend

\subsubsection{Namespace support macro}
\declaremacro{pmix_check_nspace}

Compare the string in a \refstruct{pmix_nspace_t} to a given value

\versionMarker{3.0}
\cspecificstart
\begin{codepar}
PMIX_CHECK_NSPACE(a, b)
\end{codepar}
\cspecificend

\begin{arglist}
\argin{a}{Pointer to the structure whose value is to be checked (pointer to \refstruct{pmix_nspace_t})}
\argin{b}{String value to be compared against (\code{char*})}
\end{arglist}

Returns \code{true} if the namespace matches the given value


%%%%%%%%%%%
\subsection{Rank Structure}
\declarestruct{pmix_rank_t}

The \refstruct{pmix_rank_t} structure is a \code{uint32_t} type for rank values.

\versionMarker{1.0}
\cspecificstart
\begin{codepar}
typedef uint32_t pmix_rank_t;
\end{codepar}
\cspecificend

The following constants can be used to set a variable of the type \refstruct{pmix_rank_t}. All definitions were introduced in version 1 of the standard unless otherwise marked. Valid rank values start at zero.

\begin{constantdesc}
%
\declareconstitem{PMIX_RANK_UNDEF}
A value to request job-level data where the information itself is not associated with any specific rank, or when passing a \refstruct{pmix_proc_t} identifier to an operation that only references the namespace field of that structure.
%
\declareconstitem{PMIX_RANK_WILDCARD}
A value to indicate that the user wants the data for the given key from every rank that posted that key.
%
\declareconstitem{PMIX_RANK_LOCAL_NODE}
Special rank value used to define groups of ranks.
This constant defines the group of all ranks on a local node.
%
\declareconstitem{PMIX_RANK_LOCAL_PEERS}
Special rank value used to define groups of rankss.
This constant defines the group of all ranks on a local node within the same namespace as the current process.
%
\declareconstitem{PMIX_RANK_INVALID}
An invalid rank value.
%
\declareconstitem{PMIX_RANK_VALID}
Define an upper boundary for valid rank values.
%
\end{constantdesc}


%%%%%%%%%%%
\subsection{Process Structure}
\declarestruct{pmix_proc_t}

The \refstruct{pmix_proc_t} structure is used to identify a single process in the PMIx universe.
It contains a reference to the namespace and the \refstruct{pmix_rank_t} within that namespace.

\versionMarker{1.0}
\cspecificstart
\begin{codepar}
typedef struct pmix_proc \{
    pmix_nspace_t nspace;
    pmix_rank_t rank;
\} pmix_proc_t;
\end{codepar}
\cspecificend

\subsection{Process structure support macros}
The following macros are provided to support the \refstruct{pmix_proc_t} structure.

\subsubsection{Initialize the \refstruct{pmix_proc_t} structure}
\refmacro{PMIX_PROC_CONSTRUCT}

Initialize the \refstruct{pmix_proc_t} fields

\versionMarker{1.0}
\cspecificstart
\begin{codepar}
PMIX_PROC_CONSTRUCT(m)
\end{codepar}
\cspecificend

\begin{arglist}
\argin{m}{Pointer to the structure to be initialized (pointer to \refstruct{pmix_proc_t})}
\end{arglist}

\subsubsection{Destruct the \refstruct{pmix_proc_t} structure}
\declaremacro{PMIX_PROC_DESTRUCT}

There is nothing to release here as the fields in \refstruct{pmix_proc_t} are all declared \emph{static}. However, the macro is provided for symmetry in the code and for future-proofing should some allocated field be included some day.

\subsubsection{Create a \refstruct{pmix_proc_t} array}
\declaremacro{PMIX_PROC_CREATE}

Allocate and initialize an array of \refstruct{pmix_proc_t} structures

\versionMarker{1.0}
\cspecificstart
\begin{codepar}
PMIX_PROC_CREATE(m, n)
\end{codepar}
\cspecificend

\begin{arglist}
\arginout{m}{Address where the pointer to the array of \refstruct{pmix_proc_t} structures shall be stored (handle)}
\argin{n}{Number of structures to be allocated (\code{size_t})}
\end{arglist}


\subsubsection{Free a\refstruct{pmix_proc_t} array}
\declaremacro{PMIX_PROC_FREE}

Release an array of \refstruct{pmix_proc_t} structures

\versionMarker{1.0}
\cspecificstart
\begin{codepar}
PMIX_PROC_FREE(m, n)
\end{codepar}
\cspecificend

\begin{arglist}
\argin{m}{Pointer to the array of \refstruct{pmix_proc_t} structures (handle)}
\argin{n}{Number of structures in the array (\code{size_t})}
\end{arglist}

\subsubsection{Load a\refstruct{pmix_proc_t} structure}
\declaremacro{PMIX_PROC_LOAD}

Load values into a \refstruct{pmix_proc_t}

\versionMarker{2.0}
\cspecificstart
\begin{codepar}
PMIX_PROC_LOAD(m, n, r)
\end{codepar}
\cspecificend

\begin{arglist}
\argin{m}{Pointer to the structure to be loaded (pointer to \refstruct{pmix_proc_t})}
\argin{n}{Namespace to be loaded (\refstruct{pmix_nspace_t})}
\argin{r}{Rank to be assigned (\refstruct{pmix_rank_t})}
\end{arglist}

\subsubsection{Compare identifiers}
\declaremacro{pmix_check_procid}

Compare two \refstruct{pmix_proc_t} identifiers

\versionMarker{3.0}
\cspecificstart
\begin{codepar}
PMIX_CHECK_PROCID(a, b)
\end{codepar}
\cspecificend

\begin{arglist}
\argin{a}{Pointer to a structure whose ID is to be compared (pointer to \refstruct{pmix_proc_t})}
\argin{b}{Pointer to a structure whose ID is to be compared (pointer to \refstruct{pmix_proc_t})}
\end{arglist}

Returns \code{true} if the two structures contain matching namespaces and:

\begin{itemize}
    \item the ranks are the same value
    \item one of the ranks is \refconst{PMIX_RANK_WILDCARD}
\end{itemize}


%%%%%%%%%%%
\subsection{Process State Structure}
\label{api:struct:processstate}
\declarestruct{pmix_proc_state_t}

\versionMarker{2.0}
The \refstruct{pmix_proc_state_t} structure is a \code{uint8_t} type for process state values. The following constants can be used to set a variable of the type \refstruct{pmix_proc_state_t}. All values were originally defined in version 2 of the standard unless otherwise marked.

\adviceuserstart
The fine-grained nature of the following constants may exceed the ability of an \ac{RM} to provide updated process state values during the process lifetime. This is particularly true of states in the launch process, and for short-lived processes.
\adviceuserend

\begin{constantdesc}
%
\declareconstitem{PMIX_PROC_STATE_UNDEF}
Undefined process state
%
\declareconstitem{PMIX_PROC_STATE_PREPPED}
Process is ready to be launched
%
\declareconstitem{PMIX_PROC_STATE_LAUNCH_UNDERWAY}
Process launch is underway
%
\declareconstitem{PMIX_PROC_STATE_RESTART}
Process is ready for restart
%
\declareconstitem{PMIX_PROC_STATE_TERMINATE}
Process is marked for termination
%
\declareconstitem{PMIX_PROC_STATE_RUNNING}
Process has been locally \code{fork}'ed by the \ac{RM}
%
\declareconstitem{PMIX_PROC_STATE_CONNECTED}
Process has connected to PMIx server
%
\declareconstitem{PMIX_PROC_STATE_UNTERMINATED}
Define a ``boundary'' between the terminated states and \refconst{PMIX_PROC_STATE_CONNECTED} so users can easily and quickly determine if a process is still running or not.
Any value less than this constant means that the process has not terminated.
%
\declareconstitem{PMIX_PROC_STATE_TERMINATED}
Process has terminated and is no longer running
%
\declareconstitem{PMIX_PROC_STATE_ERROR}
Define a boundary so users can easily and quickly determine if a process abnormally terminated.
Any value above this constant means that the process has terminated abnormally.
%
\declareconstitem{PMIX_PROC_STATE_KILLED_BY_CMD}
Process was killed by a command
%
\declareconstitem{PMIX_PROC_STATE_ABORTED}
Process was aborted by a call to \refapi{PMIx_Abort}
%
\declareconstitem{PMIX_PROC_STATE_FAILED_TO_START}
Process failed to start
%
\declareconstitem{PMIX_PROC_STATE_ABORTED_BY_SIG}
Process aborted by a signal
%
\declareconstitem{PMIX_PROC_STATE_TERM_WO_SYNC}
Process exited without calling \refapi{PMIx_Finalize}
%
\declareconstitem{PMIX_PROC_STATE_COMM_FAILED}
Process communication has failed
%
\declareconstitem{PMIX_PROC_STATE_CALLED_ABORT}
Process called \refapi{PMIx_Abort}
%
\declareconstitem{PMIX_PROC_STATE_MIGRATING}
Process failed and is waiting for resources before restarting
%
\declareconstitem{PMIX_PROC_STATE_CANNOT_RESTART}
Process failed and cannot be restarted
%
\declareconstitem{PMIX_PROC_STATE_TERM_NON_ZERO}
Process exited with a non-zero status
%
\declareconstitem{PMIX_PROC_STATE_FAILED_TO_LAUNCH}
Unable to launch process
%
\end{constantdesc}


%%%%%%%%%%%
\subsection{Process Information Structure}
\declarestruct{pmix_proc_info_t}

The \refstruct{pmix_proc_info_t} structure defines a set of information about a specific process including it's name, location, and state.

\versionMarker{2.0}
\cspecificstart
\begin{codepar}
typedef struct pmix_proc_info \{
    /** Process structure */
    pmix_proc_t proc;
    /** Hostname where process resides */
    char *hostname;
    /** Name of the executable */
    char *executable_name;
    /** Process ID on the host */
    pid_t pid;
    /** Exit code of the process. Default: 0 */
    int exit_code;
    /** Current state of the process */
    pmix_proc_state_t state;
\} pmix_proc_info_t;
\end{codepar}
\cspecificend


%%%%%%%%%%%
\subsection{Process Information Structure support macros}

The following macros are provided to support the \refstruct{pmix_proc_info_t} structure.

%%%%
\subsubsection{Initialize the \refstruct{pmix_proc_info_t} structure}
\declaremacro{PMIX_PROC_INFO_CONSTRUCT}

Initialize the \refstruct{pmix_proc_info_t} fields

\versionMarker{2.0}
\cspecificstart
\begin{codepar}
PMIX_PROC_INFO_CONSTRUCT(m)
\end{codepar}
\cspecificend

\begin{arglist}
\argin{m}{Pointer to the structure to be initialized (pointer to \refstruct{pmix_proc_info_t})}
\end{arglist}

%%%%
\subsubsection{Destruct the \refstruct{pmix_proc_info_t} structure}
\declaremacro{PMIX_PROC_INFO_DESTRUCT}

Destruct the \refstruct{pmix_proc_info_t} fields

\versionMarker{2.0}
\cspecificstart
\begin{codepar}
PMIX_PROC_INFO_DESTRUCT(m)
\end{codepar}
\cspecificend

\begin{arglist}
\argin{m}{Pointer to the structure to be destructed (pointer to \refstruct{pmix_proc_info_t})}
\end{arglist}

%%%%
\subsubsection{Create a \refstruct{pmix_proc_info_t} array}
\declaremacro{PMIX_PROC_INFO_CREATE}

Allocate and initialize a \refstruct{pmix_proc_info_t} array

\versionMarker{2.0}
\cspecificstart
\begin{codepar}
PMIX_PROC_INFO_CREATE(m, n)
\end{codepar}
\cspecificend

\begin{arglist}
\arginout{m}{Address where the pointer to the array of \refstruct{pmix_proc_info_t} structures shall be stored (handle)}
\argin{n}{Number of structures to be allocated (\code{size_t})}
\end{arglist}

%%%%
\subsubsection{Free a\refstruct{pmix_proc_info_t} array}
\declaremacro{PMIX_PROC_INFO_FREE}

Release an array of \refstruct{pmix_proc_info_t} structures

\versionMarker{2.0}
\cspecificstart
\begin{codepar}
PMIX_PROC_INFO_FREE(m, n)
\end{codepar}
\cspecificend

\begin{arglist}
\argin{m}{Pointer to the array of \refstruct{pmix_proc_info_t} structures (handle)}
\argin{n}{Number of structures in the array (\code{size_t})}
\end{arglist}


%%%%%%%
\subsection{Scope of Put Data}
\declarestruct{pmix_scope_t}

\versionMarker{1.0}
The \refstruct{pmix_scope_t} structure is a \code{uint8_t} type that defines the scope for data passed to \refapi{PMIx_Put}.
The following constants can be used to set a variable of the type \refstruct{pmix_scope_t}. All definitions were introduced in version 1 of the standard unless otherwise marked.

Specific implementations may support different scope values, but all implementations must support at least \refconst{PMIX_GLOBAL}.
If a scope value is not supported, then the \refapi{PMIx_Put} call must return \refconst{PMIX_ERR_NOT_SUPPORTED}.

\begin{constantdesc}
%
\declareconstitem{PMIX_SCOPE_UNDEF}
Undefined scope
%
\declareconstitem{PMIX_LOCAL}
The data is intended only for other application processes on the same node.
Data marked in this way will not be included in data packages sent to remote requestors --- i.e., it is only available to processes on the local node.
%
\declareconstitem{PMIX_REMOTE}
The data is intended solely for applications processes on remote nodes.
Data marked in this way will not be shared with other processes on the same node --- i.e., it is only available to  processes on remote nodes.
%
\declareconstitem{PMIX_GLOBAL}
The data is to be shared with all other requesting processes, regardless of location.
%
\versionMarker{2.0}
\declareconstitem{PMIX_INTERNAL}
The data is intended solely for this process and is not shared with other processes.
%
\end{constantdesc}



%%%%%%%%%%%
\subsection{Range of Published Data}
\declarestruct{pmix_data_range_t}

\versionMarker{1.0}
The \refstruct{pmix_data_range_t} structure is a \code{uint8_t} type that defines a range for data \textit{published} via functions other than \refapi{PMIx_Put} - e.g., the \refapi{PMIx_Publish} \ac{API}.
The following constants can be used to set a variable of the type \refstruct{pmix_data_range_t}. Several values were initially defined in version 1 of the standard but subsequently renamed and other values added in version 2. Thus, all values shown below are as they were defined in version 2 except where noted.

\begin{constantdesc}
%
\declareconstitem{PMIX_RANGE_UNDEF}
Undefined range
%
\declareconstitem{PMIX_RANGE_RM}
Data is intended for the host resource manager.
%
\declareconstitem{PMIX_RANGE_LOCAL}
Data is only available to processes on the local node.
%
\declareconstitem{PMIX_RANGE_NAMESPACE}
Data is only available to processes in the same namespace.
%
\declareconstitem{PMIX_RANGE_SESSION}
Data is only available to all processes in the session.
%
\declareconstitem{PMIX_RANGE_GLOBAL}
Data is available to all processes.
%
\declareconstitem{PMIX_RANGE_CUSTOM}
Range is specified in the \refstruct{pmix_info_t} associated with this call.
%
\declareconstitem{PMIX_RANGE_PROC_LOCAL}
Data is only available to this process.
%
\declareconstitem{PMIX_RANGE_INVALID}
Invalid value
%
\end{constantdesc}

\adviceuserstart
The names of the \refstruct{pmix_data_range_t} values changed between version 1 and version 2 of the standard, thereby breaking backward compatibility
\adviceuserend

%%%%%%%%%%%
\subsection{Data Persistence Structure}
\declarestruct{pmix_persistence_t}

\versionMarker{1.0}
The \refstruct{pmix_persistence_t} structure is a \code{uint8_t} type that defines the policy for data published by clients via the \refapi{PMIx_Publish} \ac{API}.
The following constants can be used to set a variable of the type \refstruct{pmix_persistence_t}. All definitions were introduced in version 1 of the standard unless otherwise marked.

\begin{constantdesc}
%
\declareconstitem{PMIX_PERSIST_INDEF}
Retain data until specifically deleted.
%
\declareconstitem{PMIX_PERSIST_FIRST_READ}
Retain data until the first access, then the data is deleted.
%
\declareconstitem{PMIX_PERSIST_PROC}
Retain data until the publishing process terminates.
%
\declareconstitem{PMIX_PERSIST_APP}
Retain data until the application terminates.
%
\declareconstitem{PMIX_PERSIST_SESSION}
Retain data until the session/allocation terminates.
%
\declareconstitem{PMIX_PERSIST_INVALID}
Invalid value
%
\end{constantdesc}

%%%%%%%%%%%
\subsection{Data Array Structure}
\declarestruct{pmix_data_array_t}

\versionMarker{2.0}
\cspecificstart
\begin{codepar}
typedef struct pmix_data_array {
    pmix_data_type_t type;
    size_t size;
    void *array;
} pmix_data_array_t;
\end{codepar}
\cspecificend

The \refstruct{pmix_data_array_t} structure is used to pass arrays of related values. Any \ac{PMIx} data type (including complex structures) can be included in the array.

%%%%%%%%%%%
\subsection{Data array structure support macros}
The following macros are provided to support the \refstruct{pmix_data_array_t} structure.

\subsubsection{Initialize the \refstruct{pmix_data_array_t} structure}
\declaremacro{PMIX_DATA_ARRAY_CONSTRUCT}

Initialize the \refstruct{pmix_data_array_t} fields, allocating memory for the array itself.

\versionMarker{2.2}
\cspecificstart
\begin{codepar}
PMIX_DATA_ARRAY_CONSTRUCT(m, n, t)
\end{codepar}
\cspecificend

\begin{arglist}
\argin{m}{Pointer to the structure to be initialized (pointer to \refstruct{pmix_data_array_t})}
\argin{n}{Number of elements in the array (\code{size_t})}
\argin{t}{\ac{PMIx} data type for the array elements (\refstruct{pmix_data_type_t})}
\end{arglist}

\subsubsection{Destruct the \refstruct{pmix_data_array_t} structure}
\declaremacro{PMIX_DATA_ARRAY_DESTRUCT}

Destruct the \refstruct{pmix_data_array_t} fields, releasing the array's memory.

\versionMarker{2.2}
\cspecificstart
\begin{codepar}
PMIX_DATA_ARRAY_DESTRUCT(m)
\end{codepar}
\cspecificend

\begin{arglist}
\argin{m}{Pointer to the structure to be destructed (pointer to \refstruct{pmix_data_array_t})}
\end{arglist}

%%%%%%%%%%%
\subsubsection{Create and initialize a \refstruct{pmix_data_array_t} object}
\declaremacro{PMIX_DATA_ARRAY_CREATE}

Allocate and initialize a \refstruct{pmix_data_array_t} structure and initialize it, allocating memory for the array itself as well.

\versionMarker{2.2}
\cspecificstart
\begin{codepar}
PMIX_DATA_ARRAY_CREATE(m, n, t)
\end{codepar}
\cspecificend

\begin{arglist}
\arginout{m}{Address where the pointer to the \refstruct{pmix_data_array_t} structure shall be stored (handle)}
\argin{n}{Number of elements in the array (\code{size_t})}
\argin{t}{\ac{PMIx} data type for the array elements (\refstruct{pmix_data_type_t})}
\end{arglist}


%%%%%%%%%%%
\subsubsection{Free a\refstruct{pmix_data_array_t} object}
\declaremacro{PMIX_DATA_ARRAY_FREE}

Release a \refstruct{pmix_data_array_t} structure, including releasing the array's memory.

\versionMarker{2.2}
\cspecificstart
\begin{codepar}
PMIX_DATA_ARRAY_FREE(m)
\end{codepar}
\cspecificend

\begin{arglist}
\argin{m}{Pointer to the \refstruct{pmix_data_array_t} structure (handle)}
\end{arglist}

%%%%%%%%%%%
\subsection{Value Structure}
\declarestruct{pmix_value_t}

The \refstruct{pmix_value_t} structure is used to represent the value passed to \refapi{PMIx_Put} and retrieved by \refapi{PMIx_Get}, as well as many of the other \ac{PMIx} functions.

A collection of values may be specified under a single key by passing a \refstruct{pmix_value_t} containing an array of type \refstruct{pmix_data_array_t}, with each array element containing its own object. All members shown below were introduced in version 1 of the standard unless otherwise marked.

\versionMarker{1.0}
\cspecificstart
\begin{codepar}
typedef struct pmix_value \{
    pmix_data_type_t type;
    union \{
        bool flag;
        uint8_t byte;
        char *string;
        size_t size;
        pid_t pid;
        int integer;
        int8_t int8;
        int16_t int16;
        int32_t int32;
        int64_t int64;
        unsigned int uint;
        uint8_t uint8;
        uint16_t uint16;
        uint32_t uint32;
        uint64_t uint64;
        float fval;
        double dval;
        struct timeval tv;
        time_t time;                    // version 2.0
        pmix_status_t status;           // version 2.0
        pmix_rank_t rank;               // version 2.0
        pmix_proc_t *proc;              // version 2.0
        pmix_byte_object_t bo;
        pmix_persistence_t persist;     // version 2.0
        pmix_scope_t scope;             // version 2.0
        pmix_data_range_t range;        // version 2.0
        pmix_proc_state_t state;        // version 2.0
        pmix_proc_info_t *pinfo;        // version 2.0
        pmix_data_array_t *darray;      // version 2.0
        void *ptr;                      // version 2.0
        pmix_alloc_directive_t adir;    // version 2.0
    \} data;
\} pmix_value_t;
\end{codepar}
\cspecificend

%%%%%%%%%%%
\subsection{Value structure support macros}
The following macros are provided to support the \refstruct{pmix_value_t} structure.

\subsubsection{Initialize the \refstruct{pmix_value_t} structure}
\declaremacro{PMIX_VALUE_CONSTRUCT}

Initialize the \refstruct{pmix_value_t} fields

\versionMarker{1.0}
\cspecificstart
\begin{codepar}
PMIX_VALUE_CONSTRUCT(m)
\end{codepar}
\cspecificend

\begin{arglist}
\argin{m}{Pointer to the structure to be initialized (pointer to \refstruct{pmix_value_t})}
\end{arglist}

\subsubsection{Destruct the \refstruct{pmix_value_t} structure}
\declaremacro{PMIX_VALUE_DESTRUCT}

Destruct the \refstruct{pmix_value_t} fields

\versionMarker{1.0}
\cspecificstart
\begin{codepar}
PMIX_VALUE_DESTRUCT(m)
\end{codepar}
\cspecificend

\begin{arglist}
\argin{m}{Pointer to the structure to be destructed (pointer to \refstruct{pmix_value_t})}
\end{arglist}

%%%%%%%%%%%
\subsubsection{Create a \refstruct{pmix_value_t} array}
\declaremacro{PMIX_VALUE_CREATE}

Allocate and initialize an array of \refstruct{pmix_value_t} structures

\versionMarker{1.0}
\cspecificstart
\begin{codepar}
PMIX_VALUE_CREATE(m, n)
\end{codepar}
\cspecificend

\begin{arglist}
\arginout{m}{Address where the pointer to the array of \refstruct{pmix_value_t} structures shall be stored (handle)}
\argin{n}{Number of structures to be allocated (\code{size_t})}
\end{arglist}


%%%%%%%%%%%
\subsubsection{Free a\refstruct{pmix_value_t} array}
\declaremacro{PMIX_VALUE_FREE}

Release an array of \refstruct{pmix_value_t} structures

\versionMarker{1.0}
\cspecificstart
\begin{codepar}
PMIX_VALUE_FREE(m, n)
\end{codepar}
\cspecificend

\begin{arglist}
\argin{m}{Pointer to the array of \refstruct{pmix_value_t} structures (handle)}
\argin{n}{Number of structures in the array (\code{size_t})}
\end{arglist}

%%%%%%%%%%%
\subsubsection{Load a value structure}
\declaremacro{PMIX_VALUE_LOAD}

%%%%
\summary

Load data into a \refstruct{pmix_value_t} structure.

\versionMarker{2.0}
\cspecificstart
\begin{codepar}
PMIX_VALUE_LOAD(v, d, t);
\end{codepar}
\cspecificend

\begin{arglist}
\argin{v}{The \refstruct{pmix_value_t} into which the data is to be loaded (pointer to \refstruct{pmix_value_t})}
\argin{d}{Pointer to the data value to be loaded (handle)}
\argin{t}{Type of the provided data value (\refstruct{pmix_data_type_t})}
\end{arglist}

%%%%
\descr

This macro simplifies the loading of data into a \refstruct{pmix_value_t} by correctly assigning values to the structure's fields.

\adviceuserstart
The data will be copied into the \refstruct{pmix_value_t} - thus, any data stored in the source value can be modified or free'd without affecting the copied data once the macro has completed.
\adviceuserend

%%%%%%%%%%%
\subsubsection{Unload a \refstruct{pmix_value_t} structure}
\declaremacro{PMIX_VALUE_UNLOAD}

%%%%
\summary

Unload data from a \refstruct{pmix_value_t} structure.

\versionMarker{2.2}
\cspecificstart
\begin{codepar}
PMIX_VALUE_UNLOAD(r, v, d, t);
\end{codepar}
\cspecificend

\begin{arglist}
\argout{r}{Status code indicating result of the operation {\refstruct{pmix_status_t}}}
\argin{v}{The \refstruct{pmix_value_t} from which the data is to be unloaded (pointer to \refstruct{pmix_value_t})}
\arginout{d}{Pointer to the location where the data value is to be returned (handle)}
\arginout{t}{Pointer to return the data type of the unloaded value (handle)}
\end{arglist}

%%%%
\descr

This macro simplifies the unloading of data from a \refstruct{pmix_value_t}.

\adviceuserstart
Memory will be allocated and the data will be in the \refstruct{pmix_value_t} returned - the source \refstruct{pmix_value_t} will not be altered.
\adviceuserend


%%%%%%%%%%%
\subsubsection{Transfer data between \refstruct{pmix_value_t} structures}
\declaremacro{PMIX_VALUE_XFER}

%%%%
\summary

Transfer the data value between two \refstruct{pmix_value_t} structures.

\versionMarker{2.0}
\cspecificstart
\begin{codepar}
PMIX_VALUE_XFER(r, d, s);
\end{codepar}
\cspecificend

\begin{arglist}
\argout{r}{Status code indicating success or failure of the transfer (\refstruct{pmix_status_t})}
\argin{d}{Pointer to the \refstruct{pmix_value_t} destination (handle)}
\argin{s}{Pointer to the \refstruct{pmix_value_t} source (handle)}
\end{arglist}

%%%%
\descr

This macro simplifies the transfer of data between two \refstruct{pmix_value_t} structures, ensuring that all fields are properly copied.

\adviceuserstart
The data will be copied into the destination \refstruct{pmix_value_t} - thus, any data stored in the source value can be modified or free'd without affecting the copied data once the macro has completed.
\adviceuserend


%%%%%%%%%%%
\subsubsection{Retrieve a numerical value from a \refstruct{pmix_value_t}}
\declaremacro{PMIX_VALUE_GET_NUMBER}

Retrieve a numerical value from a \refstruct{pmix_value_t} structure

\versionMarker{3.0}
\cspecificstart
\begin{codepar}
PMIX_VALUE_GET_NUMBER(s, m, n, t)
\end{codepar}
\cspecificend

\begin{arglist}
\argout{s}{Status code for the request (\refstruct{pmix_status_t})}
\argin{m}{Pointer to the\refstruct{pmix_value_t} structure (handle)}
\argout{n}{Variable to be set to the value (match expected type)}
\argin{t}{Type of number expected in \refarg{m} (\refstruct{pmix_data_type_t})}
\end{arglist}

Sets the provided variable equal to the numerical value contained in the given \refstruct{pmix_value_t}, returning success if the data type of the value matches the expected type and \refconst{PMIX_ERR_BAD_PARAM} if it doesn't

%%%%%%%%%%%
\subsection{Info Structure}
\declarestruct{pmix_info_t}

The \refstruct{pmix_info_t} structure defines a key/value pair with associated directive. All fields were defined in version 1.0 unless otherwise marked.

\versionMarker{1.0}
\cspecificstart
\begin{codepar}
typedef struct pmix_info_t \{
    pmix_key_t key;
    pmix_info_directives_t flags;    // version 2.0
    pmix_value_t value;
\} pmix_info_t;
\end{codepar}
\cspecificend

%%%%%%%%%%%
\subsection{Info structure support macros}
The following macros are provided to support the \refstruct{pmix_info_t} structure.

\subsubsection{Initialize the \refstruct{pmix_info_t} structure}
\declaremacro{PMIX_INFO_CONSTRUCT}

Initialize the \refstruct{pmix_info_t} fields

\versionMarker{1.0}
\cspecificstart
\begin{codepar}
PMIX_INFO_CONSTRUCT(m)
\end{codepar}
\cspecificend

\begin{arglist}
\argin{m}{Pointer to the structure to be initialized (pointer to \refstruct{pmix_info_t})}
\end{arglist}

\subsubsection{Destruct the \refstruct{pmix_info_t} structure}
\declaremacro{PMIX_INFO_DESTRUCT}

Destruct the \refstruct{pmix_info_t} fields

\versionMarker{1.0}
\cspecificstart
\begin{codepar}
PMIX_INFO_DESTRUCT(m)
\end{codepar}
\cspecificend

\begin{arglist}
\argin{m}{Pointer to the structure to be destructed (pointer to \refstruct{pmix_info_t})}
\end{arglist}

%%%%%%%%%%%
\subsubsection{Create a \refstruct{pmix_info_t} array}
\declaremacro{PMIX_INFO_CREATE}

Allocate and initialize an array of \refstruct{pmix_info_t} structures

\versionMarker{1.0}
\cspecificstart
\begin{codepar}
PMIX_INFO_CREATE(m, n)
\end{codepar}
\cspecificend

\begin{arglist}
\arginout{m}{Address where the pointer to the array of \refstruct{pmix_info_t} structures shall be stored (handle)}
\argin{n}{Number of structures to be allocated (\code{size_t})}
\end{arglist}


%%%%%%%%%%%
\subsubsection{Free a\refstruct{pmix_info_t} array}
\declaremacro{PMIX_INFO_FREE}

Release an array of \refstruct{pmix_info_t} structures

\versionMarker{1.0}
\cspecificstart
\begin{codepar}
PMIX_INFO_FREE(m, n)
\end{codepar}
\cspecificend

\begin{arglist}
\argin{m}{Pointer to the array of \refstruct{pmix_info_t} structures (handle)}
\argin{n}{Number of structures in the array (\code{size_t})}
\end{arglist}

%%%%%%%%%%%
\subsubsection{Load key and value data into a \refstruct{pmix_info_t}}
\declaremacro{PMIX_INFO_LOAD}

\versionMarker{1.0}
\cspecificstart
\begin{codepar}
PMIX_INFO_LOAD(v, k, d, t);
\end{codepar}
\cspecificend

\begin{arglist}
\argin{v}{Pointer to the \refstruct{pmix_info_t} into which the key and data are to be loaded (pointer to \refstruct{pmix_info_t})}
\argin{k}{String key to be loaded - must be less than or equal to \refconst{PMIX_MAX_KEYLEN} in length (handle)}
\argin{d}{Pointer to the data value to be loaded (handle)}
\argin{t}{Type of the provided data value (\refstruct{pmix_data_type_t})}
\end{arglist}

This macro simplifies the loading of key and data into a \refstruct{pmix_info_t} by correctly assigning values to the structure's fields.

\adviceuserstart
Both key and data will be copied into the \refstruct{pmix_info_t} - thus, the key and any data stored in the source value can be modified or free'd without affecting the copied data once the macro has completed.
\adviceuserend

%%%%%%%%%%%
\subsubsection{Copy data between \refstruct{pmix_info_t} structures}
\declaremacro{PMIX_INFO_XFER}

Copy all data (including key, value, and directives) between two \refstruct{pmix_info_t} structures.

\versionMarker{2.0}
\cspecificstart
\begin{codepar}
PMIX_INFO_XFER(d, s);
\end{codepar}
\cspecificend

\begin{arglist}
\argin{d}{Pointer to the destination \refstruct{pmix_info_t} (pointer to \refstruct{pmix_info_t})}
\argin{s}{Pointer to the source \refstruct{pmix_info_t} (pointer to \refstruct{pmix_info_t})}
\end{arglist}

This macro simplifies the transfer of data between two\refstruct{pmix_info_t} structures.

\adviceuserstart
All data (including key, value, and directives) will be copied into the destination \refstruct{pmix_info_t} - thus, the source \refstruct{pmix_info_t} may be free'd without affecting the copied data once the macro has completed.
\adviceuserend


%%%%%%%%%%%
\subsubsection{Test a boolean \refstruct{pmix_info_t}}
\declaremacro{PMIX_INFO_TRUE}

A special macro for checking if a boolean \refstruct{pmix_info_t} is \code{true}

\versionMarker{2.0}
\cspecificstart
\begin{codepar}
PMIX_INFO_TRUE(m)
\end{codepar}
\cspecificend

\begin{arglist}
\argin{m}{Pointer to a \refstruct{pmix_info_t} structure (handle)}
\end{arglist}

A \refstruct{pmix_info_t} structure is considered to be of type \refconst{PMIX_BOOL} and value \code{true} if:

\begin{itemize}
    \item the structure reports a type of \refconst{PMIX_UNDEF}, or
    \item the structure reports a type of \refconst{PMIX_BOOL} and the data flag is \code{true}
\end{itemize}

%%%%%%%%%%%
\subsection{Info Type Directives}
\declarestruct{pmix_info_directives_t}

\versionMarker{2.0}
The \refstruct{pmix_info_directives_t} structure is a \code{uint32_t} type that defines the behavior of command directives via \refstruct{pmix_info_t} arrays.
By default, the values in the \refstruct{pmix_info_t} array passed to a PMIx are \emph{optional}.

\adviceuserstart
A PMIx implementation or PMIx-enabled \ac{RM} may ignore any \refstruct{pmix_info_t} value passed to a \ac{PMIx} \ac{API} if it is not explicitly marked as \refconst{PMIX_INFO_REQD}.
This is because the values specified default to optional, meaning they can be ignored.
This may lead to unexpected behavior if the user is relying on the behavior specified by the \refstruct{pmix_info_t} value.
If the user relies on the behavior defined by the \refstruct{pmix_info_t} then they must set the \refconst{PMIX_INFO_REQD} flag using the \refmacro{PMIX_INFO_REQUIRED} macro.
\adviceuserend

\adviceimplstart
The top 16-bits of the \refstruct{pmix_info_directives_t} are reserved for internal use by \ac{PMIx} library implementers - the \ac{PMIx} standard will \textit{not} specify their intent, leaving them for customized use by implementers. Implementers are advised to use the provided \refmacro{PMIX_INFO_IS_REQUIRED} macro for testing this flag, and must return \refconst{PMIX_ERR_NOT_SUPPORTED} as soon as possible to the caller if the required behavior is not supported.
\adviceimplend

The following constants were introduced in version 2.0 (unless otherwise marked) and can be used to set a variable of the type \refstruct{pmix_info_directives_t}.

\begin{constantdesc}
%
\declareconstitem{PMIX_INFO_REQD}
The behavior defined in the \refstruct{pmix_info_t} array is required, and not optional. This is a bit-mask value.
%
\declareconstitem{PMIX_INFO_ARRAY_END}
Mark that this \refstruct{pmix_info_t} struct is at the end of an array created by the \refmacro{PMIX_INFO_CREATE} macro. This is a bit-mask value.
%
\end{constantdesc}

\advicermstart
Host environments are advised to use the provided \refmacro{PMIX_INFO_IS_REQUIRED} macro for testing this flag and must return \refconst{PMIX_ERR_NOT_SUPPORTED} as soon as possible to the caller if the required behavior is not supported.
\advicermend


\subsection{Info Directive support macros}

The following macros are provided to support the setting and testing of \refstruct{pmix_info_t} directives.

%%%%
\subsubsection{Mark an info structure as required}
\declaremacro{PMIX_INFO_REQUIRED}

%%%%
\summary

Set the \refconst{PMIX_INFO_REQD} flag in a \refstruct{pmix_info_t} structure.

\versionMarker{2.0}
\cspecificstart
\begin{codepar}
PMIX_INFO_REQUIRED(info);
\end{codepar}
\cspecificend

\begin{arglist}
\argin{info}{Pointer to the \refstruct{pmix_info_t} (pointer to \refstruct{pmix_info_t})}
\end{arglist}

This macro simplifies the setting of the \refconst{PMIX_INFO_REQD} flag in \refstruct{pmix_info_t} structures.

%%%%
\subsubsection{Mark an info structure as optional}
\declaremacro{PMIX_INFO_OPTIONAL}

%%%%
\summary
Unsets the \refconst{PMIX_INFO_REQD} flag in a \refstruct{pmix_info_t} structure.

\versionMarker{3.0}
\cspecificstart
\begin{codepar}
PMIX_INFO_OPTIONAL(info);
\end{codepar}
\cspecificend

\begin{arglist}
\argin{info}{Pointer to the \refstruct{pmix_info_t} (pointer to \refstruct{pmix_info_t})}
\end{arglist}

This macro simplifies marking a \refstruct{pmix_info_t} structure as \textit{optional}.

%%%%%%%%%%%
\subsubsection{Test an info structure for \textit{required} directive}
\declaremacro{PMIX_INFO_IS_REQUIRED}

%%%%
\summary

Test the \refconst{PMIX_INFO_REQD} flag in a \refstruct{pmix_info_t} structure, returning \code{true} if the flag is set.

\versionMarker{2.0}
\cspecificstart
\begin{codepar}
PMIX_INFO_IS_REQUIRED(info);
\end{codepar}
\cspecificend

\begin{arglist}
\argin{info}{Pointer to the \refstruct{pmix_info_t} (pointer to \refstruct{pmix_info_t})}
\end{arglist}

This macro simplifies the testing of the required flag in \refstruct{pmix_info_t} structures.

%%%%%%%%%%%
\subsubsection{Test an info structure for \textit{optional} directive}
\declaremacro{PMIX_INFO_IS_OPTIONAL}

%%%%
\summary

Test a \refstruct{pmix_info_t} structure, returning \code{true} if the structure is \textit{optional}.

\versionMarker{2.0}
\cspecificstart
\begin{codepar}
PMIX_INFO_IS_OPTIONAL(info);
\end{codepar}
\cspecificend

\begin{arglist}
\argin{info}{Pointer to the \refstruct{pmix_info_t} (pointer to \refstruct{pmix_info_t})}
\end{arglist}

Test the \refconst{PMIX_INFO_REQD} flag in a \refstruct{pmix_info_t} structure, returning \code{true} if the flag is \textit{not} set.

%%%%%%%%%%%
\subsubsection{Test an info structure for \textit{end of array} directive}
\declaremacro{PMIX_INFO_IS_END}

%%%%
\summary

Test a \refstruct{pmix_info_t} structure, returning \code{true} if the structure is at the end of an array created by the \refmacro{PMIX_INFO_CREATE} macro.

\versionMarker{2.2}
\cspecificstart
\begin{codepar}
PMIX_INFO_IS_END(info);
\end{codepar}
\cspecificend

\begin{arglist}
\argin{info}{Pointer to the \refstruct{pmix_info_t} (pointer to \refstruct{pmix_info_t})}
\end{arglist}

This macro simplifies the testing of the end-of-array flag in \refstruct{pmix_info_t} structures.

%%%%%%%%%%%
\subsection{Job Allocation Directives}
\declarestruct{pmix_alloc_directive_t}

\versionMarker{2.0}
The \refstruct{pmix_alloc_directive_t} structure is a \code{uint8_t} type that defines the behavior of allocation requests.
The following constants can be used to set a variable of the type \refstruct{pmix_alloc_directive_t}. All definitions were introduced in version 2 of the standard unless otherwise marked.

\begin{constantdesc}
%
\declareconstitem{PMIX_ALLOC_NEW}
A new allocation is being requested.
The resulting allocation will be disjoint (i.e., not connected in a job sense) from the requesting allocation.
%
\declareconstitem{PMIX_ALLOC_EXTEND}
Extend the existing allocation, either in time or as additional resources.
%
\declareconstitem{PMIX_ALLOC_RELEASE}
Release part of the existing allocation.
Attributes in the accompanying \refstruct{pmix_info_t} array may be used to specify permanent release of the identified resources, or ``lending'' of those resources for some period of time.
%
\declareconstitem{PMIX_ALLOC_REAQUIRE}
Reacquire resources that were previously ``lent'' back to the scheduler.
%
\declareconstitem{PMIX_ALLOC_EXTERNAL}
A value boundary above which implementers are free to define their own directive values.
%
\end{constantdesc}


%%%%%%%%%%%
\subsection{IO Forwarding Channels}
\declarestruct{pmix_iof_channel_t}

\versionMarker{3.0}
The \refstruct{pmix_iof_channel_t} structure is a \code{uint16_t} type that defines a set of bit-mask flags for specifying IO forwarding channels. These can be bitwise OR'd together to reference multiple channels.

\begin{constantdesc}
%
\declareconstitemNEW{PMIX_FWD_NO_CHANNELS}
Forward no channels
%
\declareconstitemNEW{PMIX_FWD_STDIN_CHANNEL}
Forward stdin
%
\declareconstitemNEW{PMIX_FWD_STDOUT_CHANNEL}
Forward stdout
%
\declareconstitemNEW{PMIX_FWD_STDERR_CHANNEL}
Forward stderr
%
\declareconstitemNEW{PMIX_FWD_STDDIAG_CHANNEL}
Forward stddiag, if available
%
\declareconstitemNEW{PMIX_FWD_ALL_CHANNELS}
Forward all available channels
%
\end{constantdesc}


%%%%%%%%%%%
\subsection{Environmental Variable Structure}
\declarestruct{pmix_envar_t}

\versionMarker{3.0}
Define a structure for specifying environment variable modifications.
Standard environment variables (e.g., \code{PATH}, \code{LD_LIBRARY_PATH}, and \code{LD_PRELOAD})
take multiple arguments separated by delimiters. Unfortunately, the delimiters
depend upon the variable itself - some use semi-colons, some colons, etc. Thus,
the operation requires not only the name of the variable to be modified and
the value to be inserted, but also the separator to be used when composing
the aggregate value.

\cspecificstart
\begin{codepar}
typedef struct {
    char *envar;
    char *value;
    char separator;
} pmix_envar_t;
\end{codepar}
\cspecificend


\subsection{Environmental variable support macros}

The following macros are provided to support the \refstruct{pmix_envar_t} structure.

\subsubsection{Initialize the \refstruct{pmix_envar_t} structure}
\declaremacro{PMIX_ENVAR_CONSTRUCT}

Initialize the \refstruct{pmix_envar_t} fields

\versionMarker{3.0}
\cspecificstart
\begin{codepar}
PMIX_ENVAR_CONSTRUCT(m)
\end{codepar}
\cspecificend

\begin{arglist}
\argin{m}{Pointer to the structure to be initialized (pointer to \refstruct{pmix_envar_t})}
\end{arglist}

\subsubsection{Destruct the \refstruct{pmix_envar_t} structure}
\declaremacro{PMIX_ENVAR_DESTRUCT}

Clear the \refstruct{pmix_envar_t} fields

\versionMarker{3.0}
\cspecificstart
\begin{codepar}
PMIX_ENVAR_DESTRUCT(m)
\end{codepar}
\cspecificend

\begin{arglist}
\argin{m}{Pointer to the structure to be destructed (pointer to \refstruct{pmix_envar_t})}
\end{arglist}


\subsubsection{Create a \refstruct{pmix_envar_t} array}
\declaremacro{PMIX_ENVAR_CREATE}

Allocate and initialize an array of \refstruct{pmix_envar_t} structures

\versionMarker{3.0}
\cspecificstart
\begin{codepar}
PMIX_ENVAR_CREATE(m, n)
\end{codepar}
\cspecificend

\begin{arglist}
\arginout{m}{Address where the pointer to the array of \refstruct{pmix_envar_t} structures shall be stored (handle)}
\argin{n}{Number of structures to be allocated (\code{size_t})}
\end{arglist}


\subsubsection{Free a\refstruct{pmix_envar_t} array}
\declaremacro{PMIX_ENVAR_FREE}

Release an array of \refstruct{pmix_envar_t} structures

\versionMarker{3.0}
\cspecificstart
\begin{codepar}
PMIX_ENVAR_FREE(m, n)
\end{codepar}
\cspecificend

\begin{arglist}
\argin{m}{Pointer to the array of \refstruct{pmix_envar_t} structures (handle)}
\argin{n}{Number of structures in the array (\code{size_t})}
\end{arglist}

\subsubsection{Load a\refstruct{pmix_envar_t} structure}
\declaremacro{PMIX_ENVAR_LOAD}

Load values into a \refstruct{pmix_envar_t}

\versionMarker{2.0}
\cspecificstart
\begin{codepar}
PMIX_ENVAR_LOAD(m, e, v, s)
\end{codepar}
\cspecificend

\begin{arglist}
\argin{m}{Pointer to the structure to be loaded (pointer to \refstruct{pmix_envar_t})}
\argin{e}{Environmental variable name (\code{char*})}
\argin{v}{Value of variable (\code{char*})}
\argin{v}{Separator character (\code{char})}
\end{arglist}


%%%%%%%%%%%
\subsection{Lookup Returned Data Structure}
\declarestruct{pmix_pdata_t}

The \refstruct{pmix_pdata_t} structure is used by \refapi{PMIx_Lookup} to describe the data being accessed.

\versionMarker{1.0}
\cspecificstart
\begin{codepar}
typedef struct pmix_pdata \{
    pmix_proc_t proc;
    pmix_key_t key;
    pmix_value_t value;
\} pmix_pdata_t;
\end{codepar}
\cspecificend

\subsection{Lookup data structure support macros}

The following macros are provided to support the \refstruct{pmix_pdata_t} structure.

\subsubsection{Initialize the \refstruct{pmix_pdata_t} structure}
\declaremacro{PMIX_PDATA_CONSTRUCT}

Initialize the \refstruct{pmix_pdata_t} fields

\versionMarker{1.0}
\cspecificstart
\begin{codepar}
PMIX_PDATA_CONSTRUCT(m)
\end{codepar}
\cspecificend

\begin{arglist}
\argin{m}{Pointer to the structure to be initialized (pointer to \refstruct{pmix_pdata_t})}
\end{arglist}

\subsubsection{Destruct the \refstruct{pmix_pdata_t} structure}
\declaremacro{PMIX_PDATA_DESTRUCT}

Destruct the \refstruct{pmix_pdata_t} fields

\versionMarker{1.0}
\cspecificstart
\begin{codepar}
PMIX_PDATA_DESTRUCT(m)
\end{codepar}
\cspecificend

\begin{arglist}
\argin{m}{Pointer to the structure to be destructed (pointer to \refstruct{pmix_pdata_t})}
\end{arglist}

%%%%%%%%%%%
\subsubsection{Create a \refstruct{pmix_pdata_t} array}
\declaremacro{PMIX_PDATA_CREATE}

Allocate and initialize an array of \refstruct{pmix_pdata_t} structures

\versionMarker{1.0}
\cspecificstart
\begin{codepar}
PMIX_PDATA_CREATE(m, n)
\end{codepar}
\cspecificend

\begin{arglist}
\arginout{m}{Address where the pointer to the array of \refstruct{pmix_pdata_t} structures shall be stored (handle)}
\argin{n}{Number of structures to be allocated (\code{size_t})}
\end{arglist}


%%%%%%%%%%%
\subsubsection{Free a\refstruct{pmix_pdata_t} array}
\declaremacro{PMIX_PDATA_FREE}

Release an array of \refstruct{pmix_pdata_t} structures

\versionMarker{1.0}
\cspecificstart
\begin{codepar}
PMIX_PDATA_FREE(m, n)
\end{codepar}
\cspecificend

\begin{arglist}
\argin{m}{Pointer to the array of \refstruct{pmix_pdata_t} structures (handle)}
\argin{n}{Number of structures in the array (\code{size_t})}
\end{arglist}

%%%%%%%%%%%
\subsubsection{Load a lookup data structure}
\declaremacro{PMIX_PDATA_LOAD}

%%%%
\summary

Load key, process identifier, and data value into a \refstruct{pmix_pdata_t} structure.

\versionMarker{1.0}
\cspecificstart
\begin{codepar}
PMIX_PDATA_LOAD(m, p, k, d, t);
\end{codepar}
\cspecificend

\begin{arglist}
\argin{m}{Pointer to the \refstruct{pmix_pdata_t} structure into which the key and data are to be loaded (pointer to \refstruct{pmix_pdata_t})}
\argin{p}{Pointer to the \refstruct{pmix_proc_t} structure containing the identifier of the process being referenced (pointer to \refstruct{pmix_proc_t})}
\argin{k}{String key to be loaded - must be less than or equal to \refconst{PMIX_MAX_KEYLEN} in length (handle)}
\argin{d}{Pointer to the data value to be loaded (handle)}
\argin{t}{Type of the provided data value (\refstruct{pmix_data_type_t})}
\end{arglist}

This macro simplifies the loading of key, process identifier, and data into a \refstruct{pmix_proc_t} by correctly assigning values to the structure's fields.

\adviceuserstart
Key, process identifier, and data will all be copied into the \refstruct{pmix_pdata_t} - thus, the source information can be modified or free'd without affecting the copied data once the macro has completed.
\adviceuserend

%%%%%%%%%%%
\subsubsection{Transfer a lookup data structure}
\declaremacro{PMIX_PDATA_XFER}

%%%%
\summary

Transfer key, process identifier, and data value between two \refstruct{pmix_pdata_t} structures.

\versionMarker{2.0}
\cspecificstart
\begin{codepar}
PMIX_PDATA_XFER(d, s);
\end{codepar}
\cspecificend

\begin{arglist}
\argin{d}{Pointer to the destination \refstruct{pmix_pdata_t} (pointer to \refstruct{pmix_pdata_t})}
\argin{s}{Pointer to the source \refstruct{pmix_pdata_t} (pointer to \refstruct{pmix_pdata_t})}
\end{arglist}

This macro simplifies the transfer of key and data between two\refstruct{pmix_pdata_t} structures.

\adviceuserstart
Key, process identifier, and data will all be copied into the destination \refstruct{pmix_pdata_t} - thus, the source \refstruct{pmix_pdata_t} may free'd without affecting the copied data once the macro has completed.
\adviceuserend


%%%%%%%%%%%
\subsection{Application Structure}
\declarestruct{pmix_app_t}

The \refstruct{pmix_app_t} structure describes the application context for the \refapi{PMIx_Spawn} and \refapi{PMIx_Spawn_nb} operations.

\versionMarker{1.0}
\cspecificstart
\begin{codepar}
typedef struct pmix_app \{
    /** Executable */
    char *cmd;
    /** Argument set, NULL terminated */
    char **argv;
    /** Environment set, NULL terminated */
    char **env;
    /** Current working directory */
    char *cwd;
    /** Maximum processes with this profile */
    int maxprocs;
    /** Array of info keys describing this application*/
    pmix_info_t *info;
    /** Number of info keys in 'info' array */
    size_t ninfo;
\} pmix_app_t;
\end{codepar}
\cspecificend

\subsection{App structure support macros}
The following macros are provided to support the \refstruct{pmix_app_t} structure.

\subsubsection{Initialize the \refstruct{pmix_app_t} structure}
\declaremacro{PMIX_APP_CONSTRUCT}

Initialize the \refstruct{pmix_app_t} fields

\versionMarker{1.0}
\cspecificstart
\begin{codepar}
PMIX_APP_CONSTRUCT(m)
\end{codepar}
\cspecificend

\begin{arglist}
\argin{m}{Pointer to the structure to be initialized (pointer to \refstruct{pmix_app_t})}
\end{arglist}

\subsubsection{Destruct the \refstruct{pmix_app_t} structure}
\declaremacro{PMIX_APP_DESTRUCT}

Destruct the \refstruct{pmix_app_t} fields

\versionMarker{1.0}
\cspecificstart
\begin{codepar}
PMIX_APP_DESTRUCT(m)
\end{codepar}
\cspecificend

\begin{arglist}
\argin{m}{Pointer to the structure to be destructed (pointer to \refstruct{pmix_app_t})}
\end{arglist}

%%%%%%%%%%%
\subsubsection{Create a \refstruct{pmix_app_t} array}
\declaremacro{PMIX_APP_CREATE}

Allocate and initialize an array of \refstruct{pmix_app_t} structures

\versionMarker{1.0}
\cspecificstart
\begin{codepar}
PMIX_APP_CREATE(m, n)
\end{codepar}
\cspecificend

\begin{arglist}
\arginout{m}{Address where the pointer to the array of \refstruct{pmix_app_t} structures shall be stored (handle)}
\argin{n}{Number of structures to be allocated (\code{size_t})}
\end{arglist}


%%%%%%%%%%%
\subsubsection{Free a\refstruct{pmix_app_t} array}
\declaremacro{PMIX_APP_FREE}

Release an array of \refstruct{pmix_app_t} structures

\versionMarker{1.0}
\cspecificstart
\begin{codepar}
PMIX_APP_FREE(m, n)
\end{codepar}
\cspecificend

\begin{arglist}
\argin{m}{Pointer to the array of \refstruct{pmix_app_t} structures (handle)}
\argin{n}{Number of structures in the array (\code{size_t})}
\end{arglist}

%%%%%%%%%%%
\subsubsection{Create the \refstruct{pmix_info_t} array of application directives}
\declaremacro{PMIX_APP_INFO_CREATE}

Create an array of \refstruct{pmix_info_t} structures for passing application-level directives, updating the \refarg{ninfo} field of the \refstruct{pmix_app_t} structure.

\versionMarker{2.2}
\cspecificstart
\begin{codepar}
PMIX_APP_INFO_CREATE(m, n)
\end{codepar}
\cspecificend

\begin{arglist}
\argin{m}{Pointer to the \refstruct{pmix_app_t} structure (handle)}
\argin{n}{Number of directives to be allocated (\code{size_t})}
\end{arglist}


%%%%%%%%%%%
\subsection{Query Structure}
\declarestruct{pmix_query_t}

The \refstruct{pmix_query_t} structure is used by \refapi{PMIx_Query_info_nb} to describe a single query operation.

\versionMarker{2.0}
\cspecificstart
\begin{codepar}
typedef struct pmix_query \{
    char **keys;
    pmix_info_t *qualifiers;
    size_t nqual;
\} pmix_query_t;
\end{codepar}
\cspecificend

\subsection{Query structure support macros}
The following macros are provided to support the \refstruct{pmix_query_t} structure.

\subsubsection{Initialize the \refstruct{pmix_query_t} structure}
\declaremacro{PMIX_QUERY_CONSTRUCT}

Initialize the \refstruct{pmix_query_t} fields

\versionMarker{2.0}
\cspecificstart
\begin{codepar}
PMIX_QUERY_CONSTRUCT(m)
\end{codepar}
\cspecificend

\begin{arglist}
\argin{m}{Pointer to the structure to be initialized (pointer to \refstruct{pmix_query_t})}
\end{arglist}

\subsubsection{Destruct the \refstruct{pmix_query_t} structure}
\declaremacro{PMIX_QUERY_DESTRUCT}

Destruct the \refstruct{pmix_query_t} fields

\versionMarker{2.0}
\cspecificstart
\begin{codepar}
PMIX_QUERY_DESTRUCT(m)
\end{codepar}
\cspecificend

\begin{arglist}
\argin{m}{Pointer to the structure to be destructed (pointer to \refstruct{pmix_query_t})}
\end{arglist}

%%%%%%%%%%%
\subsubsection{Create a \refstruct{pmix_query_t} array}
\declaremacro{PMIX_QUERY_CREATE}

Allocate and initialize an array of \refstruct{pmix_query_t} structures

\versionMarker{2.0}
\cspecificstart
\begin{codepar}
PMIX_QUERY_CREATE(m, n)
\end{codepar}
\cspecificend

\begin{arglist}
\arginout{m}{Address where the pointer to the array of \refstruct{pmix_query_t} structures shall be stored (handle)}
\argin{n}{Number of structures to be allocated (\code{size_t})}
\end{arglist}


%%%%%%%%%%%
\subsubsection{Free a \refstruct{pmix_query_t} array}
\declaremacro{PMIX_QUERY_FREE}

Release an array of \refstruct{pmix_query_t} structures

\versionMarker{2.0}
\cspecificstart
\begin{codepar}
PMIX_QUERY_FREE(m, n)
\end{codepar}
\cspecificend

\begin{arglist}
\argin{m}{Pointer to the array of \refstruct{pmix_query_t} structures (handle)}
\argin{n}{Number of structures in the array (\code{size_t})}
\end{arglist}

%%%%%%%%%%%
\subsubsection{Create the\refstruct{pmix_info_t} array of query qualifiers}
\declaremacro{PMIX_QUERY_QUALIFIERS_CREATE}

Create an array of \refstruct{pmix_info_t} structures for passing query qualifiers, updating the \refarg{nqual} field of the \refstruct{pmix_query_t} structure.

\versionMarker{2.2}
\cspecificstart
\begin{codepar}
PMIX_QUERY_QUALIFIERS_CREATE(m, n)
\end{codepar}
\cspecificend

\begin{arglist}
\argin{m}{Pointer to the \refstruct{pmix_query_t} structure (handle)}
\argin{n}{Number of qualifiers to be allocated (\code{size_t})}
\end{arglist}


%%%%%%%%%%%
\section{Packing/Unpacking Types \& Structures}

This section defines types and structures used to pack and unpack data passed through the PMIx API.

%%%%%%%%%%%
\subsection{Byte Object Type}
\declarestruct{pmix_byte_object_t}

The \refstruct{pmix_byte_object_t} structure describes a raw byte sequence.

\versionMarker{1.0}
\cspecificstart
\begin{codepar}
typedef struct pmix_byte_object \{
    char *bytes;
    size_t size;
\} pmix_byte_object_t;
\end{codepar}
\cspecificend

\subsection{Byte object support macros}
The following macros support the \refstruct{pmix_byte_object_t} structure.

\subsubsection{Initialize the \refstruct{pmix_byte_object_t} structure}
\declaremacro{PMIX_PROC_CONSTRUCT}

Initialize the \refstruct{pmix_byte_object_t} fields

\versionMarker{2.0}
\cspecificstart
\begin{codepar}
PMIX_BYTE_OBJECT_CONSTRUCT(m)
\end{codepar}
\cspecificend

\begin{arglist}
\argin{m}{Pointer to the structure to be initialized (pointer to \refstruct{pmix_byte_object_t})}
\end{arglist}

\subsubsection{Destruct the \refstruct{pmix_byte_object_t} structure}
\declaremacro{PMIX_BYTE_OBJECT_DESTRUCT}

Clear the \refstruct{pmix_byte_object_t} fields

\versionMarker{2.0}
\cspecificstart
\begin{codepar}
PMIX_BYTE_OBJECT_DESTRUCT(m)
\end{codepar}
\cspecificend

\begin{arglist}
\argin{m}{Pointer to the structure to be destructed (pointer to \refstruct{pmix_byte_object_t})}
\end{arglist}

\subsubsection{Create a \refstruct{pmix_byte_object_t} structure}
\declaremacro{PMIX_BYTE_OBJECT_CREATE}

Allocate and intitialize an array of \refstruct{pmix_byte_object_t} structures

\versionMarker{2.0}
\cspecificstart
\begin{codepar}
PMIX_BYTE_OBJECT_CREATE(m, n)
\end{codepar}
\cspecificend

\begin{arglist}
\arginout{m}{Address where the pointer to the array of \refstruct{pmix_byte_object_t} structures shall be stored (handle)}
\argin{n}{Number of structures to be allocated (\code{size_t})}
\end{arglist}

\subsubsection{Free a \refstruct{pmix_byte_object_t} array}
\declaremacro{PMIX_BYTE_OBJECT_FREE}

Release an array of \refstruct{pmix_byte_object_t} structures

\versionMarker{2.0}
\cspecificstart
\begin{codepar}
PMIX_BYTE_OBJECT_FREE(m, n)
\end{codepar}
\cspecificend

\begin{arglist}
\argin{m}{Pointer to the array of \refstruct{pmix_byte_object_t} structures (handle)}
\argin{n}{Number of structures in the array (\code{size_t})}
\end{arglist}

\subsubsection{Load a\refstruct{pmix_byte_object_t} structure}
\declaremacro{PMIX_BYTE_OBJECT_LOAD}

Load values into a \refstruct{pmix_byte_object_t}

\versionMarker{2.0}
\cspecificstart
\begin{codepar}
PMIX_BYTE_OBJECT_LOAD(b, d, s)
\end{codepar}
\cspecificend

\begin{arglist}
\argin{b}{Pointer to the structure to be loaded (pointer to \refstruct{pmix_byte_object_t})}
\argin{d}{Pointer to the data to be loaded (\code{char*})}
\argin{s}{Number of bytes in the data array (\code{size_t})}
\end{arglist}


%%%%%%%%%%
\subsection{Data Array Structure}
\declarestruct{pmix_data_array_t}

The \refstruct{pmix_data_array_t} structure defines an array data structure.

\versionMarker{2.0}
\cspecificstart
\begin{codepar}
typedef struct pmix_data_array \{
    pmix_data_type_t type;
    size_t size;
    void *array;
\} pmix_data_array_t;
\end{codepar}
\cspecificend


\subsection{Data array support macros}
The following macros support the \refstruct{pmix_data_array_t} structure.

\subsubsection{Initialize a \refstruct{pmix_data_array_t} structure}
\declaremacro{PMIX_DATA_ARRAY_CONSTRUCT}

Initialize the \refstruct{pmix_data_array_t} fields, allocating memory for the array of the indicated type.

\versionMarker{2.2}
\cspecificstart
\begin{codepar}
PMIX_DATA_ARRAY_CONSTRUCT(m, n, t)
\end{codepar}
\cspecificend

\begin{arglist}
\argin{m}{Pointer to the structure to be initialized (pointer to \refstruct{pmix_data_array_t})}
\argin{n}{Number of elements in the array (\code{size_t})}
\argin{t}{\ac{PMIx} data type of the array elements (\refstruct{pmix_data_type_t})}
\end{arglist}


\subsubsection{Destruct a \refstruct{pmix_data_array_t} structure}
\declaremacro{PMIX_DATA_ARRAY_DESTRUCT}

Destruct the \refstruct{pmix_data_array_t}, releasing the memory in the array.

\versionMarker{2.2}
\cspecificstart
\begin{codepar}
PMIX_DATA_ARRAY_CONSTRUCT(m)
\end{codepar}
\cspecificend

\begin{arglist}
\argin{m}{Pointer to the structure to be destructed (pointer to \refstruct{pmix_data_array_t})}
\end{arglist}


\subsubsection{Create a \refstruct{pmix_data_array_t} structure}
\declaremacro{PMIX_DATA_ARRAY_CREATE}

Allocate memory for the \refstruct{pmix_data_array_t} object itself, and then allocate memory for the array of the indicated type.

\versionMarker{2.2}
\cspecificstart
\begin{codepar}
PMIX_DATA_ARRAY_CREATE(m, n, t)
\end{codepar}
\cspecificend

\begin{arglist}
\arginout{m}{Variable to be set to the address of the structure (pointer to \refstruct{pmix_data_array_t})}
\argin{n}{Number of elements in the array (\code{size_t})}
\argin{t}{\ac{PMIx} data type of the array elements (\refstruct{pmix_data_type_t})}
\end{arglist}


\subsubsection{Free a \refstruct{pmix_data_array_t} structure}
\declaremacro{PMIX_DATA_ARRAY_RELEASE}

Release the memory in the array, and then release the \refstruct{pmix_data_array_t} object itself.

\versionMarker{2.2}
\cspecificstart
\begin{codepar}
PMIX_DATA_ARRAY_FREE(m)
\end{codepar}
\cspecificend

\begin{arglist}
\argin{m}{Pointer to the structure to be released (pointer to \refstruct{pmix_data_array_t})}
\end{arglist}


%%%%%%%%%%
\subsection{Generalized Data Types Used for Packing/Unpacking}
\declarestruct{pmix_data_type_t}

The \refstruct{pmix_data_type_t} structure is a \code{uint16_t} type for identifying the data type for packing/unpacking purposes. New data type values introduced in this version of the Standard are shown in \textbf{\color{magenta}magenta}.

\adviceimplstart
The following constants can be used to set a variable of the type \refstruct{pmix_data_type_t}. Data types in the \ac{PMIx} Standard are defined in terms of the C-programming language. Implementers wishing to support other languages should provide the equivalent definitions in a language-appropriate manner. Additionally, a PMIx implementation may choose to add additional types.
\adviceimplend

\begin{constantdesc}
%
\declareconstitem{PMIX_UNDEF}
Undefined
%
\declareconstitem{PMIX_BOOL}
Boolean (converted to/from native \code{true}/\code{false}) (\code{bool})
%
\declareconstitem{PMIX_BYTE}
A byte of data (\code{uint8_t})
%
\declareconstitem{PMIX_STRING}
\code{NULL} terminated string (\code{char*})
%
\declareconstitem{PMIX_SIZE}
Size \code{size_t}
%
\declareconstitem{PMIX_PID}
Operating \ac{PID} (\code{pid_t})
%
\declareconstitem{PMIX_INT}
Integer (\code{int})
%
\declareconstitem{PMIX_INT8}
8-byte integer (\code{int8_t})
%
\declareconstitem{PMIX_INT16}
16-byte integer (\code{int16_t})
%
\declareconstitem{PMIX_INT32}
32-byte integer (\code{int32_t})
%
\declareconstitem{PMIX_INT64}
64-byte integer (\code{int64_t})
%
\declareconstitem{PMIX_UINT}
Unsigned integer (\code{unsigned int})
%
\declareconstitem{PMIX_UINT8}
Unsigned 8-byte integer (\code{uint8_t})
%
\declareconstitem{PMIX_UINT16}
Unsigned 16-byte integer (\code{uint16_t})
%
\declareconstitem{PMIX_UINT32}
Unsigned 32-byte integer (\code{uint32_t})
%
\declareconstitem{PMIX_UINT64}
Unsigned 64-byte integer (\code{uint64_t})
%
\declareconstitem{PMIX_FLOAT}
Float (\code{float})
%
\declareconstitem{PMIX_DOUBLE}
Double (\code{double})
%
\declareconstitem{PMIX_TIMEVAL}
Time value (\code{struct timeval})
%
\declareconstitem{PMIX_TIME}
Time (\code{time_t})
%
\declareconstitem{PMIX_STATUS}
Status code {\refstruct{pmix_status_t}}
%
\declareconstitem{PMIX_VALUE}
Value (\refstruct{pmix_value_t})
%
\declareconstitem{PMIX_PROC}
Process (\refstruct{pmix_proc_t})
%
\declareconstitem{PMIX_APP}
Application context
%
\declareconstitem{PMIX_INFO}
Info object
%
\declareconstitem{PMIX_PDATA}
Pointer to data
%
\declareconstitem{PMIX_BUFFER}
Buffer
%
\declareconstitem{PMIX_BYTE_OBJECT}
Byte object (\refstruct{pmix_byte_object_t})
%
\declareconstitem{PMIX_KVAL}
Key/value pair
%
\declareconstitemDEP{PMIX_MODEX}{2.0}
Modex
%
\declareconstitem{PMIX_PERSIST}
Persistance (\refstruct{pmix_persistence_t})
%
\declareconstitem{PMIX_POINTER}
Pointer to an object (\code{void*})
%
\declareconstitem{PMIX_SCOPE}
Scope (\refstruct{pmix_scope_t})
%
\declareconstitem{PMIX_DATA_RANGE}
Range for data (\refstruct{pmix_data_range_t})
%
\declareconstitem{PMIX_COMMAND}
PMIx command code (used internally)
%
\declareconstitem{PMIX_INFO_DIRECTIVES}
Directives flag for \refstruct{pmix_info_t} (\refstruct{pmix_info_directives_t})
%
\declareconstitem{PMIX_DATA_TYPE}
Data type code (\refstruct{pmix_data_type_t})
%
\declareconstitem{PMIX_PROC_STATE}
Process state (\refstruct{pmix_proc_state_t})
%
\declareconstitem{PMIX_PROC_INFO}
Process information (\refstruct{pmix_proc_info_t})
%
\declareconstitem{PMIX_DATA_ARRAY}
Data array (\refstruct{pmix_data_array_t})
%
\declareconstitem{PMIX_PROC_RANK}
Process rank (\refstruct{pmix_rank_t})
%
\declareconstitem{PMIX_QUERY}
Query structure (\refstruct{pmix_query_t})
%
\declareconstitem{PMIX_COMPRESSED_STRING}
String compressed with zlib (\code{char*})
%
\declareconstitem{PMIX_ALLOC_DIRECTIVE}
Allocation directive (\refstruct{pmix_alloc_directive_t})
%
\declareconstitemNEW{PMIX_IOF_CHANNEL}
Input/output forwarding channel (\refstruct{pmix_iof_channel_t})
%
\declareconstitemNEW{PMIX_ENVAR}
Environmental variable structure (\refstruct{pmix_envar_t})
%
\declareconstitemNEW{PMIX_REGEX}
Regular expressions - can be a valid NULL-terminated string or an arbitrary array of bytes
%

\end{constantdesc}

%%%%%%%%%%%
\section{Reserved attributes}
\label{api:struct:attributes}

The PMIx standard defines a relatively small set of APIs and the caller may customize the behavior of the API by passing one or more attributes to that API.
Additionally, attributes may be keys passed to \refapi{PMIx_Get} calls to access the specified values from the system.

Each attribute is represented by a \var{key} string, and a type for the associated \var{value}.
This section defines a set of \textbf{reserved} keys which are prefixed with \code{pmix.} to designate them as PMIx standard reserved keys. All definitions were introduced in version 1 of the standard unless otherwise marked.

Applications or associated libraries (e.g., \ac{MPI}) may choose to define additional attributes.
The attributes defined in this section are of the system and job as opposed to the attributes that the application (or associated libraries) might choose to expose.
Due to this extensibility the \refapi{PMIx_Get} API will return \refconst{PMIX_ERR_NOT_FOUND} if the provided \var{key} cannot be found.

Attributes added in this version of the standard are shown in \textit{\textbf{\color{magenta}magenta}} to distinguish them from those defined in prior versions, which are shown in \textit{\textbf{black}}. Deprecated attributes are shown in \textit{\textbf{\color{green!80!black}green}} and will be removed in future versions of the standard.

\declareAttribute{PMIX_ATTR_UNDEF}{NULL}{NULL}{
Constant representing an undefined attribute.
}

%%%%%%%%%%%
\subsection{Initialization attributes}
\label{api:struct:attributes:init}

These attributes are defined to assist the caller with initialization by passing them into the appropriate initialization \ac{API} - thus, they are not typically accessed via the \refapi{PMIx_Get} \ac{API}.

%
\declareAttribute{PMIX_EVENT_BASE}{"pmix.evbase"}{struct event_base *}{
Pointer to libevent\footnote{\url{http://libevent.org/}} \code{event_base} to use in place of the internal progress thread.
}

%
\declareAttribute{PMIX_SERVER_TOOL_SUPPORT}{"pmix.srvr.tool"}{bool}{
The host \ac{RM} wants to declare itself as willing to accept tool connection requests.
}

%
\declareAttribute{PMIX_SERVER_REMOTE_CONNECTIONS}{"pmix.srvr.remote"}{bool}{
Allow connections from remote tools. Forces the PMIx server to not exclusively use loopback device.
}

%
\declareAttribute{PMIX_SERVER_SYSTEM_SUPPORT}{"pmix.srvr.sys"}{bool}{
The host \ac{RM} wants to declare itself as being the local system server for PMIx connection requests.
}

%
\declareAttribute{PMIX_SERVER_TMPDIR}{"pmix.srvr.tmpdir"}{char*}{
Top-level temporary directory for all client processes connected to this server, and where the PMIx server will place its tool rendezvous point and contact information.
}

%
\declareAttribute{PMIX_SYSTEM_TMPDIR}{"pmix.sys.tmpdir"}{char*}{
Temporary directory for this system, and where a PMIx server that declares itself to be a system-level server will place a tool rendezvous point and contact information.
}

%
\declareAttribute{PMIX_SERVER_ENABLE_MONITORING}{"pmix.srv.monitor"}{bool}{
Enable PMIx internal monitoring by the PMIx server.
}

%
\declareAttribute{PMIX_SERVER_NSPACE}{"pmix.srv.nspace"}{char*}{
Name of the namespace to use for this PMIx server.
}

%
\declareAttribute{PMIX_SERVER_RANK}{"pmix.srv.rank"}{pmix_rank_t}{
Rank of this PMIx server
}

%
\declareNewAttribute{PMIX_SERVER_GATEWAY}{"pmix.srv.gway"}{bool}{
Server is acting as a gateway for PMIx requests that cannot be serviced on backend nodes (e.g., logging to email)
}


%%%%%%%%%%%
\subsection{Tool-related attributes}
\label{api:struct:attributes:tool}

These attributes are defined to assist PMIx-enabled tools to connect with the PMIx server by passing them into the \refapi{PMIx_tool_init} \ac{API} - thus, they are not typically accessed via the \refapi{PMIx_Get} \ac{API}.


%
\declareAttribute{PMIX_TOOL_NSPACE}{"pmix.tool.nspace"}{char*}{
Name of the namespace to use for this tool.
}

%
\declareAttribute{PMIX_TOOL_RANK}{"pmix.tool.rank"}{uint32_t}{
Rank of this tool.
}

%
\declareAttribute{PMIX_SERVER_PIDINFO}{"pmix.srvr.pidinfo"}{pid_t}{
\ac{PID} of the target PMIx server for a tool.
}

%
\declareAttribute{PMIX_CONNECT_TO_SYSTEM}{"pmix.cnct.sys"}{bool}{
The requestor requires that a connection be made only to a local, system-level PMIx server.
}

%
\declareAttribute{PMIX_CONNECT_SYSTEM_FIRST}{"pmix.cnct.sys.first"}{bool}{
Preferentially, look for a system-level PMIx server first.
}

%
\declareAttribute{PMIX_SERVER_URI}{"pmix.srvr.uri"}{char*}{
\ac{URI} of the PMIx server to be contacted.
}

%
\declareAttribute{PMIX_SERVER_HOSTNAME}{"pmix.srvr.host"}{char*}{
Host where target PMIx server is located.
}

%
\declareAttribute{PMIX_CONNECT_MAX_RETRIES}{"pmix.tool.mretries"}{uint32_t}{
Maximum number of times to try to connect to PMIx server.
}

%
\declareAttribute{PMIX_CONNECT_RETRY_DELAY}{"pmix.tool.retry"}{uint32_t}{
Time in seconds between connection attempts to a PMIx server.
}

%
\declareAttribute{PMIX_TOOL_DO_NOT_CONNECT}{"pmix.tool.nocon"}{bool}{
The tool wants to use internal PMIx support, but does not want to connect to a PMIx server.
}

%
\declareNewAttribute{PMIX_RECONNECT_SERVER}{"pmix.tool.recon"}{bool}{
Tool is requesting to change server connections
}

%
\declareNewAttribute{PMIX_LAUNCHER}{"pmix.tool.launcher"}{bool}{
Tool is a launcher and needs rendezvous files created
}


%%%%%%%%%%%
\subsection{Identification attributes}
\label{api:struct:attributes:ident}

These attributes are defined to identify a process and it's associated PMIx-enabled library. They are not typically accessed via the \refapi{PMIx_Get} \ac{API}, and thus are not associated with a particular rank.

%
\declareAttribute{PMIX_USERID}{"pmix.euid"}{uint32_t}{
Effective user id.
}

%
\declareAttribute{PMIX_GRPID}{"pmix.egid"}{uint32_t}{
Effective group id.
}

%
\declareAttribute{PMIX_DSTPATH}{"pmix.dstpath"}{char*}{
Path to shared memory data storage (dstore) files.
}

%
\declareAttribute{PMIX_VERSION_INFO}{"pmix.version"}{char*}{
PMIx version of contractor.
}

%
\declareAttribute{PMIX_REQUESTOR_IS_TOOL}{"pmix.req.tool"}{bool}{
The requesting process is a PMIx tool.
}

%
\declareAttribute{PMIX_REQUESTOR_IS_CLIENT}{"pmix.req.client"}{bool}{
The requesting process is a PMIx client.
}

%%%%%%%%%%%
\subsection{Programming model attributes}
\label{api:struct:attributes:model}

These attributes are associated with programming models.

%
\declareAttribute{PMIX_PROGRAMMING_MODEL}{"pmix.pgm.model"}{char*}{
Programming model being initialized (e.g., ``MPI'' or ``OpenMP'')
}

%
\declareAttribute{PMIX_MODEL_LIBRARY_NAME}{"pmix.mdl.name"}{char*}{
Programming model implementation ID (e.g., ``OpenMPI'' or ``MPICH'')
}

%
\declareAttribute{PMIX_MODEL_LIBRARY_VERSION}{"pmix.mld.vrs"}{char*}{
Programming model version string (e.g., ``2.1.1'')
}

%
\declareAttribute{PMIX_THREADING_MODEL}{"pmix.threads"}{char*}{
Threading model used (e.g., ``pthreads'')
}

%
\declareNewAttribute{PMIX_MODEL_NUM_THREADS}{"pmix.mdl.nthrds"}{uint64_t}{
Number of active threads being used by the model
}

%
\declareNewAttribute{PMIX_MODEL_NUM_CPUS}{"pmix.mdl.ncpu"}{uint64_t}{
Number of cpus being used by the model
}

%
\declareNewAttribute{PMIX_MODEL_CPU_TYPE}{"pmix.mdl.cputype"}{char*}{
Granularity - ``hwthread'', ``core'', etc.
}

%
\declareNewAttribute{PMIX_MODEL_PHASE_NAME}{"pmix.mdl.phase"}{char*}{
User-assigned name for a phase in the application execution (e.g., ``cfd reduction'')
}

%
\declareNewAttribute{PMIX_MODEL_PHASE_TYPE}{"pmix.mdl.ptype"}{char*}{
Type of phase being executed (e.g., ``matrix multiply'')
}

%
\declareNewAttribute{PMIX_MODEL_AFFINITY_POLICY}{"pmix.mdl.tap"}{char*}{
Thread affinity policy - e.g.:
         "master" (thread co-located with master thread),
         "close" (thread located on cpu close to master thread),
         "spread" (threads load-balanced across available cpus)
}



%%%%%%%%%%%
\subsection{UNIX socket rendezvous socket attributes}
\label{api:struct:attributes:usock}

These attributes are used to describe a UNIX socket for rendezvous with the local \ac{RM} by passing them into the relevant initialization \ac{API} - thus, they are not typically accessed via the \refapi{PMIx_Get} \ac{API}.

%
\declareAttribute{PMIX_USOCK_DISABLE}{"pmix.usock.disable"}{bool}{
Disable legacy UNIX socket (usock) support
}

%
\declareAttribute{PMIX_SOCKET_MODE}{"pmix.sockmode"}{uint32_t}{
POSIX \var{mode_t} (9 bits valid)
}

%
\declareAttribute{PMIX_SINGLE_LISTENER}{"pmix.sing.listnr"}{bool}{
Use only one rendezvous socket, letting priorities and/or environment parameters select the active transport.
}


%%%%%%%%%%%
\subsection{TCP connection attributes}
\label{api:struct:attributes:tcp}

These attributes are used to describe a TCP socket for rendezvous with the local \ac{RM} by passing them into the relevant initialization \ac{API} - thus, they are not typically accessed via the \refapi{PMIx_Get} \ac{API}.

%
\declareAttribute{PMIX_TCP_REPORT_URI}{"pmix.tcp.repuri"}{char*}{
If provided, directs that the TCP \ac{URI} be reported and indicates the desired method of reporting: \code{'-'} for stdout, \code{'+'} for stderr, or filename.
}

%
\declareAttribute{PMIX_TCP_URI}{"pmix.tcp.uri"}{char*}{
The \ac{URI} of the PMIx server to connect to, or a file name containing it in the form of \code{file:<name of file containing it>}.
}

%
\declareAttribute{PMIX_TCP_IF_INCLUDE}{"pmix.tcp.ifinclude"}{char*}{
Comma-delimited list of devices and/or \ac{CIDR} notation to include when establishing the TCP connection.
}

%
\declareAttribute{PMIX_TCP_IF_EXCLUDE}{"pmix.tcp.ifexclude"}{char*}{
Comma-delimited list of devices and/or \ac{CIDR} notation to exclude when establishing the TCP connection.
}

%
\declareAttribute{PMIX_TCP_IPV4_PORT}{"pmix.tcp.ipv4"}{int}{
The IPv4 port to be used.
}

%
\declareAttribute{PMIX_TCP_IPV6_PORT}{"pmix.tcp.ipv6"}{int}{
The IPv6 port to be used.
}

%
\declareAttribute{PMIX_TCP_DISABLE_IPV4}{"pmix.tcp.disipv4"}{bool}{
Set to \code{true} to disable IPv4 family of addresses.
}

%
\declareAttribute{PMIX_TCP_DISABLE_IPV6}{"pmix.tcp.disipv6"}{bool}{
Set to \code{true} to disable IPv6 family of addresses.
}


%%%%%%%%%%%
\subsection{Global Data Storage (GDS) attributes}
\label{api:struct:attributes:gds}

These attributes are used to define the behavior of the \ac{GDS} used to manage key/value pairs by passing them into the relevant initialization \ac{API} - thus, they are not typically accessed via the \refapi{PMIx_Get} \ac{API}.

%
\declareAttribute{PMIX_GDS_MODULE}{"pmix.gds.mod"}{char*}{
Comma-delimited string of desired modules.
}


%%%%%%%%%%%
\subsection{General process-level attributes}
\label{api:struct:attributes:gproc}

These attributes are used to define process attributes and are referenced by their process rank.

%
\declareAttribute{PMIX_CPUSET}{"pmix.cpuset"}{char*}{
hwloc\footnote{\url{https://www.open-mpi.org/projects/hwloc/}} bitmap to be applied to the process upon launch.
}

%
\declareAttribute{PMIX_CREDENTIAL}{"pmix.cred"}{char*}{
Security credential assigned to the process.
}

%
\declareAttribute{PMIX_SPAWNED}{"pmix.spawned"}{bool}{
\code{true} if this process resulted from a call to \refapi{PMIx_Spawn}.
}

%
\declareAttribute{PMIX_ARCH}{"pmix.arch"}{uint32_t}{
Architecture flag.
}


%%%%%%%%%%%
\subsection{Scratch directory attributes}
\label{api:struct:attributes:scratchdir}

These attributes are used to define an application scratch directory and are referenced using the \refconst{PMIX_RANK_WILDCARD} rank.

%
\declareAttribute{PMIX_TMPDIR}{"pmix.tmpdir"}{char*}{
Full path to the top-level temporary directory assigned to the session.
}

%
\declareAttribute{PMIX_NSDIR}{"pmix.nsdir"}{char*}{
Full path to the temporary directory assigned to the namespace, under \refattr{PMIX_TMPDIR}.
}

%
\declareAttribute{PMIX_PROCDIR}{"pmix.pdir"}{char*}{
Full path to the subdirectory under \refattr{PMIX_NSDIR} assigned to the process.
}

%
\declareAttribute{PMIX_TDIR_RMCLEAN}{"pmix.tdir.rmclean"}{bool}{
Resource Manager will clean session directories
}


%%%%%%%%%%%
\subsection{Relative Rank Descriptive Attributes}
\label{api:struct:attributes:relrankinfo}

These attributes are used to describe information about relative ranks as assigned by the \ac{RM}, and thus are referenced using the process rank except where noted.

%
\declareNewAttribute{PMIX_CLUSTER_ID}{"pmix.clid"}{char*}{
A string name for the cluster this proc is executing on
}

%
\declareAttribute{PMIX_PROCID}{"pmix.procid"}{pmix_proc_t}{
Process identifier
}

%
\declareAttribute{PMIX_NSPACE}{"pmix.nspace"}{char*}{
Namespace of the job.
}

%
\declareAttribute{PMIX_JOBID}{"pmix.jobid"}{char*}{
Job identifier assigned by the scheduler.
}

%
\declareAttribute{PMIX_APPNUM}{"pmix.appnum"}{uint32_t}{
Application number within the job.
}

%
\declareAttribute{PMIX_RANK}{"pmix.rank"}{pmix_rank_t}{
Process rank within the job.
}

%
\declareAttribute{PMIX_GLOBAL_RANK}{"pmix.grank"}{pmix_rank_t}{
Process rank spanning across all jobs in this session.
}

%
\declareAttribute{PMIX_APP_RANK}{"pmix.apprank"}{pmix_rank_t}{
Process rank within this application.
}

%
\declareAttribute{PMIX_NPROC_OFFSET}{"pmix.offset"}{pmix_rank_t}{
Starting global rank of this job - referenced using \refconst{PMIX_RANK_WILDCARD}.
}

%
\declareAttribute{PMIX_LOCAL_RANK}{"pmix.lrank"}{uint16_t}{
Local rank on this node within this job.
}

%
\declareAttribute{PMIX_NODE_RANK}{"pmix.nrank"}{uint16_t}{
Process rank on this node spanning all jobs.
}

%
\declareAttribute{PMIX_LOCALLDR}{"pmix.lldr"}{pmix_rank_t}{
Lowest rank on this node within this job - referenced using \refconst{PMIX_RANK_WILDCARD}.
}

%
\declareAttribute{PMIX_APPLDR}{"pmix.aldr"}{pmix_rank_t}{
Lowest rank in this application within this job - referenced using \refconst{PMIX_RANK_WILDCARD}.
}

%
\declareAttribute{PMIX_PROC_PID}{"pmix.ppid"}{pid_t}{
\ac{PID} of specified process.
}

%
\declareAttribute{PMIX_SESSION_ID}{"pmix.session.id"}{uint32_t}{
Session identifier - referenced using \refconst{PMIX_RANK_WILDCARD}.
}

%
\declareAttribute{PMIX_NODE_LIST}{"pmix.nlist"}{char*}{
Comma-delimited list of nodes running processes for the specified namespace - referenced using \refconst{PMIX_RANK_WILDCARD}.
}

%
\declareAttribute{PMIX_ALLOCATED_NODELIST}{"pmix.alist"}{char*}{
Comma-delimited list of all nodes in this allocation regardless of whether or not they currently host processes - referenced using \refconst{PMIX_RANK_WILDCARD}.
}

%
\declareAttribute{PMIX_HOSTNAME}{"pmix.hname"}{char*}{
Name of the host where the specified process is running.
}

%
\declareAttribute{PMIX_NODEID}{"pmix.nodeid"}{uint32_t}{
Node identifier where the specified process is located, expressed as the node's index (beginning at zero) in the array resulting from expansion of the \refattr{PMIX_NODE_MAP} regular expression for the \refterm{job}
}

%
\declareAttribute{PMIX_LOCAL_PEERS}{"pmix.lpeers"}{char*}{
Comma-delimited list of ranks on this node within the specified namespace - referenced using \refconst{PMIX_RANK_WILDCARD}.
}

%
\declareAttribute{PMIX_LOCAL_PROCS}{"pmix.lprocs"}{pmix_proc_t array}{
Array of \refstruct{pmix_proc_t} of all processes on the specified node - referenced using \refconst{PMIX_RANK_WILDCARD}.
}

%
\declareAttribute{PMIX_LOCAL_CPUSETS}{"pmix.lcpus"}{char*}{
Colon-delimited cpusets of local peers within the specified namespace - referenced using \refconst{PMIX_RANK_WILDCARD}.
}

%
\declareAttribute{PMIX_PROC_URI}{"pmix.puri"}{char*}{
\ac{URI} containing contact information for a given process.
}

%
\declareAttribute{PMIX_LOCALITY}{"pmix.loc"}{uint16_t}{
Relative locality of the specified process to the requestor.
}

%
\declareAttribute{PMIX_PARENT_ID}{"pmix.parent"}{pmix_proc_t}{
Process identifier of the parent process of the calling process.
}

%
\declareNewAttribute{PMIX_EXIT_CODE}{"pmix.exit.code"}{int}{
Exit code returned when process terminated
}


%%%%%%%%%%%
\subsection{Information retrieval attributes}
\label{api:struct:attributes:retrieval}

The following attributes are used to specify the level of information (e.g., \refterm{session}, \refterm{job}, or \refterm{application}) being requested where ambiguity may exist - see \ref{chap:api_kv:getex} for examples of their use.

%
\declareNewAttribute{PMIX_SESSION_INFO}{"pmix.ssn.info"}{bool}{
Return information about the specified session. If information about a session other than the one containing the requesting process is desired, then the attribute array must contain a \refattr{PMIX_SESSION_ID} attribute identifying the desired target.
}

%
\declareNewAttribute{PMIX_JOB_INFO}{"pmix.job.info"}{bool}{
Return information about the specified job or namespace. If information about a job or namespace other than the one containing the requesting process is desired, then the attribute array must contain a \refattr{PMIX_JOBID} or \refattr{PMIX_NSPACE} attribute identifying the desired target. Similarly, if information is requested about a job or namespace in a session other than the one containing the requesting process, then an attribute identifying the target session must be provided.
}

%
\declareNewAttribute{PMIX_APP_INFO}{"pmix.app.info"}{bool}{
Return information about the specified application. If information about an application other than the one containing the requesting process is desired, then the attribute array must contain a \refattr{PMIX_APPNUM} attribute identifying the desired target. Similarly, if information is requested about an application in a job or session other than the one containing the requesting process, then attributes identifying the target job and/or session must be provided.
}

%
\declareNewAttribute{PMIX_NODE_INFO}{"pmix.node.info"}{bool}{
Return information about the specified node. If information about a node other than the one containing the requesting process is desired, then the attribute array must contain either the \refattr{PMIX_NODEID} or \refattr{PMIX_HOSTNAME} attribute identifying the desired target.
}

%%%%%%%%%%%
\subsection{Information storage attributes}
\label{api:struct:attributes:storage}

The following attributes are used to assemble information by its level (e.g., \refterm{session}, \refterm{job}, or \refterm{application}) for storage where ambiguity may exist - see \ref{chap:api_server:assemble} for examples of their use.

%
\declareNewAttribute{PMIX_SESSION_INFO_ARRAY}{"pmix.ssn.arr"}{pmix_data_array_t}{
Provide an array of \refstruct{pmix_info_t} containing session-level information. The \refattr{PMIX_SESSION_ID} attribute is required to be included in the array.
}

%
\declareNewAttribute{PMIX_JOB_INFO_ARRAY}{"pmix.job.arr"}{pmix_data_array_t}{
Provide an array of \refstruct{pmix_info_t} containing job-level information. The \refattr{PMIX_SESSION_ID} attribute of the \refterm{session} containing the \refterm{job} is required to be included in the array whenever the \ac{PMIx} server library may host multiple sessions (e.g., when executing with a host \ac{RM} daemon). As information is registered one job (aka namespace) at a time via the \refapi{PMIx_server_register_nspace} \ac{API}, there is no requirement that the array contain either the \refattr{PMIX_NSPACE} or \refattr{PMIX_JOBID} attributes when used in that context (though either or both of them may be included). At least one of the job identifiers must be provided in all other contexts where the job being referenced is ambiguous.
}

%
\declareNewAttribute{PMIX_APP_INFO_ARRAY}{"pmix.app.arr"}{pmix_data_array_t}{
Provide an array of \refstruct{pmix_info_t} containing app-level information. The \refattr{PMIX_NSPACE} or \refattr{PMIX_JOBID} attributes of the \refterm{job} containing the application, plus its \refattr{PMIX_APPNUM} attribute, are must to be included in the array when the array is \textit{not} included as part of a call to \refapi{PMIx_server_register_nspace} - i.e., when the job containing the application is ambiguous. The job identification is otherwise optional.
}

%
\declareNewAttribute{PMIX_NODE_INFO_ARRAY}{"pmix.node.arr"}{pmix_data_array_t}{
Provide an array of \refstruct{pmix_info_t} containing node-level information. At a minimum, either the \refattr{PMIX_NODEID} or \refattr{PMIX_HOSTNAME} attribute is required to be included in the array, though both may be included.
}

Note that these assemblages can be used hierarchically:

\begin{itemize}
\item a \refattr{PMIX_JOB_INFO_ARRAY} might contain multiple \refattr{PMIX_APP_INFO_ARRAY} elements, each describing values for a specific application within the job
\item a \refattr{PMIX_JOB_INFO_ARRAY} could contain a \refattr{PMIX_NODE_INFO_ARRAY} for each node hosting processes from that job, each array describing job-level values for that node
\item a \refattr{PMIX_SESSION_INFO_ARRAY} might contain multiple \refattr{PMIX_JOB_INFO_ARRAY} elements, each describing a job executing within the session. Each job array could, in turn, contain both application and node arrays, thus providing a complete picture of the active operations within the allocation
\end{itemize}

\adviceimplstart
\ac{PMIx} implementations must be capable of properly parsing and storing any hierarchical depth of information arrays. The resulting stored values are must to be accessible via both \refapi{PMIx_Get} and \refapi{PMIx_Query_info_nb} \acp{API}, assuming appropriate directives are provided by the caller.
\adviceimplend

%%%%%%%%%%%
\subsection{Size information attributes}
\label{api:struct:attributes:sizeinfo}

These attributes are used to describe the size of various dimensions of the PMIx universe - all are referenced using \refconst{PMIX_RANK_WILDCARD}.

%
\declareAttribute{PMIX_UNIV_SIZE}{"pmix.univ.size"}{uint32_t}{
Number of allocated slots in a session - each slot may or may not be occupied by an executing process. Note that this attribute is the equivalent to the combination of \refattr{PMIX_SESSION_INFO_ARRAY} with the \refattr{PMIX_MAX_PROCS} entry in the array - it is included in the Standard for historical reasons.
}

%
\declareAttribute{PMIX_JOB_SIZE}{"pmix.job.size"}{uint32_t}{
Total number of processes in this job across all contained applications. Note that this value can be different from \refattr{PMIX_MAX_PROCS}. For example, users may choose to subdivide an allocation (running several jobs in parallel within it), and dynamic programming models may support adding and removing processes from a running \refterm{job} on-they-fly. In the latter case, \ac{PMIx} events must be used to notify processes within the job that the job size has changed.
}

%
\declareAttribute{PMIX_JOB_NUM_APPS}{"pmix.job.napps"}{uint32_t}{
Number of applications in this job.
}

%
\declareAttribute{PMIX_APP_SIZE}{"pmix.app.size"}{uint32_t}{
Number of processes in this application.
}

%
\declareAttribute{PMIX_LOCAL_SIZE}{"pmix.local.size"}{uint32_t}{
Number of processes in this job or application on this node.
}

%
\declareAttribute{PMIX_NODE_SIZE}{"pmix.node.size"}{uint32_t}{
Number of processes across all jobs on this node.
}

%
\declareAttribute{PMIX_MAX_PROCS}{"pmix.max.size"}{uint32_t}{
Maximum number of processes that can be executed in this context (session, namespace, application, or node). Typically, this is a constraint imposed by a scheduler or by user settings in a hostfile or other resource description.
}

%
\declareAttribute{PMIX_NUM_SLOTS}{"pmix.num.slots"}{uint32_t}{
Number of slots allocated in this context (session, namespace, application, or node). Note that this attribute is the equivalent to \refattr{PMIX_MAX_PROCS} used in the corresponding context - it is included in the Standard for historical reasons.
}

\declareNewAttribute{PMIX_NUM_NODES}{"pmix.num.nodes"}{uint32_t}{
Number of nodes in this session, or that are currently executing processes from the associated namespace or application.
}

%%%%%%%%%%%
\subsection{Memory information attributes}
\label{api:struct:attributes:meminfo}

These attributes are used to describe memory available and used in the system - all are referenced using \refconst{PMIX_RANK_WILDCARD}.

%
\declareAttribute{PMIX_AVAIL_PHYS_MEMORY}{"pmix.pmem"}{uint64_t}{
Total available physical memory on this node.
}

%
\declareAttribute{PMIX_DAEMON_MEMORY}{"pmix.dmn.mem"}{float}{
Megabytes of memory currently used by the \ac{RM} daemon.
}

%
\declareAttribute{PMIX_CLIENT_AVG_MEMORY}{"pmix.cl.mem.avg"}{float}{
Average Megabytes of memory used by client processes.
}


%%%%%%%%%%%
\subsection{Topology information attributes}
\label{api:struct:attributes:topoinfo}

These attributes are used to describe topology information in the PMIx universe - all are referenced using \refconst{PMIX_RANK_WILDCARD} except where noted.

%
\declareAttribute{PMIX_NET_TOPO}{"pmix.ntopo"}{char*}{
\ac{XML} representation of the network topology.
}

%
\declareAttribute{PMIX_LOCAL_TOPO}{"pmix.ltopo"}{char*}{
\ac{XML} representation of local node topology.
}

%
\declareAttribute{PMIX_TOPOLOGY}{"pmix.topo"}{hwloc_topology_t}{
Pointer to the PMIx client's internal hwloc topology object.
}

%
\declareNewAttribute{PMIX_TOPOLOGY_XML}{"pmix.topo.xml"}{char*}{
\ac{XML}-based description of topology
}

%
\declareNewAttribute{PMIX_TOPOLOGY_FILE}{"pmix.topo.file"}{char*}{
Full path to file containing \ac{XML} topology description
}

%
\declareAttribute{PMIX_TOPOLOGY_SIGNATURE}{"pmix.toposig"}{char*}{
Topology signature string.
}

%
\declareAttribute{PMIX_LOCALITY_STRING}{"pmix.locstr"}{char*}{
String describing a process's bound location - referenced using the process's rank.
The string is of the form:\\
\code{NM\%s:SK\%s:L3\%s:L2\%s:L1\%s:CR\%s:HT\%s}\\
Where the \code{\%s} is replaced with an integer index or inclusive range for hwloc.
\code{NM} identifies the numa node(s).
\code{SK} identifies the socket(s).
\code{L3} identifies the L3 cache(s).
\code{L2} identifies the L2 cache(s).
\code{L1} identifies the L1 cache(s).
\code{CR} identifies the cores(s).
\code{HT} identifies the hardware thread(s).
If your architecture does not have the specified hardware designation then it can be omitted from the signature.
\\
For example: \code{NM0:SK0:L30-4:L20-4:L10-4:CR0-4:HT0-39}.\\
This means numa node \code{0}, socket \code{0}, L3 caches \code{0,1,2,3,4}, L2 caches \code{0-4}, L1 caches \code{0-4}, cores \code{0,1,2,3,4}, and hardware threads \code{0-39}.
}

%
\declareAttribute{PMIX_HWLOC_SHMEM_ADDR}{"pmix.hwlocaddr"}{size_t}{
Address of the HWLOC shared memory segment.
}

%
\declareAttribute{PMIX_HWLOC_SHMEM_SIZE}{"pmix.hwlocsize"}{size_t}{
Size of the HWLOC shared memory segment.
}

%
\declareAttribute{PMIX_HWLOC_SHMEM_FILE}{"pmix.hwlocfile"}{char*}{
Path to the HWLOC shared memory file.
}

%
\declareAttribute{PMIX_HWLOC_XML_V1}{"pmix.hwlocxml1"}{char*}{
\ac{XML} representation of local topology using HWLOC's v1.x format.
}

%
\declareAttribute{PMIX_HWLOC_XML_V2}{"pmix.hwlocxml2"}{char*}{
\ac{XML} representation of local topology using HWLOC's v2.x format.
}

%
\declareNewAttribute{PMIX_HWLOC_SHARE_TOPO}{"pmix.hwlocsh"}{bool}{
Share the HWLOC topology via shared memory
}

%
\declareNewAttribute{PMIX_HWLOC_HOLE_KIND}{"pmix.hwlocholek"}{char*}{
Kind of VM ``hole'' HWLOC should use for shared memory
}


%%%%%%%%%%%
\subsection{Request-related attributes}
\label{api:struct:attributes:request}

These attributes are used to influence the behavior of various PMIx operations - they do not represent values accessed using the \refapi{PMIx_Get} \ac{API}.

%
\declareAttribute{PMIX_COLLECT_DATA}{"pmix.collect"}{bool}{
Collect data and return it at the end of the operation.
}

%
\declareAttribute{PMIX_TIMEOUT}{"pmix.timeout"}{int}{
Time in seconds before the specified operation should time out (\var{0} indicating infinite) in error.
The timeout parameter can help avoid ``hangs'' due to programming errors that prevent the target process from ever exposing its data.
}

%
\declareAttribute{PMIX_IMMEDIATE}{"pmix.immediate"}{bool}{
Specified operation should immediately return an error from the PMIx server if the requested data cannot be found - do not request it from the host \ac{RM}.
}

%
\declareAttribute{PMIX_WAIT}{"pmix.wait"}{int}{
Caller requests that the PMIx server wait until at least the specified number of values are found (\var{0} indicates all and is the default).
}

%
\declareAttribute{PMIX_COLLECTIVE_ALGO}{"pmix.calgo"}{char*}{
Comma-delimited list of algorithms to use for the collective operation. \ac{PMIx} does not impose any requirements on a host environment's collective algorithms. Thus, the acceptable values for this attribute will be environment-dependent - users are encouraged to check their host environment for supported values.
}

%
\declareDepAttribute{PMIX_COLLECTIVE_ALGO_REQD}{"pmix.calreqd"}{bool}{
If \code{true}, indicates that the requested choice of algorithm is mandatory.
}

%
\declareAttribute{PMIX_NOTIFY_COMPLETION}{"pmix.notecomp"}{bool}{
Notify the parent process upon termination of child job.
}

%
\declareAttribute{PMIX_RANGE}{"pmix.range"}{pmix_data_range_t}{
Value for calls to publish/lookup/unpublish or for monitoring event notifications.
}

%
\declareAttribute{PMIX_PERSISTENCE}{"pmix.persist"}{pmix_persistence_t}{
Value for calls to \refapi{PMIx_Publish}.
}

%
\declareAttribute{PMIX_DATA_SCOPE}{"pmix.scope"}{pmix_scope_t}{
Scope of the data to be found in a \refapi{PMIx_Get} call.
}

%
\declareAttribute{PMIX_OPTIONAL}{"pmix.optional"}{bool}{
Look only in the client's local data store for the requested value - do not request data from the PMIx server if not found.
}

%
\declareAttribute{PMIX_EMBED_BARRIER}{"pmix.embed.barrier"}{bool}{
Execute a blocking fence operation before executing the specified operation.
For example, \refapi{PMIx_Finalize} does not include an internal barrier operation by default.
This attribute would direct \refapi{PMIx_Finalize} to execute a barrier as part of the finalize operation.
}

%
\declareAttribute{PMIX_JOB_TERM_STATUS}{"pmix.job.term.status"}{pmix_status_t}{
Status to be returned upon job termination.
}

%
\declareAttribute{PMIX_PROC_STATE_STATUS}{"pmix.proc.state"}{pmix_proc_state_t}{
Process state
}


%%%%%%%%%%%
\subsection{Server-to-PMIx library attributes}
\label{api:struct:attributes:server2cl}

Attributes used by the host environment to pass data to its PMIx server library.
The data will then be parsed and provided to the local PMIx clients. These attributes
are all referenced using \refconst{PMIX_RANK_WILDCARD} except where noted.

%
\declareAttribute{PMIX_REGISTER_NODATA}{"pmix.reg.nodata"}{bool}{
Registration is for this namespace only, do not copy job data - this attribute is not accessed using the \refapi{PMIx_Get}
}

%
\declareAttribute{PMIX_PROC_DATA}{"pmix.pdata"}{pmix_data_array_t}{
Array of process data. Starts with rank, then contains more data.
}

%
\declareAttribute{PMIX_NODE_MAP}{"pmix.nmap"}{char*}{
Regular expression of nodes - see \ref{cptr:api_server:noderegex} for an explanation of its generation.
}

%
\declareAttribute{PMIX_PROC_MAP}{"pmix.pmap"}{char*}{
Regular expression describing processes on each node  - see \ref{cptr:api_server:ppnregex} for an explanation of its generation.
}

%
\declareAttribute{PMIX_ANL_MAP}{"pmix.anlmap"}{char*}{
Process mapping in Argonne National Laboratory's PMI-1/PMI-2 notation.
}

%
\declareAttribute{PMIX_APP_MAP_TYPE}{"pmix.apmap.type"}{char*}{
Type of mapping used to layout the application (e.g., \code{cyclic}).
}

%
\declareAttribute{PMIX_APP_MAP_REGEX}{"pmix.apmap.regex"}{char*}{
Regular expression describing the result of the process mapping.
}


%%%%%%%%%%%
\subsection{Server-to-Client attributes}
\label{api:struct:attributes:server2client}

Attributes used internally to communicate data from the PMIx server to the PMIx client - they do not represent values accessed using the \refapi{PMIx_Get} \ac{API}.

%
\declareAttribute{PMIX_PROC_BLOB}{"pmix.pblob"}{pmix_byte_object_t}{
Packed blob of process data.
}

%
\declareAttribute{PMIX_MAP_BLOB}{"pmix.mblob"}{pmix_byte_object_t}{
Packed blob of process location.
}


%%%%%%%%%%%
\subsection{Event handler registration and notification attributes}
\label{api:struct:attributes:event}

Attributes to support event registration and notification - they are values passed to the event registration and notification \acp{API} and therefore are not accessed using the \refapi{PMIx_Get} \ac{API}.

%
\declareAttribute{PMIX_EVENT_HDLR_NAME}{"pmix.evname"}{char*}{
String name identifying this handler.
}

%
\declareAttribute{PMIX_EVENT_HDLR_FIRST}{"pmix.evfirst"}{bool}{
Invoke this event handler before any other handlers.
}

%
\declareAttribute{PMIX_EVENT_HDLR_LAST}{"pmix.evlast"}{bool}{
Invoke this event handler after all other handlers have been called.
}

%
\declareAttribute{PMIX_EVENT_HDLR_FIRST_IN_CATEGORY}{"pmix.evfirstcat"}{bool}{
Invoke this event handler before any other handlers in this category.
}

%
\declareAttribute{PMIX_EVENT_HDLR_LAST_IN_CATEGORY}{"pmix.evlastcat"}{bool}{
Invoke this event handler after all other handlers in this category have been called.
}

%
\declareAttribute{PMIX_EVENT_HDLR_BEFORE}{"pmix.evbefore"}{char*}{
Put this event handler immediately before the one specified in the \code{(char*)} value.
}

%
\declareAttribute{PMIX_EVENT_HDLR_AFTER}{"pmix.evafter"}{char*}{
Put this event handler immediately after the one specified in the \code{(char*)} value.
}

%
\declareAttribute{PMIX_EVENT_HDLR_PREPEND}{"pmix.evprepend"}{bool}{
Prepend this handler to the precedence list within its category.
}

%
\declareAttribute{PMIX_EVENT_HDLR_APPEND}{"pmix.evappend"}{bool}{
Append this handler to the precedence list within its category.
}

%
\declareAttribute{PMIX_EVENT_CUSTOM_RANGE}{"pmix.evrange"}{pmix_data_array_t*}{
Array of \refstruct{pmix_proc_t} defining range of event notification.
}

%
\declareAttribute{PMIX_EVENT_AFFECTED_PROC}{"pmix.evproc"}{pmix_proc_t}{
The single process that was affected.
}

%
\declareAttribute{PMIX_EVENT_AFFECTED_PROCS}{"pmix.evaffected"}{pmix_data_array_t*}{
Array of \refstruct{pmix_proc_t} defining affected processes.
}

%
\declareAttribute{PMIX_EVENT_NON_DEFAULT}{"pmix.evnondef"}{bool}{
Event is not to be delivered to default event handlers.
}

%
\declareAttribute{PMIX_EVENT_RETURN_OBJECT}{"pmix.evobject"}{void *}{
Object to be returned whenever the registered callback function \code{cbfunc} is invoked.
The object will only be returned to the process that registered it.
}

%
\declareAttribute{PMIX_EVENT_DO_NOT_CACHE}{"pmix.evnocache"}{bool}{
Instruct the PMIx server not to cache the event.
}

%
\declareAttribute{PMIX_EVENT_SILENT_TERMINATION}{"pmix.evsilentterm"}{bool}{
Do not generate an event when this job normally terminates.
}

%
\declareAttribute{PMIX_EVENT_PROXY}{"pmix.evproxy"}{pmix_proc_t*}{
\ac{PMIx} server that sourced the event
}

%
\declareAttribute{PMIX_EVENT_TEXT_MESSAGE}{"pmix.evtext"}{char*}{
Text message suitable for output by recipient - e.g., describing the cause of the event
}


%%%%%%%%%%%
\subsection{Fault tolerance attributes}
\label{api:struct:attributes:faulttolerance}

Attributes to support fault tolerance behaviors - they are values passed to the event notification \ac{API} and therefore are not accessed using the \refapi{PMIx_Get} \ac{API}.

%
\declareAttribute{PMIX_EVENT_TERMINATE_SESSION}{"pmix.evterm.sess"}{bool}{
The \ac{RM} intends to terminate this session.
}

%
\declareAttribute{PMIX_EVENT_TERMINATE_JOB}{"pmix.evterm.job"}{bool}{
The \ac{RM} intends to terminate this job.
}

%
\declareAttribute{PMIX_EVENT_TERMINATE_NODE}{"pmix.evterm.node"}{bool}{
The \ac{RM} intends to terminate all processes on this node.
}

%
\declareAttribute{PMIX_EVENT_TERMINATE_PROC}{"pmix.evterm.proc"}{bool}{
The \ac{RM} intends to terminate just this process.
}

%
\declareAttribute{PMIX_EVENT_ACTION_TIMEOUT}{"pmix.evtimeout"}{int}{
The time in seconds before the \ac{RM} will execute error response.
}

%
\declareAttribute{PMIX_EVENT_NO_TERMINATION}{"pmix.evnoterm"}{bool}{
Indicates that the handler has satisfactorily handled the event and believes termination of the application is not required.
}

%
\declareAttribute{PMIX_EVENT_WANT_TERMINATION}{"pmix.evterm"}{bool}{
Indicates that the handler has determined that the application should be terminated
}


%%%%%%%%%%%
\subsection{Spawn attributes}
\label{api:struct:attributes:spawn}

Attributes used to describe \refapi{PMIx_Spawn} behavior - they are values passed to the \refapi{PMIx_Spawn} \ac{API} and therefore are not accessed using the \refapi{PMIx_Get} \ac{API} when used in that context. However, some of the attributes defined in this section can be provided by the host environment for other purposes - e.g., the environment might provide the \refattr{PMIX_MAPPER} attribute in the job-related information so that an application can use \refapi{PMIx_Get} to discover the layout algorithm used for determining process locations. Multi-use attributes and their respective access reference rank are denoted below.

%
\declareAttribute{PMIX_PERSONALITY}{"pmix.pers"}{char*}{
Name of personality to use.
}

%
\declareAttribute{PMIX_HOST}{"pmix.host"}{char*}{
Comma-delimited list of hosts to use for spawned processes.
}

%
\declareAttribute{PMIX_HOSTFILE}{"pmix.hostfile"}{char*}{
Hostfile to use for spawned processes.
}

%
\declareAttribute{PMIX_ADD_HOST}{"pmix.addhost"}{char*}{
Comma-delimited list of hosts to add to the allocation.
}

%
\declareAttribute{PMIX_ADD_HOSTFILE}{"pmix.addhostfile"}{char*}{
Hostfile listing hosts to add to existing allocation.
}

%
\declareAttribute{PMIX_PREFIX}{"pmix.prefix"}{char*}{
Prefix to use for starting spawned processes.
}

%
\declareAttribute{PMIX_WDIR}{"pmix.wdir"}{char*}{
Working directory for spawned processes.
}

%
\declareAttribute{PMIX_MAPPER}{"pmix.mapper"}{char*}{
Mapping mechanism to use for placing spawned processes - when accessed using \refapi{PMIx_Get}, use the \refconst{PMIX_RANK_WILDCARD} value for the rank to discover the mapping mechanism used for the provided namespace.
}

%
\declareAttribute{PMIX_DISPLAY_MAP}{"pmix.dispmap"}{bool}{
Display process mapping upon spawn.
}

%
\declareAttribute{PMIX_PPR}{"pmix.ppr"}{char*}{
Number of processes to spawn on each identified resource.
}

%
\declareAttribute{PMIX_MAPBY}{"pmix.mapby"}{char*}{
Process mapping policy - when accessed using \refapi{PMIx_Get}, use the \refconst{PMIX_RANK_WILDCARD} value for the rank to discover the mapping policy used for the provided namespace
}

%
\declareAttribute{PMIX_RANKBY}{"pmix.rankby"}{char*}{
Process ranking policy - when accessed using \refapi{PMIx_Get}, use the \refconst{PMIX_RANK_WILDCARD} value for the rank to discover the ranking algorithm used for the provided namespace
}

%
\declareAttribute{PMIX_BINDTO}{"pmix.bindto"}{char*}{
Process binding policy - when accessed using \refapi{PMIx_Get}, use the \refconst{PMIX_RANK_WILDCARD} value for the rank to discover the binding policy used for the provided namespace
}

%
\declareAttribute{PMIX_PRELOAD_BIN}{"pmix.preloadbin"}{bool}{
Preload binaries onto nodes.
}

%
\declareAttribute{PMIX_PRELOAD_FILES}{"pmix.preloadfiles"}{char*}{
Comma-delimited list of files to pre-position on nodes.
}

%
\declareAttribute{PMIX_NON_PMI}{"pmix.nonpmi"}{bool}{
Spawned processes will not call \refapi{PMIx_Init}.
}

%
\declareAttribute{PMIX_STDIN_TGT}{"pmix.stdin"}{uint32_t}{
Spawned process rank that is to receive \code{stdin}.
}

%
\declareAttribute{PMIX_FWD_STDIN}{"pmix.fwd.stdin"}{bool}{
Forward this process's \code{stdin} to the designated process.
}

%
\declareAttribute{PMIX_FWD_STDOUT}{"pmix.fwd.stdout"}{bool}{
Forward \code{stdout} from spawned processes to this process.
}

%
\declareAttribute{PMIX_FWD_STDERR}{"pmix.fwd.stderr"}{bool}{
Forward \code{stderr} from spawned processes to this process.
}

%
\declareNewAttribute{PMIX_FWD_STDDIAG}{"pmix.fwd.stddiag"}{bool}{
If a diagnostic channel exists, forward any output on it from the spawned processes to this process (typically used by a tool)
}

%
\declareAttribute{PMIX_DEBUGGER_DAEMONS}{"pmix.debugger"}{bool}{
Spawned application consists of debugger daemons.
}

%
\declareAttribute{PMIX_COSPAWN_APP}{"pmix.cospawn"}{bool}{
Designated application is to be spawned as a disconnected job.
Meaning that it is not part of the ``comm_world'' of the parent process.
}

%
\declareAttribute{PMIX_SET_SESSION_CWD}{"pmix.ssncwd"}{bool}{
Set the application's current working directory to the session working directory assigned by the \ac{RM} - when accessed using \refapi{PMIx_Get}, use the \refconst{PMIX_RANK_WILDCARD} value for the rank to discover the session working directory assigned to the provided namespace
}

%
\declareAttribute{PMIX_TAG_OUTPUT}{"pmix.tagout"}{bool}{
Tag application output with the identity of the source process.
}

%
\declareAttribute{PMIX_TIMESTAMP_OUTPUT}{"pmix.tsout"}{bool}{
Timestamp output from applications.
}

%
\declareAttribute{PMIX_MERGE_STDERR_STDOUT}{"pmix.mergeerrout"}{bool}{
Merge \code{stdout} and \code{stderr} streams from application processes.
}

%
\declareAttribute{PMIX_OUTPUT_TO_FILE}{"pmix.outfile"}{char*}{
Output application output to the specified file.
}

%
\declareAttribute{PMIX_INDEX_ARGV}{"pmix.indxargv"}{bool}{
Mark the \code{argv} with the rank of the process.
}

%
\declareAttribute{PMIX_CPUS_PER_PROC}{"pmix.cpuperproc"}{uint32_t}{
Number of cpus to assign to each rank - when accessed using \refapi{PMIx_Get}, use the \refconst{PMIX_RANK_WILDCARD} value for the rank to discover the cpus/process assigned to the provided namespace
}

%
\declareAttribute{PMIX_NO_PROCS_ON_HEAD}{"pmix.nolocal"}{bool}{
Do not place processes on the head node.
}

%
\declareAttribute{PMIX_NO_OVERSUBSCRIBE}{"pmix.noover"}{bool}{
Do not oversubscribe the cpus.
}

%
\declareAttribute{PMIX_REPORT_BINDINGS}{"pmix.repbind"}{bool}{
Report bindings of the individual processes.
}

%
\declareAttribute{PMIX_CPU_LIST}{"pmix.cpulist"}{char*}{
List of cpus to use for this job - when accessed using \refapi{PMIx_Get}, use the \refconst{PMIX_RANK_WILDCARD} value for the rank to discover the cpu list used for the provided namespace
}

%
\declareAttribute{PMIX_JOB_RECOVERABLE}{"pmix.recover"}{bool}{
Application supports recoverable operations.
}

%
\declareAttribute{PMIX_JOB_CONTINUOUS}{"pmix.continuous"}{bool}{
Application is continuous, all failed processes should be immediately restarted.
}

%
\declareAttribute{PMIX_MAX_RESTARTS}{"pmix.maxrestarts"}{uint32_t}{
Maximum number of times to restart a job - when accessed using \refapi{PMIx_Get}, use the \refconst{PMIX_RANK_WILDCARD} value for the rank to discover the max restarts for the provided namespace
}

%
\declareNewAttribute{PMIX_SPAWN_TOOL}{"pmix.spwn.tool"}{bool}{
Indicate that the job being spawned is a tool
}


%%%%%%%%%%%
\subsection{Query attributes}
\label{api:struct:attributes:query}

Attributes used to describe \refapi{PMIx_Query_info_nb} behavior - these are values passed to the \refapi{PMIx_Query_info_nb} \ac{API} and therefore are not passed to the \refapi{PMIx_Get} \ac{API}.

%
\declareAttribute{PMIX_QUERY_REFRESH_CACHE}{"pmix.qry.rfsh"}{bool}{
Retrieve updated information from server.
}

%
\declareAttribute{PMIX_QUERY_NAMESPACES}{"pmix.qry.ns"}{char*}{
Request a comma-delimited list of active namespaces.
}

%
\declareAttribute{PMIX_QUERY_JOB_STATUS}{"pmix.qry.jst"}{pmix_status_t}{
Status of a specified, currently executing job.
}

%
\declareAttribute{PMIX_QUERY_QUEUE_LIST}{"pmix.qry.qlst"}{char*}{
Request a comma-delimited list of scheduler queues.
}

%
\declareAttribute{PMIX_QUERY_QUEUE_STATUS}{"pmix.qry.qst"}{TBD}{
Status of a specified scheduler queue.
}

%
\declareAttribute{PMIX_QUERY_PROC_TABLE}{"pmix.qry.ptable"}{char*}{
Input namespace of the job whose information is being requested returns (\refstruct{pmix_data_array_t}) an array of \refstruct{pmix_proc_info_t}.
}

%
\declareAttribute{PMIX_QUERY_LOCAL_PROC_TABLE}{"pmix.qry.lptable"}{char*}{
Input namespace of the job whose information is being requested returns (\refstruct{pmix_data_array_t}) an array of \refstruct{pmix_proc_info_t} for processes in job on same node.
}

%
\declareAttribute{PMIX_QUERY_AUTHORIZATIONS}{"pmix.qry.auths"}{bool}{
Return operations the PMIx tool is authorized to perform.
}

%
\declareAttribute{PMIX_QUERY_SPAWN_SUPPORT}{"pmix.qry.spawn"}{bool}{
Return a comma-delimited list of supported spawn attributes.
}

%
\declareAttribute{PMIX_QUERY_DEBUG_SUPPORT}{"pmix.qry.debug"}{bool}{
Return a comma-delimited list of supported debug attributes.
}

%
\declareAttribute{PMIX_QUERY_MEMORY_USAGE}{"pmix.qry.mem"}{bool}{
Return information on memory usage for the processes indicated in the qualifiers.
}

%
\declareAttribute{PMIX_QUERY_LOCAL_ONLY}{"pmix.qry.local"}{bool}{
Constrain the query to local information only.
}

%
\declareAttribute{PMIX_QUERY_REPORT_AVG}{"pmix.qry.avg"}{bool}{
Report only average values for sampled information.
}

%
\declareAttribute{PMIX_QUERY_REPORT_MINMAX}{"pmix.qry.minmax"}{bool}{
Report minimum and maximum values.
}

%
\declareAttribute{PMIX_QUERY_ALLOC_STATUS}{"pmix.query.alloc"}{char*}{
String identifier of the allocation whose status is being requested.
}

%
\declareAttribute{PMIX_TIME_REMAINING}{"pmix.time.remaining"}{char*}{
Query number of seconds (\code{uint32_t}) remaining in allocation for the specified namespace.
}


%%%%%%%%%%%
\subsection{Log attributes}
\label{api:struct:attributes:log}

Attributes used to describe \refapi{PMIx_Log_nb} behavior - these are values passed to the \refapi{PMIx_Log_nb} \ac{API} and therefore are not accessed using the \refapi{PMIx_Get} \ac{API}.

%
\declareNewAttribute{PMIX_LOG_SOURCE}{"pmix.log.source"}{pmix_proc_t*}{
ID of source of the log request
}

%
\declareAttribute{PMIX_LOG_STDERR}{"pmix.log.stderr"}{char*}{
Log string to \code{stderr}.
}

%
\declareAttribute{PMIX_LOG_STDOUT}{"pmix.log.stdout"}{char*}{
Log string to \code{stdout}.
}

%
\declareAttribute{PMIX_LOG_SYSLOG}{"pmix.log.syslog"}{char*}{
Log data to syslog.
Defaults to \code{ERROR} priority.  Will log to global syslog if available, otherwise to local syslog
}

%
\declareNewAttribute{PMIX_LOG_LOCAL_SYSLOG}{"pmix.log.lsys"}{char*}{
Log data to local syslog.
Defaults to \code{ERROR} priority.
}

%
\declareNewAttribute{PMIX_LOG_GLOBAL_SYSLOG}{"pmix.log.gsys"}{char*}{
Forward data to system ``gateway'' and log msg to that syslog
Defaults to \code{ERROR} priority.
}

%
\declareNewAttribute{PMIX_LOG_SYSLOG_PRI}{"pmix.log.syspri"}{int}{
Syslog priority level
}

%
\declareNewAttribute{PMIX_LOG_TIMESTAMP}{"pmix.log.tstmp"}{time_t}{
Timestamp for log report
}

%
\declareNewAttribute{PMIX_LOG_GENERATE_TIMESTAMP}{"pmix.log.gtstmp"}{bool}{
Generate timestamp for log
}

%
\declareNewAttribute{PMIX_LOG_TAG_OUTPUT}{"pmix.log.tag"}{bool}{
Label the output stream with the channel name (e.g., ``stdout'')
}

%
\declareNewAttribute{PMIX_LOG_TIMESTAMP_OUTPUT}{"pmix.log.tsout"}{bool}{
Print timestamp in output string
}

%
\declareNewAttribute{PMIX_LOG_XML_OUTPUT}{"pmix.log.xml"}{bool}{
Print the output stream in \ac{XML} format
}

%
\declareNewAttribute{PMIX_LOG_ONCE}{"pmix.log.once"}{bool}{
Only log this once with whichever channel can first support it, taking the channels in priority order
}

%
\declareAttribute{PMIX_LOG_MSG}{"pmix.log.msg"}{pmix_byte_object_t}{
Message blob to be sent somewhere.
}

%
\declareAttribute{PMIX_LOG_EMAIL}{"pmix.log.email"}{pmix_data_array_t}{
Log via email based on \refstruct{pmix_info_t} containing directives.
}

%
\declareAttribute{PMIX_LOG_EMAIL_ADDR}{"pmix.log.emaddr"}{char*}{
Comma-delimited list of email addresses that are to receive the message.
}

%
\declareNewAttribute{PMIX_LOG_EMAIL_SENDER_ADDR}{"pmix.log.emfaddr"}{char*}{
Return email address of sender
}

%
\declareAttribute{PMIX_LOG_EMAIL_SUBJECT}{"pmix.log.emsub"}{char*}{
Subject line for email.
}

%
\declareAttribute{PMIX_LOG_EMAIL_MSG}{"pmix.log.emmsg"}{char*}{
Message to be included in email.
}

%
\declareNewAttribute{PMIX_LOG_EMAIL_SERVER}{"pmix.log.esrvr"}{char*}{
Hostname (or IP address) of estmp server
}

%
\declareNewAttribute{PMIX_LOG_EMAIL_SRVR_PORT}{"pmix.log.esrvrprt"}{int32_t}{
Port the email server is listening to
}

%
\declareNewAttribute{PMIX_LOG_GLOBAL_DATASTORE}{"pmix.log.gstore"}{bool}{
Store the log data in a global data store (e.g., database)
}

%
\declareNewAttribute{PMIX_LOG_JOB_RECORD}{"pmix.log.jrec"}{bool}{
Log the provided information to the host environment's job record
}


%%%%%%%%%%%
\subsection{Debugger attributes}
\label{api:struct:attributes:debugger}

Attributes used to assist debuggers - these are values that can be passed to the \refapi{PMIx_Spawn} or \refapi{PMIx_Init} \acp{API}. Some may be accessed using the \refapi{PMIx_Get} \ac{API} with the \refconst{PMIX_RANK_WILDCARD} rank.

%
\declareNewAttribute{PMIX_DEBUG_STOP_ON_EXEC}{"pmix.dbg.exec"}{bool}{
Passed to \refapi{PMIx_Spawn} to indicate that the specified application is being spawned under debugger, and that the launcher is to pause the resulting application processes on first instruction for debugger attach.
}

%
\declareAttribute{PMIX_DEBUG_STOP_IN_INIT}{"pmix.dbg.init"}{bool}{
Passed to \refapi{PMIx_Spawn} to indicate that the specified application is being spawned under debugger, and that the \ac{PMIx} client library is to pause the resulting application processes during \refapi{PMIx_Init} until debugger attach and release.
}

%
\declareAttribute{PMIX_DEBUG_WAIT_FOR_NOTIFY}{"pmix.dbg.notify"}{bool}{
Passed to \refapi{PMIx_Spawn} to indicate that the specified application is being spawned under debugger, and that the resulting application processes are to pause at some application-determined location until debugger attach and release.
}

%
\declareAttribute{PMIX_DEBUG_JOB}{"pmix.dbg.job"}{char*}{
Namespace of the job to be debugged - provided to the debugger upon launch.
}

%
\declareAttribute{PMIX_DEBUG_WAITING_FOR_NOTIFY}{"pmix.dbg.waiting"}{bool}{
Job to be debugged is waiting for a release - this is not a value accessed using the \refapi{PMIx_Get} \ac{API}.
}

%
\declareNewAttribute{PMIX_DEBUG_JOB_DIRECTIVES}{"pmix.dbg.jdirs"}{pmix_data_array_t*}{
Array of job-level directives
}

%
\declareNewAttribute{PMIX_DEBUG_APP_DIRECTIVES}{"pmix.dbg.adirs"}{pmix_data_array_t*}{
Array of app-level directives
}


%%%%%%%%%%%
\subsection{Resource manager attributes}
\label{api:struct:attributes:rm}

Attributes used to describe the \ac{RM} - these are values assigned by the host environment and accessed using the \refapi{PMIx_Get} \ac{API}. The value of the provided namespace is unimportant but should be given as the namespace of the requesting process and a rank of \refconst{PMIX_RANK_WILDCARD} used to indicate that the information will be found with the job-level information.

%
\declareAttribute{PMIX_RM_NAME}{"pmix.rm.name"}{char*}{
String name of the \ac{RM}.
}

%
\declareAttribute{PMIX_RM_VERSION}{"pmix.rm.version"}{char*}{
\ac{RM} version string.
}


%%%%%%%%%%%
\subsection{Environment variable attributes}
\label{api:struct:attributes:envar}

Attributes used to adjust environment variables - these are values passed to the \refapi{PMIx_Spawn} \ac{API} and are not accessed using the \refapi{PMIx_Get} \ac{API}.

%
\declareAttribute{PMIX_SET_ENVAR}{"pmix.envar.set"}{pmix_envar_t*}{
Set the envar to the given value, overwriting any pre-existing one
}

%
\declareAttribute{PMIX_UNSET_ENVAR}{"pmix.envar.unset"}{char*}{
Unset the environment variable specified in the string.
}

%
\declareNewAttribute{PMIX_ADD_ENVAR}{"pmix.envar.add"}{pmix_envar_t*}{
Add the environment variable, but do not overwrite any pre-existing one
}

%
\declareNewAttribute{PMIX_PREPEND_ENVAR}{"pmix.envar.prepnd"}{pmix_envar_t*}{
Prepend the given value to the specified environmental value using the given separator character, creating the variable if it doesn't already exist
}

%
\declareNewAttribute{PMIX_APPEND_ENVAR}{"pmix.envar.appnd"}{pmix_envar_t*}{
Append the given value to the specified environmental value using the given separator character, creating the variable if it doesn't already exist
}


%%%%%%%%%%%
\subsection{Job Allocation attributes}
\label{api:struct:attributes:joballoc}

Attributes used to describe the job allocation - these are values passed to and/or returned by the \refapi{PMIx_Allocation_request_nb} and \refapi{PMIx_Allocation_request} \acp{API} and are not accessed using the \refapi{PMIx_Get} \ac{API}

%
\declareAttribute{PMIX_ALLOC_REQ_ID}{"pmix.alloc.reqid"}{char*}{
User-provided string identifier for this allocation request which can later be used to query status of the request.
}

%
\declareNewAttribute{PMIX_ALLOC_ID}{"pmix.alloc.id"}{char*}{
A string identifier (provided by the host environment) for the resulting allocation which can later be used to reference the allocated resources in, for example, a call to \refapi{PMIx_Spawn}.
}

%
\declareAttribute{PMIX_ALLOC_NUM_NODES}{"pmix.alloc.nnodes"}{uint64_t}{
The number of nodes.
}

%
\declareAttribute{PMIX_ALLOC_NODE_LIST}{"pmix.alloc.nlist"}{char*}{
Regular expression of the specific nodes.
}

%
\declareAttribute{PMIX_ALLOC_NUM_CPUS}{"pmix.alloc.ncpus"}{uint64_t}{
Number of cpus.
}

%
\declareAttribute{PMIX_ALLOC_NUM_CPU_LIST}{"pmix.alloc.ncpulist"}{char*}{
Regular expression of the number of cpus for each node.
}

%
\declareAttribute{PMIX_ALLOC_CPU_LIST}{"pmix.alloc.cpulist"}{char*}{
Regular expression of the specific cpus indicating the cpus involved.
}

%
\declareAttribute{PMIX_ALLOC_MEM_SIZE}{"pmix.alloc.msize"}{float}{
Number of Megabytes.
}

%
\declareAttribute{PMIX_ALLOC_NETWORK}{"pmix.alloc.net"}{array}{
Array of \refstruct{pmix_info_t} describing requested network resources. This must include at least: \refattr{PMIX_ALLOC_NETWORK_ID}, \refattr{PMIX_ALLOC_NETWORK_TYPE}, and \refattr{PMIX_ALLOC_NETWORK_ENDPTS}, plus whatever other descriptors are desired.
}

%
\declareAttribute{PMIX_ALLOC_NETWORK_ID}{"pmix.alloc.netid"}{char*}{
The key to be used when accessing this requested network allocation. The allocation will be returned/stored as a \refstruct{pmix_data_array_t} of \refstruct{pmix_info_t} indexed by this key and containing at least one entry with the same key and the allocated resource description.
The type of the included value depends upon the network support. For example, a TCP allocation might consist of a comma-delimited string of socket ranges such as \var{"32000-32100,33005,38123-38146"}. Additional entries will consist of any provided resource request directives, along with their assigned values. Examples include: \refattr{PMIX_ALLOC_NETWORK_TYPE} - the type of resources provided; \refattr{PMIX_ALLOC_NETWORK_PLANE} - if applicable, what plane the resources were assigned from; \refattr{PMIX_ALLOC_NETWORK_QOS} - the assigned QoS; \refattr{PMIX_ALLOC_BANDWIDTH} - the allocated bandwidth; \refattr{PMIX_ALLOC_NETWORK_SEC_KEY} - a security key for the requested network allocation. NOTE: the assigned values may differ from those requested, especially if \refconst{PMIX_INFO_REQD} was not set in the request.
}

%
\declareAttribute{PMIX_ALLOC_BANDWIDTH}{"pmix.alloc.bw"}{float}{
Mbits/sec.
}

%
\declareAttribute{PMIX_ALLOC_NETWORK_QOS}{"pmix.alloc.netqos"}{char*}{
Quality of service level.
}

%
\declareAttribute{PMIX_ALLOC_TIME}{"pmix.alloc.time"}{uint32_t}{
Time in seconds.
}

%
\declareNewAttribute{PMIX_ALLOC_NETWORK_TYPE}{"pmix.alloc.nettype"}{char*}{
Type of desired transport (e.g., \var{``tcp''}, \var{``udp''})
}

%
\declareNewAttribute{PMIX_ALLOC_NETWORK_PLANE}{"pmix.alloc.netplane"}{char*}{
ID string for the NIC (aka \textit{plane}) to be used for this allocation (e.g., CIDR for Ethernet)
}

%
\declareNewAttribute{PMIX_ALLOC_NETWORK_ENDPTS}{"pmix.alloc.endpts"}{size_t}{
Number of endpoints to allocate per process
}

%
\declareNewAttribute{PMIX_ALLOC_NETWORK_ENDPTS_NODE}{"pmix.alloc.endpts.nd"}{size_t}{
Number of endpoints to allocate per node
}

%
\declareNewAttribute{PMIX_ALLOC_NETWORK_SEC_KEY}{"pmix.alloc.nsec"}{pmix_byte_object_t}{
Network security key
}


%%%%%%%%%%%
\subsection{Job control attributes}
\label{api:struct:attributes:jobcontrol}

Attributes used to request control operations on an executing application - these are values passed to the \refapi{PMIx_Job_control_nb} \ac{API} and are not accessed using the \refapi{PMIx_Get} \ac{API}.

%
\declareAttribute{PMIX_JOB_CTRL_ID}{"pmix.jctrl.id"}{char*}{
Provide a string identifier for this request.
}

%
\declareAttribute{PMIX_JOB_CTRL_PAUSE}{"pmix.jctrl.pause"}{bool}{
Pause the specified processes.
}

%
\declareAttribute{PMIX_JOB_CTRL_RESUME}{"pmix.jctrl.resume"}{bool}{
Resume (``un-pause'') the specified processes.
}

%
\declareAttribute{PMIX_JOB_CTRL_CANCEL}{"pmix.jctrl.cancel"}{char*}{
Cancel the specified request (\code{NULL} implies cancel all requests from this requestor).
}

%
\declareAttribute{PMIX_JOB_CTRL_KILL}{"pmix.jctrl.kill"}{bool}{
Forcibly terminate the specified processes and cleanup.
}

%
\declareAttribute{PMIX_JOB_CTRL_RESTART}{"pmix.jctrl.restart"}{char*}{
Restart the specified processes using the given checkpoint ID.
}

%
\declareAttribute{PMIX_JOB_CTRL_CHECKPOINT}{"pmix.jctrl.ckpt"}{char*}{
Checkpoint the specified processes and assign the given ID to it.
}

%
\declareAttribute{PMIX_JOB_CTRL_CHECKPOINT_EVENT}{"pmix.jctrl.ckptev"}{bool}{
Use event notification to trigger a process checkpoint.
}

%
\declareAttribute{PMIX_JOB_CTRL_CHECKPOINT_SIGNAL}{"pmix.jctrl.ckptsig"}{int}{
Use the given signal to trigger a process checkpoint.
}

%
\declareAttribute{PMIX_JOB_CTRL_CHECKPOINT_TIMEOUT}{"pmix.jctrl.ckptsig"}{int}{
Time in seconds to wait for a checkpoint to complete.
}

%
\declareAttribute{PMIX_JOB_CTRL_CHECKPOINT_METHOD}{"pmix.jctrl.ckmethod"}{pmix_data_array_t}{
Array of \refstruct{pmix_info_t} declaring each method and value supported by this application.
}

%
\declareAttribute{PMIX_JOB_CTRL_SIGNAL}{"pmix.jctrl.sig"}{int}{
Send given signal to specified processes.
}

%
\declareAttribute{PMIX_JOB_CTRL_PROVISION}{"pmix.jctrl.pvn"}{char*}{
Regular expression identifying nodes that are to be provisioned.
}

%
\declareAttribute{PMIX_JOB_CTRL_PROVISION_IMAGE}{"pmix.jctrl.pvnimg"}{char*}{
Name of the image that is to be provisioned.
}

%
\declareAttribute{PMIX_JOB_CTRL_PREEMPTIBLE}{"pmix.jctrl.preempt"}{bool}{
Indicate that the job can be pre-empted.
}

%
\declareAttribute{PMIX_JOB_CTRL_TERMINATE}{"pmix.jctrl.term"}{bool}{
Politely terminate the specified processes.
}

%
\declareNewAttribute{PMIX_REGISTER_CLEANUP}{"pmix.reg.cleanup"}{char*}{
Comma-delimited list of files to be removed upon process termination
}

%
\declareNewAttribute{PMIX_REGISTER_CLEANUP_DIR}{"pmix.reg.cleanupdir"}{char*}{
Comma-delimited list of directories to be removed upon process termination
}

%
\declareNewAttribute{PMIX_CLEANUP_RECURSIVE}{"pmix.clnup.recurse"}{bool}{
Recursively cleanup all subdirectories under the specified one(s)
}

%
\declareNewAttribute{PMIX_CLEANUP_EMPTY}{"pmix.clnup.empty"}{bool}{
Only remove empty subdirectories
}

%
\declareNewAttribute{PMIX_CLEANUP_IGNORE}{"pmix.clnup.ignore"}{char*}{
Comma-delimited list of filenames that are not to be removed
}

%
\declareNewAttribute{PMIX_CLEANUP_LEAVE_TOPDIR}{"pmix.clnup.lvtop"}{bool}{
When recursively cleaning subdirectories, do not remove the top-level directory (the one given in the cleanup request)
}


%%%%%%%%%%%
\subsection{Monitoring attributes}
\label{api:struct:attributes:monitor}

Attributes used to control monitoring of an executing application- these are values passed to the \refapi{PMIx_Process_monitor_nb} \ac{API} and are not accessed using the \refapi{PMIx_Get} \ac{API}.

%
\declareAttribute{PMIX_MONITOR_ID}{"pmix.monitor.id"}{char*}{
Provide a string identifier for this request.
}

%
\declareAttribute{PMIX_MONITOR_CANCEL}{"pmix.monitor.cancel"}{char*}{
Identifier to be canceled (\code{NULL} means cancel all monitoring for this process).
}

%
\declareAttribute{PMIX_MONITOR_APP_CONTROL}{"pmix.monitor.appctrl"}{bool}{
The application desires to control the response to a monitoring event.
}

%
\declareAttribute{PMIX_MONITOR_HEARTBEAT}{"pmix.monitor.mbeat"}{void}{
Register to have the PMIx server monitor the requestor for heartbeats.
}

%
\declareAttribute{PMIX_SEND_HEARTBEAT}{"pmix.monitor.beat"}{void}{
Send heartbeat to local PMIx server.
}

%
\declareAttribute{PMIX_MONITOR_HEARTBEAT_TIME}{"pmix.monitor.btime"}{uint32_t}{
Time in seconds before declaring heartbeat missed.
}

%
\declareAttribute{PMIX_MONITOR_HEARTBEAT_DROPS}{"pmix.monitor.bdrop"}{uint32_t}{
Number of heartbeats that can be missed before generating the event.
}

%
\declareAttribute{PMIX_MONITOR_FILE}{"pmix.monitor.fmon"}{char*}{
Register to monitor file for signs of life.
}

%
\declareAttribute{PMIX_MONITOR_FILE_SIZE}{"pmix.monitor.fsize"}{bool}{
Monitor size of given file is growing to determine if the application is running.
}

%
\declareAttribute{PMIX_MONITOR_FILE_ACCESS}{"pmix.monitor.faccess"}{char*}{
Monitor time since last access of given file to determine if the application is running.
}

%
\declareAttribute{PMIX_MONITOR_FILE_MODIFY}{"pmix.monitor.fmod"}{char*}{
Monitor time since last modified of given file to determine if the application is running.
}

%
\declareAttribute{PMIX_MONITOR_FILE_CHECK_TIME}{"pmix.monitor.ftime"}{uint32_t}{
Time in seconds between checking the file.
}

%
\declareAttribute{PMIX_MONITOR_FILE_DROPS}{"pmix.monitor.fdrop"}{uint32_t}{
Number of file checks that can be missed before generating the event.
}

%%%%%%%%%%%
\subsection{Security attributes}
\label{api:struct:attributes:security}

\versionMarker{3.0}
Attributes for managing security credentials

%
\declareNewAttribute{PMIX_CRED_TYPE}{"pmix.sec.ctype"}{char*}{
When passed in \refapi{PMIx_Get_credential}, a prioritized, comma-delimited list of desired credential types for use
in environments where multiple authentication mechanisms may be available. When returned in a callback function, a
string identifier of the credential type.
}

%
\declareNewAttribute{PMIX_CRYPTO_KEY}{"pmix.sec.key"}{pmix_byte_object_t}{
Blob containing crypto key
}


%%%%%%%%%%%
\subsection{IO Forwarding attributes}
\label{api:struct:attributes:security}

\versionMarker{3.0}
Attributes used to control IO forwarding behavior

%
\declareNewAttribute{PMIX_IOF_CACHE_SIZE}{"pmix.iof.csize"}{uint32_t}{
The requested size of the server cache in bytes for each specified channel. By default, the server is allowed (but not required) to drop all bytes received beyond the max size.
}

%
\declareNewAttribute{PMIX_IOF_DROP_OLDEST}{"pmix.iof.old"}{bool}{
In an overflow situation, drop the oldest bytes to make room in the cache.
}

%
\declareNewAttribute{PMIX_IOF_DROP_NEWEST}{"pmix.iof.new"}{bool}{
In an overflow situation, drop any new bytes received until room becomes available in the cache (default).
}

%
\declareNewAttribute{PMIX_IOF_BUFFERING_SIZE}{"pmix.iof.bsize"}{uint32_t}{
Controls grouping of IO on the specified channel(s) to avoid being called every time a bit of IO arrives. The library will execute the callback whenever the specified number of bytes becomes available. Any remaining buffered data will be ``flushed'' upon call to deregister the respective channel.
}

%
\declareNewAttribute{PMIX_IOF_BUFFERING_TIME}{"pmix.iof.btime"}{uint32_t}{
Max time in seconds to buffer IO before delivering it. Used in conjunction with buffering size, this
prevents IO from being held indefinitely while waiting for another payload to arrive.
}

%
\declareNewAttribute{PMIX_IOF_COMPLETE}{"pmix.iof.cmp"}{bool}{
Indicates whether or not the specified IO channel has been closed by the source.
}

%
\declareNewAttribute{PMIX_IOF_TAG_OUTPUT}{"pmix.iof.tag"}{bool}{
Tag output with the channel it comes from.
}

%
\declareNewAttribute{PMIX_IOF_TIMESTAMP_OUTPUT}{"pmix.iof.ts"}{bool}{
Timestamp output
}

%
\declareNewAttribute{PMIX_IOF_XML_OUTPUT}{"pmix.iof.xml"}{bool}{
Format output in \ac{XML}
}

%%%%%%%%%%%
\subsection{Application setup attributes}
\label{api:struct:attributes:security}

\versionMarker{3.0}
Attributes for controlling contents of application setup data

%
\declareNewAttribute{PMIX_SETUP_APP_ENVARS}{"pmix.setup.env"}{bool}{
Harvest and include relevant environmental variables
}

%
\declareNewAttribute{PMIX_SETUP_APP_NONENVARS}{""pmix.setup.nenv"}{bool}{
Include all relevant data other than environmental variables
}

%
\declareNewAttribute{PMIX_SETUP_APP_ALL}{"pmix.setup.all"}{bool}{
Include all relevant data
}

%%%%%%%%%%%
\section{Callback Functions}

PMIx provides blocking and nonblocking versions of most APIs.
In the nonblocking versions, a callback is activated upon completion of the the operation.
This section describes many of those callbacks.


%%%%%%%%%%%
\subsection{Release Callback Function}
\declareapi{pmix_release_cbfunc_t}

%%%%
\summary

The \refapi{pmix_release_cbfunc_t} is used by the \refapi{pmix_modex_cbfunc_t} and \refapi{pmix_info_cbfunc_t} operations to indicate that the callback data may be reclaimed/freed by the caller.

%%%%
\format

\versionMarker{1.0}
\cspecificstart
\begin{codepar}
typedef void (*pmix_release_cbfunc_t)
    (void *cbdata)
\end{codepar}
\cspecificend

\begin{arglist}
\arginout{cbdata}{Callback data passed to original API call (memory reference)}
\end{arglist}

%%%%
\descr

Since the data is ``owned'' by the host server, provide a callback function to notify the host server that we are done with the data so it can be released.


%%%%%%%%%%%
\subsection{Modex Callback Function}
\declareapi{pmix_modex_cbfunc_t}

%%%%
\summary

The \refapi{pmix_modex_cbfunc_t} is used by the \refapi{pmix_server_fencenb_fn_t} and \refapi{pmix_server_dmodex_req_fn_t} PMIx server operations to return modex \ac{BCX} data.

\versionMarker{1.0}
\cspecificstart
\begin{codepar}
typedef void (*pmix_modex_cbfunc_t)
    (pmix_status_t status,
     const char *data, size_t ndata,
     void *cbdata,
     pmix_release_cbfunc_t release_fn,
     void *release_cbdata)
\end{codepar}
\cspecificend

\begin{arglist}
\argin{status}{Status associated with the operation (handle)}
\argin{data}{Data to be passed (pointer)}
\argin{ndata}{size of the data (\code{size_t})}
\argin{cbdata}{Callback data passed to original API call (memory reference)}
\argin{release_fn}{Callback for releasing \argref{data} (function pointer)}
\argin{release_cbdata}{Pointer to be passed to \argref{release_fn} (memory reference)}
\end{arglist}

%%%%
\descr

A callback function that is solely used by PMIx servers, and not clients, to return modex \ac{BCX} data in response to ``fence'' and ``get'' operations.
The returned blob contains the data collected from each server participating in the operation.



%%%%%%%%%%%
\subsection{Spawn Callback Function}
\declareapi{pmix_spawn_cbfunc_t}

%%%%
\summary

The \refapi{pmix_spawn_cbfunc_t} is used on the PMIx client side by \refapi{PMIx_Spawn_nb} and on the PMIx server side by \refapi{pmix_server_spawn_fn_t}.

\versionMarker{1.0}
\cspecificstart
\begin{codepar}
typedef void (*pmix_spawn_cbfunc_t)
    (pmix_status_t status,
     pmix_nspace_t nspace, void *cbdata);
\end{codepar}
\cspecificend

\begin{arglist}
\argin{status}{Status associated with the operation (handle)}
\argin{nspace}{Namespace string (\refstruct{pmix_nspace_t})}
\argin{cbdata}{Callback data passed to original API call (memory reference)}
\end{arglist}


%%%%
\descr

The callback will be executed upon launch of the specified applications in \refapi{PMIx_Spawn_nb}, or upon failure to launch any of them.

The \refarg{status} of the callback will indicate whether or not the spawn succeeded.
The \refarg{nspace} of the spawned processes will be returned, along with any provided callback data.
Note that the returned \refarg{nspace} value will not be protected by the \ac{PRI} upon return from the callback function, so the receiver must copy it if it needs to be retained.


%%%%%%%%%%%
\subsection{Op Callback Function}
\declareapi{pmix_op_cbfunc_t}

%%%%
\summary

The \refapi{pmix_op_cbfunc_t} is used by operations that simply return a status.

\versionMarker{1.0}
\cspecificstart
\begin{codepar}
typedef void (*pmix_op_cbfunc_t)
    (pmix_status_t status, void *cbdata);
\end{codepar}
\cspecificend

\begin{arglist}
\argin{status}{Status associated with the operation (handle)}
\argin{cbdata}{Callback data passed to original API call (memory reference)}
\end{arglist}

%%%%
\descr

Used by a wide range of \ac{PMIx} API's including \refapi{PMIx_Fence_nb}, \refapi{pmix_server_client_connected_fn_t}, \refapi{PMIx_server_register_nspace}.
This callback function is used to return a status to an often nonblocking operation.


%%%%%%%%%%%
\subsection{Lookup Callback Function}
\declareapi{pmix_lookup_cbfunc_t}

%%%%
\summary

The \refapi{pmix_lookup_cbfunc_t} is used by \refapi{PMIx_Lookup_nb} to return data.

\versionMarker{1.0}
\cspecificstart
\begin{codepar}
typedef void (*pmix_lookup_cbfunc_t)
    (pmix_status_t status,
     pmix_pdata_t data[], size_t ndata,
     void *cbdata);
\end{codepar}
\cspecificend

\begin{arglist}
\argin{status}{Status associated with the operation (handle)}
\argin{data}{Array of data returned (\refstruct{pmix_pdata_t})}
\argin{ndata}{Number of elements in the \argref{data} array (\code{size_t})}
\argin{cbdata}{Callback data passed to original API call (memory reference)}
\end{arglist}


%%%%
\descr

A callback function for calls to \refapi{PMIx_Lookup_nb}
The function will be called upon completion of the command with the \refarg{status} indicating the success or failure of the request.
Any retrieved data will be returned in an array of \refstruct{pmix_pdata_t} structs.
The namespace and rank of the process that provided each data element is also returned.

Note that these structures will be released upon return from the callback function, so the receiver must copy/protect the data prior to returning if it needs to be retained.


%%%%%%%%%%%
\subsection{Value Callback Function}
\declareapi{pmix_value_cbfunc_t}

%%%%
\summary

The \refapi{pmix_value_cbfunc_t} is used by \refapi{PMIx_Get_nb} to return data.

\versionMarker{1.0}
\cspecificstart
\begin{codepar}
typedef void (*pmix_value_cbfunc_t)
    (pmix_status_t status,
     pmix_value_t *kv, void *cbdata);
\end{codepar}
\cspecificend

\begin{arglist}
\argin{status}{Status associated with the operation (handle)}
\argin{kv}{Key/value pair representing the data (\refstruct{pmix_value_t})}
\argin{cbdata}{Callback data passed to original API call (memory reference)}
\end{arglist}


%%%%
\descr

A callback function for calls to \refapi{PMIx_Get_nb}.
The \refarg{status} indicates if the requested data was found or not.
A pointer to the \refstruct{pmix_value_t} structure containing the found data is returned.
The pointer will be \code{NULL} if the requested data was not found.


%%%%%%%%%%%
\subsection{Info Callback Function}
\declareapi{pmix_info_cbfunc_t}

%%%%
\summary

The \refapi{pmix_info_cbfunc_t} is a general information callback used by various APIs.

\versionMarker{2.0}
\cspecificstart
\begin{codepar}
typedef void (*pmix_info_cbfunc_t)
    (pmix_status_t status,
     pmix_info_t info[], size_t ninfo,
     void *cbdata,
     pmix_release_cbfunc_t release_fn,
     void *release_cbdata);
\end{codepar}
\cspecificend

\begin{arglist}
\argin{status}{Status associated with the operation (\refstruct{pmix_status_t})}
\argin{info}{Array of \refstruct{pmix_info_t} returned by the operation (pointer)}
\argin{ninfo}{Number of elements in the \argref{info} array (\code{size_t})}
\argin{cbdata}{Callback data passed to original API call (memory reference)}
\argin{release_fn}{Function to be called when done with the \argref{info} data (function pointer)}
\argin{release_cbdata}{Callback data to be passed to \argref{release_fn} (memory reference)}
\end{arglist}


%%%%
\descr

The \refarg{status} indicates if requested data was found or not.
An array of \refstruct{pmix_info_t} will contain the key/value pairs.


%%%%%%%%%%%
\subsection{Event Handler Registration Callback Function}
\declareapi{pmix_evhdlr_reg_cbfunc_t}

The \refapi{pmix_evhdlr_reg_cbfunc_t} callback function.

\adviceuserstart
The \ac{PMIx} \textit{ad hoc} v1.0 Standard defined an error handler registration callback function with a compatible signature, but with a different type definition function name (pmix_errhandler_reg_cbfunc_t). It was removed from the v2.0 Standard and is not included in this document to avoid confusion.
\adviceuserend

\versionMarker{2.0}
\cspecificstart
\begin{codepar}
typedef void (*pmix_evhdlr_reg_cbfunc_t)
    (pmix_status_t status,
     size_t evhdlr_ref,
     void *cbdata)
\end{codepar}
\cspecificend

\begin{arglist}
\argin{status}{Status indicates if the request was successful or not (\refstruct{pmix_status_t})}
\argin{evhdlr_ref}{Reference assigned to the event handler by \ac{PMIx} --- this reference
 * must be used to deregister the err handler (\code{size_t})}
\argin{cbdata}{Callback data passed to original API call (memory reference)}
\end{arglist}


%%%%
\descr

Define a callback function for calls to \refapi{PMIx_Register_event_handler}


%%%%%%%%%%%
\subsection{Notification Handler Completion Callback Function}
\declareapi{pmix_event_notification_cbfunc_fn_t}

%%%%
\summary

The \refapi{pmix_event_notification_cbfunc_fn_t} is called by event handlers to indicate completion of their operations.

\versionMarker{2.0}
\cspecificstart
\begin{codepar}
typedef void (*pmix_event_notification_cbfunc_fn_t)
    (pmix_status_t status,
     pmix_info_t *results, size_t nresults,
     pmix_op_cbfunc_t cbfunc, void *thiscbdata,
     void *notification_cbdata);
\end{codepar}
\cspecificend

\begin{arglist}
\argin{status}{Status returned by the event handler's operation (\refstruct{pmix_status_t})}
\argin{results}{Results from this event handler's operation on the event (\refstruct{pmix_info_t})}
\argin{nresults}{Number of elements in the results array (\code{size_t})}
\argin{cbfunc}{\refapi{pmix_op_cbfunc_t} function to be executed when \ac{PMIx} completes processing the callback (function reference)}
\argin{thiscbdata}{Callback data that was passed in to the handler (memory reference)}
\argin{cbdata}{Callback data to be returned when \ac{PMIx} executes cbfunc (memory reference)}
\end{arglist}

%%%%
\descr

Define a callback by which an event handler can notify the \ac{PMIx} library that it has completed its response to the notification. The handler is \textit{required} to execute this callback so the library can determine if additional handlers need to be called. The handler shall return \refconst{PMIX_EVENT_ACTION_COMPLETE} if no further action is required. The return status of each event handler and any returned \refstruct{pmix_info_t} structures will be added to the \textit{results} array of \refstruct{pmix_info_t} passed to any subsequent event handlers to help guide their operation.

If non-NULL, the provided callback function will be called to allow the event handler to release the provided info array and execute any other required cleanup operations.


%%%%%%%%%%%
\subsection{Notification Function}
\declareapi{pmix_notification_fn_t}

%%%%
\summary

The \refapi{pmix_notification_fn_t} is called by \ac{PMIx} to deliver notification of an event.

\adviceuserstart
The \ac{PMIx} \textit{ad hoc} v1.0 Standard defined an error notification function with an identical name, but different signature than the v2.0 Standard described below. The \textit{ad hoc} v1.0 version was removed from the v2.0 Standard is not included in this document to avoid confusion.
\adviceuserend


\versionMarker{2.0}
\cspecificstart
\begin{codepar}
typedef void (*pmix_notification_fn_t)
    (size_t evhdlr_registration_id,
     pmix_status_t status,
     const pmix_proc_t *source,
     pmix_info_t info[], size_t ninfo,
     pmix_info_t results[], size_t nresults,
     pmix_event_notification_cbfunc_fn_t cbfunc,
     void *cbdata);
\end{codepar}
\cspecificend

\begin{arglist}
\argin{evhdlr_registration_id}{Registration number of the handler being called (\code{size_t})}
\argin{status}{Status associated with the operation (\refstruct{pmix_status_t})}
\argin{source}{Identifier of the process that generated the event (\refstruct{pmix_proc_t})}. If the source is the \ac{SMS}, then the nspace will be empty and the rank will be PMIX_RANK_UNDEF
\argin{info}{Information describing the event (\refstruct{pmix_info_t})}. This argument will be NULL if no additional information was provided by the event generator.
\argin{ninfo}{Number of elements in the info array (\code{size_t})}
\argin{results}{Aggregated results from prior event handlers servicing this event (\refstruct{pmix_info_t})}. This argument will be \code{NULL} if this is the first handler servicing the event, or if no prior handlers provided results.
\argin{nresults}{Number of elements in the results array (\code{size_t})}
\argin{cbfunc}{\refapi{pmix_event_notification_cbfunc_fn_t} callback function to be executed upon completion of the handler's operation and prior to handler return (function reference)}.
\argin{cbdata}{Callback data to be passed to cbfunc (memory reference)}
\end{arglist}

%%%%
\descr

Note that different \acp{RM} may provide differing levels of support for event notification to application processes. Thus, the \refarg{info} array may be \code{NULL} or may contain detailed information of the event. It is the responsibility of the application to parse any provided info array for defined key-values if it so desires.

\adviceuserstart
Possible uses of the \refarg{info} array include:

\begin{itemize}
\item for the host \ac{RM} to alert the process as to planned actions, such as aborting the session, in response to the reported event

\item provide a timeout for alternative action to occur, such as for the application to request an alternate response to the event
\end{itemize}

For example, the \ac{RM} might alert the application to the failure of a node that resulted in termination of several processes, and indicate that the overall session will be aborted unless the application requests an alternative behavior in the next 5 seconds. The application then has time to respond with a checkpoint request, or a request to recover from the failure by obtaining replacement nodes and restarting from some earlier checkpoint.

Support for these options is left to the discretion of the host \ac{RM}. Info keys are included in the common definitions above but may be augmented by environment vendors.
\adviceuserend

\advicermstart
On the server side, the notification function is used to inform the \ac{PMIx} server library's host of a detected event in the \ac{PMIx} server library. Events generated by \ac{PMIx} clients are communicated to the \ac{PMIx} server library, but will be relayed to the host via the \refapi{pmix_server_notify_event_fn_t} function pointer, if provided.
\advicermend

%%%%%%%%%%%
\subsection{Server Setup Application Callback Function}
\declareapi{pmix_setup_application_cbfunc_t}

The \refapi{PMIx_server_setup_application} callback function.

%%%%
\summary

Provide a function by which the resource manager can receive application-specific environmental variables and other setup data prior to launch of an application.

%%%%
\format

\versionMarker{2.0}
\cspecificstart
\begin{codepar}
typedef void (*pmix_setup_application_cbfunc_t)(
                        pmix_status_t status,
                        pmix_info_t info[], size_t ninfo,
                        void *provided_cbdata,
                        pmix_op_cbfunc_t cbfunc, void *cbdata)
\end{codepar}
\cspecificend

\begin{arglist}
\argin{status}{returned status of the request (\refstruct{pmix_status_t})}
\argin{info}{Array of info structures (array of handles)}
\argin{ninfo}{Number of elements in the \refarg{info} array (integer)}
\argin{provided_cbdata}{Data originally passed to call to \refapi{PMIx_server_setup_application} (memory reference)}
\argin{cbfunc}{\refapi{pmix_op_cbfunc_t} function to be called when processing completed (function reference)}
\argin{cbdata}{Data to be passed to the \refarg{cbfunc} callback function (memory reference)}
\end{arglist}

\descr

Define a function to be called by the \ac{PMIx} server library for return of application-specific setup data in response to a request from the host \ac{RM}. The returned \refarg{info} array is owned by the \ac{PMIx} server library and will be free'd when the provided \refarg{cbfunc} is called.


%%%%%%%%%%%
\subsection{Server Direct Modex Response Callback Function}
\declareapi{pmix_dmodex_response_fn_t}

The \refapi{PMIx_server_dmodex_request} callback function.

%%%%
\summary

Provide a function by which the local \ac{PMIx} server library can return connection and other data posted by local application processes to the host resource manager.

%%%%
\format

\versionMarker{1.0}
\cspecificstart
\begin{codepar}
typedef void (*pmix_dmodex_response_fn_t)(pmix_status_t status,
                        char *data, size_t sz,
                        void *cbdata);
\end{codepar}
\cspecificend


\begin{arglist}
\argin{status}{Returned status of the request (\refstruct{pmix_status_t})}
\argin{data}{Pointer to a data "blob" containing the requested information (handle)}
\argin{sz}{Number of bytes in the \refarg{data} blob (integer)}
\argin{cbdata}{Data passed into the initial call to \refapi{PMIx_server_dmodex_request} (memory reference)}
\end{arglist}


\descr
Define a function to be called by the PMIx server library for return of information posted by a local application process (via \refapi{PMIx_Put} with subsequent \refapi{PMIx_Commit}) in response to a request from the host RM. The returned \refarg{data} blob is owned by the PMIx server library and will be free’d upon return from the function.

%%%%%%%%%%%
\subsection{PMIx Client Connection Callback Function}
\declareapi{pmix_connection_cbfunc_t}

%%%%
\summary

Callback function for incoming connection request from a local client

%%%%
\format

\versionMarker{1.0}
\cspecificstart
\begin{codepar}
typedef void (*pmix_connection_cbfunc_t)(
                             int incoming_sd, void *cbdata)
\end{codepar}
\cspecificend

\begin{arglist}
\argin{incoming_sd}{(integer)}
\argin{cbdata}{ (memory reference)}
\end{arglist}

%%%%
\descr

Callback function for incoming connection requests from local clients - only used by host environments that wish to directly handle socket connection requests.


%%%%%%%%%%%
\subsection{PMIx Tool Connection Callback Function}
\declareapi{pmix_tool_connection_cbfunc_t}

%%%%
\summary

Callback function for incoming tool connections.

%%%%
\format

\versionMarker{2.0}
\cspecificstart
\begin{codepar}
typedef void (*pmix_tool_connection_cbfunc_t)(
                             pmix_status_t status,
                             pmix_proc_t *proc, void *cbdata)
\end{codepar}
\cspecificend

\begin{arglist}
\argin{status}{\refstruct{pmix_status_t} value (handle)}
\argin{proc}{\refstruct{pmix_proc_t} structure containing the identifier assigned to the tool (handle)}
\argin{cbdata}{Data to be passed (memory reference)}
\end{arglist}

%%%%
\descr

Callback function for incoming tool connections.
The host environment shall provide a namespace/rank identifier for the connecting tool.

\advicermstart
It is assumed that \code{rank=0} will be the normal assignment, but allow for the future possibility of a parallel set of tools connecting, and thus each process requiring a unique rank.
\advicermend

%%%%%%%%%%%
\subsection{Credential callback function}
\declareapi{pmix_credential_cbfunc_t}

%%%%
\summary

Callback function to return a requested security credential

%%%%
\format

\versionMarker{3.0}
\cspecificstart
\begin{codepar}
typedef void (*pmix_credential_cbfunc_t)(
                             pmix_status_t status,
                             pmix_byte_object_t *credential,
                             pmix_info_t info[], size_t ninfo,
                             void *cbdata)
\end{codepar}
\cspecificend

\begin{arglist}
\argin{status}{\refstruct{pmix_status_t} value (handle)}
\argin{credential}{\refstruct{pmix_byte_object_t} structure containing the security credential (handle)}
\argin{info}{Array of provided by the system to pass any additional information about the credential - e.g., the identity of the issuing agent. (handle)}
\argin{ninfo}{Number of elements in \refarg{info} (\code{size_t})}
\argin{cbdata}{Object passed in original request (memory reference)}
\end{arglist}

%%%%
\descr

Define a callback function to return a requested security credential. Information provided by the issuing agent can subsequently be used
by the application for a variety of purposes. Examples include:

\begin{itemize}
    \item checking identified authorizations to determine what requests/operations are feasible as a means to steering workflows
    \item compare the credential type to that of the local SMS for compatibility
\end{itemize}

\adviceuserstart
The credential is opaque and therefore understandable only by a service compatible with the issuer. The \refarg{info} array is owned by the \ac{PMIx} library and is not to be released or altered by the receiving party.
\adviceuserend

%%%%%%%%%%%
\subsection{Credential validation callback function}
\declareapi{pmix_validation_cbfunc_t}

%%%%
\summary

Callback function for security credential validation

%%%%
\format

\versionMarker{3.0}
\cspecificstart
\begin{codepar}
typedef void (*pmix_validation_cbfunc_t)(
                             pmix_status_t status,
                             pmix_info_t info[], size_t ninfo,
                             void *cbdata);
\end{codepar}
\cspecificend

\begin{arglist}
\argin{status}{\refstruct{pmix_status_t} value (handle)}
\argin{info}{Array of \refstruct{pmix_info_t} provided by the system to pass any additional information about the authentication - e.g., the effective userid and group id of the certificate holder, and any related authorizations (handle)}
\argin{ninfo}{Number of elements in \refarg{info} (\code{size_t})}
\argin{cbdata}{Object passed in original request (memory reference)}
\end{arglist}

%%%%
\descr

Define a validation callback function to indicate if a provided credential is valid, and any corresponding information regarding authorizations and other security matters.

\adviceuserstart
The precise contents of the array will depend on the host environment and its associated security system. At the minimum, it is expected (but not required) that the array will contain entries for the \refattr{PMIX_USERID} and \refattr{PMIX_GRPID} of the client described in the credential. The \refarg{info} array is owned by the \ac{PMIx} library and is not to be released or altered by the receiving party.
\adviceuserend

%%%%%%%%%%%
\subsection{IOF delivery function}
\declareapi{pmix_iof_cbfunc_t}

%%%%
\summary

Callback function for delivering forwarded \ac{IO} to a process

%%%%
\format

\versionMarker{3.0}
\cspecificstart
\begin{codepar}
typedef void (*pmix_iof_cbfunc_t)(
                             size_t iofhdlr, pmix_iof_channel_t channel,
                             pmix_proc_t *source, char *payload,
                             pmix_info_t info[], size_t ninfo);
\end{codepar}
\cspecificend

\begin{arglist}
\argin{iofhdlr}{Registration number of the handler being invoked (\code{size_t})}
\argin{channel}{bitmask identifying the channel the data arrived on (\refstruct{pmix_iof_channel_t})}
\argin{source}{Pointer to a \refstruct{pmix_proc_t} identifying the namespace/rank of the process that generated the data (\code{char*})}
\argin{payload}{Pointer to character array containing the data.}
\argin{info}{Array of \refstruct{pmix_info_t} provided by the source containing metadata about the payload. This could include \refattr{PMIX_IOF_COMPLETE} (handle)}
\argin{ninfo}{Number of elements in \refarg{info} (\code{size_t})}
\end{arglist}

%%%%
\descr

Define a callback function for delivering forwarded \ac{IO} to a process. This function will be called whenever data becomes available, or a
specified buffering size and/or time has been met.

\adviceuserstart
Multiple strings may be included in a given \refarg{payload}, and the \refarg{payload} may \textit{not} be \code{NULL} terminated. The user is responsible for releasing the \refarg{payload} memory. The \refarg{info} array is owned by the \ac{PMIx} library and is not to be released or altered by the receiving party.
\adviceuserend


%%%%%%%%%%%
\subsection{IOF and Event registration function}
\declareapi{pmix_hdlr_reg_cbfunc_t}

%%%%
\summary

Callback function for calls to register handlers, e.g., event notification and IOF requests.

%%%%
\format

\versionMarker{3.0}
\cspecificstart
\begin{codepar}
typedef void (*pmix_hdlr_reg_cbfunc_t)(pmix_status_t status,
                                       size_t refid,
                                       void *cbdata);
\end{codepar}
\cspecificend

\begin{arglist}
\argin{status}{\refconst{PMIX_SUCCESS} or an appropriate error constant (\refstruct{pmix_status_t})}
\argin{refid}{reference identifier assigned to the handler by PMIx, used to deregister the handler (\code{size_t})}
\argin{cbdata}{object provided to the registration call (pointer)}
\end{arglist}

%%%%
\descr

Callback function for calls to register handlers, e.g., event notification and IOF requests.


%%%%%%%%%%%
\section{Constant String Functions}

Provide a string representation for several types of values.
Note that the provided string is statically defined and must NOT be \code{free}'d.

%%%%
\summary
\declareapi{PMIx_Error_string}

String representation of a \refstruct{pmix_status_t}.

\versionMarker{1.0}
\cspecificstart
\begin{codepar}
const char*
PMIx_Error_string(pmix_status_t status);
\end{codepar}
\cspecificend

%%%%
\summary
\declareapi{PMIx_Proc_state_string}

String representation of a \refstruct{pmix_proc_state_t}.

\versionMarker{2.0}
\cspecificstart
\begin{codepar}
const char*
PMIx_Proc_state_string(pmix_proc_state_t state);
\end{codepar}
\cspecificend

%%%%
\summary
\declareapi{PMIx_Scope_string}

String representation of a \refstruct{pmix_scope_t}.

\versionMarker{2.0}
\cspecificstart
\begin{codepar}
const char*
PMIx_Scope_string(pmix_scope_t scope);
\end{codepar}
\cspecificend

%%%%
\summary
\declareapi{PMIx_Persistence_string}

String representation of a \refstruct{pmix_persistence_t}.

\versionMarker{2.0}
\cspecificstart
\begin{codepar}
const char*
PMIx_Persistence_string(pmix_persistence_t persist);
\end{codepar}
\cspecificend

%%%%
\summary
\declareapi{PMIx_Data_range_string}

String representation of a \refstruct{pmix_data_range_t}.

\versionMarker{2.0}
\cspecificstart
\begin{codepar}
const char*
PMIx_Data_range_string(pmix_data_range_t range);
\end{codepar}
\cspecificend

%%%%
\summary
\declareapi{PMIx_Info_directives_string}

String representation of a \refstruct{pmix_info_directives_t}.

\versionMarker{2.0}
\cspecificstart
\begin{codepar}
const char*
PMIx_Info_directives_string(pmix_info_directives_t directives);
\end{codepar}
\cspecificend

%%%%
\summary
\declareapi{PMIx_Data_type_string}

String representation of a \refstruct{pmix_data_type_t}.

\versionMarker{2.0}
\cspecificstart
\begin{codepar}
const char*
PMIx_Data_type_string(pmix_data_type_t type);
\end{codepar}
\cspecificend

%%%%
\summary
\declareapi{PMIx_Alloc_directive_string}

String representation of a \refstruct{pmix_alloc_directive_t}.

\versionMarker{2.0}
\cspecificstart
\begin{codepar}
const char*
PMIx_Alloc_directive_string(pmix_alloc_directive_t directive);
\end{codepar}
\cspecificend

%%%%
\summary
\declareapi{PMIx_IOF_channel_string}

String representation of a \refstruct{pmix_iof_channel_t}.

\versionMarker{3.0}
\cspecificstart
\begin{codepar}
const char*
PMIx_IOF_channel_string(pmix_iof_channel_t channel);
\end{codepar}
\cspecificend



%%%%%%%%%%%%%%%%%%%%%%%%%%%%%%%%%%%%%%%%%%%%%%%%%


    % Initialization & Finalization
    %  - Client, Server, Tool interfaces
    %%%%%%%%%%%%%%%%%%%%%%%%%%%%%%%%%%%%%%%%%%%%%%%%%
% Chapter: Initialization & Finalization
%%%%%%%%%%%%%%%%%%%%%%%%%%%%%%%%%%%%%%%%%%%%%%%%%
\chapter{Client Initialization and Finalization}
\label{chap:api_init}

The \ac{PMIx} library is required to be initialized and finalized around the usage of most \ac{PMIx} functions or macros.
The \acp{API} that may be used outside of the initialized and finalized region are noted.
All other \acp{API} must be used inside this region.

There are three sets of initialization and finalization functions depending upon the role of the process in the \ac{PMIx} Standard - those associated with the \ac{PMIx} \refterm{client} are defined in this chapter. Similar functions corresponding to the roles of \emph{server} and \emph{tool} are defined in Chapters \ref{chap:api_server} and \ref{chap:api_tools}, respectively.

Note that a process can only call \textit{one} of the initialization/finalization functional pairs from the set of three - e.g., a process that calls the client initialization function cannot also call the tool or server initialization functions, and must call the corresponding client finalization function. Regardless of the role assumed by the process, all processes have access to the client \acp{API}. Thus, the \emph{server} and \emph{tool} roles can be considered supersets of the \ac{PMIx} client.

%%%%%%%%%%%%%%%%%%%%%%%%%%%%%%%%%%%%%%%%%%%%%%%%%
%%%%%%%%%%%%%%%%%%%%%%%%%%%%%%%%%%%%%%%%%%%%%%%%%
\section{\code{PMIx_Initialized}}
\declareapi{PMIx_Initialized}

%%%%
\summary

Determine if the \ac{PMIx} library has been initialized. This function may be used outside of the initialized and finalized region, and is usable by servers and tools in addition to clients.

%%%%
\format

\versionMarker{1.0}
\cspecificstart
\begin{codepar}
int PMIx_Initialized(void)
\end{codepar}
\cspecificend

A value of \code{1} (true) will be returned if the \ac{PMIx} library has been initialized, and \code{0} (false) otherwise.

\rationalestart
The return value is an integer for historical reasons as that was the signature of prior PMI libraries.
\rationaleend

%%%%
\descr

Check to see if the \ac{PMIx} library has been initialized using any of the init functions:
\refapi{PMIx_Init}, \refapi{PMIx_server_init}, or \refapi{PMIx_tool_init}.


%%%%%%%%%%%%%%%%%%%%%%%%%%%%%%%%%%%%%%%%%%%%%%%%%
%%%%%%%%%%%%%%%%%%%%%%%%%%%%%%%%%%%%%%%%%%%%%%%%%
\section{\code{PMIx_Get_version}}
\declareapi{PMIx_Get_version}

%%%%
\summary

Get the \ac{PMIx} version information. This function may be used outside of the initialized and finalized region, and is usable by servers and tools in addition to clients.

%%%%
\format

\versionMarker{1.0}
\cspecificstart
\begin{codepar}
const char* PMIx_Get_version(void)
\end{codepar}
\cspecificend

%%%%
\descr

Get the \ac{PMIx} version string.
Note that the provided string is statically defined and must \textit{not} be free'd.


%%%%%%%%%%%%%%%%%%%%%%%%%%%%%%%%%%%%%%%%%%%%%%%%%
%%%%%%%%%%%%%%%%%%%%%%%%%%%%%%%%%%%%%%%%%%%%%%%%%
\section{\code{PMIx_Init}}
\declareapi{PMIx_Init}

%%%%
\summary

Initialize the \ac{PMIx} client library

%%%%
\format

\versionMarker{1.2}
\cspecificstart
\begin{codepar}
pmix_status_t
PMIx_Init(pmix_proc_t *proc,
          pmix_info_t info[], size_t ninfo)
\end{codepar}
\cspecificend

\begin{arglist}
\arginout{proc}{proc structure (handle)}
\argin{info}{Array of \refstruct{pmix_info_t} structures (array of handles)}
\argin{ninfo}{Number of elements in the \refarg{info} array (\code{size_t})}
\end{arglist}

Returns \refconst{PMIX_SUCCESS} or a negative value corresponding to a \ac{PMIx} error constant.

\optattrstart
The following attributes are optional for implementers of \ac{PMIx} libraries:

\pasteAttributeItemBegin{PMIX_USOCK_DISABLE} If the library supports Unix socket connections, this attribute may be supported for disabling it.
\pasteAttributeItemEnd{}
\pasteAttributeItemBegin{PMIX_SOCKET_MODE} If the library supports socket connections, this attribute may be supported for setting the socket mode.
\pasteAttributeItemEnd{}
\pasteAttributeItemBegin{PMIX_SINGLE_LISTENER} If the library supports multiple methods for clients to connect to servers, this attribute may be supported for disabling all but one of them.
\pasteAttributeItemEnd{}
\pasteAttributeItemBegin{PMIX_TCP_REPORT_URI} If the library supports TCP socket connections, this attribute may be supported for reporting the URI.
\pasteAttributeItemEnd{}
\pasteAttributeItemBegin{PMIX_TCP_IF_INCLUDE} If the library supports TCP socket connections, this attribute may be supported for specifying the interfaces to be used.
\pasteAttributeItemEnd{}
\pasteAttributeItemBegin{PMIX_TCP_IF_EXCLUDE} If the library supports TCP socket connections, this attribute may be supported for specifying the interfaces that are \textit{not} to be used.
\pasteAttributeItemEnd{}
\pasteAttributeItemBegin{PMIX_TCP_IPV4_PORT} If the library supports IPV4 connections, this attribute may be supported for specifying the port to be used.
\pasteAttributeItemEnd{}
\pasteAttributeItemBegin{PMIX_TCP_IPV6_PORT} If the library supports IPV6 connections, this attribute may be supported for specifying the port to be used.
\pasteAttributeItemEnd{}
\pasteAttributeItemBegin{PMIX_TCP_DISABLE_IPV4} If the library supports IPV4 connections, this attribute may be supported for disabling it.
\pasteAttributeItemEnd{}
\pasteAttributeItemBegin{PMIX_TCP_DISABLE_IPV6} If the library supports IPV6 connections, this attribute may be supported for disabling it.
\pasteAttributeItemEnd{}
\pasteAttributeItem{PMIX_EXTERNAL_PROGRESS}
%
\declareAttribute{PMIX_EVENT_BASE}{"pmix.evbase"}{void*}{
Pointer to an \code{event_base} to use in place of the internal progress thread. All \ac{PMIx} library events are to be assigned to the provided event base. The event base \emph{must} be compatible with the event library used by the \ac{PMIx} implementation - e.g., either both the host and \ac{PMIx} library must use libevent, or both must use libev. Cross-matches are unlikely to work and should be avoided - it is the responsibility of the host to ensure that the \ac{PMIx} implementation supports (and was built with) the appropriate event library.
}

\vspace{\baselineskip}
If provided, the following attributes are used by the event notification system for inter-library coordination:

\pasteAttributeItem{PMIX_PROGRAMMING_MODEL}
\pasteAttributeItem{PMIX_MODEL_LIBRARY_NAME}
\pasteAttributeItem{PMIX_MODEL_LIBRARY_VERSION}
\pasteAttributeItem{PMIX_THREADING_MODEL}
\pasteAttributeItem{PMIX_MODEL_NUM_THREADS}
\pasteAttributeItem{PMIX_MODEL_NUM_CPUS}
\pasteAttributeItem{PMIX_MODEL_CPU_TYPE}
\pasteAttributeItem{PMIX_MODEL_AFFINITY_POLICY}

\optattrend

%%%%
\descr

Initialize the \ac{PMIx} client, returning the process identifier assigned to this client's application in the provided \refstruct{pmix_proc_t} struct.
Passing a value of \code{NULL} for this parameter is allowed if the user wishes solely to initialize the \ac{PMIx} system and does not require return of the identifier at that time.

When called, the \ac{PMIx} client shall check for the required connection information of the local \ac{PMIx} server and establish the connection.
If the information is not found, or the server connection fails, then an appropriate error constant shall be returned.

If successful, the function shall return \refconst{PMIX_SUCCESS} and fill the \refarg{proc} structure (if provided) with the server-assigned namespace and rank of the process within the application.
In addition, all startup information provided by the resource manager shall be made available to the client process via subsequent calls to \refapi{PMIx_Get}.

The \ac{PMIx} client library shall be reference counted, and so multiple calls to \refapi{PMIx_Init} are allowed by the standard.
Thus, one way for an application process to obtain its namespace and rank is to simply call \refapi{PMIx_Init} with a non-NULL \refarg{proc} parameter.
Note that each call to \refapi{PMIx_Init} must be balanced with a call to \refapi{PMIx_Finalize} to maintain the reference count.

Each call to \refapi{PMIx_Init} may contain an array of \refstruct{pmix_info_t} structures passing directives to the \ac{PMIx} client library as per the above attributes.

Multiple calls to \refapi{PMIx_Init} shall not include conflicting directives.
The \refapi{PMIx_Init} function will return an error when directives that conflict with prior directives are encountered.

%%%%%%%%%%%%%%%%%%%%%%%%%%%%%%%%%%%%%%%%%%%%%%%%%
\subsection{Initialization events}

The following events are typically associated with calls to \refapi{PMIx_Init}:

\begin{constantdesc}
%
\declareconstitem{PMIX_MODEL_DECLARED}
Model declared.
%
\declareconstitem{PMIX_MODEL_RESOURCES}
Resource usage by a programming model has changed.
%
\declareconstitem{PMIX_OPENMP_PARALLEL_ENTERED}
An OpenMP parallel code region has been entered.
%
\declareconstitem{PMIX_OPENMP_PARALLEL_EXITED}
An OpenMP parallel code region has completed.
%
\end{constantdesc}

%%%%%%%%%%%%%%%%%%%%%%%%%%%%%%%%%%%%%%%%%%%%%%%%%
\subsection{Initialization attributes}

The following attributes influence the behavior of \refapi{PMIx_Init}.

%%%%%%%%%%%%%%%%%%%%%%%%%%%%%%%%%%%%%%%%%%%%%%%%%
\subsubsection{Connection attributes}

These attributes are used to describe a TCP socket for rendezvous with the local \ac{RM} by passing them into the relevant initialization \ac{API} - thus, they are not typically accessed via the \refapi{PMIx_Get} \ac{API}.

%
\declareAttribute{PMIX_TCP_REPORT_URI}{"pmix.tcp.repuri"}{char*}{
If provided, directs that the TCP \ac{URI} be reported and indicates the desired method of reporting: \code{'-'} for stdout, \code{'+'} for stderr, or filename.
}
%
\declareAttribute{PMIX_TCP_URI}{"pmix.tcp.uri"}{char*}{
The \ac{URI} of the PMIx server to connect to, or a file name containing it in the form of \code{file:<name of file containing it>}.
}
%
\declareAttribute{PMIX_TCP_IF_INCLUDE}{"pmix.tcp.ifinclude"}{char*}{
Comma-delimited list of devices and/or \ac{CIDR} notation to include when establishing the TCP connection.
}
%
\declareAttribute{PMIX_TCP_IF_EXCLUDE}{"pmix.tcp.ifexclude"}{char*}{
Comma-delimited list of devices and/or \ac{CIDR} notation to exclude when establishing the TCP connection.
}
%
\declareAttribute{PMIX_TCP_IPV4_PORT}{"pmix.tcp.ipv4"}{int}{
The IPv4 port to be used..
}
%
\declareAttribute{PMIX_TCP_IPV6_PORT}{"pmix.tcp.ipv6"}{int}{
The IPv6 port to be used.
}
%
\declareAttribute{PMIX_TCP_DISABLE_IPV4}{"pmix.tcp.disipv4"}{bool}{
Set to \code{true} to disable IPv4 family of addresses.
}
%
\declareAttribute{PMIX_TCP_DISABLE_IPV6}{"pmix.tcp.disipv6"}{bool}{
Set to \code{true} to disable IPv6 family of addresses.
}

%%%%%%%%%%%%%%%%%%%%%%%%%%%%%%%%%%%%%%%%%%%%%%%%%
\subsubsection{Programming model attributes}
\label{api:struct:attributes:model}

These attributes are associated with programming models.

%
\declareAttribute{PMIX_PROGRAMMING_MODEL}{"pmix.pgm.model"}{char*}{
Programming model being initialized (e.g., ``MPI'' or ``OpenMP'').
}
%
\declareAttribute{PMIX_MODEL_LIBRARY_NAME}{"pmix.mdl.name"}{char*}{
Programming model implementation ID (e.g., ``OpenMPI'' or ``MPICH'').
}
%
\declareAttribute{PMIX_MODEL_LIBRARY_VERSION}{"pmix.mld.vrs"}{char*}{
Programming model version string (e.g., ``2.1.1'').
}
%
\declareAttribute{PMIX_THREADING_MODEL}{"pmix.threads"}{char*}{
Threading model used (e.g., ``pthreads'').
}
%
\declareAttribute{PMIX_MODEL_NUM_THREADS}{"pmix.mdl.nthrds"}{uint64_t}{
Number of active threads being used by the model.
}
%
\declareAttribute{PMIX_MODEL_NUM_CPUS}{"pmix.mdl.ncpu"}{uint64_t}{
Number of cpus being used by the model.
}
%
\declareAttribute{PMIX_MODEL_CPU_TYPE}{"pmix.mdl.cputype"}{char*}{
Granularity - ``hwthread'', ``core'', etc.
}
%
\declareAttribute{PMIX_MODEL_PHASE_NAME}{"pmix.mdl.phase"}{char*}{
User-assigned name for a phase in the application execution (e.g., ``cfd reduction'').
}
%
\declareAttribute{PMIX_MODEL_PHASE_TYPE}{"pmix.mdl.ptype"}{char*}{
Type of phase being executed (e.g., ``matrix multiply'').
}
%
\declareAttribute{PMIX_MODEL_AFFINITY_POLICY}{"pmix.mdl.tap"}{char*}{
Thread affinity policy - e.g.:
         "master" (thread co-located with master thread),
         "close" (thread located on cpu close to master thread),
         "spread" (threads load-balanced across available cpus).
}


%%%%%%%%%%%%%%%%%%%%%%%%%%%%%%%%%%%%%%%%%%%%%%%%%
%%%%%%%%%%%%%%%%%%%%%%%%%%%%%%%%%%%%%%%%%%%%%%%%%
\section{\code{PMIx_Finalize}}
\declareapi{PMIx_Finalize}

%%%%
\summary

Finalize the PMIx client library.

%%%%
\format

\versionMarker{1.0}
\cspecificstart
\begin{codepar}
pmix_status_t
PMIx_Finalize(const pmix_info_t info[], size_t ninfo)
\end{codepar}
\cspecificend

\begin{arglist}
\argin{info}{Array of \refstruct{pmix_info_t} structures (array of handles)}
\argin{ninfo}{Number of elements in the \refarg{info} array (\code{size_t})}
\end{arglist}

Returns \refconst{PMIX_SUCCESS} or a negative value corresponding to a PMIx error constant.

\optattrstart
The following attributes are optional for implementers of \ac{PMIx} libraries:

\pasteAttributeItem{PMIX_EMBED_BARRIER}
\optattrend

%%%%
\descr

Decrement the \ac{PMIx} client library reference count.
When the reference count reaches zero, the library will finalize the \ac{PMIx} client, closing the connection with the local \ac{PMIx} server and releasing all internally allocated memory.

\subsection{Finalize attributes}

The following attribute influences the behavior of \refapi{PMIx_Finalize}.

%
\declareAttribute{PMIX_EMBED_BARRIER}{"pmix.embed.barrier"}{bool}{
Execute a blocking fence operation before executing the specified operation.
\refapi{PMIx_Finalize} does not include an internal barrier operation by default.
This attribute directs \refapi{PMIx_Finalize} to execute a barrier as part of the finalize operation.
}


%%%%%%%%%%%%%%%%%%%%%%%%%%%%%%%%%%%%%%%%%%%%%%%%%
%%%%%%%%%%%%%%%%%%%%%%%%%%%%%%%%%%%%%%%%%%%%%%%%%
\section{\code{PMIx_Progress}}
\declareapi{PMIx_Progress}

%%%%
\summary

Progress the \ac{PMIx} library.

%%%%
\format

\versionMarker{4.0}
\cspecificstart
\begin{codepar}
void
PMIx_Progress(void)
\end{codepar}
\cspecificend


%%%%
\descr

Progress the \ac{PMIx} library. Note that special care must be taken to avoid deadlocking in \ac{PMIx} callback functions and acp{API}.

%%%%%%%%%%%%%%%%%%%%%%%%%%%%%%%%%%%%%%%%%%%%%%%%%


    % Key/Value Management
    %  - put, get, commit, fence, (un)publish, lookup
    %%%%%%%%%%%%%%%%%%%%%%%%%%%%%%%%%%%%%%%%%%%%%%%%%
% Chapter: Key/Value Management
%%%%%%%%%%%%%%%%%%%%%%%%%%%%%%%%%%%%%%%%%%%%%%%%%
\chapter{Key/Value Management}
\label{chap:api_kv_mgmt}

Management of key-value pairs in \ac{PMIx} is a distributed responsibility. While the stated objective of the \ac{PMIx} community is to eliminate collective operations, it is recognized that the traditional method of publishing/exchanging data must be supported until that objective can be met. This method relies on processes to discover and publish their local information which is collected by the local PMIx server library.
Global exchange of the published information is then executed via a collective operation performed by the host \ac{SMS} servers.

Keys are required to be unique within a specific level of information as defined in \ref{api:struct:attributes:retrieval}. For example, a value for \refattr{PMIX_NUM_NODES} can be specified for each of the \refterm{session}, \refterm{job}, and \refterm{application} levels. However, subsequently specifying another value for that attribute in the \refterm{session} level will overwrite the prior value.

%%%%%%%%%%%%%%%%%%%%%%%%%%%%%%%%%%%%%%%%%%%%%%
%%%%%%%%%%%%%%%%%%%%%%%%%%%%%%%%%%%%%%%%%%%%%%
\section{Setting and Accessing Key/Value Pairs}
\label{chap:api_kv_mgmt:access}


%%%%%%%%%%%
\subsection{\code{PMIx_Put}}
\declareapi{PMIx_Put}

%%%%
\summary

Push a key/value pair into the client's namespace.

%%%%
\format

\versionMarker{1.0}
\cspecificstart
\begin{codepar}
pmix_status_t
PMIx_Put(pmix_scope_t scope,
         const pmix_key_t key,
         pmix_value_t *val)
\end{codepar}
\cspecificend

\begin{arglist}
\argin{scope}{Distribution scope of the provided value (handle)}
\argin{key}{key (\refstruct{pmix_key_t})}
\argin{value}{Reference to a \refstruct{pmix_value_t} structure (handle)}
\end{arglist}

Returns \refconst{PMIX_SUCCESS} or a negative value corresponding to a PMIx error constant.

%%%%
\descr

Push a value into the client's namespace.
The client's \ac{PMIx} library will cache the information locally until \refapi{PMIx_Commit} is called.

The provided \refarg{scope} is passed to the local PMIx server, which will distribute the data to other processes according to the provided scope.
The \refstruct{pmix_scope_t} values are defined in \specrefstruct{pmix_scope_t}.
Specific implementations may support different scope values, but all implementations must support at least \code{PMIX_GLOBAL}.

The \refstruct{pmix_value_t} structure supports both string and binary values.
PMIx implementations will support heterogeneous environments by properly converting binary values between host architectures, and will copy the provided \refarg{value} into internal memory.

\adviceimplstart
The PMIx server library will properly pack/unpack data to accommodate heterogeneous environments. The host \ac{SMS} is not involved in this action. The \refarg{value} argument must be copied - the caller is free to release it following return from the function.
\adviceimplend

\adviceuserstart
The value is copied by the PMIx client library. Thus, the application is free to release and/or modify the value once the call to \refapi{PMIx_Put} has completed.

Note that keys starting with a string of ``\code{pmix}'' are exclusively reserved for the \ac{PMIx} standard and must not be used in calls to \refapi{PMIx_Put}. Thus, applications should never use a defined ``PMIX_'' attribute as the key in a call to \refapi{PMIx_Put}.
\adviceuserend


%%%%%%%%%%%
\subsection{\code{PMIx_Get}}
\declareapi{PMIx_Get}

%%%%
\summary

Retrieve a key/value pair from the client's namespace.

%%%%
\format

\versionMarker{1.0}
\cspecificstart
\begin{codepar}
pmix_status_t
PMIx_Get(const pmix_proc_t *proc, const pmix_key_t key,
         const pmix_info_t info[], size_t ninfo,
         pmix_value_t **val)
\end{codepar}
\cspecificend

\begin{arglist}
\argin{proc}{process reference (handle)}
\argin{key}{key to retrieve (\refstruct{pmix_key_t})}
\argin{info}{Array of info structures (array of handles)}
\argin{ninfo}{Number of element in the \refarg{info} array (integer)}
\argout{val}{value (handle)}
\end{arglist}

Returns \refconst{PMIX_SUCCESS} or a negative value corresponding to a PMIx error constant.

\reqattrstart
The following attributes are required to be supported by all \ac{PMIx} libraries:

\pastePRIAttributeItem{PMIX_OPTIONAL}
\pastePRIAttributeItem{PMIX_IMMEDIATE}
\pastePRIAttributeItem{PMIX_DATA_SCOPE}
\pastePRIAttributeItem{PMIX_SESSION_INFO}
\pastePRIAttributeItem{PMIX_JOB_INFO}
\pastePRIAttributeItem{PMIX_APP_INFO}
\pastePRIAttributeItem{PMIX_NODE_INFO}

\reqattrend

\optattrstart
The following attributes are optional for host environments:

\pastePRRTEAttributeItem{PMIX_TIMEOUT}

\optattrend

\adviceimplstart
We recommend that implementation of the \refattr{PMIX_TIMEOUT} attribute be left to the host environment due to race condition considerations between delivery of the data by the host environment versus internal timeout in the \ac{PMIx} server library. Implementers that choose to support \refattr{PMIX_TIMEOUT} directly in the \ac{PMIx} server library must take care to resolve the race condition and should avoid passing \refattr{PMIX_TIMEOUT} to the host environment so that multiple competing timeouts are not created.
\adviceimplend

%%%%
\descr

Retrieve information for the specified \refarg{key} as published by the process identified in the given \refstruct{pmix_proc_t}, returning a pointer to the value in the given address.

This is a blocking operation - the caller will block until either the specified data becomes available from the specified rank in the \refarg{proc} structure or the operation times out should the \refattr{PMIX_TIMEOUT} attribute have been given.
The caller is responsible for freeing all memory associated with the returned \refarg{value} when no longer required.

The \refarg{info} array is used to pass user requests regarding the get operation.

\adviceuserstart
Information provided by the \ac{PMIx} server at time of process start is accessed by providing the namespace of the job with the rank set to \refconst{PMIX_RANK_WILDCARD}. The list of data referenced in this way is maintained on the \ac{PMIx} web site at \url{https://pmix.org/support/faq/wildcard-rank-access/} but includes items such as the number of processes in the namespace (\refattr{PMIX_JOB_SIZE}), total available slots in the allocation (\refattr{PMIX_UNIV_SIZE}), and the number of nodes in the allocation (\refattr{PMIX_NUM_NODES}).

Data posted by a process via \refapi{PMIx_Put} needs to be retrieved by specifying the rank of the posting process. All other information is retrievable using a rank of \refconst{PMIX_RANK_WILDCARD} when the information being retrieved refers to something non-rank specific (e.g., number of processes on a node, number of processes in a job), and using the rank of the relevant process when requesting information that is rank-specific (e.g., the \ac{URI} of the process, or the node upon which it is executing). Each subsection of Section \ref{api:struct:attributes} indicates the appropriate rank value for referencing the defined attribute.
\adviceuserend

%%%%%%%%%%%
\subsection{\code{PMIx_Get_nb}}
\declareapi{PMIx_Get_nb}

%%%%
\summary

Nonblocking \refapi{PMIx_Get} operation.

%%%%
\format

\versionMarker{1.0}
\cspecificstart
\begin{codepar}
pmix_status_t
PMIx_Get_nb(const pmix_proc_t *proc, const char key[],
            const pmix_info_t info[], size_t ninfo,
            pmix_value_cbfunc_t cbfunc, void *cbdata)
\end{codepar}
\cspecificend

\begin{arglist}
\argin{proc}{process reference (handle)}
\argin{key}{key to retrieve (string)}
\argin{info}{Array of info structures (array of handles)}
\argin{ninfo}{Number of elements in the \refarg{info} array (integer)}
\argin{cbfunc}{Callback function (function reference)}
\argin{cbdata}{Data to be passed to the callback function (memory reference)}
\end{arglist}

Returns one of the following:

\begin{itemize}
    \item \refconst{PMIX_SUCCESS}, indicating that the request is being processed by the host environment - result will be returned in the provided \refarg{cbfunc}. Note that the library must not invoke the callback function prior to returning from the \ac{API}.
    \item \refconst{PMIX_OPERATION_SUCCEEDED}, indicating that the request was immediately processed and returned \textit{success} - the \refarg{cbfunc} will \textit{not} be called
    \item a PMIx error constant indicating either an error in the input or that the request was immediately processed and failed - the \refarg{cbfunc} will \textit{not} be called
\end{itemize}

If executed, the status returned in the provided callback function will be one of the following constants:

\begin{itemize}
\item \refconst{PMIX_SUCCESS} The requested data has been returned
\item \refconst{PMIX_ERR_NOT_FOUND} The requested data was not available
\item a non-zero \ac{PMIx} error constant indicating a reason for the request's failure
\end{itemize}

\reqattrstart
The following attributes are required to be supported by all \ac{PMIx} libraries:

\pastePRIAttributeItem{PMIX_OPTIONAL}
\pastePRIAttributeItem{PMIX_IMMEDIATE}
\pastePRIAttributeItem{PMIX_DATA_SCOPE}
\pastePRIAttributeItem{PMIX_SESSION_INFO}
\pastePRIAttributeItem{PMIX_JOB_INFO}
\pastePRIAttributeItem{PMIX_APP_INFO}
\pastePRIAttributeItem{PMIX_NODE_INFO}

\reqattrend

\optattrstart
The following attributes are optional for host environments that support this operation:

\pastePRRTEAttributeItem{PMIX_TIMEOUT}

\optattrend

\adviceimplstart
We recommend that implementation of the \refattr{PMIX_TIMEOUT} attribute be left to the host environment due to race condition considerations between delivery of the data by the host environment versus internal timeout in the \ac{PMIx} server library. Implementers that choose to support \refattr{PMIX_TIMEOUT} directly in the \ac{PMIx} server library must take care to resolve the race condition and should avoid passing \refattr{PMIX_TIMEOUT} to the host environment so that multiple competing timeouts are not created.
\adviceimplend

%%%%
\descr

The callback function will be executed once the specified data becomes available from the identified process and retrieved by the local server.
The \argref{info} array is used as described by the \refapi{PMIx_Get} routine.

\adviceuserstart
Information provided by the \ac{PMIx} server at time of process start is accessed by providing the namespace of the job with the rank set to \refconst{PMIX_RANK_WILDCARD}. Attributes referenced in this way are identified in \ref{api:struct:attributes} but includes items such as the number of processes in the namespace (\refattr{PMIX_JOB_SIZE}), total available slots in the allocation (\refattr{PMIX_UNIV_SIZE}), and the number of nodes in the allocation (\refattr{PMIX_NUM_NODES}).

In general, data posted by a process via \refapi{PMIx_Put} and data that refers directly to a process-related value needs to be retrieved by specifying the rank of the posting process. All other information is retrievable using a rank of \refconst{PMIX_RANK_WILDCARD}, as illustrated in \ref{chap:api_kv:getex}. See \ref{api:struct:attributes:retrieval} for an explanation regarding use of the \emph{level} attributes.
\adviceuserend


%%%%%%%%%%%
\subsection{\code{PMIx_Store_internal}}
\declareapi{PMIx_Store_internal}

%%%%
\summary

Store some data locally for retrieval by other areas of the proc.

%%%%
\format

\versionMarker{1.0}
\cspecificstart
\begin{codepar}
pmix_status_t
PMIx_Store_internal(const pmix_proc_t *proc,
                    const pmix_key_t key,
                    pmix_value_t *val);
\end{codepar}
\cspecificend

\begin{arglist}
\argin{proc}{process reference (handle)}
\argin{key}{key to retrieve (string)}
\argin{val}{Value to store (handle)}
\end{arglist}

Returns \refconst{PMIX_SUCCESS} or a negative value corresponding to a PMIx error constant.

%%%%
\descr

Store some data locally for retrieval by other areas of the proc.
This is data that has only internal scope - it will never be ``pushed'' externally.

%%%%%%%%%%%
\subsection{Accessing information: examples}
\label{chap:api_kv:getex}

This section provides examples illustrating methods for accessing information at various levels. The intent of the examples is not to provide comprehensive coding guidance, but rather to illustrate how \refapi{PMIx_Get} can be used to obtain information on a \refterm{session}, \refterm{job}, \refterm{application}, process, and node.

\subsubsection{Session-level information}

The \refapi{PMIx_Get} \ac{API} does not include an argument for specifying the \refterm{session} associated with the information being requested. Information regarding the session containing the requestor can be obtained by the following methods:

\begin{itemize}
\item for session-level attributes (e.g., \refattr{PMIX_UNIV_SIZE}), specifying the requestor's namespace and a rank of \refconst{PMIX_RANK_WILDCARD}; or
\item for non-specific attributes (e.g., \refattr{PMIX_NUM_NODES}), including the \refattr{PMIX_SESSION_INFO} attribute to indicate that the session-level information for that attribute is being requested
\end{itemize}

Example requests are shown below:

\cspecificstart
\begin{codepar}
pmix_info_t info;
pmix_value_t *value;
pmix_status_t rc;
pmix_proc_t myproc, wildcard;

/* initialize the client library */
PMIx_Init(&myproc, NULL, 0);

/* get the #slots in our session */
PMIX_PROC_LOAD(&wildcard, myproc.nspace, PMIX_RANK_WILDCARD);
rc = PMIx_Get(&wildcard, PMIX_UNIV_SIZE, NULL, 0, &value);

/* get the #nodes in our session */
PMIX_INFO_LOAD(&info, PMIX_SESSION_INFO, NULL, PMIX_BOOL);
rc = PMIx_Get(&wildcard, PMIX_NUM_NODES, &info, 1, &value);
\end{codepar}
\cspecificend

Information regarding a different session can be requested by either specifying the namespace and a rank of \refconst{PMIX_RANK_WILDCARD} for a process in the target session, or adding the \refattr{PMIX_SESSION_ID} attribute identifying the target session. In the latter case, the \refarg{proc} argument to \refapi{PMIx_Get} will be ignored:

\cspecificstart
\begin{codepar}
pmix_info_t info[2];
pmix_value_t *value;
pmix_status_t rc;
pmix_proc_t myproc;
uint32_t sid;

/* initialize the client library */
PMIx_Init(&myproc, NULL, 0);

/* get the #nodes in a different session */
sid = 12345;
PMIX_INFO_LOAD(&info[0], PMIX_SESSION_INFO, NULL, PMIX_BOOL);
PMIX_INFO_LOAD(&info[1], PMIX_SESSION_ID, &sid, PMIX_UINT32);
rc = PMIx_Get(&myproc, PMIX_NUM_NODES, info, 2, &value);
\end{codepar}
\cspecificend

\subsubsection{Job-level information}

Information regarding a job can be obtained by the following methods:

\begin{itemize}
\item for job-level attributes (e.g., \refattr{PMIX_JOB_SIZE} or \refattr{PMIX_JOB_NUM_APPS}), specifying the namespace of the job and a rank of \refconst{PMIX_RANK_WILDCARD} for the \refarg{proc} argument to \refapi{PMIx_Get}; or
\item for non-specific attributes (e.g., \refattr{PMIX_NUM_NODES}), including the \refattr{PMIX_JOB_INFO} attribute to indicate that the job-level information for that attribute is being requested
\end{itemize}

Example requests are shown below:

\cspecificstart
\begin{codepar}
pmix_info_t info;
pmix_value_t *value;
pmix_status_t rc;
pmix_proc_t myproc, wildcard;

/* initialize the client library */
PMIx_Init(&myproc, NULL, 0);

/* get the #apps in our job */
PMIX_PROC_LOAD(&wildcard, myproc.nspace, PMIX_RANK_WILDCARD);
rc = PMIx_Get(&wildcard, PMIX_JOB_NUM_APPS, NULL, 0, &value);

/* get the #nodes in our job */
PMIX_INFO_LOAD(&info, PMIX_JOB_INFO, NULL, PMIX_BOOL);
rc = PMIx_Get(&wildcard, PMIX_NUM_NODES, &info, 1, &value);
\end{codepar}
\cspecificend


\subsubsection{Application-level information}

Information regarding an application can be obtained by the following methods:

\begin{itemize}
\item for application-level attributes (e.g., \refattr{PMIX_APP_SIZE}), specifying the namespace and rank of a process within that application;
\item for application-level attributes (e.g., \refattr{PMIX_APP_SIZE}), including the \refattr{PMIX_APPNUM} attribute specifying the application whose information is being requested. In this case, the namespace field of the \refarg{proc} argument is used to reference the \refterm{job} containing the application - the \refterm{rank} field is ignored;
\item or application-level attributes (e.g., \refattr{PMIX_APP_SIZE}), including the \refattr{PMIX_APPNUM} and \refattr{PMIX_NSPACE} or \refattr{PMIX_JOBID} attributes specifying the job/application whose information is being requested. In this case, the \refarg{proc} argument is ignored;
\item for non-specific attributes (e.g., \refattr{PMIX_NUM_NODES}), including the \refattr{PMIX_APP_INFO} attribute to indicate that the application-level information for that attribute is being requested
\end{itemize}

Example requests are shown below:

\cspecificstart
\begin{codepar}
pmix_info_t info;
pmix_value_t *value;
pmix_status_t rc;
pmix_proc_t myproc, otherproc;
uint32_t appsize, appnum;

/* initialize the client library */
PMIx_Init(&myproc, NULL, 0);

/* get the #processes in our application */
rc = PMIx_Get(&myproc, PMIX_APP_SIZE, NULL, 0, &value);
appsize = value->data.uint32;

/* get the #nodes in an application containing "otherproc".
 * Note that the rank of a process in the other application
 * must be obtained first - a simple method is shown here */

/* assume for this example that we are in the first application
 * and we want the #nodes in the second application - use the
 * rank of the first process in that application, remembering
 * that ranks start at zero */
PMIX_PROC_LOAD(&otherproc, myproc.nspace, appsize);

PMIX_INFO_LOAD(&info, PMIX_APP_INFO, NULL, PMIX_BOOL);
rc = PMIx_Get(&otherproc, PMIX_NUM_NODES, &info, 1, &value);

/* alternatively, we can directly ask for the #nodes in
 * the second application in our job, again remembering that
 * application numbers start with zero */
appnum = 1;
PMIX_INFO_LOAD(&appinfo[0], PMIX_APP_INFO, NULL, PMIX_BOOL);
PMIX_INFO_LOAD(&appinfo[1], PMIX_APPNUM, &appnum, PMIX_UINT32);
rc = PMIx_Get(&myproc, PMIX_NUM_NODES, appinfo, 2, &value);

\end{codepar}
\cspecificend

\subsubsection{Process-level information}

Process-level information is accessed by providing the namespace and rank of the target process. In the absence of any directive as to the level of information being requested, the \ac{PMIx} library will always return the process-level value.

\subsubsection{Node-level information}

Information regarding a node within the system can be obtained by the following methods:

\begin{itemize}
\item for node-level attributes (e.g., \refattr{PMIX_NODE_SIZE}), specifying the namespace and rank of a process executing on the target node;
\item for node-level attributes (e.g., \refattr{PMIX_NODE_SIZE}), including the \refattr{PMIX_NODEID} or \refattr{PMIX_HOSTNAME} attribute specifying the node whose information is being requested. In this case, the \refarg{proc} argument's values are ignored; or
\item for non-specific attributes (e.g., \refattr{PMIX_MAX_PROCS}), including the \refattr{PMIX_NODE_INFO} attribute to indicate that the node-level information for that attribute is being requested
\end{itemize}

Example requests are shown below:

\cspecificstart
\begin{codepar}
pmix_info_t info[2];
pmix_value_t *value;
pmix_status_t rc;
pmix_proc_t myproc, otherproc;
uint32_t nodeid;

/* initialize the client library */
PMIx_Init(&myproc, NULL, 0);

/* get the #procs on our node */
rc = PMIx_Get(&myproc, PMIX_NODE_SIZE, NULL, 0, &value);

/* get the #slots on another node */
PMIX_INFO_LOAD(&info[0], PMIX_NODE_INFO, NULL, PMIX_BOOL);
PMIX_INFO_LOAD(&info[1], PMIX_HOSTNAME, "remotehost", PMIX_STRING);
rc = PMIx_Get(&myproc, PMIX_MAX_PROCS, info, 2, &value);

\end{codepar}
\cspecificend

\adviceuserstart
An explanation of the use of \refapi{PMIx_Get} versus \refapi{PMIx_Query_info_nb} is provided in \ref{chap:api_job_mgmt:query}.
\adviceuserend

%%%%%%%%%%%%%%%%%%%%%%%%%%%%%%%%%%%%%%%%%%%%%%
%%%%%%%%%%%%%%%%%%%%%%%%%%%%%%%%%%%%%%%%%%%%%%
\section{Exchanging Key/Value Pairs}
\label{chap:api_kv_mgmt:exchange}

The APIs defined in this section push key/value pairs from the client to the local \ac{PMIx} server, and circulate the data between \ac{PMIx} servers for subsequent retrieval by the local clients.

%%%%%%%%%%%
\subsection{\code{PMIx_Commit}}
\declareapi{PMIx_Commit}

%%%%
\summary

Push all previously \refapi{PMIx_Put} values to the local PMIx server.

%%%%
\format

\versionMarker{1.0}
\cspecificstart
\begin{codepar}
pmix_status_t PMIx_Commit(void)
\end{codepar}
\cspecificend

Returns \refconst{PMIX_SUCCESS} or a negative value corresponding to a PMIx error constant.

%%%%
\descr

This is an asynchronous operation.
The \ac{PRI} will immediately return to the caller while the data is transmitted to the local server in the background.

\adviceuserstart
The local PMIx server will cache the information locally - i.e., the committed data will not be circulated during \refapi{PMIx_Commit}.
Availability of the data upon completion of \refapi{PMIx_Commit} is therefore implementation-dependent.
\adviceuserend


%%%%%%%%%%%
\subsection{\code{PMIx_Fence}}
\declareapi{PMIx_Fence}

%%%%
\summary

Execute a blocking barrier across the processes identified in the specified array, collecting information posted via \refapi{PMIx_Put} as directed.

%%%%
\format

\versionMarker{1.0}
\cspecificstart
\begin{codepar}
pmix_status_t
PMIx_Fence(const pmix_proc_t procs[], size_t nprocs,
           const pmix_info_t info[], size_t ninfo)
\end{codepar}
\cspecificend

\begin{arglist}
\argin{procs}{Array of \refstruct{pmix_proc_t} structures (array of handles)}
\argin{nprocs}{Number of element in the \refarg{procs} array (integer)}
\argin{info}{Array of info structures (array of handles)}
\argin{ninfo}{Number of element in the \refarg{info} array (integer)}
\end{arglist}

Returns \refconst{PMIX_SUCCESS} or a negative value corresponding to a PMIx error constant.

\reqattrstart
The following attributes are required to be supported by all \ac{PMIx} libraries:

\pastePRIAttributeItem{PMIX_COLLECT_DATA}

\reqattrend

\optattrstart
The following attributes are optional for host environments:

\pastePRRTEAttributeItem{PMIX_TIMEOUT}
\pasteAttributeItem{PMIX_COLLECTIVE_ALGO}
\pasteAttributeItem{PMIX_COLLECTIVE_ALGO_REQD}

\optattrend

\adviceimplstart
We recommend that implementation of the \refattr{PMIX_TIMEOUT} attribute be left to the host environment due to race condition considerations between completion of the operation versus internal timeout in the \ac{PMIx} server library. Implementers that choose to support \refattr{PMIX_TIMEOUT} directly in the \ac{PMIx} server library must take care to resolve the race condition and should avoid passing \refattr{PMIX_TIMEOUT} to the host environment so that multiple competing timeouts are not created.
\adviceimplend

%%%%
\descr

Passing a \code{NULL} pointer as the \refarg{procs} parameter indicates that the fence is to span all processes in the client's namespace.
Each provided \refstruct{pmix_proc_t} struct can pass \refconst{PMIX_RANK_WILDCARD} to indicate that all processes in the given namespace are participating.

The \refarg{info} array is used to pass user requests regarding the fence operation.

Note that for scalability reasons, the default behavior for \refapi{PMIx_Fence} is to not collect the data.

\adviceimplstart
\refapi{PMIx_Fence} and its non-blocking form are both \emph{collective} operations. Accordingly, the \ac{PMIx} server library is required to aggregate participation by local clients, passing the request to the host environment once all local participants have executed the \ac{API}.
\adviceimplend

\advicermstart
The host will receive a single call for each collective operation. It is the responsibility of the host to identify the nodes containing participating processes, execute the collective across all participating nodes, and notify the local \ac{PMIx} server library upon completion of the global collective.
\advicermend

%%%%%%%%%%%
\subsection{\code{PMIx_Fence_nb}}
\declareapi{PMIx_Fence_nb}

%%%%
\summary

Execute a nonblocking \refapi{PMIx_Fence} across the processes identified in the specified array of processes, collecting information posted via \refapi{PMIx_Put} as directed.

%%%%
\format

\versionMarker{1.0}
\cspecificstart
\begin{codepar}
pmix_status_t
PMIx_Fence_nb(const pmix_proc_t procs[], size_t nprocs,
              const pmix_info_t info[], size_t ninfo,
              pmix_op_cbfunc_t cbfunc, void *cbdata)
\end{codepar}
\cspecificend

\begin{arglist}
\argin{procs}{Array of \refstruct{pmix_proc_t} structures (array of handles)}
\argin{nprocs}{Number of element in the \refarg{procs} array (integer)}
\argin{info}{Array of info structures (array of handles)}
\argin{ninfo}{Number of element in the \refarg{info} array (integer)}
\argin{cbfunc}{Callback function (function reference)}
\argin{cbdata}{Data to be passed to the callback function (memory reference)}
\end{arglist}

Returns one of the following:

\begin{itemize}
    \item \refconst{PMIX_SUCCESS}, indicating that the request is being processed by the host environment - result will be returned in the provided \refarg{cbfunc}. Note that the library must not invoke the callback function prior to returning from the \ac{API}.
    \item \refconst{PMIX_OPERATION_SUCCEEDED}, indicating that the request was immediately processed and returned \textit{success} - the \refarg{cbfunc} will \textit{not} be called. This can occur if the collective involved only processes on the local node.
    \item a PMIx error constant indicating either an error in the input or that the request was immediately processed and failed - the \refarg{cbfunc} will \textit{not} be called
\end{itemize}


\reqattrstart
The following attributes are required to be supported by all \ac{PMIx} libraries:

\pastePRIAttributeItem{PMIX_COLLECT_DATA}

\reqattrend

\optattrstart
The following attributes are optional for host environments that support this operation:

\pastePRRTEAttributeItem{PMIX_TIMEOUT}
\pasteAttributeItem{PMIX_COLLECTIVE_ALGO}
\pasteAttributeItem{PMIX_COLLECTIVE_ALGO_REQD}

\optattrend

\adviceimplstart
We recommend that implementation of the \refattr{PMIX_TIMEOUT} attribute be left to the host environment due to race condition considerations between completion of the operation versus internal timeout in the \ac{PMIx} server library. Implementers that choose to support \refattr{PMIX_TIMEOUT} directly in the \ac{PMIx} server library must take care to resolve the race condition and should avoid passing \refattr{PMIX_TIMEOUT} to the host environment so that multiple competing timeouts are not created.
\adviceimplend

%%%%
\descr

Nonblocking \refapi{PMIx_Fence} routine.
Note that the function will return an error if a \code{NULL} callback function is given.

Note that for scalability reasons, the default behavior for \refapi{PMIx_Fence_nb} is to not collect the data.

See the \refapi{PMIx_Fence} description for further details.

%%%%%%%%%%%%%%%%%%%%%%%%%%%%%%%%%%%%%%%%%%%%%%
%%%%%%%%%%%%%%%%%%%%%%%%%%%%%%%%%%%%%%%%%%%%%%
\section{Publish and Lookup Data}
\label{chap:api_kv_mgmt:publish}

The APIs defined in this section publish data from one client that can be later exchanged and looked up by another client.

\adviceimplstart
\ac{PMIx} libraries that support any of the functions in this section are required to support \textit{all} of them.
\adviceimplend

\advicermstart
Host environments that support any of the functions in this section are required to support \textit{all} of them.
\advicermend

%%%%%%%%%%%
\subsection{\code{PMIx_Publish}}
\declareapi{PMIx_Publish}

%%%%
\summary

Publish data for later access via \refapi{PMIx_Lookup}.

%%%%
\format

\versionMarker{1.0}
\cspecificstart
\begin{codepar}
pmix_status_t
PMIx_Publish(const pmix_info_t info[], size_t ninfo)
\end{codepar}
\cspecificend

\begin{arglist}
\argin{info}{Array of info structures (array of handles)}
\argin{ninfo}{Number of element in the \refarg{info} array (integer)}
\end{arglist}

Returns \refconst{PMIX_SUCCESS} or a negative value corresponding to a PMIx error constant.

\reqattrstart
\ac{PMIx} libraries are not required to directly support any attributes for this function. However, any provided attributes must be passed to the host \ac{SMS} daemon for processing, and the \ac{PMIx} library is \textit{required} to add the \refPRIAttributeItem{PMIX_USERID} and the \refPRIAttributeItem{PMIX_GRPID} attributes of the client process that published the info.

\reqattrend

\optattrstart
The following attributes are optional for host environments that support this operation:

\pastePRRTEAttributeItem{PMIX_TIMEOUT}
\pastePRRTEAttributeItem{PMIX_RANGE}
\pastePRRTEAttributeItem{PMIX_PERSISTENCE}

\optattrend

\adviceimplstart
We recommend that implementation of the \refattr{PMIX_TIMEOUT} attribute be left to the host environment due to race condition considerations between completion of the operation versus internal timeout in the \ac{PMIx} server library. Implementers that choose to support \refattr{PMIX_TIMEOUT} directly in the \ac{PMIx} server library must take care to resolve the race condition and should avoid passing \refattr{PMIX_TIMEOUT} to the host environment so that multiple competing timeouts are not created.
\adviceimplend

%%%%
\descr

Publish the data in the \refarg{info} array for subsequent lookup.
By default, the data will be published into the \refconst{PMIX_RANGE_SESSION} range and with \refconst{PMIX_PERSIST_APP} persistence.
Changes to those values, and any additional directives, can be included in the \refstruct{pmix_info_t} array. Attempts to access the data by processes outside of the provided data range will be rejected. The persistence parameter instructs the server as to how long the data is to be retained.

The blocking form will block until the server confirms that the data has been sent to the \ac{PMIx} server and that it has obtained confirmation from its host \ac{SMS} daemon that the data is ready to be looked up. Data is copied into the backing key-value data store, and therefore the \refarg{info} array can be released upon return from the blocking function call.

\adviceuserstart
Duplicate keys within the specified data range may lead to unexpected behavior depending
on host RM implementation of the backing key-value store.
\adviceuserend

\adviceimplstart
Implementations should, to the best of their ability, detect duplicate keys and protect the
user from unexpected behavior - preferably returning an error. This version of the
standard does not define a specific error code to be returned, so the implementation must
make it clear to the user what to expect in this scenario. One suggestion is to define an RM
specific error code beyond the \refconst{PMIX_EXTERNAL_ERR_BASE} boundary. Future versions of the standard
will clarify that a specific \ac{PMIx} error be returned when conflicting values are published for a given
key, and will provide attributes to allow modified behaviors such as overwrite.
\adviceimplend

%%%%%%%%%%%
\subsection{\code{PMIx_Publish_nb}}
\declareapi{PMIx_Publish_nb}

%%%%
\summary

Nonblocking \refapi{PMIx_Publish} routine.

%%%%
\format

\versionMarker{1.0}
\cspecificstart
\begin{codepar}
pmix_status_t
PMIx_Publish_nb(const pmix_info_t info[], size_t ninfo,
                pmix_op_cbfunc_t cbfunc, void *cbdata)
\end{codepar}
\cspecificend

\begin{arglist}
\argin{info}{Array of info structures (array of handles)}
\argin{ninfo}{Number of element in the \refarg{info} array (integer)}
\argin{cbfunc}{Callback function \refapi{pmix_op_cbfunc_t} (function reference)}
\argin{cbdata}{Data to be passed to the callback function (memory reference)}
\end{arglist}

Returns one of the following:

\begin{itemize}
    \item \refconst{PMIX_SUCCESS}, indicating that the request is being processed by the host environment - result will be returned in the provided \refarg{cbfunc}. Note that the library must not invoke the callback function prior to returning from the \ac{API}.
    \item \refconst{PMIX_OPERATION_SUCCEEDED}, indicating that the request was immediately processed and returned \textit{success} - the \refarg{cbfunc} will \textit{not} be called
    \item a PMIx error constant indicating either an error in the input or that the request was immediately processed and failed - the \refarg{cbfunc} will \textit{not} be called
\end{itemize}

\reqattrstart
\ac{PMIx} libraries are not required to directly support any attributes for this function. However, any provided attributes must be passed to the host \ac{SMS} daemon for processing, and the \ac{PMIx} library is \textit{required} to add the \refPRIAttributeItem{PMIX_USERID} and the \refPRIAttributeItem{PMIX_GRPID} attributes of the client process that published the info.

\reqattrend

\optattrstart
The following attributes are optional for host environments that support this operation:

\pastePRRTEAttributeItem{PMIX_TIMEOUT}
\pastePRRTEAttributeItem{PMIX_RANGE}
\pastePRRTEAttributeItem{PMIX_PERSISTENCE}

\optattrend

\adviceimplstart
We recommend that implementation of the \refattr{PMIX_TIMEOUT} attribute be left to the host environment due to race condition considerations between completion of the operation versus internal timeout in the \ac{PMIx} server library. Implementers that choose to support \refattr{PMIX_TIMEOUT} directly in the \ac{PMIx} server library must take care to resolve the race condition and should avoid passing \refattr{PMIX_TIMEOUT} to the host environment so that multiple competing timeouts are not created.
\adviceimplend

%%%%
\descr

Nonblocking \refapi{PMIx_Publish} routine. The non-blocking form will return immediately, executing the callback when the \ac{PMIx} server receives confirmation from its host \ac{SMS} daemon.

Note that the function will return an error if a \code{NULL} callback function is given, and that the \refarg{info} array must be maintained until the callback is provided.


%%%%%%%%%%%
\subsection{\code{PMIx_Lookup}}
\declareapi{PMIx_Lookup}

%%%%
\summary

Lookup information published by this or another process with \refapi{PMIx_Publish} or \refapi{PMIx_Publish_nb}.

%%%%
\format

\versionMarker{1.0}
\cspecificstart
\begin{codepar}
pmix_status_t
PMIx_Lookup(pmix_pdata_t data[], size_t ndata,
            const pmix_info_t info[], size_t ninfo)
\end{codepar}
\cspecificend

\begin{arglist}
\arginout{data}{Array of publishable data structures (array of handles)}
\argin{ndata}{Number of elements in the \refarg{data} array (integer)}
\argin{info}{Array of info structures (array of handles)}
\argin{ninfo}{Number of elements in the \refarg{info} array (integer)}
\end{arglist}

Returns \refconst{PMIX_SUCCESS} or a negative value corresponding to a PMIx error constant.

\reqattrstart
\ac{PMIx} libraries are not required to directly support any attributes for this function. However, any provided attributes must be passed to the host \ac{SMS} daemon for processing, and the \ac{PMIx} library is \textit{required} to add the \refPRIAttributeItem{PMIX_USERID} and the \refPRIAttributeItem{PMIX_GRPID} attributes of the client process that is requesting the info.

\reqattrend

\optattrstart
The following attributes are optional for host environments that support this operation:

\pastePRRTEAttributeItem{PMIX_TIMEOUT}
\pastePRRTEAttributeItem{PMIX_RANGE}
\pastePRRTEAttributeItem{PMIX_WAIT}

\optattrend

\adviceimplstart
We recommend that implementation of the \refattr{PMIX_TIMEOUT} attribute be left to the host environment due to race condition considerations between completion of the operation versus internal timeout in the \ac{PMIx} server library. Implementers that choose to support \refattr{PMIX_TIMEOUT} directly in the \ac{PMIx} server library must take care to resolve the race condition and should avoid passing \refattr{PMIX_TIMEOUT} to the host environment so that multiple competing timeouts are not created.
\adviceimplend

%%%%
\descr

Lookup information published by this or another process.
By default, the search will be conducted across the \refconst{PMIX_RANGE_SESSION} range.
Changes to the range, and any additional directives, can be provided in the \refstruct{pmix_info_t} array.

Note that the search is also constrained to only data published by the current user (i.e., the search will not return data published by an application being executed by another user).
There currently is no option to override this behavior - such an option may become available later via an appropriate \refstruct{pmix_info_t} directive.

The \argref{data} parameter consists of an array of \refstruct{pmix_pdata_t} struct with the keys specifying the requested information.
Data will be returned for each key in the associated \refarg{value} struct.
Any key that cannot be found will return with a data type of \refconst{PMIX_UNDEF}.
The function will return \refconst{PMIX_SUCCESS} if any values can be found, so the caller must check each data element to ensure it was returned.

The proc field in each \refstruct{pmix_pdata_t} struct will contain the namespace/rank of the process that published the data.

\adviceuserstart
Although this is a blocking function, it will not wait by default for the requested data to be published.
Instead, it will block for the time required by the server to lookup its current data and return any found items.
Thus, the caller is responsible for ensuring that data is published prior to executing a lookup, using \refattr{PMIX_WAIT} to instruct the server to wait for the data to be published, or for retrying until the requested data is found.
\adviceuserend

%%%%%%%%%%%
\subsection{\code{PMIx_Lookup_nb}}
\declareapi{PMIx_Lookup_nb}

%%%%
\summary

Nonblocking version of \refapi{PMIx_Lookup}.

%%%%
\format

\versionMarker{1.0}
\cspecificstart
\begin{codepar}
pmix_status_t
PMIx_Lookup_nb(char **keys,
               const pmix_info_t info[], size_t ninfo,
               pmix_lookup_cbfunc_t cbfunc, void *cbdata)
\end{codepar}
\cspecificend

\begin{arglist}
\argin{keys}{Array to be provided to the callback (array of strings)}
\argin{info}{Array of info structures (array of handles)}
\argin{ninfo}{Number of element in the \refarg{info} array (integer)}
\argin{cbfunc}{Callback function (handle)}
\argin{cbdata}{Callback data to be provided to the callback function (pointer)}
\end{arglist}

Returns one of the following:

\begin{itemize}
    \item \refconst{PMIX_SUCCESS}, indicating that the request is being processed by the host environment - result will be returned in the provided \refarg{cbfunc}. Note that the library must not invoke the callback function prior to returning from the \ac{API}.
    \item a PMIx error constant indicating an error in the input - the \refarg{cbfunc} will \textit{not} be called
\end{itemize}


\reqattrstart
\ac{PMIx} libraries are not required to directly support any attributes for this function. However, any provided attributes must be passed to the host \ac{SMS} daemon for processing, and the \ac{PMIx} library is \textit{required} to add the \refPRIAttributeItem{PMIX_USERID} and the \refPRIAttributeItem{PMIX_GRPID} attributes of the client process that is requesting the info.

\reqattrend

\optattrstart
The following attributes are optional for host environments that support this operation:

\pastePRRTEAttributeItem{PMIX_TIMEOUT}
\pastePRRTEAttributeItem{PMIX_RANGE}
\pastePRRTEAttributeItem{PMIX_WAIT}

\optattrend

\adviceimplstart
We recommend that implementation of the \refattr{PMIX_TIMEOUT} attribute be left to the host environment due to race condition considerations between completion of the operation versus internal timeout in the \ac{PMIx} server library. Implementers that choose to support \refattr{PMIX_TIMEOUT} directly in the \ac{PMIx} server library must take care to resolve the race condition and should avoid passing \refattr{PMIX_TIMEOUT} to the host environment so that multiple competing timeouts are not created.
\adviceimplend


%%%%
\descr

Non-blocking form of the \refapi{PMIx_Lookup} function.
Data for the provided NULL-terminated \refarg{keys} array will be returned in the provided callback function.
As with \refapi{PMIx_Lookup}, the default behavior is to not wait for data to be published.
The \refarg{info} array can be used to modify the behavior as previously described by \refapi{PMIx_Lookup}. Both the \refarg{info} and \refarg{keys} arrays must be maintained until the callback is provided.



%%%%%%%%%%%
\subsection{\code{PMIx_Unpublish}}
\declareapi{PMIx_Unpublish}

%%%%
\summary

Unpublish data posted by this process using the given keys.

%%%%
\format

\versionMarker{1.0}
\cspecificstart
\begin{codepar}
pmix_status_t
PMIx_Unpublish(char **keys,
               const pmix_info_t info[], size_t ninfo)
\end{codepar}
\cspecificend

\begin{arglist}
\argin{info}{Array of info structures (array of handles)}
\argin{ninfo}{Number of element in the \refarg{info} array (integer)}
\end{arglist}

Returns \refconst{PMIX_SUCCESS} or a negative value corresponding to a PMIx error constant.

\reqattrstart
\ac{PMIx} libraries are not required to directly support any attributes for this function. However, any provided attributes must be passed to the host \ac{SMS} daemon for processing, and the \ac{PMIx} library is \textit{required} to add the \refPRIAttributeItem{PMIX_USERID} and the \refPRIAttributeItem{PMIX_GRPID} attributes of the client process that is requesting the operation.

\reqattrend

\optattrstart
The following attributes are optional for host environments that support this operation:

\pastePRRTEAttributeItem{PMIX_TIMEOUT}
\pastePRRTEAttributeItem{PMIX_RANGE}

\optattrend

\adviceimplstart
We recommend that implementation of the \refattr{PMIX_TIMEOUT} attribute be left to the host environment due to race condition considerations between completion of the operation versus internal timeout in the \ac{PMIx} server library. Implementers that choose to support \refattr{PMIX_TIMEOUT} directly in the \ac{PMIx} server library must take care to resolve the race condition and should avoid passing \refattr{PMIX_TIMEOUT} to the host environment so that multiple competing timeouts are not created.
\adviceimplend


%%%%
\descr

Unpublish data posted by this process using the given \refarg{keys}.
The function will block until the data has been removed by the server (i.e., it is safe to publish that key again).
A value of \code{NULL} for the \refarg{keys} parameter instructs the server to remove all data published by this process.

By default, the range is assumed to be \refconst{PMIX_RANGE_SESSION}.
Changes to the range, and any additional directives, can be provided in the \refarg{info} array.


%%%%%%%%%%%
\subsection{\code{PMIx_Unpublish_nb}}
\declareapi{PMIx_Unpublish_nb}

%%%%
\summary

Nonblocking version of \refapi{PMIx_Unpublish}.

%%%%
\format

\versionMarker{1.0}
\cspecificstart
\begin{codepar}
pmix_status_t
PMIx_Unpublish_nb(char **keys,
                  const pmix_info_t info[], size_t ninfo,
                  pmix_op_cbfunc_t cbfunc, void *cbdata)
\end{codepar}
\cspecificend

\begin{arglist}
\argin{keys}{(array of strings)}
\argin{info}{Array of info structures (array of handles)}
\argin{ninfo}{Number of element in the \refarg{info} array (integer)}
\argin{cbfunc}{Callback function \refapi{pmix_op_cbfunc_t} (function reference)}
\argin{cbdata}{Data to be passed to the callback function (memory reference)}
\end{arglist}

Returns one of the following:

\begin{itemize}
    \item \refconst{PMIX_SUCCESS}, indicating that the request is being processed by the host environment - result will be returned in the provided \refarg{cbfunc}. Note that the library must not invoke the callback function prior to returning from the \ac{API}.
    \item \refconst{PMIX_OPERATION_SUCCEEDED}, indicating that the request was immediately processed and returned \textit{success} - the \refarg{cbfunc} will \textit{not} be called
    \item a PMIx error constant indicating either an error in the input or that the request was immediately processed and failed - the \refarg{cbfunc} will \textit{not} be called
\end{itemize}

\reqattrstart
\ac{PMIx} libraries are not required to directly support any attributes for this function. However, any provided attributes must be passed to the host \ac{SMS} daemon for processing, and the \ac{PMIx} library is \textit{required} to add the \refPRIAttributeItem{PMIX_USERID} and the \refPRIAttributeItem{PMIX_GRPID} attributes of the client process that is requesting the operation.

\reqattrend

\optattrstart
The following attributes are optional for host environments that support this operation:

\pastePRRTEAttributeItem{PMIX_TIMEOUT}
\pastePRRTEAttributeItem{PMIX_RANGE}

\optattrend

\adviceimplstart
We recommend that implementation of the \refattr{PMIX_TIMEOUT} attribute be left to the host environment due to race condition considerations between completion of the operation versus internal timeout in the \ac{PMIx} server library. Implementers that choose to support \refattr{PMIX_TIMEOUT} directly in the \ac{PMIx} server library must take care to resolve the race condition and should avoid passing \refattr{PMIX_TIMEOUT} to the host environment so that multiple competing timeouts are not created.
\adviceimplend

%%%%
\descr

Non-blocking form of the \refapi{PMIx_Unpublish} function.
The callback function will be executed once the server confirms removal of the specified data. The \refarg{info} array must be maintained until the callback is provided.



%%%%%%%%%%%%%%%%%%%%%%%%%%%%%%%%%%%%%%%%%%%%%%%%%


    % Process Management
    %  - spawn, (dis)connect, resolve_peers
    %%%%%%%%%%%%%%%%%%%%%%%%%%%%%%%%%%%%%%%%%%%%%%%%%
% Chapter: Process Management
%%%%%%%%%%%%%%%%%%%%%%%%%%%%%%%%%%%%%%%%%%%%%%%%%
\chapter{Process Management}
\label{chap:api_proc_mgmt}

This chapter defines functionality processes can use to abort processes, spawn processes, and determine the relative locality of local processes.

%%%%%%%%%%%%%%%%%%%%%%%%%%%%%%%%%%%%%%%%%%%%%%%%%
%%%%%%%%%%%%%%%%%%%%%%%%%%%%%%%%%%%%%%%%%%%%%%%%%
\section{Abort}
\label{chap:api_proc_mgmt:abort}

\ac{PMIx} provides a dedicated API by which an application can request that specified processes be aborted by the system.

%%%%%%%%%%%%%%%%%%%%%%%%%%%%%%%%%%%%%%%%%%%%%%%%%
\subsection{\code{PMIx_Abort}}
\declareapi{PMIx_Abort}

%%%%
\summary

Abort the specified processes

%%%%
\format

\versionMarker{1.0}
\cspecificstart
\begin{codepar}
pmix_status_t
PMIx_Abort(int status, const char msg[],
           pmix_proc_t procs[], size_t nprocs)
\end{codepar}
\cspecificend

\begin{arglist}
\argin{status}{Error code to return to invoking environment (integer)}
\argin{msg}{String message to be returned to user (string)}
\argin{procs}{Array of \refstruct{pmix_proc_t} structures (array of handles)}
\argin{nprocs}{Number of elements in the \refarg{procs} array (integer)}
\end{arglist}

Returns one of the following:

\begin{itemize}
    \item \refconst{PMIX_SUCCESS} if the operation was successfully completed. Note that the function shall not return in this situation if the caller's own process was included in the request.
    \item \refconst{PMIX_ERR_PARAM_VALUE_NOT_SUPPORTED} if the \ac{PMIx} implementation and host environment support this \ac{API}, but the request includes processes that the host environment cannot abort - e.g., if the request is to abort subsets of processes from a namespace, or processes outside of the caller's own namespace, and the host environment does not permit such operations. In this case, none of the specified processes will be terminated.
    \item a PMIx error constant indicating an error in the request.
\end{itemize}

%%%%
\descr

Request that the host resource manager print the provided message and abort the provided array of \refarg{procs}.
A Unix or POSIX environment should handle the provided status as a return error code from the main program that launched the application.
A \code{NULL} for the \refarg{procs} array indicates that all processes in the caller's namespace are to be aborted, including itself - this is the equivalent of passing a \refstruct{pmix_proc_t} array element containing the caller's namespace and a rank value of \refconst{PMIX_RANK_WILDCARD}. While it is permitted for a caller to request abort of processes from namespaces other than its own, not all environments will support such requests.
Passing a \code{NULL} \refarg{msg} parameter is allowed.

The function shall not return until the host environment has carried out the operation on the specified processes. If the caller is included in the array of targets, then the function will not return unless the host is unable to execute the operation.

\adviceuserstart
The response to this request is somewhat dependent on the specific \ac{RM} and its configuration (e.g., some resource managers will not abort the application if the provided status is zero unless specifically configured to do so, some cannot abort subsets of processes in an application, and some may not permit termination of processes outside of the caller's own namespace), and thus lies outside the control of PMIx itself.
However, the PMIx client library shall inform the \ac{RM} of the request that the specified \refarg{procs} be aborted, regardless of the value of the provided status.

Note that race conditions caused by multiple processes calling \refapi{PMIx_Abort} are left to the server implementation to resolve with regard to which status is returned and what messages (if any) are printed.
\adviceuserend


%%%%%%%%%%%%%%%%%%%%%%%%%%%%%%%%%%%%%%%%%%%%%%%%%
%%%%%%%%%%%%%%%%%%%%%%%%%%%%%%%%%%%%%%%%%%%%%%%%%
\section{Process Creation}
\label{chap:api_proc_mgmt:spawn}

The \refapi{PMIx_Spawn} commands spawn new processes and/or applications in the \ac{PMIx} universe. This may include requests to extend the existing resource allocation or obtain a new one, depending upon provided and supported attributes.

%%%%%%%%%%%%%%%%%%%%%%%%%%%%%%%%%%%%%%%%%%%%%%%%%
\subsection{\code{PMIx_Spawn}}
\declareapi{PMIx_Spawn}

%%%%
\summary

Spawn a new job.

%%%%
\format

\versionMarker{1.0}
\cspecificstart
\begin{codepar}
pmix_status_t
PMIx_Spawn(const pmix_info_t job_info[], size_t ninfo,
           const pmix_app_t apps[], size_t napps,
           char nspace[])
\end{codepar}
\cspecificend

\begin{arglist}
\argin{job_info}{Array of info structures (array of handles)}
\argin{ninfo}{Number of elements in the \refarg{job_info} array (integer)}
\argin{apps}{Array of \refstruct{pmix_app_t} structures (array of handles)}
\argin{napps}{Number of elements in the \refarg{apps} array (integer)}
\argout{nspace}{Namespace of the new job (string)}
\end{arglist}

Returns \refconst{PMIX_SUCCESS} or a negative value corresponding to a PMIx error constant.

\reqattrstart
\ac{PMIx} libraries are not required to directly support any attributes for this function. However, any provided attributes must be passed to the host environment for processing.

Host environments are required to support the following attributes when present in either the \refarg{job_info} or the \textit{info} array of an element of the \refarg{apps} array:

\pasteAttributeItem{PMIX_WDIR}
\pasteAttributeItem{PMIX_SET_SESSION_CWD}
\pasteAttributeItem{PMIX_PREFIX}
\pasteAttributeItem{PMIX_HOST}
\pasteAttributeItem{PMIX_HOSTFILE}

\reqattrend

\optattrstart
The following attributes are optional for host environments that support this operation:

\pasteAttributeItem{PMIX_ADD_HOSTFILE}
\pasteAttributeItem{PMIX_ADD_HOST}
\pasteAttributeItem{PMIX_PRELOAD_BIN}
\pasteAttributeItem{PMIX_PRELOAD_FILES}
\pasteAttributeItem{PMIX_PERSONALITY}
\pasteAttributeItem{PMIX_DISPLAY_MAP}
\pasteAttributeItem{PMIX_PPR}
\pasteAttributeItem{PMIX_MAPBY}
\pasteAttributeItem{PMIX_RANKBY}
\pasteAttributeItem{PMIX_BINDTO}
\pasteAttributeItem{PMIX_STDIN_TGT}
\pasteAttributeItem{PMIX_TAG_OUTPUT}
\pasteAttributeItem{PMIX_TIMESTAMP_OUTPUT}
\pasteAttributeItem{PMIX_MERGE_STDERR_STDOUT}
\pasteAttributeItem{PMIX_OUTPUT_TO_FILE}
\pasteAttributeItem{PMIX_INDEX_ARGV}
\pasteAttributeItem{PMIX_CPUS_PER_PROC}
\pasteAttributeItem{PMIX_NO_PROCS_ON_HEAD}
\pasteAttributeItem{PMIX_NO_OVERSUBSCRIBE}
\pasteAttributeItem{PMIX_REPORT_BINDINGS}
\pasteAttributeItem{PMIX_CPU_LIST}
\pasteAttributeItem{PMIX_JOB_RECOVERABLE}
\pasteAttributeItem{PMIX_JOB_CONTINUOUS}
\pasteAttributeItem{PMIX_MAX_RESTARTS}
\pasteAttributeItem{PMIX_SET_ENVAR}
\pasteAttributeItem{PMIX_UNSET_ENVAR}
\pasteAttributeItem{PMIX_ADD_ENVAR}
\pasteAttributeItem{PMIX_PREPEND_ENVAR}
\pasteAttributeItem{PMIX_APPEND_ENVAR}
\pasteAttributeItem{PMIX_FIRST_ENVAR}
\pasteAttributeItem{PMIX_ALLOC_QUEUE}
\pasteAttributeItem{PMIX_ALLOC_TIME}
\pasteAttributeItem{PMIX_ALLOC_NUM_NODES}
\pasteAttributeItem{PMIX_ALLOC_NODE_LIST}
\pasteAttributeItem{PMIX_ALLOC_NUM_CPUS}
\pasteAttributeItem{PMIX_ALLOC_NUM_CPU_LIST}
\pasteAttributeItem{PMIX_ALLOC_CPU_LIST}
\pasteAttributeItem{PMIX_ALLOC_MEM_SIZE}
\pasteAttributeItem{PMIX_ALLOC_BANDWIDTH}
\pasteAttributeItem{PMIX_ALLOC_FABRIC_QOS}
\pasteAttributeItem{PMIX_ALLOC_FABRIC_TYPE}
\pasteAttributeItem{PMIX_ALLOC_FABRIC_PLANE}
\pasteAttributeItem{PMIX_ALLOC_FABRIC_ENDPTS}
\pasteAttributeItem{PMIX_ALLOC_FABRIC_ENDPTS_NODE}
\pasteAttributeItem{PMIX_COSPAWN_APP}
\pasteAttributeItem{PMIX_SPAWN_TOOL}
\pasteAttributeItem{PMIX_EVENT_SILENT_TERMINATION}

\optattrend

%%%%
\descr

Spawn a new job.
The assigned namespace of the spawned applications is returned in the \refarg{nspace} parameter.
A \code{NULL} value in that location indicates that the caller doesn't wish to have the namespace returned.
The \refarg{nspace} array must be at least of size one more than \refconst{PMIX_MAX_NSLEN}.

By default, the spawned processes will be PMIx ``connected'' to the parent process upon successful launch (see Section \ref{chap:api_proc_mgmt:connect}
for details). This includes that (a) the parent process will be given a copy of the new job's
information so it can query job-level info without incurring any communication penalties, (b) newly spawned child processes will receive a copy of the parent processes job-level info, and (c) both the parent process and members of the child job will receive notification of errors from processes in their combined assemblage.

\adviceuserstart
Behavior of individual resource managers may differ, but it is expected that failure of any application process to start will result in termination/cleanup of all processes in the newly spawned job and return of an error code to the caller.
\adviceuserend

\adviceimplstart
Tools may utilize \refapi{PMIx_Spawn} to start intermediate launchers as described in Section \ref{chap:api_tools:indirect}. For times where the tool is not attached to a \ac{PMIx} server, internal support for fork/exec of the specified applications would allow the tool to maintain a single code path for both the connected and disconnected cases. Inclusion of such support is recommended, but not required.
\adviceimplend


%%%%%%%%%%%%%%%%%%%%%%%%%%%%%%%%%%%%%%%%%%%%%%%%%
\subsection{\code{PMIx_Spawn_nb}}
\declareapi{PMIx_Spawn_nb}

%%%%
\summary

Nonblocking version of the \refapi{PMIx_Spawn} routine.

%%%%
\format

\versionMarker{1.0}
\cspecificstart
\begin{codepar}
pmix_status_t
PMIx_Spawn_nb(const pmix_info_t job_info[], size_t ninfo,
              const pmix_app_t apps[], size_t napps,
              pmix_spawn_cbfunc_t cbfunc, void *cbdata)
\end{codepar}
\cspecificend

\begin{arglist}
\argin{job_info}{Array of info structures (array of handles)}
\argin{ninfo}{Number of elements in the \refarg{job_info} array (integer)}
\argin{apps}{Array of \refstruct{pmix_app_t} structures (array of handles)}
\argin{cbfunc}{Callback function \refapi{pmix_spawn_cbfunc_t} (function reference)}
\argin{cbdata}{Data to be passed to the callback function (memory reference)}
\end{arglist}

Returns one of the following:

\begin{itemize}
    \item \refconst{PMIX_SUCCESS}, indicating that the request is being processed by the host environment - result will be returned in the provided \refarg{cbfunc}. Note that the library must not invoke the callback function prior to returning from the \ac{API}.
    \item a PMIx error constant indicating an error in the request - the \refarg{cbfunc} will \textit{not} be called
\end{itemize}


\reqattrstart
\ac{PMIx} libraries are not required to directly support any attributes for this function. However, any provided attributes must be passed to the host \ac{SMS} daemon for processing.

Host environments are required to support the following attributes when present in either the \refarg{job_info} or the \textit{info} array of an element of the \refarg{apps} array:

\pasteAttributeItem{PMIX_WDIR}
\pasteAttributeItem{PMIX_SET_SESSION_CWD}
\pasteAttributeItem{PMIX_PREFIX}
\pasteAttributeItem{PMIX_HOST}
\pasteAttributeItem{PMIX_HOSTFILE}

\reqattrend

\optattrstart
The following attributes are optional for host environments that support this operation:

\pasteAttributeItem{PMIX_ADD_HOSTFILE}
\pasteAttributeItem{PMIX_ADD_HOST}
\pasteAttributeItem{PMIX_PRELOAD_BIN}
\pasteAttributeItem{PMIX_PRELOAD_FILES}
\pasteAttributeItem{PMIX_PERSONALITY}
\pasteAttributeItem{PMIX_DISPLAY_MAP}
\pasteAttributeItem{PMIX_PPR}
\pasteAttributeItem{PMIX_MAPBY}
\pasteAttributeItem{PMIX_RANKBY}
\pasteAttributeItem{PMIX_BINDTO}
\pasteAttributeItem{PMIX_STDIN_TGT}
\pasteAttributeItem{PMIX_TAG_OUTPUT}
\pasteAttributeItem{PMIX_TIMESTAMP_OUTPUT}
\pasteAttributeItem{PMIX_MERGE_STDERR_STDOUT}
\pasteAttributeItem{PMIX_OUTPUT_TO_FILE}
\pasteAttributeItem{PMIX_INDEX_ARGV}
\pasteAttributeItem{PMIX_CPUS_PER_PROC}
\pasteAttributeItem{PMIX_NO_PROCS_ON_HEAD}
\pasteAttributeItem{PMIX_NO_OVERSUBSCRIBE}
\pasteAttributeItem{PMIX_REPORT_BINDINGS}
\pasteAttributeItem{PMIX_CPU_LIST}
\pasteAttributeItem{PMIX_JOB_RECOVERABLE}
\pasteAttributeItem{PMIX_JOB_CONTINUOUS}
\pasteAttributeItem{PMIX_MAX_RESTARTS}
\pasteAttributeItem{PMIX_SET_ENVAR}
\pasteAttributeItem{PMIX_UNSET_ENVAR}
\pasteAttributeItem{PMIX_ADD_ENVAR}
\pasteAttributeItem{PMIX_PREPEND_ENVAR}
\pasteAttributeItem{PMIX_APPEND_ENVAR}
\pasteAttributeItem{PMIX_FIRST_ENVAR}
\pasteAttributeItem{PMIX_ALLOC_QUEUE}
\pasteAttributeItem{PMIX_ALLOC_TIME}
\pasteAttributeItem{PMIX_ALLOC_NUM_NODES}
\pasteAttributeItem{PMIX_ALLOC_NODE_LIST}
\pasteAttributeItem{PMIX_ALLOC_NUM_CPUS}
\pasteAttributeItem{PMIX_ALLOC_NUM_CPU_LIST}
\pasteAttributeItem{PMIX_ALLOC_CPU_LIST}
\pasteAttributeItem{PMIX_ALLOC_MEM_SIZE}
\pasteAttributeItem{PMIX_ALLOC_BANDWIDTH}
\pasteAttributeItem{PMIX_ALLOC_FABRIC_QOS}
\pasteAttributeItem{PMIX_ALLOC_FABRIC_TYPE}
\pasteAttributeItem{PMIX_ALLOC_FABRIC_PLANE}
\pasteAttributeItem{PMIX_ALLOC_FABRIC_ENDPTS}
\pasteAttributeItem{PMIX_ALLOC_FABRIC_ENDPTS_NODE}
\pasteAttributeItem{PMIX_COSPAWN_APP}
\pasteAttributeItem{PMIX_SPAWN_TOOL}
\pasteAttributeItem{PMIX_EVENT_SILENT_TERMINATION}

\optattrend

%%%%
\descr

Nonblocking version of the \refapi{PMIx_Spawn} routine. The provided callback function will be executed upon successful start of \textit{all} specified application processes.

\adviceuserstart
Behavior of individual resource managers may differ, but it is expected that failure of any application process to start will result in termination/cleanup of all processes in the newly spawned job and return of an error code to the caller.
\adviceuserend

%%%%%%%%%%%%%%%%%%%%%%%%%%%%%%%%%%%%%%%%%%%%%%%%%
\subsection{Spawn-specific constants}
\label{api:struct:constants:spawn}

In addition to the generic error constants, the following spawn-specific error constants may be returned by the spawn \acp{API}:

\begin{constantdesc}
%
\declareconstitemNEW{PMIX_ERR_JOB_ALLOC_FAILED}
The job request could not be executed due to failure to obtain the specified allocation
%
\declareconstitemNEW{PMIX_ERR_JOB_APP_NOT_EXECUTABLE}
The specified application executable either could not be found, or lacks execution privileges.
%
\declareconstitemNEW{PMIX_ERR_JOB_NO_EXE_SPECIFIED}
The job request did not specify an executable.
%
\declareconstitemNEW{PMIX_ERR_JOB_FAILED_TO_MAP}
The launcher was unable to map the processes for the specified job request.
%
\declareconstitemNEW{PMIX_ERR_JOB_FAILED_TO_LAUNCH}
One or more processes in the job request failed to launch
%
\end{constantdesc}

%%%%%%%%%%%%%%%%%%%%%%%%%%%%%%%%%%%%%%%%%%%%%%%%%
\subsection{Spawn attributes}
\label{api:struct:attributes:spawn}

Attributes used to describe \refapi{PMIx_Spawn} behavior - they are values passed to the \refapi{PMIx_Spawn} \ac{API} and therefore are not accessed using the \refapi{PMIx_Get} \acp{API} when used in that context. However, some of the attributes defined in this section can be provided by the host environment for other purposes - e.g., the host might provide the \refattr{PMIX_MAPPER} attribute in the job-related information so that an application can use \refapi{PMIx_Get} to discover the layout algorithm used for determining process locations. Multi-use attributes and their respective access reference rank are denoted below.

%
\declareAttribute{PMIX_PERSONALITY}{"pmix.pers"}{char*}{
Name of personality corresponding to programming model used by application - supported values depend upon \ac{PMIx} implementation.
}
%
\declareAttribute{PMIX_HOST}{"pmix.host"}{char*}{
Comma-delimited list of hosts to use for spawned processes.
}
%
\declareAttribute{PMIX_HOSTFILE}{"pmix.hostfile"}{char*}{
Hostfile to use for spawned processes.
}
%
\declareAttribute{PMIX_ADD_HOST}{"pmix.addhost"}{char*}{
Comma-delimited list of hosts to add to the allocation.
}
%
\declareAttribute{PMIX_ADD_HOSTFILE}{"pmix.addhostfile"}{char*}{
Hostfile containing hosts to add to existing allocation.
}
%
\declareAttribute{PMIX_PREFIX}{"pmix.prefix"}{char*}{
Prefix to use for starting spawned processes - i.e., the directory where the executables can be found.
}
%
\declareAttribute{PMIX_WDIR}{"pmix.wdir"}{char*}{
Working directory for spawned processes.
}
%
\declareAttribute{PMIX_DISPLAY_MAP}{"pmix.dispmap"}{bool}{
Display process mapping upon spawn.
}
%
\declareAttribute{PMIX_PPR}{"pmix.ppr"}{char*}{
Number of processes to spawn on each identified resource.
}
%
\declareAttribute{PMIX_MAPBY}{"pmix.mapby"}{char*}{
Process mapping policy - when accessed using \refapi{PMIx_Get}, use the \refconst{PMIX_RANK_WILDCARD} value for the rank to discover the mapping policy used for the provided namespace. Supported values are launcher specific.
}
%
\declareAttribute{PMIX_RANKBY}{"pmix.rankby"}{char*}{
Process ranking policy - when accessed using \refapi{PMIx_Get}, use the \refconst{PMIX_RANK_WILDCARD} value for the rank to discover the ranking algorithm used for the provided namespace. Supported values are launcher specific.
}
%
\declareAttribute{PMIX_BINDTO}{"pmix.bindto"}{char*}{
Process binding policy - when accessed using \refapi{PMIx_Get}, use the \refconst{PMIX_RANK_WILDCARD} value for the rank to discover the binding policy used for the provided namespace. Supported values are launcher specific.
}
%
\declareAttribute{PMIX_PRELOAD_BIN}{"pmix.preloadbin"}{bool}{
Preload executables onto nodes prior to executing launch procedure.
}
%
\declareAttribute{PMIX_PRELOAD_FILES}{"pmix.preloadfiles"}{char*}{
Comma-delimited list of files to pre-position on nodes prior to executing launch procedure.
}
%
\declareAttribute{PMIX_STDIN_TGT}{"pmix.stdin"}{uint32_t}{
Spawned process rank that is to receive any forwarded \code{stdin}.
}
%
\declareAttribute{PMIX_SET_SESSION_CWD}{"pmix.ssncwd"}{bool}{
Set the current working directory to the session working directory assigned by the \ac{RM} - can be assigned to the entire job (by including attribute in the \refarg{job_info} array) or on a per-application basis in the \refarg{info} array for each \refstruct{pmix_app_t}.
}
%
\declareAttribute{PMIX_TAG_OUTPUT}{"pmix.tagout"}{bool}{
Tag \code{stdout}/\code{stderr} with the identity of the source process - can be assigned to the entire job (by including attribute in the \refarg{job_info} array) or on a per-application basis in the \refarg{info} array for each \refstruct{pmix_app_t}.
}
%
\declareAttribute{PMIX_TIMESTAMP_OUTPUT}{"pmix.tsout"}{bool}{
Timestamp output - can be assigned to the entire job (by including attribute in the \refarg{job_info} array) or on a per-application basis in the \refarg{info} array for each \refstruct{pmix_app_t}.
}
%
\declareAttribute{PMIX_MERGE_STDERR_STDOUT}{"pmix.mergeerrout"}{bool}{
Merge \code{stdout} and \code{stderr} streams - can be assigned to the entire job (by including attribute in the \refarg{job_info} array) or on a per-application basis in the \refarg{info} array for each \refstruct{pmix_app_t}.
}
%
\declareAttribute{PMIX_OUTPUT_TO_FILE}{"pmix.outfile"}{char*}{
Direct output (both stdout and stderr) into files of form \code{"<filename>.rank"} - can be assigned to the entire job (by including attribute in the \refarg{job_info} array) or on a per-application basis in the \refarg{info} array for each \refstruct{pmix_app_t}.
}
%
\declareAttributeNEW{PMIX_OUTPUT_TO_DIRECTORY}{"pmix.outdir"}{char*}{
Direct output into files of form \code{"<directory>/\allowbreak <jobid>/\allowbreak rank.<rank>/\allowbreak stdout[err]"} - can be assigned to the entire job (by including attribute in the \refarg{job_info} array) or on a per-application basis in the \refarg{info} array for each \refstruct{pmix_app_t}.
}
%
\declareAttribute{PMIX_INDEX_ARGV}{"pmix.indxargv"}{bool}{
Mark the \code{argv} with the rank of the process.
}
%
\declareAttribute{PMIX_CPUS_PER_PROC}{"pmix.cpuperproc"}{uint32_t}{
Number of \acp{PU} to assign to each rank - when accessed using \refapi{PMIx_Get}, use the \refconst{PMIX_RANK_WILDCARD} value for the rank to discover the \acp{PU}/process assigned to the provided namespace.
}
%
\declareAttribute{PMIX_NO_PROCS_ON_HEAD}{"pmix.nolocal"}{bool}{
Do not place processes on the head node.
}
%
\declareAttribute{PMIX_NO_OVERSUBSCRIBE}{"pmix.noover"}{bool}{
Do not oversubscribe the nodes - i.e., do not place more processes than allocated slots on a node.
}
%
\declareAttribute{PMIX_REPORT_BINDINGS}{"pmix.repbind"}{bool}{
Report bindings of the individual processes.
}
%
\declareAttribute{PMIX_CPU_LIST}{"pmix.cpulist"}{char*}{
List of \acp{PU} to use for this job - when accessed using \refapi{PMIx_Get}, use the \refconst{PMIX_RANK_WILDCARD} value for the rank to discover the \ac{PU} list used for the provided namespace.
}
%
\declareAttribute{PMIX_JOB_RECOVERABLE}{"pmix.recover"}{bool}{
Application supports recoverable operations.
}
%
\declareAttribute{PMIX_JOB_CONTINUOUS}{"pmix.continuous"}{bool}{
Application is continuous, all failed processes should be immediately restarted.
}
%
\declareAttribute{PMIX_MAX_RESTARTS}{"pmix.maxrestarts"}{uint32_t}{
Maximum number of times to restart a process - when accessed using \refapi{PMIx_Get}, use the \refconst{PMIX_RANK_WILDCARD} value for the rank to discover the max restarts for the provided namespace.
}
%
\declareAttribute{PMIX_SPAWN_TOOL}{"pmix.spwn.tool"}{bool}{
Indicate that the job being spawned is a tool.
}
%
\declareAttributeNEW{PMIX_TIMEOUT_STACKTRACES}{"pmix.tim.stack"}{bool}{
Include process stacktraces in timeout report from a job.
}
%
\declareAttributeNEW{PMIX_TIMEOUT_REPORT_STATE}{"pmix.tim.state"}{bool}{
Report process states in timeout report from a job.
}
%
\declareAttributeNEW{PMIX_NOTIFY_JOB_EVENTS}{"pmix.note.jev"}{bool}{
Requests that the launcher generate the
\refconst{PMIX_EVENT_JOB_START}, \refconst{PMIX_LAUNCH_COMPLETE}, and
\refconst{PMIX_EVENT_JOB_END} events. Each event is to include at least the
namespace of the corresponding job and a \refattr{PMIX_EVENT_TIMESTAMP}
indicating the time the event occurred. Note that the requester must register
for these individual events, or capture
and process them by registering a default event handler instead of individual
handlers and then process the events based on the returned status code.
Another common method is to register one event handler for all job-related
events, with a separate handler for non-job events - see
\refapi{PMIx_Register_event_handler} for details.
}
%
\declareAttribute{PMIX_NOTIFY_COMPLETION}{"pmix.notecomp"}{bool}{
Requests that the launcher generate the \refconst{PMIX_EVENT_JOB_END} event
for normal or abnormal termination of the spawned job. The event shall include
the returned status code (\refattr{PMIX_JOB_TERM_STATUS}) for the
corresponding job; the identity (\refattr{PMIX_PROCID}) and exit status
(\refattr{PMIX_EXIT_CODE}) of the first failed process, if applicable; and a
\refattr{PMIX_EVENT_TIMESTAMP} indicating the time the termination occurred.
Note that the requester must register for the event or capture and process it
within a default event handler.
}
%
\declareAttributeNEW{PMIX_NOTIFY_PROC_TERMINATION}{"pmix.noteproc"}{bool}{
Requests that the launcher generate the \refconst{PMIX_EVENT_PROC_TERMINATED}
event whenever a process either normally or abnormally terminates.
}
%
\declareAttributeNEW{PMIX_NOTIFY_PROC_ABNORMAL_TERMINATION}{"pmix.noteabproc"}{bool}{
Requests that the launcher generate the \refconst{PMIX_EVENT_PROC_TERMINATED}
event only when a process abnormally terminates.
}
%
\declareAttributeNEW{PMIX_LOG_PROC_TERMINATION}{"pmix.logproc"}{bool}{
Requests that the launcher log the \refconst{PMIX_EVENT_PROC_TERMINATED} event
whenever a process either normally or abnormally terminates.
}
%
\declareAttributeNEW{PMIX_LOG_PROC_ABNORMAL_TERMINATION}{"pmix.logabproc"}{bool}{
Requests that the launcher log the \refconst{PMIX_EVENT_PROC_TERMINATED} event
only when a process abnormally terminates.
}
%
\declareAttributeNEW{PMIX_LOG_JOB_EVENTS}{"pmix.log.jev"}{bool}{
Requests that the launcher log the \refconst{PMIX_EVENT_JOB_START},
\refconst{PMIX_LAUNCH_COMPLETE}, and \refconst{PMIX_EVENT_JOB_END} events using
\refapi{PMIx_Log}, subject to the logging attributes of Section
\ref{api:struct:attributes:log}.
}
%
\declareAttributeNEW{PMIX_LOG_COMPLETION}{"pmix.logcomp"}{bool}{
Requests that the launcher log the \refconst{PMIX_EVENT_JOB_END} event
for normal or abnormal termination of the spawned job using
\refapi{PMIx_Log}, subject to the logging attributes of Section
\ref{api:struct:attributes:log}. The event shall include
the returned status code (\refattr{PMIX_JOB_TERM_STATUS}) for the
corresponding job; the identity (\refattr{PMIX_PROCID}) and exit status
(\refattr{PMIX_EXIT_CODE}) of the first failed process, if applicable; and a
\refattr{PMIX_EVENT_TIMESTAMP} indicating the time the termination occurred.
}
%
\declareAttribute{PMIX_EVENT_SILENT_TERMINATION}{"pmix.evsilentterm"}{bool}{
Do not generate an event when this job normally terminates.
}

\vspace{\baselineskip}
Attributes used to adjust remote environment variables prior to spawning the specified application processes.

%
\declareAttribute{PMIX_SET_ENVAR}{"pmix.envar.set"}{pmix_envar_t*}{
Set the envar to the given value, overwriting any pre-existing one
}
%
\declareAttribute{PMIX_UNSET_ENVAR}{"pmix.envar.unset"}{char*}{
Unset the environment variable specified in the string.
}
%
\declareAttribute{PMIX_ADD_ENVAR}{"pmix.envar.add"}{pmix_envar_t*}{
Add the environment variable, but do not overwrite any pre-existing one
}
%
\declareAttribute{PMIX_PREPEND_ENVAR}{"pmix.envar.prepnd"}{pmix_envar_t*}{
Prepend the given value to the specified environmental value using the given separator character, creating the variable if it doesn't already exist
}
%
\declareAttribute{PMIX_APPEND_ENVAR}{"pmix.envar.appnd"}{pmix_envar_t*}{
Append the given value to the specified environmental value using the given separator character, creating the variable if it doesn't already exist
}
%
\declareAttributeNEW{PMIX_FIRST_ENVAR}{"pmix.envar.first"}{pmix_envar_t*}{
Ensure the given value appears first in the specified envar using the separator character, creating the envar if it doesn't already exist
}

%%%%%%%%%%%%%%%%%%%%%%%%%%%%%%%%%%%%%%%%%%%%%%%%%
\subsection{Application Structure}
\declarestruct{pmix_app_t}

The \refstruct{pmix_app_t} structure describes the application context for the \refapi{PMIx_Spawn} and \refapi{PMIx_Spawn_nb} operations.

\versionMarker{1.0}
\cspecificstart
\begin{codepar}
typedef struct pmix_app \{
    /** Executable */
    char *cmd;
    /** Argument set, NULL terminated */
    char **argv;
    /** Environment set, NULL terminated */
    char **env;
    /** Current working directory */
    char *cwd;
    /** Maximum processes with this profile */
    int maxprocs;
    /** Array of info keys describing this application*/
    pmix_info_t *info;
    /** Number of info keys in 'info' array */
    size_t ninfo;
\} pmix_app_t;
\end{codepar}
\cspecificend

%%%%%%%%%%%%%%%%%%%%%%%%%%%%%%%%%%%%%%%%%%%%%%%%%
\subsubsection{App structure support macros}
The following macros are provided to support the \refstruct{pmix_app_t} structure.

%%%%%%%%%%%
\littleheader{Initialize the app structure}
\declaremacro{PMIX_APP_CONSTRUCT}

Initialize the \refstruct{pmix_app_t} fields

\versionMarker{1.0}
\cspecificstart
\begin{codepar}
PMIX_APP_CONSTRUCT(m)
\end{codepar}
\cspecificend

\begin{arglist}
\argin{m}{Pointer to the structure to be initialized (pointer to \refstruct{pmix_app_t})}
\end{arglist}

%%%%%%%%%%%
\littleheader{Destruct the app structure}
\declaremacro{PMIX_APP_DESTRUCT}

Destruct the \refstruct{pmix_app_t} fields

\versionMarker{1.0}
\cspecificstart
\begin{codepar}
PMIX_APP_DESTRUCT(m)
\end{codepar}
\cspecificend

\begin{arglist}
\argin{m}{Pointer to the structure to be destructed (pointer to \refstruct{pmix_app_t})}
\end{arglist}

%%%%%%%%%%%
\littleheader{Create an app array}
\declaremacro{PMIX_APP_CREATE}

Allocate and initialize an array of \refstruct{pmix_app_t} structures

\versionMarker{1.0}
\cspecificstart
\begin{codepar}
PMIX_APP_CREATE(m, n)
\end{codepar}
\cspecificend

\begin{arglist}
\arginout{m}{Address where the pointer to the array of \refstruct{pmix_app_t} structures shall be stored (handle)}
\argin{n}{Number of structures to be allocated (\code{size_t})}
\end{arglist}

%%%%%%%%%%%
\littleheader{Free an app structure}
\declaremacro{PMIX_APP_RELEASE}

Release a \refstruct{pmix_app_t} structure

\versionMarker{4.0}
\cspecificstart
\begin{codepar}
PMIX_APP_RELEASE(m)
\end{codepar}
\cspecificend

\begin{arglist}
\argin{m}{Pointer to a \refstruct{pmix_app_t} structure (handle)}
\end{arglist}

%%%%%%%%%%%
\littleheader{Free an app array}
\declaremacro{PMIX_APP_FREE}

Release an array of \refstruct{pmix_app_t} structures

\versionMarker{1.0}
\cspecificstart
\begin{codepar}
PMIX_APP_FREE(m, n)
\end{codepar}
\cspecificend

\begin{arglist}
\argin{m}{Pointer to the array of \refstruct{pmix_app_t} structures (handle)}
\argin{n}{Number of structures in the array (\code{size_t})}
\end{arglist}

%%%%%%%%%%%
\littleheader{Create the info array of application directives}
\declaremacro{PMIX_APP_INFO_CREATE}

Create an array of \refstruct{pmix_info_t} structures for passing application-level directives, updating the \refarg{ninfo} field of the \refstruct{pmix_app_t} structure.

\versionMarker{2.2}
\cspecificstart
\begin{codepar}
PMIX_APP_INFO_CREATE(m, n)
\end{codepar}
\cspecificend

\begin{arglist}
\argin{m}{Pointer to the \refstruct{pmix_app_t} structure (handle)}
\argin{n}{Number of directives to be allocated (\code{size_t})}
\end{arglist}


%%%%%%%%%%%%%%%%%%%%%%%%%%%%%%%%%%%%%%%%%%%%%%%%%
\subsubsection{Spawn Callback Function}
\declareapi{pmix_spawn_cbfunc_t}

%%%%
\summary

The \refapi{pmix_spawn_cbfunc_t} is used on the PMIx client side by \refapi{PMIx_Spawn_nb} and on the PMIx server side by \refapi{pmix_server_spawn_fn_t}.

\versionMarker{1.0}
\cspecificstart
\begin{codepar}
typedef void (*pmix_spawn_cbfunc_t)
    (pmix_status_t status,
     pmix_nspace_t nspace, void *cbdata);
\end{codepar}
\cspecificend

\begin{arglist}
\argin{status}{Status associated with the operation (handle)}
\argin{nspace}{Namespace string (\refstruct{pmix_nspace_t})}
\argin{cbdata}{Callback data passed to original API call (memory reference)}
\end{arglist}


%%%%
\descr

The callback will be executed upon launch of the specified applications in \refapi{PMIx_Spawn_nb}, or upon failure to launch any of them.

The \refarg{status} of the callback will indicate whether or not the spawn succeeded.
The \refarg{nspace} of the spawned processes will be returned, along with any provided callback data.
Note that the returned \refarg{nspace} value will not be protected upon return from the callback function, so the receiver must copy it if it needs to be retained.


%%%%%%%%%%%%%%%%%%%%%%%%%%%%%%%%%%%%%%%%%%%%%%%%%
%%%%%%%%%%%%%%%%%%%%%%%%%%%%%%%%%%%%%%%%%%%%%%%%%
\section{Connecting and Disconnecting Processes}
\label{chap:api_proc_mgmt:connect}

This section defines functions to connect and disconnect processes in two or more separate \ac{PMIx} namespaces. The \ac{PMIx} definition of \textit{connected} solely implies that the host environment should treat the failure of any process in the assemblage as a reportable event, taking action on the assemblage as if it were a single application. For example, if the environment defaults (in the absence of any application directives) to terminating an application upon failure of any process in that application, then the environment should terminate all processes in the connected assemblage upon failure of any member.

The host environment may choose to assign a new namespace to the connected assemblage and/or assign new ranks for its members for its own internal tracking purposes. However, it is not required to communicate such assignments to the participants (e.g., in response to an appropriate call to \refapi{PMIx_Query_info_nb}). The host environment is required to generate a \refconst{PMIX_ERR_PROC_TERM_WO_SYNC} event should any process in the assemblage terminate or call \refapi{PMIx_Finalize} without first \textit{disconnecting} from the assemblage. If the job including the process is terminated as a result of that action, then the host environment is required to also generate the \refconst{PMIX_ERR_JOB_TERM_WO_SYNC} for all jobs that were terminated as a result.

\advicermstart
The \textit{connect} operation does not require the exchange of job-level information nor the inclusion of information posted by  participating processes via \refapi{PMIx_Put}. Indeed, the callback function utilized in \refapi{pmix_server_connect_fn_t} cannot pass information back into the \ac{PMIx} server library. However, host environments are advised that collecting such information at the participating daemons represents an optimization opportunity as participating processes are likely to request such information after the connect operation completes.
\advicermend

\adviceuserstart
Attempting to \textit{connect} processes solely within the same namespace is essentially a \textit{no-op} operation. While not explicitly prohibited, users are advised that a \ac{PMIx} implementation or host environment may return an error in such cases.

Neither the \ac{PMIx} implementation nor host environment are required to provide any tracking support for the assemblage. Thus, the application is responsible for maintaining the membership list of the assemblage.
\adviceuserend


%%%%%%%%%%%%%%%%%%%%%%%%%%%%%%%%%%%%%%%%%%%%%%%%%
\subsection{\code{PMIx_Connect}}
\declareapi{PMIx_Connect}

%%%%
\summary

Connect namespaces.

%%%%
\format

\versionMarker{1.0}
\cspecificstart
\begin{codepar}
pmix_status_t
PMIx_Connect(const pmix_proc_t procs[], size_t nprocs,
             const pmix_info_t info[], size_t ninfo)
\end{codepar}
\cspecificend

\begin{arglist}
\argin{procs}{Array of proc structures (array of handles)}
\argin{nprocs}{Number of elements in the \refarg{procs} array (integer)}
\argin{info}{Array of info structures (array of handles)}
\argin{ninfo}{Number of elements in the \refarg{info} array (integer)}
\end{arglist}

Returns \refconst{PMIX_SUCCESS} or a negative value corresponding to a PMIx error constant.

\reqattrstart
\ac{PMIx} libraries are not required to directly support any attributes for this function. However, any provided attributes must be passed to the host \ac{SMS} daemon for processing.

\reqattrend

\optattrstart
The following attributes are optional for \ac{PMIx} implementations:

\pasteAttributeItem{PMIX_ALL_CLONES_PARTICIPATE}


The following attributes are optional for host environments that support this operation:

\pasteAttributeItem{PMIX_TIMEOUT}

\optattrend

%%%%
\descr

Record the processes specified by the \refarg{procs} array as \textit{connected} as per the \ac{PMIx} definition. The function will return once all processes identified in \refarg{procs} have called either \refapi{PMIx_Connect} or its non-blocking version, \textit{and} the host environment has completed any supporting operations required to meet the terms of the \ac{PMIx} definition of \textit{connected} processes.

A process can only engage in one connect operation involving the identical \refarg{procs} array at a time.
However, a process can be simultaneously engaged in multiple connect operations, each involving a different \refarg{procs} array.

As in the case of the \refapi{PMIx_Fence} operation, the \refarg{info} array can be used to pass user-level directives regarding timeout constraints and other options available from the host \ac{RM}.

\adviceuserstart
All processes engaged in a given \refapi{PMIx_Connect} operation must provide the identical \refarg{procs} array as ordering of entries in the array and the method by which those processes are identified (e.g., use of \refconst{PMIX_RANK_WILDCARD} versus listing the individual processes) \textit{may} impact the host environment's algorithm for uniquely identifying an operation.
\adviceuserend

\adviceimplstart
\refapi{PMIx_Connect} and its non-blocking form are both \emph{collective} operations. Accordingly, the \ac{PMIx} server library is required to aggregate participation by local clients, passing the request to the host environment once all local participants have executed the \ac{API}.
\adviceimplend

\advicermstart
The host will receive a single call for each collective operation. It is the responsibility of the host to identify the nodes containing participating processes, execute the collective across all participating nodes, and notify the local \ac{PMIx} server library upon completion of the global collective.
\advicermend


%%%%%%%%%%%%%%%%%%%%%%%%%%%%%%%%%%%%%%%%%%%%%%%%%
\subsection{\code{PMIx_Connect_nb}}
\declareapi{PMIx_Connect_nb}

%%%%
\summary

Nonblocking \refapi{PMIx_Connect_nb} routine.

%%%%
\format

\versionMarker{1.0}
\cspecificstart
\begin{codepar}
pmix_status_t
PMIx_Connect_nb(const pmix_proc_t procs[], size_t nprocs,
                const pmix_info_t info[], size_t ninfo,
                pmix_op_cbfunc_t cbfunc, void *cbdata)
\end{codepar}
\cspecificend

\begin{arglist}
\argin{procs}{Array of proc structures (array of handles)}
\argin{nprocs}{Number of elements in the \refarg{procs} array (integer)}
\argin{info}{Array of info structures (array of handles)}
\argin{ninfo}{Number of elements in the \refarg{info} array (integer)}
\argin{cbfunc}{Callback function \refapi{pmix_op_cbfunc_t} (function reference)}
\argin{cbdata}{Data to be passed to the callback function (memory reference)}
\end{arglist}

Returns one of the following:

\begin{itemize}
    \item \refconst{PMIX_SUCCESS}, indicating that the request is being processed by the host environment - result will be returned in the provided \refarg{cbfunc}. Note that the library must not invoke the callback function prior to returning from the \ac{API}.
    \item \refconst{PMIX_OPERATION_SUCCEEDED}, indicating that the request was immediately processed and returned \textit{success} - the \refarg{cbfunc} will \textit{not} be called
    \item a PMIx error constant indicating either an error in the input or that the request was immediately processed and failed - the \refarg{cbfunc} will \textit{not} be called
\end{itemize}


\reqattrstart
\ac{PMIx} libraries are not required to directly support any attributes for this function. However, any provided attributes must be passed to the host \ac{SMS} daemon for processing.

\reqattrend

\optattrstart
The following attributes are optional for \ac{PMIx} implementations:

\pasteAttributeItem{PMIX_ALL_CLONES_PARTICIPATE}


The following attributes are optional for host environments that support this operation:

\pasteAttributeItem{PMIX_TIMEOUT}

\optattrend

%%%%
\descr

Nonblocking version of \refapi{PMIx_Connect}. The callback function is called once all processes identified in \refarg{procs} have called either \refapi{PMIx_Connect} or its non-blocking version, \textit{and} the host environment has completed any supporting operations required to meet the terms of the \ac{PMIx} definition of \textit{connected} processes. See the advice provided in the description for \refapi{PMIx_Connect} for more information.


%%%%%%%%%%%%%%%%%%%%%%%%%%%%%%%%%%%%%%%%%%%%%%%%%
\subsection{\code{PMIx_Disconnect}}
\declareapi{PMIx_Disconnect}

%%%%
\summary

Disconnect a previously connected set of processes.

%%%%
\format

\versionMarker{1.0}
\cspecificstart
\begin{codepar}
pmix_status_t
PMIx_Disconnect(const pmix_proc_t procs[], size_t nprocs,
                const pmix_info_t info[], size_t ninfo);
\end{codepar}
\cspecificend

\begin{arglist}
\argin{procs}{Array of proc structures (array of handles)}
\argin{nprocs}{Number of elements in the \refarg{procs} array (integer)}
\argin{info}{Array of info structures (array of handles)}
\argin{ninfo}{Number of elements in the \refarg{info} array (integer)}
\end{arglist}

Returns one of the following:

\begin{itemize}
    \item \refconst{PMIX_SUCCESS}, indicating that the request was successfully executed
    \item the \refconst{PMIX_ERR_INVALID_OPERATION} error indicating that the specified set of \refarg{procs} was not previously \textit{connected} via a call to \refapi{PMIx_Connect} or its non-blocking form.
    \item a PMIx error constant indicating either an error in the input or that the request failed
\end{itemize}


\reqattrstart
\ac{PMIx} libraries are not required to directly support any attributes for this function. However, any provided attributes must be passed to the host \ac{SMS} daemon for processing.

\reqattrend

\optattrstart
The following attributes are optional for \ac{PMIx} implementations:

\pasteAttributeItem{PMIX_ALL_CLONES_PARTICIPATE}


The following attributes are optional for host environments that support this operation:

\pasteAttributeItem{PMIX_TIMEOUT}

\optattrend

%%%%
\descr

Disconnect a previously connected set of processes. The function will return once all processes identified in \refarg{procs} have called either \refapi{PMIx_Disconnect} or its non-blocking version, \textit{and} the host environment has completed any required supporting operations.

A process can only engage in one disconnect operation involving the identical \refarg{procs} array at a time.
However, a process can be simultaneously engaged in multiple disconnect operations, each involving a different \refarg{procs} array.

As in the case of the \refapi{PMIx_Fence} operation, the \refarg{info} array can be used to pass user-level directives regarding the algorithm to be used for any collective operation involved in the operation, timeout constraints, and other options available from the host \ac{RM}.

\adviceuserstart
All processes engaged in a given \refapi{PMIx_Disconnect} operation must provide the identical \refarg{procs} array as ordering of entries in the array and the method by which those processes are identified (e.g., use of \refconst{PMIX_RANK_WILDCARD} versus listing the individual processes) \textit{may} impact the host environment's algorithm for uniquely identifying an operation.
\adviceuserend

\adviceimplstart
\refapi{PMIx_Disconnect} and its non-blocking form are both \emph{collective} operations. Accordingly, the \ac{PMIx} server library is required to aggregate participation by local clients, passing the request to the host environment once all local participants have executed the \ac{API}.
\adviceimplend

\advicermstart
The host will receive a single call for each collective operation. The host will receive a single call for each collective operation. It is the responsibility of the host to identify the nodes containing participating processes, execute the collective across all participating nodes, and notify the local \ac{PMIx} server library upon completion of the global collective.

\advicermend


%%%%%%%%%%%%%%%%%%%%%%%%%%%%%%%%%%%%%%%%%%%%%%%%%
\subsection{\code{PMIx_Disconnect_nb}}
\declareapi{PMIx_Disconnect_nb}

%%%%
\summary

Nonblocking \refapi{PMIx_Disconnect} routine.

%%%%
\format

\versionMarker{1.0}
\cspecificstart
\begin{codepar}
pmix_status_t
PMIx_Disconnect_nb(const pmix_proc_t procs[], size_t nprocs,
                   const pmix_info_t info[], size_t ninfo,
                   pmix_op_cbfunc_t cbfunc, void *cbdata);
\end{codepar}
\cspecificend

\begin{arglist}
\argin{procs}{Array of proc structures (array of handles)}
\argin{nprocs}{Number of elements in the \refarg{procs} array (integer)}
\argin{info}{Array of info structures (array of handles)}
\argin{ninfo}{Number of elements in the \refarg{info} array (integer)}
\argin{cbfunc}{Callback function \refapi{pmix_op_cbfunc_t} (function reference)}
\argin{cbdata}{Data to be passed to the callback function (memory reference)}
\end{arglist}

Returns one of the following:

\begin{itemize}
    \item \refconst{PMIX_SUCCESS}, indicating that the request is being processed by the host environment - result will be returned in the provided \refarg{cbfunc}. Note that the library must not invoke the callback function prior to returning from the \ac{API}.
    \item \refconst{PMIX_OPERATION_SUCCEEDED}, indicating that the request was immediately processed and returned \textit{success} - the \refarg{cbfunc} will \textit{not} be called
    \item a PMIx error constant indicating either an error in the input or that the request was immediately processed and failed - the \refarg{cbfunc} will \textit{not} be called
\end{itemize}

\reqattrstart
\ac{PMIx} libraries are not required to directly support any attributes for this function. However, any provided attributes must be passed to the host \ac{SMS} daemon for processing.

\reqattrend

\optattrstart
The following attributes are optional for \ac{PMIx} implementations:

\pasteAttributeItem{PMIX_ALL_CLONES_PARTICIPATE}


The following attributes are optional for host environments that support this operation:

\pasteAttributeItem{PMIX_TIMEOUT}

\optattrend

%%%%
\descr

Nonblocking \refapi{PMIx_Disconnect} routine. The callback function is called either:

\begin{itemize}
    \item to return the \refconst{PMIX_ERR_INVALID_OPERATION} error indicating that the specified set of \refarg{procs} was not previously \textit{connected} via a call to \refapi{PMIx_Connect} or its non-blocking form;

    \item to return a \ac{PMIx} error constant indicating that the operation failed; or

    \item once all processes identified in \refarg{procs} have called either \refapi{PMIx_Disconnect_nb} or its blocking version, \textit{and} the host environment has completed any required supporting operations.
\end{itemize}

See the advice provided in the description for \refapi{PMIx_Disconnect} for more information.


%%%%%%%%%%%%%%%%%%%%%%%%%%%%%%%%%%%%%%%%%%%%%%%%%
%%%%%%%%%%%%%%%%%%%%%%%%%%%%%%%%%%%%%%%%%%%%%%%%%
\section{Process Locality}
\label{chap:api_proc_mgmt:locality}

The relative locality of processes is often used to optimize their interactions with the hardware and other processes. \ac{PMIx} provides a means by which the host environment can communicate the locality of a given process using the \refapi{PMIx_server_generate_locality_string} to generate an abstracted representation of that value. This provides a human-readable format and allows the client to parse the locality string with a method of its choice that may differ from the one used by the server that generated it.

There are times, however, when relative locality and other \ac{PMIx}-provided
information doesn't include some element required by the application. In these
instances, the application may need access to the full description of the
local hardware topology. \ac{PMIx} does not itself generate such descriptions
- there are multiple third-party libraries that fulfill that role. Instead,
\ac{PMIx} offers an abstraction method by which users can obtain a pointer to
the description. This transparently enables support for different methods of
sharing the topology between the host environment (which may well have already
generated it prior to local start of application processes) and the clients -
e.g., through passing of a shared memory region.

%%%%%%%%%%%%%%%%%%%%%%%%%%%%%%%%%%%%%%%%%%%%%%%%%
\subsection{\code{PMIx_Load_topology}}
\declareapi{PMIx_Load_topology}

%%%%
\summary

Load the local hardware topology description

%%%%
\format

\versionMarker{4.0}
\cspecificstart
\begin{codepar}
pmix_status_t
PMIx_Load_topology(pmix_topology_t *topo);
\end{codepar}
\cspecificend

\begin{arglist}
\arginout{topo}{Address of a \refstruct{pmix_topology_t} structure where the topology information is to be loaded (handle)}
\end{arglist}

Returns \refconst{PMIX_SUCCESS}, indicating that the \refarg{topo} was successfully loaded, or an appropriate \ac{PMIx} error constant.

%%%%
\descr

Obtain a pointer to the topology description of the local node. If the
\refarg{source} field of the provided \refstruct{pmix_topology_t} is set, then
the \ac{PMIx} library must return a description from the specified
implementation or else indicate that the implementation is not available by
returning the \refconst{PMIX_ERR_NOT_SUPPORTED} error constant.

The returned pointer may point to a shared memory region or an actual instance
of the topology description. In either case, the description shall be treated
as a "read-only" object - attempts to modify the object are likely to fail and
return an error. The \ac{PMIx} library is responsible for performing any required cleanup when the client library finalizes.

\adviceuserstart
It is the responsibility of the user to ensure that the \refarg{topo} argument
is properly initialized prior to calling this \ac{API}, and to check the
returned \refarg{source} to verify that the returned topology description is
compatible with the user's code.
\adviceuserend

%%%%%%%%%%%%%%%%%%%%%%%%%%%%%%%%%%%%%%%%%%%%%%%%%
\subsection{\code{PMIx_Get_relative_locality}}
\declareapi{PMIx_Get_relative_locality}

%%%%
\summary

Get the relative locality of two local processes given their locality strings.

%%%%
\format

\versionMarker{4.0}
\cspecificstart
\begin{codepar}
pmix_status_t
PMIx_Get_relative_locality(const char *locality1,
                           const char *locality2,
                           pmix_locality_t *locality);
\end{codepar}
\cspecificend

\begin{arglist}
\argin{locality1}{String returned by the \refapi{PMIx_server_generate_locality_string} \ac{API} (handle)}
\argin{locality2}{String returned by the \refapi{PMIx_server_generate_locality_string} \ac{API} (handle)}
\arginout{locality}{Location where the relative locality bitmask is to be constructed (memory reference)}
\end{arglist}

Returns \refconst{PMIX_SUCCESS}, indicating that the \refarg{locality} was successfully loaded, or an appropriate \ac{PMIx} error constant.

%%%%
\descr

Parse the locality strings of two processes (as returned by \refapi{PMIx_Get} using the \refattr{PMIX_LOCALITY_STRING} key) and set the appropriate \refstruct{pmix_locality_t} locality bits in the provided memory location.

%%%%%%%%%%%%%%%%%%%%%%%%%%%%%%%%%%%%%%%%%%%%%%%%%
\subsubsection{Topology description}
\declarestruct{pmix_topology_t}

The \refstruct{pmix_topology_t} structure contains a (case-insensitive)
string identifying the source of the topology (e.g., "hwloc") and a pointer
to the corresponding implementation-specific topology description.

\versionMarker{4.0}
\cspecificstart
\begin{codepar}
typedef struct pmix_topology \{
    char *source;
    void *topology;
\} pmix_topoology_t;
\end{codepar}
\cspecificend

%%%%%%%%%%%%%%%%%%%%%%%%%%%%%%%%%%%%%%%%%%%%%%%%%
\subsubsection{Topology support macros}

The following macros support the \refstruct{pmix_topology_t} structure.

\littleheader{Initialize the topology structure}
\declaremacro{PMIX_TOPOLOGY_CONSTRUCT}

Initialize the \refstruct{pmix_topology_t} fields to \code{NULL}

\versionMarker{4.0}
\cspecificstart
\begin{codepar}
PMIX_TOPOLOGY_CONSTRUCT(m)
\end{codepar}
\cspecificend

\begin{arglist}
\argin{m}{Pointer to the structure to be initialized (pointer to \refstruct{pmix_topology_t})}
\end{arglist}


\littleheader{Destruct the topology structure}
\declaremacro{PMIX_TOPOLOGY_DESTRUCT}

Destruct the \refstruct{pmix_topology_t} fields

\versionMarker{4.0}
\cspecificstart
\begin{codepar}
PMIX_TOPOLOGY_DESTRUCT(m)
\end{codepar}
\cspecificend

\begin{arglist}
\argin{m}{Pointer to the structure to be destructed (pointer to \refstruct{pmix_topology_t})}
\end{arglist}


\littleheader{Create a topology array}
\declaremacro{PMIX_TOPOLOGY_CREATE}

Allocate and initialize a \refstruct{pmix_topology_t} array.

\versionMarker{4.0}
\cspecificstart
\begin{codepar}
PMIX_TOPOLOGY_CREATE(m, n)
\end{codepar}
\cspecificend

\begin{arglist}
\arginout{m}{Address where the pointer to the array of \refstruct{pmix_topology_t} structures shall be stored (handle)}
\argin{n}{Number of structures to be allocated (size_t)}
\end{arglist}


\littleheader{Release a topology array}
\declaremacro{PMIX_TOPOLOGY_FREE}

Release a \refstruct{pmix_topology_t} array.

\versionMarker{4.0}
\cspecificstart
\begin{codepar}
PMIX_TOPOLOGY_FREE(m, n)
\end{codepar}
\cspecificend

\begin{arglist}
\arginout{m}{Address of the array of \refstruct{pmix_topology_t} structures to be released (handle)}
\argin{n}{Number of structures in the array (size_t)}
\end{arglist}


%%%%%%%%%%%%%%%%%%%%%%%%%%%%%%%%%%%%%%%%%%%%%%%%%
\subsubsection{Relative locality of two processes}
\declarestruct{pmix_locality_t}
\label{api:proc:locality}

\versionMarker{4.0}
The \refstruct{pmix_locality_t} datatype is a \code{uint16_t} bitmask that
defines the relative locality of two processes on a node. The following
constants represent specific bits in the mask and can be used to test a
locality value using standard bit-test methods.

\begin{constantdesc}
%
\declareconstitemNEW{PMIX_LOCALITY_UNKNOWN}
All bits are set to zero, indicating that the relative locality of the two processes is unknown
%
\declareconstitemNEW{PMIX_LOCALITY_NONLOCAL}
The two processes do not share any common locations
%
\declareconstitemNEW{PMIX_LOCALITY_SHARE_HWTHREAD}
The two processes share at least one hardware thread
%
\declareconstitemNEW{PMIX_LOCALITY_SHARE_CORE}
The two processes share at least one core
%
\declareconstitemNEW{PMIX_LOCALITY_SHARE_L1CACHE}
The two processes share at least an L1 cache
%
\declareconstitemNEW{PMIX_LOCALITY_SHARE_L2CACHE}
The two processes share at least an L2 cache
%
\declareconstitemNEW{PMIX_LOCALITY_SHARE_L3CACHE}
The two processes share at least an L3 cache
%
\declareconstitemNEW{PMIX_LOCALITY_SHARE_PACKAGE}
The two processes share at least a package
%
\declareconstitemNEW{PMIX_LOCALITY_SHARE_NUMA}
The two processes share at least one \ac{NUMA} region
%
\declareconstitemNEW{PMIX_LOCALITY_SHARE_NODE}
The two processes are executing on the same node
%
\end{constantdesc}


Implementers and vendors may choose to extend these definitions as needed to describe a particular system.


%%%%%%%%%%%%%%%%%%%%%%%%%%%%%%%%%%%%%%%%%%%%%%%%%
\subsubsection{Locality keys}

%
\declareAttribute{PMIX_LOCALITY_STRING}{"pmix.locstr"}{char*}{
String describing a process's bound location - referenced using the process's
rank. The string is prefixed by the implementation that created it (e.g.,
"hwloc") followed by a colon. The remainder of the string represents the
corresponding locality as expressed by the underlying implementation. The
entire string must be passed to \refapi{PMIx_Get_relative_locality} for
processing. Note that hosts are only required to provide locality strings for
local client processes - thus, a call to \refapi{PMIx_Get} for the locality
string of a process that returns \refconst{PMIX_ERR_NOT_FOUND} indicates that
the process is not executing on the same node.
}

%%%%%%%%%%%%%%%%%%%%%%%%%%%%%%%%%%%%%%%%%%%%%%%%%
\subsection{\code{PMIx_Parse_cpuset_string}}
\declareapi{PMIx_Parse_cpuset_string}

%%%%
\summary

Parse the \ac{PU} binding bitmap from its string representation.

%%%%
\format

\versionMarker{4.0}
\cspecificstart
\begin{codepar}
pmix_status_t
PMIx_Parse_cpuset_string(const char *cpuset_string,
                         pmix_cpuset_t *cpuset);
\end{codepar}
\cspecificend

\begin{arglist}
\argin{cpuset_string}{String returned by the \refapi{PMIx_server_generate_cpuset_string} \ac{API} (handle)}
\arginout{cpuset}{Address of an object where the bitmap is to be stored (memory reference)}
\end{arglist}

Returns \refconst{PMIX_SUCCESS}, indicating that the \refarg{cpuset} was successfully loaded, or an appropriate \ac{PMIx} error constant.

%%%%
\descr

Parse the string representation of the binding bitmap (as returned by \refapi{PMIx_Get} using the \refattr{PMIX_CPUSET} key) and set the appropriate \ac{PU} binding location information in the provided memory location.

%%%%%%%%%%%%%%%%%%%%%%%%%%%%%%%%%%%%%%%%%%%%%%%%%
\subsection{\code{PMIx_Get_cpuset}}
\declareapi{PMIx_Get_cpuset}

%%%%
\summary

Get the \ac{PU} binding bitmap of the current process.

%%%%
\format

\versionMarker{4.0}
\cspecificstart
\begin{codepar}
pmix_status_t
PMIx_Get_cpuset(pmix_cpuset_t *cpuset, pmix_bind_envelope_t ref);
\end{codepar}
\cspecificend

\begin{arglist}
\arginout{cpuset}{Address of an object where the bitmap is to be stored (memory reference)}
\argin{ref}{The binding envelope to be considered when formulating the bitmap (\refstruct{pmix_bind_envelope_t})}
\end{arglist}

Returns \refconst{PMIX_SUCCESS}, indicating that the \refarg{cpuset} was successfully loaded, or an appropriate \ac{PMIx} error constant.

%%%%
\descr

Obtain and set the appropriate \ac{PU} binding location information in the provided memory location based on the specified binding envelope.

%%%%%%%%%%%%%%%%%%%%%%%%%%%%%%%%%%%%%%%%%%%%%%%%%
\subsubsection{Binding envelope}
\declarestruct{pmix_bind_envelope_t}
\label{api:proc:bindenv}

\versionMarker{4.0}
The \refstruct{pmix_bind_envelope_t} data type
defines the envelope of threads within a possibly multi-threaded process that are to be considered when getting the cpuset associated with the process. Valid values include:

\begin{constantdesc}
%
\declareconstitemNEW{PMIX_CPUBIND_PROCESS}
Use the location of all threads in the possibly multi-threaded process.
%
\declareconstitemNEW{PMIX_CPUBIND_THREAD}
Use only the location of the thread calling the \ac{API}.
%
\end{constantdesc}


%%%%%%%%%%%%%%%%%%%%%%%%%%%%%%%%%%%%%%%%%%%%%%%%%
\subsection{\code{PMIx_Compute_distances}}
\declareapi{PMIx_Compute_distances}

%%%%
\summary

Compute distances from specified process location to local devices.

%%%%
\format

\versionMarker{4.0}
\cspecificstart
\begin{codepar}
pmix_status_t
PMIx_Compute_distances(pmix_topology_t *topo,
                       pmix_cpuset_t *cpuset,
                       pmix_info_t info[], size_t ninfo[],
                       pmix_device_distance_t *distances[],
                       size_t *ndist);
\end{codepar}
\cspecificend

\begin{arglist}
\argin{topo}{Pointer to the topology description of the node where the process is located (\code{NULL} indicates the local node) (\refstruct{pmix_topology_t})}
\argin{cpuset}{Pointer to the location of the process (\refstruct{pmix_cpuset_t})}
\argin{info}{Array of \refstruct{pmix_info_t} describing the devices whose distance is to be computed (handle)}
\argin{ninfo}{Number of elements in \refarg{info} (integer)}
\arginout{distances}{Pointer to an address where the array of \refstruct{pmix_device_distance_t} structures containing the distances from the caller to the specified devices is to be returned (handle)}
\arginout{ndist}{Pointer to an address where the number of elements in the \refarg{distances} array is to be returned (handle)}
\end{arglist}

Returns one of the following:

\begin{itemize}
\item \refconst{PMIX_SUCCESS} indicating that the distances were returned.
\item a non-zero \ac{PMIx} error constant indicating the reason the request failed.
\end{itemize}


%%%%
\descr

Both the minimum and maximum distance fields in the elements of the array shall be filled with the respective distances between the current process location and the types of devices or specific device identified in the \refarg{info} directives. In the absence of directives, distances to all supported device types shall be returned.

\adviceuserstart
A process whose threads are not all bound to the same location may return inconsistent results from calls to this \ac{API} by different threads if the \refconst{PMIX_CPUBIND_THREAD} binding envelope was used when generating the \refarg{cpuset}.
\adviceuserend

%%%%%%%%%%%%%%%%%%%%%%%%%%%%%%%%%%%%%%%%%%%%%%%%%
\subsection{\code{PMIx_Compute_distances_nb}}
\declareapi{PMIx_Compute_distances_nb}

%%%%
\summary

Compute distances from specified process location to local devices.

%%%%
\format

\versionMarker{4.0}
\cspecificstart
\begin{codepar}
pmix_status_t
PMIx_Compute_distances_nb(pmix_topology_t *topo,
                          pmix_cpuset_t *cpuset,
                          pmix_info_t info[], size_t ninfo[],
                          pmix_device_dist_cbfunc_t cbfunc,
                          void *cbdata);
\end{codepar}
\cspecificend

\begin{arglist}
\argin{topo}{Pointer to the topology description of the node where the process is located (\code{NULL} indicates the local node) (\refstruct{pmix_topology_t})}
\argin{cpuset}{Pointer to the location of the process (\refstruct{pmix_cpuset_t})}
\argin{info}{Array of \refstruct{pmix_info_t} describing the devices whose distance is to be computed (handle)}
\argin{ninfo}{Number of elements in \refarg{info} (integer)}
\argin{cbfunc}{Callback function \refapi{pmix_info_cbfunc_t} (function reference)}
\argin{cbdata}{Data to be passed to the callback function (memory reference)}
\end{arglist}

Returns one of the following:

\begin{itemize}
\item \refconst{PMIX_SUCCESS} indicating that the request has been accepted for processing and the provided callback function will be executed upon completion of the operation. Note that the library must not invoke the callback function prior to returning from the \ac{API}.
\item a non-zero \ac{PMIx} error constant indicating a reason for the request to have been rejected. In this case, the provided callback function will not be executed
\end{itemize}


%%%%
\descr

Non-blocking form of the \refapi{PMIx_Compute_distances} \ac{API}.

%%%%%%%%%%%%%%%%%%%%%%%%%%%%%%%%%%%%%%%%%%%%%%%%%
\subsection{Device Distance Callback Function}
\declareapi{pmix_device_dist_cbfunc_t}

%%%%
\summary

The \refapi{pmix_device_dist_cbfunc_t} is used to return an array of device distances.

\versionMarker{4.0}
\cspecificstart
\begin{codepar}
typedef void (*pmix_device_dist_cbfunc_t)
    (pmix_status_t status,
     pmix_device_distance_t *dist,
     size_t ndist,
     void *cbdata,
     pmix_release_cbfunc_t release_fn,
     void *release_cbdata);
\end{codepar}
\cspecificend

\begin{arglist}
\argin{status}{Status associated with the operation (\refstruct{pmix_status_t})}
\argin{dist}{Array of \refstruct{pmix_device_distance_t} returned by the operation (pointer)}
\argin{ndist}{Number of elements in the \argref{dist} array (\code{size_t})}
\argin{cbdata}{Callback data passed to original \ac{API} call (memory reference)}
\argin{release_fn}{Function to be called when done with the \argref{dist} data (function pointer)}
\argin{release_cbdata}{Callback data to be passed to \argref{release_fn} (memory reference)}
\end{arglist}


%%%%
\descr

The \refarg{status} indicates if requested data was found or not.
The array of \refstruct{pmix_device_distance_t} will contain the distance information.

%%%%%%%%%%%%%%%%%%%%%%%%%%%%%%%%%%%%%%%%%%%%%%%%%
\subsection{Device type}
\declarestruct{pmix_device_type_t}
\label{api:proc:devtype}

The \refstruct{pmix_device_type_t} is a \code{uint64_t} bitmask for identifying the type(s) whose distances are being requested, or the type of a specific device being referenced (e.g., in a \refstruct{pmix_device_distance_t} object).

\versionMarker{1.0}
\cspecificstart
\begin{codepar}
typedef uint16_t pmix_device_type_t;
\end{codepar}
\cspecificend

The following constants can be used to set a variable of the type \refstruct{pmix_device_type_t}.

\begin{constantdesc}
%
\declareconstitemNEW{PMIX_DEVTYPE_UNKNOWN}
The device is of an unknown type - will not be included in returned device distances.
%
\declareconstitemNEW{PMIX_DEVTYPE_BLOCK}
Operating system block device, or non-volatile memory device (e.g., "sda" or "dax2.0" on Linux).
%
\declareconstitemNEW{PMIX_DEVTYPE_GPU}
Operating system \ac{GPU} device (e.g., "card0" for a Linux \ac{DRM} device).
%
\declareconstitemNEW{PMIX_DEVTYPE_NETWORK}
Operating system network device (e.g., the "eth0" interface on Linux).
%
\declareconstitemNEW{PMIX_DEVTYPE_OPENFABRICS}
Operating system OpenFabrics device (e.g., an "mlx4_0" InfiniBand \ac{HCA}, or "hfi1_0" Omni-Path interface on Linux).
%
\declareconstitemNEW{PMIX_DEVTYPE_DMA}
Operating system \ac{DMA} engine device (e.g., the "dma0chan0" \ac{DMA} channel on Linux).
%
\declareconstitemNEW{PMIX_DEVTYPE_COPROC}
Operating system co-processor device (e.g., "mic0" for a Xeon Phi on Linux, "opencl0d0" for a OpenCL device, or "cuda0" for a \ac{CUDA} device).
%
\end{constantdesc}

%%%%%%%%%%%%%%%%%%%%%%%%%%%%%%%%%%%%%%%%%%%%%%%%%
\subsection{Device Distance Structure}
\declarestruct{pmix_device_distance_t}

The \refstruct{pmix_device_distance_t} structure contains the minimum and maximum relative distance from the caller to a given device.

\versionMarker{4.0}
\cspecificstart
\begin{codepar}
typedef struct pmix_device_distance \{
    char *uuid;
    char *osname;
    pmix_device_type_t type;
    uint16_t mindist;
    uint16_t maxdist;
\} pmix_device_distance_t;
\end{codepar}
\cspecificend

The \refarg{uuid} is a string identifier guaranteed to be unique within the cluster and is typically assembled from discovered device attributes (e.g., the \ac{IP} address of the device). The \refarg{osname} is the local operating system name of the device and is only unique to that node.

The two distance fields provide the minimum and maximum relative distance to the device from the specified location of the process, expressed as a 16-bit integer value where a smaller number indicates that this device is closer to the process than a device with a larger distance value. Note that relative distance values are not necessarily correlated to a physical property - e.g., a device at twice the distance from another device does not necessarily have twice the latency for communication with it.

Relative distances only apply to similar devices and cannot be used to compare devices of different types. Both minimum and maximum distances are provided to support cases where the process may be bound to more than one location, and the locations are at different distances from the device.

A relative distance value of \code{UINT16_MAX} indicates that the distance from the process to the device could not be provided. This may be due to lack of available information (e.g., the \ac{PMIx} library not having access to device locations) or other factors.


%%%%%%%%%%%%%%%%%%%%%%%%%%%%%%%%%%%%%%%%%%%%%%%%%
\subsection{Device distance support macros}
\label{api:netenddist:macros}

The following macros are provided to support the \refstruct{pmix_device_distance_t} structure.

%%%%
\littleheader{Initialize the device distance structure}
\declaremacro{PMIX_DEVICE_DIST_CONSTRUCT}

Initialize the \refstruct{pmix_device_distance_t} fields.

\versionMarker{4.0}
\cspecificstart
\begin{codepar}
PMIX_DEVICE_DIST_CONSTRUCT(m)
\end{codepar}
\cspecificend

\begin{arglist}
\argin{m}{Pointer to the structure to be initialized (pointer to \refstruct{pmix_device_distance_t})}
\end{arglist}

%%%%
\littleheader{Destruct the device distance structure}
\declaremacro{PMIX_DEVICE_DIST_DESTRUCT}

Destruct the \refstruct{pmix_device_distance_t} fields.

\versionMarker{4.0}
\cspecificstart
\begin{codepar}
PMIX_DEVICE_DIST_DESTRUCT(m)
\end{codepar}
\cspecificend

\begin{arglist}
\argin{m}{Pointer to the structure to be destructed (pointer to \refstruct{pmix_device_distance_t})}
\end{arglist}

%%%%
\littleheader{Create an device distance array}
\declaremacro{PMIX_DEVICE_DIST_CREATE}

Allocate and initialize a \refstruct{pmix_device_distance_t} array.

\versionMarker{4.0}
\cspecificstart
\begin{codepar}
PMIX_DEVICE_DIST_CREATE(m, n)
\end{codepar}
\cspecificend

\begin{arglist}
\arginout{m}{Address where the pointer to the array of \refstruct{pmix_device_distance_t} structures shall be stored (handle)}
\argin{n}{Number of structures to be allocated (\code{size_t})}
\end{arglist}

%%%%
\littleheader{Release an device distance array}
\declaremacro{PMIX_DEVICE_DIST_FREE}

Release an array of \refstruct{pmix_device_distance_t} structures.

\versionMarker{4.0}
\cspecificstart
\begin{codepar}
PMIX_DEVICE_DIST_FREE(m, n)
\end{codepar}
\cspecificend

\begin{arglist}
\argin{m}{Pointer to the array of \refstruct{pmix_device_distance_t} structures (handle)}
\argin{n}{Number of structures in the array (\code{size_t})}
\end{arglist}

%%%%%%%%%%%%%%%%%%%%%%%%%%%%%%%%%%%%%%%%%%%%%%%%%
\subsection{Device distance attributes}
\label{api:netenddist:attrs}

The following attributes can be used to retrieve device distances from the \ac{PMIx} data store. Note that distances stored by the host environment are based on the process location at the time of start of execution and may not reflect changes to location imposed by the process itself.
%
\declareAttributeNEW{PMIX_DEVICE_DISTANCES}{"pmix.dev.dist"}{pmix_data_array_t}{
Return an array of \refstruct{pmix_device_distance_t} containing the minimum and maximum distances of the given process location to all devices of the specified type on the local node.
}
%
\declareAttributeNEW{PMIX_DEVICE_TYPE}{"pmix.dev.type"}{pmix_device_type_t}{
Bitmask specifying the type(s) of device(s) whose information is being requested. Only used as a directive/qualifier.
}
%
\declareAttributeNEW{PMIX_DEVICE_ID}{"pmix.dev.id"}{string}{
System-wide \ac{UUID} or node-local \ac{OS} name of a particular device.
}

%%%%%%%%%%%%%%%%%%%%%%%%%%%%%%%%%%%%%%%%%%%%%%%%%
%%%%%%%%%%%%%%%%%%%%%%%%%%%%%%%%%%%%%%%%%%%%%%%%%


    % Job Allocation Management
    %  - Allocation request, process monitoring
    %%%%%%%%%%%%%%%%%%%%%%%%%%%%%%%%%%%%%%%%%%%%%%%%%
% Chapter: Job Allocation Management
%%%%%%%%%%%%%%%%%%%%%%%%%%%%%%%%%%%%%%%%%%%%%%%%%
\chapter{Job Management and Reporting}
\label{chap:api_job_mgmt}

The job management \acp{API} provide an application with the ability to orchestrate its operation in partnership with the \ac{SMS}.
Members of this category include the \refapi{PMIx_Allocation_request}, \refapi{PMIx_Job_control}, and \refapi{PMIx_Process_monitor} \acp{API}.

%%%%%%%%%%%%%%%%%%%%%%%%%%%%%%%%%%%%%%%%%%%%%%%%%
%%%%%%%%%%%%%%%%%%%%%%%%%%%%%%%%%%%%%%%%%%%%%%%%%
\section{Allocation Requests}
\label{chap:api_job_mgmt:alloc}

This section defines functionality to request new allocations from the \ac{RM}, and request modifications to existing allocations.
These are primarily used in the following scenarios:
\begin{itemize}
\item \textit{Evolving} applications that dynamically request and return resources as they execute.
\item \textit{Malleable} environments where the scheduler redirects resources away from executing applications for higher priority jobs or load balancing.
\item \textit{Resilient} applications that need to request replacement resources in the face of failures.
\item \textit{Rigid} jobs where the user has requested a static allocation of resources for a fixed period of time, but realizes that they underestimated their required time while executing.
\end{itemize}
\ac{PMIx} attempts to address this range of use-cases with a flexible \ac{API}.

%%%%%%%%%%%%%%%%%%%%%%%%%%%%%%%%%%%%%%%%%%%%%%%%%
\subsection{\code{PMIx_Allocation_request}}
\declareapi{PMIx_Allocation_request}

%%%%
\summary

Request an allocation operation from the host resource manager.

%%%%
\format

\versionMarker{3.0}
\cspecificstart
\begin{codepar}
pmix_status_t
PMIx_Allocation_request(pmix_alloc_directive_t directive,
                        pmix_info_t info[], size_t ninfo,
                        pmix_info_t *results[], size_t *nresults);
\end{codepar}
\cspecificend

\begin{arglist}
\argin{directive}{Allocation directive (\refstruct{pmix_alloc_directive_t})}
\argin{info}{Array of \refstruct{pmix_info_t} structures (array of handles)}
\argin{ninfo}{Number of elements in the \refarg{info} array (integer)}
\arginout{results}{Address where a pointer to an array of \refstruct{pmix_info_t} containing the results of the request can be returned (memory reference)}
\arginout{nresults}{Address where the number of elements in \refarg{results} can be returned (handle)}
\end{arglist}

Returns one of the following:

\begin{itemize}
    \item \refconst{PMIX_SUCCESS}, indicating that the request was processed and returned \textit{success}
    \item a PMIx error constant indicating either an error in the input or that the request was refused
\end{itemize}

\reqattrstart
\ac{PMIx} libraries are not required to directly support any attributes for this function. However, any provided attributes must be passed to the host \ac{SMS} daemon for processing, and the \ac{PMIx} library is \textit{required} to add the \refAttributeItem{PMIX_USERID} and the \refAttributeItem{PMIX_GRPID} attributes of the client process making the request.

Host environments that implement support for this operation are required to support the following attributes:

\pasteAttributeItem{PMIX_ALLOC_REQ_ID}
\pasteAttributeItem{PMIX_ALLOC_NUM_NODES}
\pasteAttributeItem{PMIX_ALLOC_NUM_CPUS}
\pasteAttributeItem{PMIX_ALLOC_TIME}

\reqattrend

\optattrstart
The following attributes are optional for host environments that support this operation:

\pasteAttributeItem{PMIX_ALLOC_NODE_LIST}
\pasteAttributeItem{PMIX_ALLOC_NUM_CPU_LIST}
\pasteAttributeItem{PMIX_ALLOC_CPU_LIST}
\pasteAttributeItem{PMIX_ALLOC_MEM_SIZE}
\pasteAttributeItem{PMIX_ALLOC_FABRIC}
\pasteAttributeItem{PMIX_ALLOC_FABRIC_ID}
\pasteAttributeItem{PMIX_ALLOC_BANDWIDTH}
\pasteAttributeItem{PMIX_ALLOC_FABRIC_QOS}
\pasteAttributeItem{PMIX_ALLOC_FABRIC_TYPE}
\pasteAttributeItem{PMIX_ALLOC_FABRIC_PLANE}
\pasteAttributeItem{PMIX_ALLOC_FABRIC_ENDPTS}
\pasteAttributeItem{PMIX_ALLOC_FABRIC_ENDPTS_NODE}
\pasteAttributeItem{PMIX_ALLOC_FABRIC_SEC_KEY}

\optattrend

%%%%
\descr

Request an allocation operation from the host resource manager.
Several broad categories are envisioned, including the ability to:

\begin{compactitem}
%
\item Request allocation of additional resources, including memory, bandwidth, and compute.
This should be accomplished in a non-blocking manner so that the application can continue to progress while waiting for resources to become available.
Note that the new allocation will be disjoint from (i.e., not affiliated with) the allocation of the requestor - thus the termination of one allocation will not impact the other.
%
\item Extend the reservation on currently allocated resources, subject to scheduling availability and priorities.
This includes extending the time limit on current resources, and/or requesting additional resources be allocated to the requesting job.
Any additional allocated resources will be considered as part of the current allocation, and thus will be released at the same time.
%
\item Return no-longer-required resources to the scheduler.
This includes the ``loan'' of resources back to the scheduler with a promise to return them upon subsequent request.
\end{compactitem}

If successful, the returned results for a request for additional resources must include the host resource manager's identifier (\refattr{PMIX_ALLOC_ID}) that the requester can use to specify the resources in, for example, a call to \refapi{PMIx_Spawn}.

%%%%%%%%%%%%%%%%%%%%%%%%%%%%%%%%%%%%%%%%%%%%%%%%%
\subsection{\code{PMIx_Allocation_request_nb}}
\declareapi{PMIx_Allocation_request_nb}

%%%%
\summary

Request an allocation operation from the host resource manager.

%%%%
\format

\versionMarker{2.0}
\cspecificstart
\begin{codepar}
pmix_status_t
PMIx_Allocation_request_nb(pmix_alloc_directive_t directive,
                           pmix_info_t info[], size_t ninfo,
                           pmix_info_cbfunc_t cbfunc, void *cbdata);
\end{codepar}
\cspecificend

\begin{arglist}
\argin{directive}{Allocation directive (\refstruct{pmix_alloc_directive_t})}
\argin{info}{Array of \refstruct{pmix_info_t} structures (array of handles)}
\argin{ninfo}{Number of elements in the \refarg{info} array (integer)}
\argin{cbfunc}{Callback function \refapi{pmix_info_cbfunc_t} (function reference)}
\argin{cbdata}{Data to be passed to the callback function (memory reference)}
\end{arglist}

Returns one of the following:

\begin{itemize}
    \item \refconst{PMIX_SUCCESS}, indicating that the request is being processed by the host environment - result will be returned in the provided \refarg{cbfunc}. Note that the library must not invoke the callback function prior to returning from the \ac{API}.
    \item \refconst{PMIX_OPERATION_SUCCEEDED}, indicating that the request was immediately processed and returned \textit{success} - the \refarg{cbfunc} will \textit{not} be called
    \item a PMIx error constant indicating either an error in the input or that the request was immediately processed and failed - the \refarg{cbfunc} will \textit{not} be called
\end{itemize}

\reqattrstart
\ac{PMIx} libraries are not required to directly support any attributes for this function. However, any provided attributes must be passed to the host \ac{SMS} daemon for processing, and the \ac{PMIx} library is \textit{required} to add the \refAttributeItem{PMIX_USERID} and the \refAttributeItem{PMIX_GRPID} attributes of the client process making the request.

Host environments that implement support for this operation are required to support the following attributes:

\pasteAttributeItem{PMIX_ALLOC_REQ_ID}
\pasteAttributeItem{PMIX_ALLOC_NUM_NODES}
\pasteAttributeItem{PMIX_ALLOC_NUM_CPUS}
\pasteAttributeItem{PMIX_ALLOC_TIME}

\reqattrend

\optattrstart
The following attributes are optional for host environments that support this operation:

\pasteAttributeItem{PMIX_ALLOC_NODE_LIST}
\pasteAttributeItem{PMIX_ALLOC_NUM_CPU_LIST}
\pasteAttributeItem{PMIX_ALLOC_CPU_LIST}
\pasteAttributeItem{PMIX_ALLOC_MEM_SIZE}
\pasteAttributeItem{PMIX_ALLOC_FABRIC}
\pasteAttributeItem{PMIX_ALLOC_FABRIC_ID}
\pasteAttributeItem{PMIX_ALLOC_BANDWIDTH}
\pasteAttributeItem{PMIX_ALLOC_FABRIC_QOS}
\pasteAttributeItem{PMIX_ALLOC_FABRIC_TYPE}
\pasteAttributeItem{PMIX_ALLOC_FABRIC_PLANE}
\pasteAttributeItem{PMIX_ALLOC_FABRIC_ENDPTS}
\pasteAttributeItem{PMIX_ALLOC_FABRIC_ENDPTS_NODE}
\pasteAttributeItem{PMIX_ALLOC_FABRIC_SEC_KEY}

\optattrend

%%%%
\descr

Non-blocking form of the \refapi{PMIx_Allocation_request} \ac{API}.


%%%%%%%%%%%%%%%%%%%%%%%%%%%%%%%%%%%%%%%%%%%%%%%%%
\subsection{Job Allocation attributes}
\label{api:struct:attributes:joballoc}

Attributes used to describe the job allocation - these are values passed to and/or returned by the \refapi{PMIx_Allocation_request_nb} and \refapi{PMIx_Allocation_request} \acp{API} and are not accessed using the \refapi{PMIx_Get} \ac{API}.

%
\declareAttribute{PMIX_ALLOC_REQ_ID}{"pmix.alloc.reqid"}{char*}{
User-provided string identifier for this allocation request which can later be used to query status of the request.
}
%
\declareAttributeNEW{PMIX_ALLOC_ID}{"pmix.alloc.id"}{char*}{
A string identifier (provided by the host environment) for the resulting allocation which can later be used to reference the allocated resources in, for example, a call to \refapi{PMIx_Spawn}.
}
%
\declareAttributeNEW{PMIX_ALLOC_QUEUE}{"pmix.alloc.queue"}{char*}{
Name of the \ac{WLM} queue to which the allocation request is to be directed, or the queue being referenced in a query.
}
%
\declareAttribute{PMIX_ALLOC_NUM_NODES}{"pmix.alloc.nnodes"}{uint64_t}{
The number of nodes being requested in an allocation request.
}
%
\declareAttribute{PMIX_ALLOC_NODE_LIST}{"pmix.alloc.nlist"}{char*}{
Regular expression of the specific nodes being requested in an allocation request.
}
%
\declareAttribute{PMIX_ALLOC_NUM_CPUS}{"pmix.alloc.ncpus"}{uint64_t}{
Number of \acp{PU} being requested in an allocation request.
}
%
\declareAttribute{PMIX_ALLOC_NUM_CPU_LIST}{"pmix.alloc.ncpulist"}{char*}{
Regular expression of the number of \acp{PU} for each node being requested in an allocation request.
}
%
\declareAttribute{PMIX_ALLOC_CPU_LIST}{"pmix.alloc.cpulist"}{char*}{
Regular expression of the specific \acp{PU}  being requested in an allocation request.
}
\declareAttribute{PMIX_ALLOC_NUM_GPUS}{"pmix.alloc.ncpus"}{uint64_t}{
Number of \agp{PU} being requested in an allocation request.
}
%
\declareAttribute{PMIX_ALLOC_NUM_GPU_LIST}{"pmix.alloc.ncpulist"}{char*}{
Regular expression of the number of \agp{PU} for each node being requested in an allocation request.
}
%
\declareAttribute{PMIX_ALLOC_GPU_LIST}{"pmix.alloc.cpulist"}{char*}{
Regular expression of the specific \agp{PU}  being requested in an allocation request.
}
%
\declareAttribute{PMIX_ALLOC_MEM_SIZE}{"pmix.alloc.msize"}{float}{
Number of Megabytes[base2] of memory (per process) being requested in an allocation request.
}
%
\declareAttribute{PMIX_ALLOC_FABRIC}{"pmix.alloc.net"}{array}{
Array of \refstruct{pmix_info_t} describing requested fabric resources. This must include at least: \refattr{PMIX_ALLOC_FABRIC_ID}, \refattr{PMIX_ALLOC_FABRIC_TYPE}, and \refattr{PMIX_ALLOC_FABRIC_ENDPTS}, plus whatever other descriptors are desired.
}
%
\declareAttribute{PMIX_ALLOC_FABRIC_ID}{"pmix.alloc.netid"}{char*}{
The key to be used when accessing this requested fabric allocation. The fabric allocation will be returned/stored as a \refstruct{pmix_data_array_t} of \refstruct{pmix_info_t} whose first element is composed of this key and the allocated resource description.
The type of the included value depends upon the fabric support. For example, a \ac{TCP} allocation might consist of a comma-delimited string of socket ranges such as \code{"32000-32100,\allowbreak 33005,38123-38146"}. Additional array entries will consist of any provided resource request directives, along with their assigned values. Examples include: \refattr{PMIX_ALLOC_FABRIC_TYPE} - the type of resources provided; \refattr{PMIX_ALLOC_FABRIC_PLANE} - if applicable, what plane the resources were assigned from; \refattr{PMIX_ALLOC_FABRIC_QOS} - the assigned QoS; \refattr{PMIX_ALLOC_BANDWIDTH} - the allocated bandwidth; \refattr{PMIX_ALLOC_FABRIC_SEC_KEY} - a security key for the requested fabric allocation. NOTE: the array contents may differ from those requested, especially if \refconst{PMIX_INFO_REQD} was not set in the request.
}
%
\declareAttribute{PMIX_ALLOC_BANDWIDTH}{"pmix.alloc.bw"}{float}{
Fabric bandwidth (in Megabits[base2]/sec) for the job being requested in an allocation request.
}
%
\declareAttribute{PMIX_ALLOC_FABRIC_QOS}{"pmix.alloc.netqos"}{char*}{
Fabric quality of service level for the job being requested in an allocation request.
}
%
\declareAttribute{PMIX_ALLOC_TIME}{"pmix.alloc.time"}{uint32_t}{
Total session time (in seconds) being requested in an allocation request.
}
%
\declareAttribute{PMIX_ALLOC_FABRIC_TYPE}{"pmix.alloc.nettype"}{char*}{
Type of desired transport (e.g., \var{``tcp''}, \var{``udp''}) being requested in an allocation request.
}
%
\declareAttribute{PMIX_ALLOC_FABRIC_PLANE}{"pmix.alloc.netplane"}{char*}{
ID string for the \refterm{fabric plane} to be used for the requested allocation.
}
%
\declareAttribute{PMIX_ALLOC_FABRIC_ENDPTS}{"pmix.alloc.endpts"}{size_t}{
Number of endpoints to allocate per \refterm{process} in the job.
}
%
\declareAttribute{PMIX_ALLOC_FABRIC_ENDPTS_NODE}{"pmix.alloc.endpts.nd"}{size_t}{
Number of endpoints to allocate per \refterm{node} for the job.
}
%
\declareAttribute{PMIX_ALLOC_FABRIC_SEC_KEY}{"pmix.alloc.nsec"}{pmix_byte_object_t}{
Request that the allocation include a fabric security key for the spawned job.
}


%%%%%%%%%%%%%%%%%%%%%%%%%%%%%%%%%%%%%%%%%%%%%%%%%
\subsection{Job Allocation Directives}
\declarestruct{pmix_alloc_directive_t}

\versionMarker{2.0}
The \refstruct{pmix_alloc_directive_t} structure is a \code{uint8_t} type that defines the behavior of allocation requests.
The following constants can be used to set a variable of the type \refstruct{pmix_alloc_directive_t}. All definitions were introduced in version 2 of the standard unless otherwise marked.

\begin{constantdesc}
%
\declareconstitem{PMIX_ALLOC_NEW}
A new allocation is being requested.
The resulting allocation will be disjoint (i.e., not connected in a job sense) from the requesting allocation.
%
\declareconstitem{PMIX_ALLOC_EXTEND}
Extend the existing allocation, either in time or as additional resources.
%
\declareconstitem{PMIX_ALLOC_RELEASE}
Release part of the existing allocation.
Attributes in the accompanying \refstruct{pmix_info_t} array may be used to specify permanent release of the identified resources, or ``lending'' of those resources for some period of time.
%
\declareconstitem{PMIX_ALLOC_REAQUIRE}
Reacquire resources that were previously ``lent'' back to the scheduler.
%
\declareconstitem{PMIX_ALLOC_EXTERNAL}
A value boundary above which implementers are free to define their own directive values.
%
\end{constantdesc}



%%%%%%%%%%%%%%%%%%%%%%%%%%%%%%%%%%%%%%%%%%%%%%%%%
%%%%%%%%%%%%%%%%%%%%%%%%%%%%%%%%%%%%%%%%%%%%%%%%%
\section{Job Control}
\label{chap:api_job_mgmt:jctrl}

This section defines \acp{API} that enable the application and host environment to coordinate the response to failures and other events.
This can include requesting termination of the entire job or a subset of processes within a job, but can
also be used in combination with other \ac{PMIx} capabilities (e.g., allocation support and event notification) for more nuanced responses. For example, an application notified of an incipient over-temperature condition on a node could use the \refapi{PMIx_Allocation_request_nb} interface to request replacement nodes while simultaneously using the \refapi{PMIx_Job_control_nb} interface to direct that a checkpoint event be delivered to all processes in the application. If replacement resources are not available, the application might use the \refapi{PMIx_Job_control_nb} interface to request that the job continue at a lower power setting, perhaps sufficient to avoid the over-temperature failure.

The job control \acp{API} can also be used by an application to register itself as available for preemption when operating in an environment such as a cloud or where incentives, financial or otherwise, are provided to jobs willing to be preempted. Registration can include attributes indicating how many resources are being offered for preemption (e.g., all or only some portion), whether the application will require time to prepare for preemption, etc. Jobs that
request a warning will receive an event notifying them of an impending preemption (possibly including information as to the resources that will be taken away, how much time the application will be given prior to being preempted, whether the preemption will be a suspension or full termination, etc.) so they have an opportunity to save
their work. Once the application is ready, it calls the provided event completion callback function to indicate that
the SMS is free to suspend or terminate it, and can include directives regarding any desired restart.

%%%%%%%%%%%%%%%%%%%%%%%%%%%%%%%%%%%%%%%%%%%%%%%%%
\subsection{\code{PMIx_Job_control}}
\declareapi{PMIx_Job_control}

%%%%
\summary

Request a job control action.

%%%%
\format

\versionMarker{3.0}
\cspecificstart
\begin{codepar}
pmix_status_t
PMIx_Job_control(const pmix_proc_t targets[], size_t ntargets,
                 const pmix_info_t directives[], size_t ndirs,
                 pmix_info_t *results[], size_t *nresults);
\end{codepar}
\cspecificend

\begin{arglist}
\argin{targets}{Array of proc structures (array of handles)}
\argin{ntargets}{Number of elements in the \refarg{targets} array (integer)}
\argin{directives}{Array of info structures (array of handles)}
\argin{ndirs}{Number of elements in the \refarg{directives} array (integer)}
\arginout{results}{Address where a pointer to an array of \refstruct{pmix_info_t} containing the results of the request can be returned (memory reference)}
\arginout{nresults}{Address where the number of elements in \refarg{results} can be returned (handle)}
\end{arglist}

Returns one of the following:

\begin{itemize}
    \item \refconst{PMIX_SUCCESS}, indicating that the request was processed by the host environment and returned \textit{success}. Details of the result will be returned in the \refarg{results} array
    \item a \ac{PMIx} error constant indicating either an error in the input or that the request was refused
\end{itemize}

\reqattrstart
\ac{PMIx} libraries are not required to directly support any attributes for this function. However, any provided attributes must be passed to the host \ac{SMS} daemon for processing, and the \ac{PMIx} library is \textit{required} to add the \refAttributeItem{PMIX_USERID} and the \refAttributeItem{PMIX_GRPID} attributes of the client process making the request.

Host environments that implement support for this operation are required to support the following attributes:

\pasteAttributeItem{PMIX_JOB_CTRL_ID}
\pasteAttributeItem{PMIX_JOB_CTRL_PAUSE}
\pasteAttributeItem{PMIX_JOB_CTRL_RESUME}
\pasteAttributeItem{PMIX_JOB_CTRL_KILL}
\pasteAttributeItem{PMIX_JOB_CTRL_SIGNAL}
\pasteAttributeItem{PMIX_JOB_CTRL_TERMINATE}
\pasteAttributeItem{PMIX_REGISTER_CLEANUP}
\pasteAttributeItem{PMIX_REGISTER_CLEANUP_DIR}
\pasteAttributeItem{PMIX_CLEANUP_RECURSIVE}
\pasteAttributeItem{PMIX_CLEANUP_EMPTY}
\pasteAttributeItem{PMIX_CLEANUP_IGNORE}
\pasteAttributeItem{PMIX_CLEANUP_LEAVE_TOPDIR}

\reqattrend

\optattrstart
The following attributes are optional for host environments that support this operation:

\pasteAttributeItem{PMIX_JOB_CTRL_CANCEL}
\pasteAttributeItem{PMIX_JOB_CTRL_RESTART}
\pasteAttributeItem{PMIX_JOB_CTRL_CHECKPOINT}
\pasteAttributeItem{PMIX_JOB_CTRL_CHECKPOINT_EVENT}
\pasteAttributeItem{PMIX_JOB_CTRL_CHECKPOINT_SIGNAL}
\pasteAttributeItem{PMIX_JOB_CTRL_CHECKPOINT_TIMEOUT}
\pasteAttributeItem{PMIX_JOB_CTRL_CHECKPOINT_METHOD}
\pasteAttributeItem{PMIX_JOB_CTRL_PROVISION}
\pasteAttributeItem{PMIX_JOB_CTRL_PROVISION_IMAGE}
\pasteAttributeItem{PMIX_JOB_CTRL_PREEMPTIBLE}

\optattrend

%%%%
\descr

Request a job control action.
The \refarg{targets} array identifies the processes to which the requested job control action is to be applied. All \refterm{clones} of an identified process are to have the requested action applied to them.
A \code{NULL} value can be used to indicate all processes in the caller's namespace.
The use of \refconst{PMIX_RANK_WILDCARD} can also be used to indicate that all processes in the given namespace are to be included.

The directives are provided as \refstruct{pmix_info_t} structures in the \refarg{directives} array.
The returned \refarg{status} indicates whether or not the request was granted, and information as to the reason for any denial of the request shall be returned in the \refarg{results} array.

%%%%%%%%%%%%%%%%%%%%%%%%%%%%%%%%%%%%%%%%%%%%%%%%%
\subsection{\code{PMIx_Job_control_nb}}
\declareapi{PMIx_Job_control_nb}

%%%%
\summary

Request a job control action.

%%%%
\format

\versionMarker{2.0}
\cspecificstart
\begin{codepar}
pmix_status_t
PMIx_Job_control_nb(const pmix_proc_t targets[], size_t ntargets,
                    const pmix_info_t directives[], size_t ndirs,
                    pmix_info_cbfunc_t cbfunc, void *cbdata);
\end{codepar}
\cspecificend

\begin{arglist}
\argin{targets}{Array of proc structures (array of handles)}
\argin{ntargets}{Number of elements in the \refarg{targets} array (integer)}
\argin{directives}{Array of info structures (array of handles)}
\argin{ndirs}{Number of elements in the \refarg{directives} array (integer)}
\argin{cbfunc}{Callback function \refapi{pmix_info_cbfunc_t} (function reference)}
\argin{cbdata}{Data to be passed to the callback function (memory reference)}
\end{arglist}

Returns one of the following:

\begin{itemize}
    \item \refconst{PMIX_SUCCESS}, indicating that the request is being processed by the host environment - result will be returned in the provided \refarg{cbfunc}. Note that the library must not invoke the callback function prior to returning from the \ac{API}.
    \item \refconst{PMIX_OPERATION_SUCCEEDED}, indicating that the request was immediately processed and returned \textit{success} - the \refarg{cbfunc} will \textit{not} be called
    \item a PMIx error constant indicating either an error in the input or that the request was immediately processed and failed - the \refarg{cbfunc} will \textit{not} be called
\end{itemize}

\reqattrstart
\ac{PMIx} libraries are not required to directly support any attributes for this function. However, any provided attributes must be passed to the host \ac{SMS} daemon for processing, and the \ac{PMIx} library is \textit{required} to add the \refAttributeItem{PMIX_USERID} and the \refAttributeItem{PMIX_GRPID} attributes of the client process making the request.

Host environments that implement support for this operation are required to support the following attributes:

\pasteAttributeItem{PMIX_JOB_CTRL_ID}
\pasteAttributeItem{PMIX_JOB_CTRL_PAUSE}
\pasteAttributeItem{PMIX_JOB_CTRL_RESUME}
\pasteAttributeItem{PMIX_JOB_CTRL_KILL}
\pasteAttributeItem{PMIX_JOB_CTRL_SIGNAL}
\pasteAttributeItem{PMIX_JOB_CTRL_TERMINATE}
\pasteAttributeItem{PMIX_REGISTER_CLEANUP}
\pasteAttributeItem{PMIX_REGISTER_CLEANUP_DIR}
\pasteAttributeItem{PMIX_CLEANUP_RECURSIVE}
\pasteAttributeItem{PMIX_CLEANUP_EMPTY}
\pasteAttributeItem{PMIX_CLEANUP_IGNORE}
\pasteAttributeItem{PMIX_CLEANUP_LEAVE_TOPDIR}

\reqattrend

\optattrstart
The following attributes are optional for host environments that support this operation:

\pasteAttributeItem{PMIX_JOB_CTRL_CANCEL}
\pasteAttributeItem{PMIX_JOB_CTRL_RESTART}
\pasteAttributeItem{PMIX_JOB_CTRL_CHECKPOINT}
\pasteAttributeItem{PMIX_JOB_CTRL_CHECKPOINT_EVENT}
\pasteAttributeItem{PMIX_JOB_CTRL_CHECKPOINT_SIGNAL}
\pasteAttributeItem{PMIX_JOB_CTRL_CHECKPOINT_TIMEOUT}
\pasteAttributeItem{PMIX_JOB_CTRL_CHECKPOINT_METHOD}
\pasteAttributeItem{PMIX_JOB_CTRL_PROVISION}
\pasteAttributeItem{PMIX_JOB_CTRL_PROVISION_IMAGE}
\pasteAttributeItem{PMIX_JOB_CTRL_PREEMPTIBLE}

\optattrend

%%%%
\descr

Non-blocking form of the \refapi{PMIx_Job_control} \ac{API}.
The \refarg{targets} array identifies the processes to which the requested job control action is to be applied. All \refterm{clones} of an identified process are to have the requested action applied to them.
A \code{NULL} value can be used to indicate all processes in the caller's namespace.
The use of \refconst{PMIX_RANK_WILDCARD} can also be used to indicate that all processes in the given namespace are to be included.

The directives are provided as \refstruct{pmix_info_t} structures in the \refarg{directives} array.
The callback function provides a \refarg{status} to indicate whether or not the request was granted, and to provide some information as to the reason for any denial in the \refapi{pmix_info_cbfunc_t} array of \refstruct{pmix_info_t} structures.

%%%%%%%%%%%%%%%%%%%%%%%%%%%%%%%%%%%%%%%%%%%%%%%%%
\subsection{Job control constants}
\label{api:struct:constants:jobcontrol}

The following constants are specifically defined for return by the job control \acp{API}:

\begin{constantdesc}

%
\declareconstitemNEW{PMIX_ERR_CONFLICTING_CLEANUP_DIRECTIVES}
Conflicting directives given for job/process cleanup.

\end{constantdesc}

%%%%%%%%%%%%%%%%%%%%%%%%%%%%%%%%%%%%%%%%%%%%%%%%%
\subsection{Job control events}
\label{api:struct:events:jobcontrol}

The following job control events may be available for registration, depending upon implementation and host environment support:

\begin{constantdesc}
%
\declareconstitem{PMIX_JCTRL_CHECKPOINT}
Monitored by \ac{PMIx} client to trigger a checkpoint operation.
%
\declareconstitem{PMIX_JCTRL_CHECKPOINT_COMPLETE}
Sent by a \ac{PMIx} client and monitored by a \ac{PMIx} server to notify that requested checkpoint operation has completed.
%
\declareconstitem{PMIX_JCTRL_PREEMPT_ALERT}
Monitored by a \ac{PMIx} client to detect that an \ac{RM} intends to preempt the job.
%
\declareconstitem{PMIX_ERR_PROC_RESTART}
Error in process restart.
%
\declareconstitem{PMIX_ERR_PROC_CHECKPOINT}
Error in process checkpoint.
%
\declareconstitem{PMIX_ERR_PROC_MIGRATE}
Error in process migration.
%
\end{constantdesc}

%%%%%%%%%%%%%%%%%%%%%%%%%%%%%%%%%%%%%%%%%%%%%%%%%
\subsection{Job control attributes}
\label{api:struct:attributes:jobcontrol}

Attributes used to request control operations on an executing application - these are values passed to the job control \acp{API} and are not accessed using the \refapi{PMIx_Get} \ac{API}.

%
\declareAttribute{PMIX_JOB_CTRL_ID}{"pmix.jctrl.id"}{char*}{
Provide a string identifier for this request. The user can provide an identifier for the requested operation, thus allowing them to later request status of the operation or to terminate it. The host, therefore, shall track it with the request for future reference.
}
%
\declareAttribute{PMIX_JOB_CTRL_PAUSE}{"pmix.jctrl.pause"}{bool}{
Pause the specified processes.
}
%
\declareAttribute{PMIX_JOB_CTRL_RESUME}{"pmix.jctrl.resume"}{bool}{
Resume (``un-pause'') the specified processes.
}
%
\declareAttribute{PMIX_JOB_CTRL_CANCEL}{"pmix.jctrl.cancel"}{char*}{
Cancel the specified request - the provided request ID must match the \refattr{PMIX_JOB_CTRL_ID} provided to a previous call to \refapi{PMIx_Job_control}. An ID of \code{NULL} implies cancel all requests from this requestor.
}
%
\declareAttribute{PMIX_JOB_CTRL_KILL}{"pmix.jctrl.kill"}{bool}{
Forcibly terminate the specified processes and cleanup.
}
%
\declareAttribute{PMIX_JOB_CTRL_RESTART}{"pmix.jctrl.restart"}{char*}{
Restart the specified processes using the given checkpoint ID.
}
%
\declareAttribute{PMIX_JOB_CTRL_CHECKPOINT}{"pmix.jctrl.ckpt"}{char*}{
Checkpoint the specified processes and assign the given ID to it.
}
%
\declareAttribute{PMIX_JOB_CTRL_CHECKPOINT_EVENT}{"pmix.jctrl.ckptev"}{bool}{
Use event notification to trigger a process checkpoint.
}
%
\declareAttribute{PMIX_JOB_CTRL_CHECKPOINT_SIGNAL}{"pmix.jctrl.ckptsig"}{int}{
Use the given signal to trigger a process checkpoint.
}
%
\declareAttribute{PMIX_JOB_CTRL_CHECKPOINT_TIMEOUT}{"pmix.jctrl.ckptsig"}{int}{
Time in seconds to wait for a checkpoint to complete.
}
%
\declareAttribute{PMIX_JOB_CTRL_CHECKPOINT_METHOD}{"pmix.jctrl.ckmethod"}{pmix_data_array_t}{
Array of \refstruct{pmix_info_t} declaring each method and value supported by this application.
}
%
\declareAttribute{PMIX_JOB_CTRL_SIGNAL}{"pmix.jctrl.sig"}{int}{
Send given signal to specified processes.
}
%
\declareAttribute{PMIX_JOB_CTRL_PROVISION}{"pmix.jctrl.pvn"}{char*}{
Regular expression identifying nodes that are to be provisioned.
}
%
\declareAttribute{PMIX_JOB_CTRL_PROVISION_IMAGE}{"pmix.jctrl.pvnimg"}{char*}{
Name of the image that is to be provisioned.
}
%
\declareAttribute{PMIX_JOB_CTRL_PREEMPTIBLE}{"pmix.jctrl.preempt"}{bool}{
Indicate that the job can be pre-empted.
}
%
\declareAttribute{PMIX_JOB_CTRL_TERMINATE}{"pmix.jctrl.term"}{bool}{
Politely terminate the specified processes.
}
%
\declareAttribute{PMIX_REGISTER_CLEANUP}{"pmix.reg.cleanup"}{char*}{
Comma-delimited list of files to be removed upon process termination.
}
%
\declareAttribute{PMIX_REGISTER_CLEANUP_DIR}{"pmix.reg.cleanupdir"}{char*}{
Comma-delimited list of directories to be removed upon process termination.
}
%
\declareAttribute{PMIX_CLEANUP_RECURSIVE}{"pmix.clnup.recurse"}{bool}{
Recursively cleanup all subdirectories under the specified one(s).
}
%
\declareAttribute{PMIX_CLEANUP_EMPTY}{"pmix.clnup.empty"}{bool}{
Only remove empty subdirectories.
}
%
\declareAttribute{PMIX_CLEANUP_IGNORE}{"pmix.clnup.ignore"}{char*}{
Comma-delimited list of filenames that are not to be removed.
}
%
\declareAttribute{PMIX_CLEANUP_LEAVE_TOPDIR}{"pmix.clnup.lvtop"}{bool}{
When recursively cleaning subdirectories, do not remove the top-level directory (the one given in the cleanup request).
}


%%%%%%%%%%%%%%%%%%%%%%%%%%%%%%%%%%%%%%%%%%%%%%%%%
%%%%%%%%%%%%%%%%%%%%%%%%%%%%%%%%%%%%%%%%%%%%%%%%%
\section{Process and Job Monitoring}
\label{chap:api_job_mgmt:monitor}

In addition to external faults, a common problem encountered in \ac{HPC} applications is a failure to make
progress due to some internal conflict in the computation. These situations can
result in a significant waste of resources as the \ac{SMS} is unaware of the problem, and thus cannot terminate the
job. Various watchdog methods have been developed for detecting this situation, including requiring a periodic ``heartbeat''
from the application and monitoring a specified file for changes in size and/or modification time.

The following \acp{API} allow applications to request monitoring, directing what is to be monitored, the frequency of the associated check, whether or not the application is to be notified (via the event notification subsystem) of stall detection, and other characteristics of the operation.

%%%%%%%%%%%%%%%%%%%%%%%%%%%%%%%%%%%%%%%%%%%%%%%%%
\subsection{\code{PMIx_Process_monitor}}
\declareapi{PMIx_Process_monitor}

%%%%
\summary

Request that application processes be monitored.

%%%%
\format

\versionMarker{3.0}
\cspecificstart
\begin{codepar}
pmix_status_t
PMIx_Process_monitor(const pmix_info_t *monitor,
                     pmix_status_t error,
                     const pmix_info_t directives[], size_t ndirs,
                     pmix_info_t *results[], size_t *nresults);
\end{codepar}
\cspecificend

\begin{arglist}
\argin{monitor}{info (handle)}
\argin{error}{status (integer)}
\argin{directives}{Array of info structures (array of handles)}
\argin{ndirs}{Number of elements in the \refarg{directives} array (integer)}
\arginout{results}{Address where a pointer to an array of \refstruct{pmix_info_t} containing the results of the request can be returned (memory reference)}
\arginout{nresults}{Address where the number of elements in \refarg{results} can be returned (handle)}
\end{arglist}

Returns one of the following:

\begin{itemize}
    \item \refconst{PMIX_SUCCESS}, indicating that the request was processed and returned \textit{success}. Details of the result will be returned in the \refarg{results} array
    \item a PMIx error constant indicating either an error in the input or that the request was refused
\end{itemize}

\optattrstart
The following attributes may be implemented by a \ac{PMIx} library or by the host environment. If supported by the \ac{PMIx} server library, then the library must not pass the supported attributes to the host environment. All attributes not directly supported by the server library must be passed to the host environment if it supports this operation, and the library is \textit{required} to add the \refAttributeItem{PMIX_USERID} and the \refAttributeItem{PMIX_GRPID} attributes of the requesting process:

\pasteAttributeItem{PMIX_MONITOR_ID}
\pasteAttributeItem{PMIX_MONITOR_CANCEL}
\pasteAttributeItem{PMIX_MONITOR_APP_CONTROL}
\pasteAttributeItem{PMIX_MONITOR_HEARTBEAT}
\pasteAttributeItem{PMIX_MONITOR_HEARTBEAT_TIME}
\pasteAttributeItem{PMIX_MONITOR_HEARTBEAT_DROPS}
\pasteAttributeItem{PMIX_MONITOR_FILE}
\pasteAttributeItem{PMIX_MONITOR_FILE_SIZE}
\pasteAttributeItem{PMIX_MONITOR_FILE_ACCESS}
\pasteAttributeItem{PMIX_MONITOR_FILE_MODIFY}
\pasteAttributeItem{PMIX_MONITOR_FILE_CHECK_TIME}
\pasteAttributeItem{PMIX_MONITOR_FILE_DROPS}
\pasteAttributeItem{PMIX_SEND_HEARTBEAT}

\optattrend

%%%%
\descr

Request that application processes be monitored via several possible methods.
For example, that the server monitor this process for periodic heartbeats as an indication that the process has not become ``wedged''.
When a monitor detects the specified alarm condition, it will generate an event notification using the provided error code and passing along any available relevant information.
It is up to the caller to register a corresponding event handler.

The \refarg{monitor} argument is an attribute indicating the type of monitor being requested.
For example, \refattr{PMIX_MONITOR_FILE} to indicate that the requestor is asking that a file be monitored.

The \refarg{error} argument is the status code to be used when generating an event notification alerting that the monitor has been triggered.
The range of the notification defaults to \refconst{PMIX_RANGE_NAMESPACE}.
This can be changed by providing a \refattr{PMIX_RANGE} directive.

The \refarg{directives} argument characterizes the monitoring request (e.g., monitor file size) and frequency of checking to be done

The returned \refarg{status} indicates whether or not the request was granted, and information as to the reason for any denial of the request shall be returned in the \refarg{results} array.

%%%%%%%%%%%%%%%%%%%%%%%%%%%%%%%%%%%%%%%%%%%%%%%%%
\subsection{\code{PMIx_Process_monitor_nb}}
\declareapi{PMIx_Process_monitor_nb}

%%%%
\summary

Request that application processes be monitored.

%%%%
\format

\versionMarker{2.0}
\cspecificstart
\begin{codepar}
pmix_status_t
PMIx_Process_monitor_nb(const pmix_info_t *monitor,
                        pmix_status_t error,
                        const pmix_info_t directives[],
                        size_t ndirs,
                        pmix_info_cbfunc_t cbfunc, void *cbdata);
\end{codepar}
\cspecificend

\begin{arglist}
\argin{monitor}{info (handle)}
\argin{error}{status (integer)}
\argin{directives}{Array of info structures (array of handles)}
\argin{ndirs}{Number of elements in the \refarg{directives} array (integer)}
\argin{cbfunc}{Callback function \refapi{pmix_info_cbfunc_t} (function reference)}
\argin{cbdata}{Data to be passed to the callback function (memory reference)}
\end{arglist}

Returns one of the following:

\begin{itemize}
    \item \refconst{PMIX_SUCCESS}, indicating that the request is being processed by the host environment - result will be returned in the provided \refarg{cbfunc}. Note that the library must not invoke the callback function prior to returning from the \ac{API}.
    \item \refconst{PMIX_OPERATION_SUCCEEDED}, indicating that the request was immediately processed and returned \textit{success} - the \refarg{cbfunc} will \textit{not} be called.
    \item a PMIx error constant indicating either an error in the input or that the request was immediately processed and failed - the \refarg{cbfunc} will \textit{not} be called.
\end{itemize}

\optattrstart
The following attributes may be implemented by a \ac{PMIx} library or by the host environment. If supported by the \ac{PMIx} server library, then the library must not pass the supported attributes to the host environment. All attributes not directly supported by the server library must be passed to the host environment if it supports this operation, and the library is \textit{required} to add the \refAttributeItem{PMIX_USERID} and the \refAttributeItem{PMIX_GRPID} attributes of the requesting process:

\pasteAttributeItem{PMIX_MONITOR_ID}
\pasteAttributeItem{PMIX_MONITOR_CANCEL}
\pasteAttributeItem{PMIX_MONITOR_APP_CONTROL}
\pasteAttributeItem{PMIX_MONITOR_HEARTBEAT}
\pasteAttributeItem{PMIX_MONITOR_HEARTBEAT_TIME}
\pasteAttributeItem{PMIX_MONITOR_HEARTBEAT_DROPS}
\pasteAttributeItem{PMIX_MONITOR_FILE}
\pasteAttributeItem{PMIX_MONITOR_FILE_SIZE}
\pasteAttributeItem{PMIX_MONITOR_FILE_ACCESS}
\pasteAttributeItem{PMIX_MONITOR_FILE_MODIFY}
\pasteAttributeItem{PMIX_MONITOR_FILE_CHECK_TIME}
\pasteAttributeItem{PMIX_MONITOR_FILE_DROPS}
\pasteAttributeItem{PMIX_SEND_HEARTBEAT}

\optattrend

%%%%
\descr

Non-blocking form of the \refapi{PMIx_Process_monitor} \ac{API}. The \refarg{cbfunc} function provides a \refarg{status} to indicate whether or not the request was granted, and to provide some information as to the reason for any denial in the \refapi{pmix_info_cbfunc_t} array of \refstruct{pmix_info_t} structures.

%%%%%%%%%%%%%%%%%%%%%%%%%%%%%%%%%%%%%%%%%%%%%%%%%
\subsection{\code{PMIx_Heartbeat}}
\declaremacro{PMIx_Heartbeat}

%%%%
\summary

Send a heartbeat to the \ac{PMIx} server library

%%%%
\format

\versionMarker{2.0}
\cspecificstart
\begin{codepar}
PMIx_Heartbeat();
\end{codepar}
\cspecificend


%%%%
\descr

A simplified macro wrapping \refapi{PMIx_Process_monitor_nb} that sends a heartbeat to the \ac{PMIx} server library.

%%%%%%%%%%%%%%%%%%%%%%%%%%%%%%%%%%%%%%%%%%%%%%%%%
\subsection{Monitoring events}
\label{api:struct:events:monitor}

The following monitoring events may be available for registration, depending upon implementation and host environment support:

\begin{constantdesc}
%
\declareconstitem{PMIX_MONITOR_HEARTBEAT_ALERT}
Heartbeat failed to arrive within specified window. The process that triggered this alert will be identified in the event.
%
\declareconstitem{PMIX_MONITOR_FILE_ALERT}
File failed its monitoring detection criteria. The file that triggered this alert will be identified in the event.
%
\end{constantdesc}

%%%%%%%%%%%
\subsection{Monitoring attributes}
\label{api:struct:attributes:monitor}

Attributes used to control monitoring of an executing application- these are values passed to the \refapi{PMIx_Process_monitor_nb} \ac{API} and are not accessed using the \refapi{PMIx_Get} \ac{API}.

%
\declareAttribute{PMIX_MONITOR_ID}{"pmix.monitor.id"}{char*}{
Provide a string identifier for this request.
}
%
\declareAttribute{PMIX_MONITOR_CANCEL}{"pmix.monitor.cancel"}{char*}{
Identifier to be canceled (\code{NULL} means cancel all monitoring for this process).
}
%
\declareAttribute{PMIX_MONITOR_APP_CONTROL}{"pmix.monitor.appctrl"}{bool}{
The application desires to control the response to a monitoring event - i.e., the application is requesting that the host environment not take immediate action in response to the event (e.g., terminating the job).
}
%
\declareAttribute{PMIX_MONITOR_HEARTBEAT}{"pmix.monitor.mbeat"}{void}{
Register to have the PMIx server monitor the requestor for heartbeats.
}
%
\declareAttribute{PMIX_SEND_HEARTBEAT}{"pmix.monitor.beat"}{void}{
Send heartbeat to local PMIx server.
}
%
\declareAttribute{PMIX_MONITOR_HEARTBEAT_TIME}{"pmix.monitor.btime"}{uint32_t}{
Time in seconds before declaring heartbeat missed.
}
%
\declareAttribute{PMIX_MONITOR_HEARTBEAT_DROPS}{"pmix.monitor.bdrop"}{uint32_t}{
Number of heartbeats that can be missed before generating the event.
}
%
\declareAttribute{PMIX_MONITOR_FILE}{"pmix.monitor.fmon"}{char*}{
Register to monitor file for signs of life.
}
%
\declareAttribute{PMIX_MONITOR_FILE_SIZE}{"pmix.monitor.fsize"}{bool}{
Monitor size of given file is growing to determine if the application is running.
}
%
\declareAttribute{PMIX_MONITOR_FILE_ACCESS}{"pmix.monitor.faccess"}{char*}{
Monitor time since last access of given file to determine if the application is running.
}
%
\declareAttribute{PMIX_MONITOR_FILE_MODIFY}{"pmix.monitor.fmod"}{char*}{
Monitor time since last modified of given file to determine if the application is running.
}
%
\declareAttribute{PMIX_MONITOR_FILE_CHECK_TIME}{"pmix.monitor.ftime"}{uint32_t}{
Time in seconds between checking the file.
}
%
\declareAttribute{PMIX_MONITOR_FILE_DROPS}{"pmix.monitor.fdrop"}{uint32_t}{
Number of file checks that can be missed before generating the event.
}

%%%%%%%%%%%%%%%%%%%%%%%%%%%%%%%%%%%%%%%%%%%%%%%%%
%%%%%%%%%%%%%%%%%%%%%%%%%%%%%%%%%%%%%%%%%%%%%%%%%
\section{Logging}
\label{chap:api_job_mgmt:logging}

The logging interface supports posting information by applications and SMS elements to persistent storage. This function is \textit{not} intended for output of computational results, but rather for reporting status and saving state information such as inserting computation progress reports into the application's \ac{SMS} job log or error reports to the local syslog.

%%%%%%%%%%%%%%%%%%%%%%%%%%%%%%%%%%%%%%%%%%%%%%%%%
\subsection{\code{PMIx_Log}}
\declareapi{PMIx_Log}

%%%%
\summary

Log data to a data service.

%%%%
\format

\versionMarker{3.0}
\cspecificstart
\begin{codepar}
pmix_status_t
PMIx_Log(const pmix_info_t data[], size_t ndata,
         const pmix_info_t directives[], size_t ndirs);
\end{codepar}
\cspecificend

\begin{arglist}
\argin{data}{Array of info structures (array of handles)}
\argin{ndata}{Number of elements in the \refarg{data} array (\code{size_t})}
\argin{directives}{Array of info structures (array of handles)}
\argin{ndirs}{Number of elements in the \refarg{directives} array (\code{size_t})}
\end{arglist}

Return codes are one of the following:

\begin{constantdesc}
    \item \refconst{PMIX_SUCCESS} The logging request was successful.
    \item \refconst{PMIX_ERR_BAD_PARAM} The logging request contains at least one incorrect entry.
    \item \refconst{PMIX_ERR_NOT_SUPPORTED} The \ac{PMIx} implementation or host environment does not support this function.
    \item other appropriate \ac{PMIx} error code
\end{constantdesc}

\reqattrstart
If the \ac{PMIx} library does not itself perform this operation, then it is required to pass any attributes provided by the client to the host environment for processing. In addition, it must include the following attributes in the passed \refarg{info} array:

\pasteAttributeItem{PMIX_USERID}
\pasteAttributeItem{PMIX_GRPID}

Host environments or \ac{PMIx} libraries that implement support for this operation are required to support the following attributes:

\pasteAttributeItem{PMIX_LOG_STDERR}
\pasteAttributeItem{PMIX_LOG_STDOUT}
\pasteAttributeItem{PMIX_LOG_SYSLOG}
\pasteAttributeItem{PMIX_LOG_LOCAL_SYSLOG}
\pasteAttributeItem{PMIX_LOG_GLOBAL_SYSLOG}
\pasteAttributeItem{PMIX_LOG_SYSLOG_PRI}
\pasteAttributeItem{PMIX_LOG_ONCE}

\reqattrend

\optattrstart
The following attributes are optional for host environments or \ac{PMIx} libraries that support this operation:

\pasteAttributeItem{PMIX_LOG_SOURCE}
\pasteAttributeItem{PMIX_LOG_TIMESTAMP}
\pasteAttributeItem{PMIX_LOG_GENERATE_TIMESTAMP}
\pasteAttributeItem{PMIX_LOG_TAG_OUTPUT}
\pasteAttributeItem{PMIX_LOG_TIMESTAMP_OUTPUT}
\pasteAttributeItem{PMIX_LOG_XML_OUTPUT}
\pasteAttributeItem{PMIX_LOG_EMAIL}
\pasteAttributeItem{PMIX_LOG_EMAIL_ADDR}
\pasteAttributeItem{PMIX_LOG_EMAIL_SENDER_ADDR}
\pasteAttributeItem{PMIX_LOG_EMAIL_SERVER}
\pasteAttributeItem{PMIX_LOG_EMAIL_SRVR_PORT}
\pasteAttributeItem{PMIX_LOG_EMAIL_SUBJECT}
\pasteAttributeItem{PMIX_LOG_EMAIL_MSG}
\pasteAttributeItem{PMIX_LOG_JOB_RECORD}
\pasteAttributeItem{PMIX_LOG_GLOBAL_DATASTORE}

\optattrend

%%%%
\descr

Log data subject to the services offered by the host environment. The data to be logged is provided in the \refarg{data} array. The (optional) \refarg{directives} can be used to direct the choice of logging channel.

\adviceuserstart
It is strongly recommended that the \refapi{PMIx_Log} API not be used by applications for streaming data as it is not a ``performant'' transport and can perturb the application since it involves the local \ac{PMIx} server and host \ac{SMS} daemon. Note that a return of \refconst{PMIX_SUCCESS} only denotes that the data was successfully handed to the appropriate system call (for local channels) or the host environment and does not indicate receipt at the final destination.
\adviceuserend

%%%%%%%%%%%%%%%%%%%%%%%%%%%%%%%%%%%%%%%%%%%%%%%%%
\subsection{\code{PMIx_Log_nb}}
\declareapi{PMIx_Log_nb}

%%%%
\summary

Log data to a data service.

%%%%
\format

\versionMarker{2.0}
\cspecificstart
\begin{codepar}
pmix_status_t
PMIx_Log_nb(const pmix_info_t data[], size_t ndata,
            const pmix_info_t directives[], size_t ndirs,
            pmix_op_cbfunc_t cbfunc, void *cbdata);
\end{codepar}
\cspecificend

\begin{arglist}
\argin{data}{Array of info structures (array of handles)}
\argin{ndata}{Number of elements in the \refarg{data} array (\code{size_t})}
\argin{directives}{Array of info structures (array of handles)}
\argin{ndirs}{Number of elements in the \refarg{directives} array (\code{size_t})}
\argin{cbfunc}{Callback function \refapi{pmix_op_cbfunc_t} (function reference)}
\argin{cbdata}{Data to be passed to the callback function (memory reference)}
\end{arglist}

Return codes are one of the following:

\begin{constantdesc}
\item \refconst{PMIX_SUCCESS} The logging request is valid and is being processed. The resulting status from the operation will be provided in the callback function. Note that the library must not invoke the callback function prior to returning from the \ac{API}.
\item \refconst{PMIX_OPERATION_SUCCEEDED}, indicating that the request was immediately processed and returned \textit{success} - the \refarg{cbfunc} will \textit{not} be called
\item \refconst{PMIX_ERR_BAD_PARAM} The logging request contains at least one incorrect entry that prevents it from being processed. The callback function will not be called.
\item \refconst{PMIX_ERR_NOT_SUPPORTED} The \ac{PMIx} implementation does not support this function. The callback function will not be called.
\item other appropriate \ac{PMIx} error code - the callback function will not be called.
\end{constantdesc}

\reqattrstart
If the \ac{PMIx} library does not itself perform this operation, then it is required to pass any attributes provided by the client to the host environment for processing. In addition, it must include the following attributes in the passed \refarg{info} array:

\pasteAttributeItem{PMIX_USERID}
\pasteAttributeItem{PMIX_GRPID}

Host environments or \ac{PMIx} libraries that implement support for this operation are required to support the following attributes:

\pasteAttributeItem{PMIX_LOG_STDERR}
\pasteAttributeItem{PMIX_LOG_STDOUT}
\pasteAttributeItem{PMIX_LOG_SYSLOG}
\pasteAttributeItem{PMIX_LOG_LOCAL_SYSLOG}
\pasteAttributeItem{PMIX_LOG_GLOBAL_SYSLOG}
\pasteAttributeItem{PMIX_LOG_SYSLOG_PRI}
\pasteAttributeItem{PMIX_LOG_ONCE}

\reqattrend

\optattrstart
The following attributes are optional for host environments or \ac{PMIx} libraries that support this operation:

\pasteAttributeItem{PMIX_LOG_SOURCE}
\pasteAttributeItem{PMIX_LOG_TIMESTAMP}
\pasteAttributeItem{PMIX_LOG_GENERATE_TIMESTAMP}
\pasteAttributeItem{PMIX_LOG_TAG_OUTPUT}
\pasteAttributeItem{PMIX_LOG_TIMESTAMP_OUTPUT}
\pasteAttributeItem{PMIX_LOG_XML_OUTPUT}
\pasteAttributeItem{PMIX_LOG_EMAIL}
\pasteAttributeItem{PMIX_LOG_EMAIL_ADDR}
\pasteAttributeItem{PMIX_LOG_EMAIL_SENDER_ADDR}
\pasteAttributeItem{PMIX_LOG_EMAIL_SERVER}
\pasteAttributeItem{PMIX_LOG_EMAIL_SRVR_PORT}
\pasteAttributeItem{PMIX_LOG_EMAIL_SUBJECT}
\pasteAttributeItem{PMIX_LOG_EMAIL_MSG}
\pasteAttributeItem{PMIX_LOG_JOB_RECORD}
\pasteAttributeItem{PMIX_LOG_GLOBAL_DATASTORE}

\optattrend

%%%%
\descr

Log data subject to the services offered by the host environment. The data to be logged is provided in the \refarg{data} array. The (optional) \refarg{directives} can be used to direct the choice of logging channel.
The callback function will be executed when the log operation has been completed. The \refarg{data} and \refarg{directives} arrays must be maintained until the callback is provided.

\adviceuserstart
It is strongly recommended that the \refapi{PMIx_Log_nb} API not be used by applications for streaming data as it is not a ``performant'' transport and can perturb the application since it involves the local \ac{PMIx} server and host \ac{SMS} daemon. Note that a return of \refconst{PMIX_SUCCESS} only denotes that the data was successfully handed to the appropriate system call (for local channels) or the host environment and does not indicate receipt at the final destination.
\adviceuserend


%%%%%%%%%%%%%%%%%%%%%%%%%%%%%%%%%%%%%%%%%%%%%%%%%
\subsection{Log attributes}
\label{api:struct:attributes:log}

Attributes used to describe \refapi{PMIx_Log} behavior - these are values passed to the \refapi{PMIx_Log} \ac{API} and therefore are not accessed using the \refapi{PMIx_Get} \ac{API}.

%
\declareAttribute{PMIX_LOG_SOURCE}{"pmix.log.source"}{pmix_proc_t*}{
ID of source of the log request.
}
%
\declareAttribute{PMIX_LOG_STDERR}{"pmix.log.stderr"}{char*}{
Log string to \code{stderr}.
}
%
\declareAttribute{PMIX_LOG_STDOUT}{"pmix.log.stdout"}{char*}{
Log string to \code{stdout}.
}
%
\declareAttribute{PMIX_LOG_SYSLOG}{"pmix.log.syslog"}{char*}{
Log data to syslog.
Defaults to \code{ERROR} priority.  Will log to global syslog if available, otherwise to local syslog.
}
%
\declareAttribute{PMIX_LOG_LOCAL_SYSLOG}{"pmix.log.lsys"}{char*}{
Log data to local syslog.
Defaults to \code{ERROR} priority.
}
%
\declareAttribute{PMIX_LOG_GLOBAL_SYSLOG}{"pmix.log.gsys"}{char*}{
Forward data to system ``gateway'' and log msg to that syslog
Defaults to \code{ERROR} priority.
}
%
\declareAttribute{PMIX_LOG_SYSLOG_PRI}{"pmix.log.syspri"}{int}{
Syslog priority level.
}
%
\declareAttribute{PMIX_LOG_TIMESTAMP}{"pmix.log.tstmp"}{time_t}{
Timestamp for log report.
}
%
\declareAttribute{PMIX_LOG_GENERATE_TIMESTAMP}{"pmix.log.gtstmp"}{bool}{
Generate timestamp for log.
}
%
\declareAttribute{PMIX_LOG_TAG_OUTPUT}{"pmix.log.tag"}{bool}{
Label the output stream with the channel name (e.g., ``stdout'').
}
%
\declareAttribute{PMIX_LOG_TIMESTAMP_OUTPUT}{"pmix.log.tsout"}{bool}{
Print timestamp in output string.
}
%
\declareAttribute{PMIX_LOG_XML_OUTPUT}{"pmix.log.xml"}{bool}{
Print the output stream in \ac{XML} format.
}
%
\declareAttribute{PMIX_LOG_ONCE}{"pmix.log.once"}{bool}{
Only log this once with whichever channel can first support it, taking the channels in priority order.
}
%
\declareAttribute{PMIX_LOG_MSG}{"pmix.log.msg"}{pmix_byte_object_t}{
Message blob to be sent somewhere.
}
%
\declareAttribute{PMIX_LOG_EMAIL}{"pmix.log.email"}{pmix_data_array_t}{
Log via email based on \refstruct{pmix_info_t} containing directives.
}
%
\declareAttribute{PMIX_LOG_EMAIL_ADDR}{"pmix.log.emaddr"}{char*}{
Comma-delimited list of email addresses that are to receive the message.
}
%
\declareAttribute{PMIX_LOG_EMAIL_SENDER_ADDR}{"pmix.log.emfaddr"}{char*}{
Return email address of sender.
}
%
\declareAttribute{PMIX_LOG_EMAIL_SUBJECT}{"pmix.log.emsub"}{char*}{
Subject line for email.
}
%
\declareAttribute{PMIX_LOG_EMAIL_MSG}{"pmix.log.emmsg"}{char*}{
Message to be included in email.
}
%
\declareAttribute{PMIX_LOG_EMAIL_SERVER}{"pmix.log.esrvr"}{char*}{
Hostname (or \ac{IP} address) of SMTP server.
}
%
\declareAttribute{PMIX_LOG_EMAIL_SRVR_PORT}{"pmix.log.esrvrprt"}{int32_t}{
Port the email server is listening to.
}
%
\declareAttribute{PMIX_LOG_GLOBAL_DATASTORE}{"pmix.log.gstore"}{bool}{
Store the log data in a global data store (e.g., database).
}
%
\declareAttribute{PMIX_LOG_JOB_RECORD}{"pmix.log.jrec"}{bool}{
Log the provided information to the host environment's job record.
}

%%%%%%%%%%%%%%%%%%%%%%%%%%%%%%%%%%%%%%%%%%%%%%%%%


    % Event Handling
    %  - (de)register_event, notify_event
    %%%%%%%%%%%%%%%%%%%%%%%%%%%%%%%%%%%%%%%%%%%%%%%%%
% Chapter: Events
%%%%%%%%%%%%%%%%%%%%%%%%%%%%%%%%%%%%%%%%%%%%%%%%%
\chapter{Event Notification}
\label{chap:api_event}

This chapter defines the \ac{PMIx} event notification system.
These interfaces are designed to support the reporting of events to/from clients and servers, and between library layers within a single process.

%%%%%%%%%%%%%%%%%%%%%%%%%%%%%%%%%%%%%%%%%%%%%%%%%
%%%%%%%%%%%%%%%%%%%%%%%%%%%%%%%%%%%%%%%%%%%%%%%%%
\section{Notification and Management}
\label{chap:api_event:notify}

\ac{PMIx} event notification provides an asynchronous out-of-band mechanism for communicating events between application processes and/or elements of the \ac{SMS}. Its uses span a wide range including fault notification, coordination between multiple programming libraries within a single process, and workflow orchestration for non-synchronous programming models. Events can be divided into two distinct classes:

\begin{itemize}
\item \textit{Job-specific events} directly relate to a job executing within the session, such as a debugger attachment, process failure within a related job, or events generated by an application process. Events in this category are to be immediately delivered to the \ac{PMIx} server library for relay to the related local processes.

\item \textit{Environment events} indirectly relate to a job but do not specifically target the job itself. This category includes \ac{SMS}-generated events such as \ac{ECC} errors, temperature excursions, and other non-job conditions that might directly affect a session's resources, but would never include an event generated by an application process. Note that although these do potentially impact the session's jobs, they are not directly tied to those jobs. Thus, events in this category are to be delivered to the \ac{PMIx} server library only upon request.
\end{itemize}

Both \ac{SMS} elements and applications can register for events of either type.

\adviceimplstart
Race conditions can cause the registration to come after events of possible interest (e.g., a memory \ac{ECC} event that occurs after start of execution but prior to registration, or an application process generating an event prior to another process registering to receive it). \ac{SMS} vendors are \textit{requested} to cache environment events for some time to mitigate this situation, but are not \textit{required} to do so. However, \ac{PMIx} implementers are \textit{required} to cache all events received by the \ac{PMIx} server library and to deliver them to registering clients in the same order in which they were received
\adviceimplend

\adviceuserstart
Applications must be aware that they may not receive environment events that occur prior to registration, depending upon the capabilities of the host \ac{SMS}.
\adviceuserend

The generator of an event can specify the \textit{target range} for delivery of that event. Thus, the generator can choose to limit notification to processes on the local node, processes within the same job as the generator, processes within the same allocation, other threads within the same process, only the \ac{SMS} (i.e., not to any application processes), all application processes, or to a custom range based on specific process identifiers. Only processes within the given range that register for the provided event code will be notified. In addition, the generator can use attributes to direct that the event not be delivered to any default event handlers, or to any multi-code handler (as defined below).

Event notifications provide the process identifier of the source of the event plus the event code and any additional information provided by the generator. When an event notification is received by a process, the registered handlers are scanned for their event code(s), with matching handlers assembled into an \textit{event chain} for servicing. Note that users can also specify a \textit{source range} when registering an event (using the same range designators described above) to further limit when they are to be invoked. When assembled, PMIx event chains are ordered based on both the specificity of the event handler and user directives at time of handler registration. By default, handlers are grouped into three categories based on the number of event codes that can trigger the callback:
\begin{itemize}
%
\item \textit{single-code} handlers are serviced first as they are the most specific. These are handlers that are registered against one specific event code.
%
\item \textit{multi-code} handlers are serviced once all single-code handlers have completed. The handler will be included in the chain upon receipt of an event matching any of the provided codes.
%
\item \textit{default} handlers are serviced once all multi-code handlers have completed. These handlers are always included in the chain unless the generator specifically excludes them.
%
\end{itemize}

Users can specify the callback order of a handler within its category at the time of registration. Ordering can be specified either by providing the relevant returned event handler registration ID or using event handler names, if the user specified an event handler name when registering the corresponding event. Thus, users can specify that a given handler be executed before or after another handler should both handlers appear in an event chain (the ordering is ignored if the other handler isn't included). Note that ordering does not imply immediate relationships. For example, multiple handlers registered to be serviced after event handler \textit{A} will all be executed after \textit{A}, but are not guaranteed to be executed in any particular order amongst themselves.

In addition, one event handler can be declared as the \textit{first} handler to be executed in the chain. This handler will \textit{always} be called prior to any other handler, regardless of category, provided the incoming event matches both the specified range and event code. Only one handler can be so designated --- attempts to designate additional handlers as \textit{first} will return an error. Deregistration of the declared \textit{first} handler will re-open the position for subsequent assignment.

Similarly, one event handler can be declared as the \textit{last} handler to be executed in the chain. This handler will \textit{always} be called after all other handlers have executed, regardless of category, provided the incoming event matches both the specified range and event code. Note that this handler will not be called if the chain is terminated by an earlier handler. Only one handler can be designated as \textit{last} --- attempts to designate additional handlers as \textit{last} will return an error. Deregistration of the declared \textit{last} handler will re-open the position for subsequent assignment.

\adviceuserstart
Note that the \textit{last} handler is called \textit{after} all registered default handlers that match the specified range of the incoming event unless a handler prior to it terminates the chain. Thus, if the application intends to define a \textit{last} handler, it should ensure that no default handler aborts the process before it.
\adviceuserend

Upon completing its work and prior to returning, each handler \textit{must} call the event handler completion function provided when it was invoked (including a status code plus any information to be passed to later handlers) so that the chain can continue being progressed. \ac{PMIx} automatically aggregates the status and any results of each handler (as provided in the completion callback) with status from all prior handlers so that each step in the chain has full knowledge of what preceded it. An event handler can terminate all further progress along the chain by passing the \refconst{PMIX_EVENT_ACTION_COMPLETE} status to the completion callback function.

\subsection{Events versus status constants}
\label{api:event:evssc}

Return status constants (see Section \ref{api:struct:errors}) represent values that can be returned from or passed into \ac{PMIx}
\acp{API}. These are distinct from \ac{PMIx} \emph{events} in that they are
not values that can be registered against event handlers. In general, the two
types of constants are distinguished by inclusion of an "ERR" in the name of
error constants versus an "EVENT" in events, though there are exceptions (e.g,
the \refconst{PMIX_SUCCESS} constant).


%%%%%%%%%%%%%%%%%%%%%%%%%%%%%%%%%%%%%%%%%%%%%%%%%
\subsection{\code{PMIx_Register_event_handler}}
\declareapi{PMIx_Register_event_handler}

%%%%
\summary

Register an event handler.

%%%%
\format

\versionMarker{2.0}
\cspecificstart
\begin{codepar}
pmix_status_t
PMIx_Register_event_handler(pmix_status_t codes[], size_t ncodes,
                            pmix_info_t info[], size_t ninfo,
                            pmix_notification_fn_t evhdlr,
                            pmix_hdlr_reg_cbfunc_t cbfunc,
                            void *cbdata);
\end{codepar}
\cspecificend

\begin{arglist}
\argin{codes}{Array of status codes (array of \refstruct{pmix_status_t})}
\argin{ncodes}{Number of elements in the \refarg{codes} array (\code{size_t})}
\argin{info}{Array of info structures (array of handles)}
\argin{ninfo}{Number of elements in the \refarg{info} array (\code{size_t})}
\argin{evhdlr}{Event handler to be called \refapi{pmix_notification_fn_t} (function reference)}
\argin{cbfunc}{Callback function \refapi{pmix_hdlr_reg_cbfunc_t} (function reference)}
\argin{cbdata}{Data to be passed to the cbfunc callback function (memory reference)}
\end{arglist}


If \refarg{cbfunc} is \code{NULL}, the function call will be treated as a \emph{blocking} call. In this case, the returned status will be either (a) the event handler reference identifier if the value is greater than or equal to zero, or (b) a negative error code indicative of the reason for the failure.

If the \refarg{cbfunc} is non-\code{NULL}, the function call will be treated as a \emph{non-blocking} call and will return the following:

\begin{itemize}
\item \refconst{PMIX_SUCCESS} indicating that the request has been accepted for processing and the provided callback function will be executed upon completion of the operation. Note that the library must not invoke the callback function prior to returning from the \ac{API}. The result of the registration operation shall be returned in the provided callback function along with the assigned event handler identifier.
\item \refconst{PMIX_ERR_EVENT_REGISTRATION} indicating that the registration
has failed for an undetermined reason.
\item a non-zero \ac{PMIx} error constant indicating a reason for the request to have been rejected. In this case, the provided callback function will not be executed.
\end{itemize}

The callback function must not be executed prior to returning from the \ac{API}, and no events corresponding to this registration may be delivered prior to the completion of the registration callback function (\refarg{cbfunc}).

\reqattrstart
The following attributes are required to be supported by all \ac{PMIx} libraries:

\pasteAttributeItem{PMIX_EVENT_HDLR_NAME}
\pasteAttributeItem{PMIX_EVENT_HDLR_FIRST}
\pasteAttributeItem{PMIX_EVENT_HDLR_LAST}
\pasteAttributeItem{PMIX_EVENT_HDLR_FIRST_IN_CATEGORY}
\pasteAttributeItem{PMIX_EVENT_HDLR_LAST_IN_CATEGORY}
\pasteAttributeItem{PMIX_EVENT_HDLR_BEFORE}
\pasteAttributeItem{PMIX_EVENT_HDLR_AFTER}
\pasteAttributeItem{PMIX_EVENT_HDLR_PREPEND}
\pasteAttributeItem{PMIX_EVENT_HDLR_APPEND}
\pasteAttributeItem{PMIX_EVENT_CUSTOM_RANGE}
\pasteAttributeItem{PMIX_RANGE}
\pasteAttributeItem{PMIX_EVENT_RETURN_OBJECT}

\divider

Host environments that implement support for \ac{PMIx} event notification are required to support the following attributes when registering handlers - these attributes are used to direct that the handler should be invoked only when the event affects the indicated process(es):

\pasteAttributeItem{PMIX_EVENT_AFFECTED_PROC}
\pasteAttributeItem{PMIX_EVENT_AFFECTED_PROCS}

\reqattrend


%%%%
\descr

Register an event handler to report events. Note that the codes being registered do \textit{not} need to be \ac{PMIx} error constants --- any integer value can be registered. This allows for registration of non-PMIx events such as those defined by a particular \ac{SMS} vendor or by an application itself.

\adviceuserstart
In order to avoid potential conflicts, users are advised to only define codes that lie outside the range of the \ac{PMIx} standard's error codes. Thus, \ac{SMS} vendors and application developers should constrain their definitions to positive values or negative values beyond the \refconst{PMIX_EXTERNAL_ERR_BASE} boundary.
\adviceuserend


\adviceuserstart
As previously stated, upon completing its work, and prior to returning, each handler \textit{must} call the event handler completion function provided when it was invoked (including a status code plus any information to be passed to later handlers) so that the chain can continue being progressed. An event handler can terminate all further progress along the chain by passing the \refconst{PMIX_EVENT_ACTION_COMPLETE} status to the completion callback function. Note that the parameters passed to the event handler (e.g., the \refarg{info} and \refarg{results} arrays) will cease to be valid once the completion function has been called - thus, any information in the incoming parameters that will be referenced following the call to the completion function must be copied.
\adviceuserend

%%%%%%%%%%%%%%%%%%%%%%%%%%%%%%%%%%%%%%%%%%%%%%%%%
\subsection{Event registration constants}
\label{api:struct:constants:event}

\begin{constantdesc}
%
\declareconstitem{PMIX_ERR_EVENT_REGISTRATION}
Error in event registration.
%
\end{constantdesc}

%%%%%%%%%%%%%%%%%%%%%%%%%%%%%%%%%%%%%%%%%%%%%%%%%
\subsection{System events}
\label{api:struct:sys:event}

\begin{constantdesc}
%
\declareconstitemNEW{PMIX_EVENT_SYS_BASE}
Mark the beginning of a dedicated range of constants for system event reporting.
%
\declareconstitemNEW{PMIX_EVENT_NODE_DOWN}
A node has gone down - the identifier of the affected node will be included in the notification.
%
\declareconstitemNEW{PMIX_EVENT_NODE_OFFLINE}
A node has been marked as \emph{offline} - the identifier of the affected node will be included in the notification.
%
\declareconstitemNEW{PMIX_EVENT_SYS_OTHER}
Mark the end of a dedicated range of constants for system event reporting.
%
\end{constantdesc}

\littleheader{Detect system event constant}
\declaremacro{PMIX_SYSTEM_EVENT}

Test a given event constant to see if it falls within the dedicated range of constants for system event reporting.

\versionMarker{2.2}
\cspecificstart
\begin{codepar}
PMIX_SYSTEM_EVENT(a)
\end{codepar}
\cspecificend

\begin{arglist}
\argin{a}{Error constant to be checked (\refstruct{pmix_status_t})}
\end{arglist}

Returns \code{true} if the provided values falls within the dedicated range of events for system event reporting.

%%%%%%%%%%%%%%%%%%%%%%%%%%%%%%%%%%%%%%%%%%%%%%%%%
\subsection{Event handler registration and notification attributes}
\label{api:struct:attributes:event}

Attributes to support event registration and notification.

%
\declareAttribute{PMIX_EVENT_HDLR_NAME}{"pmix.evname"}{char*}{
String name identifying this handler.
}
%
\declareAttribute{PMIX_EVENT_HDLR_FIRST}{"pmix.evfirst"}{bool}{
Invoke this event handler before any other handlers.
}
%
\declareAttribute{PMIX_EVENT_HDLR_LAST}{"pmix.evlast"}{bool}{
Invoke this event handler after all other handlers have been called.
}
%
\declareAttribute{PMIX_EVENT_HDLR_FIRST_IN_CATEGORY}{"pmix.evfirstcat"}{bool}{
Invoke this event handler before any other handlers in this category.
}
%
\declareAttribute{PMIX_EVENT_HDLR_LAST_IN_CATEGORY}{"pmix.evlastcat"}{bool}{
Invoke this event handler after all other handlers in this category have been called.
}
%
\declareAttribute{PMIX_EVENT_HDLR_BEFORE}{"pmix.evbefore"}{char*}{
Put this event handler immediately before the one specified in the \code{(char*)} value.
}
%
\declareAttribute{PMIX_EVENT_HDLR_AFTER}{"pmix.evafter"}{char*}{
Put this event handler immediately after the one specified in the \code{(char*)} value.
}
%
\declareAttribute{PMIX_EVENT_HDLR_PREPEND}{"pmix.evprepend"}{bool}{
Prepend this handler to the precedence list within its category.
}
%
\declareAttribute{PMIX_EVENT_HDLR_APPEND}{"pmix.evappend"}{bool}{
Append this handler to the precedence list within its category.
}
%
\declareAttribute{PMIX_EVENT_CUSTOM_RANGE}{"pmix.evrange"}{pmix_data_array_t*}{
Array of \refstruct{pmix_proc_t} defining range of event notification.
}
%
\declareAttribute{PMIX_EVENT_AFFECTED_PROC}{"pmix.evproc"}{pmix_proc_t}{
The single process that was affected.
}
%
\declareAttribute{PMIX_EVENT_AFFECTED_PROCS}{"pmix.evaffected"}{pmix_data_array_t*}{
Array of \refstruct{pmix_proc_t} defining affected processes.
}
%
\declareAttribute{PMIX_EVENT_NON_DEFAULT}{"pmix.evnondef"}{bool}{
Event is not to be delivered to default event handlers.
}
%
\declareAttribute{PMIX_EVENT_RETURN_OBJECT}{"pmix.evobject"}{void *}{
Object to be returned whenever the registered callback function \code{cbfunc} is invoked.
The object will only be returned to the process that registered it.
}
%
\declareAttribute{PMIX_EVENT_DO_NOT_CACHE}{"pmix.evnocache"}{bool}{
Instruct the \ac{PMIx} server not to cache the event.
}
%
\declareAttribute{PMIX_EVENT_PROXY}{"pmix.evproxy"}{pmix_proc_t*}{
\ac{PMIx} server that sourced the event.
}
%
\declareAttribute{PMIX_EVENT_TEXT_MESSAGE}{"pmix.evtext"}{char*}{
Text message suitable for output by recipient - e.g., describing the cause of the event.
}
%
\declareAttributeNEW{PMIX_EVENT_TIMESTAMP}{"pmix.evtstamp"}{time_t}{
System time when the associated event occurred.
}

%%%%%%%%%%%%%%%%%%%%%%%%%%%%%%%%%%%%%%%%%%%%%%%%%
\subsubsection{Fault tolerance event attributes}
\label{api:struct:attributes:ft}

The following attributes may be used by the host environment when providing an event notification as qualifiers indicating the action it intends to take in response to the event:

%
\declareAttribute{PMIX_EVENT_TERMINATE_SESSION}{"pmix.evterm.sess"}{bool}{
The \ac{RM} intends to terminate this session.
}
%
\declareAttribute{PMIX_EVENT_TERMINATE_JOB}{"pmix.evterm.job"}{bool}{
The \ac{RM} intends to terminate this job.
}
%
\declareAttribute{PMIX_EVENT_TERMINATE_NODE}{"pmix.evterm.node"}{bool}{
The \ac{RM} intends to terminate all processes on this node.
}
%
\declareAttribute{PMIX_EVENT_TERMINATE_PROC}{"pmix.evterm.proc"}{bool}{
The \ac{RM} intends to terminate just this process.
}
%
\declareAttribute{PMIX_EVENT_ACTION_TIMEOUT}{"pmix.evtimeout"}{int}{
The time in seconds before the \ac{RM} will execute the indicated operation.
}

%%%%%%%%%%%%%%%%%%%%%%%%%%%%%%%%%%%%%%%%%%%%%%%%%
\subsubsection{Hybrid programming event attributes}
\label{api:struct:attributes:hybrid}

The following attributes may be used by programming models to coordinate their use of common resources within a process in conjunction with the \refconst{PMIX_OPENMP_PARALLEL_ENTERED} event:
%
\pasteAttributeItem{PMIX_MODEL_PHASE_NAME}
\pasteAttributeItem{PMIX_MODEL_PHASE_TYPE}

%%%%%%%%%%%%%%%%%%%%%%%%%%%%%%%%%%%%%%%%%%%%%%%%%
\subsection{Notification Function}
\declareapi{pmix_notification_fn_t}

%%%%
\summary

The \refapi{pmix_notification_fn_t} is called by \ac{PMIx} to deliver notification of an event.

\adviceuserstart
The \ac{PMIx} \textit{ad hoc} v1.0 Standard defined an error notification function with an identical name, but different signature than the v2.0 Standard described below. The \textit{ad hoc} v1.0 version was removed from the v2.0 Standard is not included in this document to avoid confusion.
\adviceuserend


\versionMarker{2.0}
\cspecificstart
\begin{codepar}
typedef void (*pmix_notification_fn_t)
    (size_t evhdlr_registration_id,
     pmix_status_t status,
     const pmix_proc_t *source,
     pmix_info_t info[], size_t ninfo,
     pmix_info_t results[], size_t nresults,
     pmix_event_notification_cbfunc_fn_t cbfunc,
     void *cbdata);
\end{codepar}
\cspecificend

\begin{arglist}
\argin{evhdlr_registration_id}{Registration number of the handler being called (\code{size_t})}
\argin{status}{Status associated with the operation (\refstruct{pmix_status_t})}
\argin{source}{Identifier of the process that generated the event (\refstruct{pmix_proc_t})}. If the source is the \ac{SMS}, then the nspace will be empty and the rank will be PMIX_RANK_UNDEF
\argin{info}{Information describing the event (\refstruct{pmix_info_t})}. This argument will be NULL if no additional information was provided by the event generator.
\argin{ninfo}{Number of elements in the info array (\code{size_t})}
\argin{results}{Aggregated results from prior event handlers servicing this event (\refstruct{pmix_info_t})}. This argument will be \code{NULL} if this is the first handler servicing the event, or if no prior handlers provided results.
\argin{nresults}{Number of elements in the results array (\code{size_t})}
\argin{cbfunc}{\refapi{pmix_event_notification_cbfunc_fn_t} callback function to be executed upon completion of the handler's operation and prior to handler return (function reference)}.
\argin{cbdata}{Callback data to be passed to cbfunc (memory reference)}
\end{arglist}

%%%%
\descr

Note that different \acp{RM} may provide differing levels of support for event notification to application processes. Thus, the \refarg{info} array may be \code{NULL} or may contain detailed information of the event. It is the responsibility of the application to parse any provided info array for defined key-values if it so desires.

\adviceuserstart
Possible uses of the \refarg{info} array include:

\begin{itemize}
%
\item for the host \ac{RM} to alert the process as to planned actions, such as aborting the session, in response to the reported event
%
\item provide a timeout for alternative action to occur, such as for the application to request an alternate response to the event
%
\end{itemize}

For example, the \ac{RM} might alert the application to the failure of a node that resulted in termination of several processes, and indicate that the overall session will be aborted unless the application requests an alternative behavior in the next 5 seconds. The application then has time to respond with a checkpoint request, or a request to recover from the failure by obtaining replacement nodes and restarting from some earlier checkpoint.

Support for these options is left to the discretion of the host \ac{RM}. Info keys are included in the common definitions above but may be augmented by environment vendors.
\adviceuserend

\advicermstart
On the server side, the notification function is used to inform the \ac{PMIx} server library's host of a detected event in the \ac{PMIx} server library. Events generated by \ac{PMIx} clients are communicated to the \ac{PMIx} server library, but will be relayed to the host via the \refapi{pmix_server_notify_event_fn_t} function pointer, if provided.
\advicermend


%%%%%%%%%%%%%%%%%%%%%%%%%%%%%%%%%%%%%%%%%%%%%%%%%
\subsection{\code{PMIx_Deregister_event_handler}}
\declareapi{PMIx_Deregister_event_handler}

%%%%
\summary

Deregister an event handler.

%%%%
\format

\versionMarker{2.0}
\cspecificstart
\begin{codepar}
pmix_status_t
PMIx_Deregister_event_handler(size_t evhdlr_ref,
                              pmix_op_cbfunc_t cbfunc,
                              void *cbdata);
\end{codepar}
\cspecificend

\begin{arglist}
\argin{evhdlr_ref}{Event handler ID returned by registration (\code{size_t})}
\argin{cbfunc}{Callback function to be executed upon completion of operation \refapi{pmix_op_cbfunc_t} (function reference)}
\argin{cbdata}{Data to be passed to the cbfunc callback function (memory reference)}
\end{arglist}

If \refarg{cbfunc} is \code{NULL}, the function will be treated as a \emph{blocking} call and the result of the operation returned in the status code.

If \refarg{cbfunc} is non-\code{NULL}, the function will be treated as a \emph{non-blocking} call and return one of the following:

\begin{itemize}
\item \refconst{PMIX_SUCCESS}, indicating that the request is being processed - result will be returned in the provided \refarg{cbfunc}. Note that the library must not invoke the callback function prior to returning from the \ac{API}.
\item \refconst{PMIX_OPERATION_SUCCEEDED}, indicating that the request was immediately processed and returned \textit{success} - the \refarg{cbfunc} will \textit{not} be called
\item a PMIx error constant indicating either an error in the input or that the request was immediately processed and failed - the \refarg{cbfunc} will \textit{not} be called
\end{itemize}

The returned status code will be one of the following:

\begin{itemize}
\item \refconst{PMIX_SUCCESS} The event handler was successfully deregistered.
\item \refconst{PMIX_ERR_BAD_PARAM} The provided \refarg{evhdlr_ref} was unrecognized.
\item \refconst{PMIX_ERR_NOT_SUPPORTED} The \ac{PMIx} implementation does not support event notification.
\end{itemize}

%%%%
\descr

Deregister an event handler. Note that no events corresponding to the referenced registration may be delivered following completion of the deregistration operation (either return from the \ac{API} with \refconst{PMIX_OPERATION_SUCCEEDED} or execution of the \refarg{cbfunc}).

%%%%%%%%%%%%%%%%%%%%%%%%%%%%%%%%%%%%%%%%%%%%%%%%%
\subsection{\code{PMIx_Notify_event}}
\declareapi{PMIx_Notify_event}

%%%%
\summary

Report an event for notification via any
registered event handler.

%%%%
\format

\versionMarker{2.0}
\cspecificstart
\begin{codepar}
pmix_status_t
PMIx_Notify_event(pmix_status_t status,
                  const pmix_proc_t *source,
                  pmix_data_range_t range,
                  pmix_info_t info[], size_t ninfo,
                  pmix_op_cbfunc_t cbfunc, void *cbdata);
\end{codepar}
\cspecificend

\begin{arglist}
\argin{status}{Status code of the event (\refstruct{pmix_status_t})}
\argin{source}{Pointer to a \refstruct{pmix_proc_t} identifying the original reporter of the event (handle)}
\argin{range}{Range across which this notification shall be delivered (\refstruct{pmix_data_range_t})}
\argin{info}{Array of \refstruct{pmix_info_t} structures containing any further info provided by the originator of the event (array of handles)}
\argin{ninfo}{Number of elements in the \refarg{info} array (\code{size_t})}
\argin{cbfunc}{Callback function to be executed upon completion of operation \refapi{pmix_op_cbfunc_t} (function reference)}
\argin{cbdata}{Data to be passed to the cbfunc callback function (memory reference)}
\end{arglist}

If \refarg{cbfunc} is \code{NULL}, the function will be treated as a \emph{blocking} call and the result of the operation returned in the status code.

If \refarg{cbfunc} is non-\code{NULL}, the function will be treated as a \emph{non-blocking} call and return one of the following:

\begin{itemize}
\item \refconst{PMIX_SUCCESS} The notification request is valid and is being processed. The callback function will be called when the process-local operation is complete and will provide the resulting status of that operation. Note that this does \textit{not} reflect the success or failure of delivering the event to any recipients. The callback function must not be executed prior to returning from the \ac{API}.
\item \refconst{PMIX_OPERATION_SUCCEEDED}, indicating that the request was immediately processed and returned \textit{success} - the \refarg{cbfunc} will \textit{not} be called
\item \refconst{PMIX_ERR_BAD_PARAM} The request contains at least one incorrect entry that prevents it from being processed. The callback function will \textit{not} be called.
\item \refconst{PMIX_ERR_NOT_SUPPORTED} The \ac{PMIx} implementation does not support event notification, or in the case of a \ac{PMIx} server calling the API, the range extended beyond the local node and the host \ac{SMS} environment does not support event notification. The callback function will \textit{not} be called.
\end{itemize}

\reqattrstart
The following attributes are required to be supported by all \ac{PMIx} libraries:

\pasteAttributeItem{PMIX_EVENT_NON_DEFAULT}
\pasteAttributeItem{PMIX_EVENT_CUSTOM_RANGE}
\pasteAttributeItem{PMIX_EVENT_DO_NOT_CACHE}
\pasteAttributeItem{PMIX_EVENT_PROXY}
\pasteAttributeItem{PMIX_EVENT_TEXT_MESSAGE}

\divider

Host environments that implement support for \ac{PMIx} event notification are required to provide the following attributes for all events generated by the environment:

\pasteAttributeItem{PMIX_EVENT_AFFECTED_PROC}
\pasteAttributeItem{PMIX_EVENT_AFFECTED_PROCS}

\reqattrend

\optattrstart
Host environments that support \ac{PMIx} event notification may offer notifications for environmental events impacting the job and for \ac{SMS} events relating to the job. The following attributes may optionally be included to indicate the host environment's intended response to the event:

\pasteAttributeItem{PMIX_EVENT_TERMINATE_SESSION}
\pasteAttributeItem{PMIX_EVENT_TERMINATE_JOB}
\pasteAttributeItem{PMIX_EVENT_TERMINATE_NODE}
\pasteAttributeItem{PMIX_EVENT_TERMINATE_PROC}
\pasteAttributeItem{PMIX_EVENT_ACTION_TIMEOUT}

\optattrend

%%%%
\descr

Report an event for notification via any registered event handler. This function can be called by any \ac{PMIx} process, including application processes, \ac{PMIx} servers, and \ac{SMS} elements. The \ac{PMIx} server calls this \ac{API} to report events it detected itself so that the host \ac{SMS} daemon distribute and handle them, and to pass events given to it by its host down to any attached client processes for processing. Examples might include notification of the failure of another process, detection of an impending node failure due to rising temperatures, or an intent to preempt the application. Events may be locally generated or come from anywhere in the system.

Host \ac{SMS} daemons call the \ac{API} to pass events down to its embedded \ac{PMIx} server both for transmittal to local client processes and for the host's own internal processing where the host has registered its own event handlers. The \ac{PMIx} server library is not allowed to echo any event given to it by its host via this \ac{API} back to the host through the \refapi{pmix_server_notify_event_fn_t} server module function. The host is required to deliver the event to all \ac{PMIx} servers where the targeted processes either are currently running, or (if they haven't started yet) might be running at some point in the future as the events are required to be cached by the \ac{PMIx} server library.

Client application processes can call this function to notify the \ac{SMS} and/or other application processes of an event it encountered. Note that processes are not constrained to report status values defined in the official \ac{PMIx} standard --- any integer value can be used. Thus, applications are free to define their own internal events and use the notification system for their own internal purposes.

\adviceuserstart
The callback function will be called upon completion of the
\code{notify_event} function's actions. At that time, any messages required for executing the operation (e.g., to send the notification to the local \ac{PMIx} server) will
have been queued, but may not yet have been transmitted. The caller is required to maintain the input
data until the callback function has been executed --- the sole purpose of the callback function is to indicate when the input data is no longer required.
\adviceuserend

%%%%%%%%%%%%%%%%%%%%%%%%%%%%%%%%%%%%%%%%%%%%%%%%%
\subsection{Notification Handler Completion Callback Function}
\declareapi{pmix_event_notification_cbfunc_fn_t}

%%%%
\summary

The \refapi{pmix_event_notification_cbfunc_fn_t} is called by event handlers to indicate completion of their operations.

\versionMarker{2.0}
\cspecificstart
\begin{codepar}
typedef void (*pmix_event_notification_cbfunc_fn_t)
    (pmix_status_t status,
     pmix_info_t *results, size_t nresults,
     pmix_op_cbfunc_t cbfunc, void *thiscbdata,
     void *notification_cbdata);
\end{codepar}
\cspecificend

\begin{arglist}
\argin{status}{Status returned by the event handler's operation (\refstruct{pmix_status_t})}
\argin{results}{Results from this event handler's operation on the event (\refstruct{pmix_info_t})}
\argin{nresults}{Number of elements in the results array (\code{size_t})}
\argin{cbfunc}{\refapi{pmix_op_cbfunc_t} function to be executed when \ac{PMIx} completes processing the callback (function reference)}
\argin{thiscbdata}{Callback data that was passed in to the handler (memory reference)}
\argin{cbdata}{Callback data to be returned when \ac{PMIx} executes cbfunc (memory reference)}
\end{arglist}

%%%%
\descr

Define a callback by which an event handler can notify the \ac{PMIx} library that it has completed its response to the notification. The handler is \textit{required} to execute this callback so the library can determine if additional handlers need to be called. The handler shall return \refconst{PMIX_EVENT_ACTION_COMPLETE} if no further action is required. The return status of each event handler and any returned \refstruct{pmix_info_t} structures will be added to the \refarg{results} array of \refstruct{pmix_info_t} passed to any subsequent event handlers to help guide their operation.

If non-\code{NULL}, the provided callback function will be called to allow the event handler to release the provided info array and execute any other required cleanup operations.

%%%%%%%%%%%%%%%%%%%%%%%%%%%%%%%%%%%%%%%%%%%%%%%%%
\subsubsection{Completion Callback Function Status Codes}

The following status code may be returned indicating various actions taken by other event handlers.

\begin{constantdesc}
%
\declareconstitem{PMIX_EVENT_NO_ACTION_TAKEN}
Event handler: No action taken.
%
\declareconstitem{PMIX_EVENT_PARTIAL_ACTION_TAKEN}
Event handler: Partial action taken.
%
\declareconstitem{PMIX_EVENT_ACTION_DEFERRED}
Event handler: Action deferred.
%
\declareconstitem{PMIX_EVENT_ACTION_COMPLETE}
Event handler: Action complete.
%
\end{constantdesc}

%%%%%%%%%%%%%%%%%%%%%%%%%%%%%%%%%%%%%%%%%%%%%%%%%


    % Data Packing & Unpacking
    %  - (un)pack, copy
    %%%%%%%%%%%%%%%%%%%%%%%%%%%%%%%%%%%%%%%%%%%%%%%%%
% Chapter: Data Packing and Unpacking
%%%%%%%%%%%%%%%%%%%%%%%%%%%%%%%%%%%%%%%%%%%%%%%%%
\chapter{Data Packing and Unpacking}
\label{chap:api_data_mgmt}

\ac{PMIx} intentionally does not include support for internode communications in the standard, instead relying on its host \ac{SMS} environment to transfer any needed data and/or requests between nodes. These operations frequently involve PMIx-defined public data structures that include binary data. Many \ac{HPC} clusters are homogeneous, and so transferring the structures can be done rather simply. However, greater effort is required in heterogeneous environments to ensure binary data is correctly transferred. \ac{PMIx} buffer manipulation functions are provided for this purpose via standardized interfaces to ease adoption.

%%%%%%%%%%%
\section{Data Buffer Type}
\declarestruct{pmix_data_buffer_t}

The \refstruct{pmix_data_buffer_t} structure describes a data buffer used for packing and unpacking.

\versionMarker{2.0}
\cspecificstart
\begin{codepar}
typedef struct pmix_data_buffer \{
    /** Start of my memory */
    char *base_ptr;
    /** Where the next data will be packed to
        (within the allocated memory starting
        at base_ptr) */
    char *pack_ptr;
    /** Where the next data will be unpacked
        from (within the allocated memory
        starting as base_ptr) */
    char *unpack_ptr;
    /** Number of bytes allocated (starting
        at base_ptr) */
    size_t bytes_allocated;
    /** Number of bytes used by the buffer
        (i.e., amount of data -- including
        overhead -- packed in the buffer) */
    size_t bytes_used;
\} pmix_data_buffer_t;
\end{codepar}
\cspecificend


%%%%%%%%%%%%%%%%%%%%%%%%%%%%%%%%%%%%%%%%%%%%%%
%%%%%%%%%%%%%%%%%%%%%%%%%%%%%%%%%%%%%%%%%%%%%%
\section{Support Macros}
\label{chap:datamgt:macros}

\ac{PMIx} provides a set of convenience macros for creating, initiating, and releasing data buffers.

%%%%%%%%%%%
\subsection{\code{PMIX_DATA_BUFFER_CREATE}}
\declaremacro{PMIX_DATA_BUFFER_CREATE}

%%%%
\summary

Allocate memory for a \refstruct{pmix_data_buffer_t} object and initialize it

%%%%
\format

\versionMarker{2.0}
\cspecificstart
\begin{codepar}
PMIX_DATA_BUFFER_CREATE(buffer);
\end{codepar}
\cspecificend

\begin{arglist}
\argout{buffer}{Variable to be assigned the pointer to the allocated \refstruct{pmix_data_buffer_t} (handle)}
\end{arglist}

%%%%
\descr

This macro uses \textit{calloc} to allocate memory for the buffer and initialize all fields in it


%%%%%%%%%%%
\subsection{\code{PMIX_DATA_BUFFER_RELEASE}}
\declaremacro{PMIX_DATA_BUFFER_RELEASE}

%%%%
\summary

Free a \refstruct{pmix_data_buffer_t} object and the data it contains

%%%%
\format

\versionMarker{2.0}
\cspecificstart
\begin{codepar}
PMIX_DATA_BUFFER_RELEASE(buffer);
\end{codepar}
\cspecificend

\begin{arglist}
\argin{buffer}{Pointer to the \refstruct{pmix_data_buffer_t} to be released (handle)}
\end{arglist}

%%%%
\descr

Free's the data contained in the buffer, and then free's the buffer itself


%%%%%%%%%%%
\subsection{\code{PMIX_DATA_BUFFER_CONSTRUCT}}
\declaremacro{PMIX_DATA_BUFFER_CONSTRUCT}

%%%%
\summary

Initialize a statically declared \refstruct{pmix_data_buffer_t} object

%%%%
\format

\versionMarker{2.0}
\cspecificstart
\begin{codepar}
PMIX_DATA_BUFFER_CONSTRUCT(buffer);
\end{codepar}
\cspecificend

\begin{arglist}
\argin{buffer}{Pointer to the allocated \refstruct{pmix_data_buffer_t} that is to be initialized (handle)}
\end{arglist}

%%%%
\descr

Initialize a pre-allocated buffer object


%%%%%%%%%%%
\subsection{\code{PMIX_DATA_BUFFER_DESTRUCT}}
\declaremacro{PMIX_DATA_BUFFER_DESTRUCT}

%%%%
\summary

Release the data contained in a \refstruct{pmix_data_buffer_t} object

%%%%
\format

\versionMarker{2.0}
\cspecificstart
\begin{codepar}
PMIX_DATA_BUFFER_DESTRUCT(buffer);
\end{codepar}
\cspecificend

\begin{arglist}
\argin{buffer}{Pointer to the \refstruct{pmix_data_buffer_t} whose data is to be released (handle)}
\end{arglist}

%%%%
\descr

Free's the data contained in a \refstruct{pmix_data_buffer_t} object


%%%%%%%%%%%
\subsection{\code{PMIX_DATA_BUFFER_LOAD}}
\declaremacro{PMIX_DATA_BUFFER_LOAD}

%%%%
\summary

Load a blob into a \refstruct{pmix_data_buffer_t} object

%%%%
\format

\versionMarker{2.0}
\cspecificstart
\begin{codepar}
PMIX_DATA_BUFFER_LOAD(buffer, data, size);
\end{codepar}
\cspecificend

\begin{arglist}
\argin{buffer}{Pointer to a pre-allocated \refstruct{pmix_data_buffer_t} (handle)}
\argin{data}{Pointer to a blob (\code{char*})}
\argin{size}{Number of bytes in the blob {\code{size_t}}}
\end{arglist}

%%%%
\descr

Load the given data into the provided \refstruct{pmix_data_buffer_t} object, usually done in preparation for unpacking the provided data. Note that the data is \textit{not} copied into the buffer - thus, the blob must not be released until after operations on the buffer have completed.


%%%%%%%%%%%
\subsection{\code{PMIX_DATA_BUFFER_UNLOAD}}
\declaremacro{PMIX_DATA_BUFFER_UNLOAD}

%%%%
\summary

Unload the data from a \refstruct{pmix_data_buffer_t} object

%%%%
\format

\versionMarker{2.0}
\cspecificstart
\begin{codepar}
PMIX_DATA_BUFFER_UNLOAD(buffer, data, size);
\end{codepar}
\cspecificend

\begin{arglist}
\argin{buffer}{Pointer to the \refstruct{pmix_data_buffer_t} whose data is to be extracted (handle)}
\argout{data}{Variable to be assigned the pointer to the extracted blob (\code{void*})}
\argout{size}{Variable to be assigned the number of bytes in the blob {\code{size_t}}}
\end{arglist}

%%%%
\descr

Extract the data in a buffer, assigning the pointer to the data (and the number of bytes in the blob) to the provided variables, usually done to transmit the blob to a remote process for unpacking. The buffer's internal pointer will be set to NULL to protect the data upon buffer destruct or release - thus, the user is responsible for releasing the blob when done with it.


%%%%%%%%%%%%%%%%%%%%%%%%%%%%%%%%%%%%%%%%%%%%%%
%%%%%%%%%%%%%%%%%%%%%%%%%%%%%%%%%%%%%%%%%%%%%%
\section{General Routines}
\label{chap:data_mgmt:general}

The following routines are provided to support internode transfers in heterogeneous environments.

%%%%%%%%%%%
\subsection{\code{PMIx_Data_pack}}
\declareapi{PMIx_Data_pack}

%%%%
\summary

Pack one or more values of a specified type into a buffer, usually for transmission to another process

%%%%
\format

\versionMarker{2.0}
\cspecificstart
\begin{codepar}
pmix_status_t
PMIx_Data_pack(const pmix_proc_t *target,
               pmix_data_buffer_t *buffer,
               void *src, int32_t num_vals,
               pmix_data_type_t type);
\end{codepar}
\cspecificend

\begin{arglist}
\argin{target}{Pointer to a \refstruct{pmix_proc_t} containing the nspace/rank of the process that will be unpacking the final buffer. A NULL value may be used to indicate that the target is based on the same \ac{PMIx} version as the caller. Note that only the target's nspace is relevant. (handle)}
\argin{buffer}{Pointer to a \refstruct{pmix_data_buffer_t} where the packed data is to be stored (handle)}
\argin{src}{Pointer to a location where the data resides. Strings are to be passed as (char **) --- i.e., the caller must pass the address of the pointer to the string as the (void*). This allows the caller to pass multiple strings in a single call. (memory reference)}
\argin{num_vals}{Number of elements pointed to by the \refarg{src} pointer. A string value is counted as a single value regardless of length. The values must be contiguous in memory. Arrays of pointers (e.g., string arrays) should be contiguous, although the data pointed to need not be contiguous across array entries.(\code{int32_t})}
\argin{type}{The type of the data to be packed (\refstruct{pmix_data_type_t})}
\end{arglist}

Returns one of the following:
\begin{constantdesc}
\item \refconst{PMIX_SUCCESS} The data has been packed as requested
\item \refconst{PMIX_ERR_NOT_SUPPORTED} The \ac{PMIx} implementation does not support this function.
\item \refconst{PMIX_ERR_BAD_PARAM} The provided buffer or src is \code{NULL}
\item \refconst{PMIX_ERR_UNKNOWN_DATA_TYPE} The specified data type is not known to this implementation
\item \refconst{PMIX_ERR_OUT_OF_RESOURCE} Not enough memory to support the operation
\item \refconst{PMIX_ERROR} General error
\end{constantdesc}

%%%%
\descr

The pack function packs one or more values of a specified type into the specified buffer.  The buffer must have already been
initialized via the \refmacro{PMIX_DATA_BUFFER_CREATE} or \refmacro{PMIX_DATA_BUFFER_CONSTRUCT}
macros --- otherwise, \refapi{PMIx_Data_pack} will return an error.
Providing an unsupported type flag will likewise be reported as an error.

Note that any data to be packed that is not hard type cast (i.e.,
not type cast to a specific size) may lose precision when unpacked
by a non-homogeneous recipient.  The \refapi{PMIx_Data_pack} function will do its best to deal
with heterogeneity issues between the packer and unpacker in such
cases. Sending a number larger than can be handled by the recipient
will return an error code (generated upon unpacking) ---
the error cannot be detected during packing.

The namespace of the intended recipient of the packed buffer (i.e., the
process that will be unpacking it) is used solely to resolve any data type
differences between \ac{PMIx} versions. The recipient must, therefore, be
known to the user prior to calling the pack function so that the
\ac{PMIx} library is aware of the version the recipient is using. Note that
all processes in a given namespace are \textit{required} to use the same \ac{PMIx}
version --- thus, the caller must only know at least one process from the
target's namespace.


%%%%%%%%%%%
\subsection{\code{PMIx_Data_unpack}}
\declareapi{PMIx_Data_unpack}

%%%%
\summary

Unpack values from a \refstruct{pmix_data_buffer_t}

%%%%
\format

\versionMarker{2.0}
\cspecificstart
\begin{codepar}
pmix_status_t
PMIx_Data_unpack(const pmix_proc_t *source,
                 pmix_data_buffer_t *buffer, void *dest,
                 int32_t *max_num_values,
                 pmix_data_type_t type);

\end{codepar}
\cspecificend


\begin{arglist}
\argin{source}{Pointer to a \refstruct{pmix_proc_t} structure containing the nspace/rank of the process that packed the provided buffer. A NULL value may be used to indicate that the source is based on the same \ac{PMIx} version as the caller. Note that only the source's nspace is relevant. (handle)}
\argin{buffer}{A pointer to the buffer from which the value will be extracted. (handle)}
\arginout{dest}{A pointer to the memory location into which the data is to be stored. Note that these values will be stored contiguously in memory. For strings, this pointer must be to (char**) to provide a means of supporting multiple string operations. The unpack function will allocate memory for each string in the array - the caller must only provide adequate memory for the array of pointers. (\code{void*})}
\arginout{max_num_values}{The number of values to be unpacked --- upon completion, the parameter will be set to the actual number of values unpacked. In most cases, this should match the maximum number provided in the parameters --- but in no case will it exceed the value of this parameter.  Note that unpacking fewer values than are actually available will leave the buffer in an unpackable state --- the function will return an error code to warn of this condition.(\code{int32_t})}
\argin{type}{The type of the data to be unpacked --- must be one of the \ac{PMIx} defined data types (\refstruct{pmix_data_type_t})}
\end{arglist}

Returns one of the following:
\begin{constantdesc}
\item \refconst{PMIX_SUCCESS} The data has been unpacked as requested
\item \refconst{PMIX_ERR_NOT_SUPPORTED} The \ac{PMIx} implementation does not support this function.
\item \refconst{PMIX_ERR_BAD_PARAM} The provided buffer or dest is \code{NULL}
\item \refconst{PMIX_ERR_UNKNOWN_DATA_TYPE} The specified data type is not known to this implementation
\item \refconst{PMIX_ERR_OUT_OF_RESOURCE} Not enough memory to support the operation
\item \refconst{PMIX_ERROR} General error
\end{constantdesc}

%%%%
\descr

The unpack function unpacks the next value (or values) of a specified type from the given buffer. The buffer must have already been initialized via an \refmacro{PMIX_DATA_BUFFER_CREATE} or \refmacro{PMIX_DATA_BUFFER_CONSTRUCT} call (and assumedly filled with some data) --- otherwise, the unpack_value function will return an error. Providing an unsupported type flag will likewise be reported as an error, as will specifying a data type that \textit{does not} match the type of the next item in the buffer. An attempt to read beyond the end of the stored data held in the buffer will also return an error.

NOTE: it is possible for the buffer to be corrupted and that \ac{PMIx} will \textit{think} there is a proper variable type at the beginning of an unpack region --- but that the value is bogus (e.g., just a byte field in a string array that so happens to have a value that matches the specified data type flag). Therefore, the data type error check is \textit{not} completely safe.

Unpacking values is a "nondestructive" process --- i.e., the values are not removed from the buffer. It is therefore possible for the caller to re-unpack a value from the same buffer by resetting the unpack_ptr.

Warning: The caller is responsible for providing adequate memory storage for the requested data. The user must provide a parameter indicating the maximum number of values that can be unpacked into the allocated memory. If more values exist in the buffer than can fit into the memory storage, then the function will unpack what it can fit into that location and return an error code indicating that the buffer was only partially unpacked.

Note that any data that was not hard type cast (i.e., not type cast to a specific size) when packed may lose precision when unpacked by a non-homogeneous recipient. \ac{PMIx} will do its best to deal with heterogeneity issues between the packer and unpacker in such cases. Sending a number larger than can be handled by the recipient will return an error code generated upon unpacking --- these errors cannot be detected during packing.

The namespace of the process that packed the buffer is used solely to resolve any data type
differences between \ac{PMIx} versions. The packer must, therefore, be
known to the user prior to calling the pack function so that the
\ac{PMIx} library is aware of the version the packer is using. Note that
all processes in a given namespace are \textit{required} to use the same \ac{PMIx}
version --- thus, the caller must only know at least one process from the
packer's namespace.


%%%%%%%%%%%
\subsection{\code{PMIx_Data_copy}}
\declareapi{PMIx_Data_copy}

%%%%
\summary

Copy a data value from one location to another.

%%%%
\format

\versionMarker{2.0}
\cspecificstart
\begin{codepar}
pmix_status_t
PMIx_Data_copy(void **dest, void *src,
               pmix_data_type_t type);
\end{codepar}
\cspecificend

\begin{arglist}
\argin{dest}{The address of a pointer into which the address of the resulting data is to be stored. (\code{void**})}
\argin{src}{A pointer to the memory location from which the data is to be copied (handle)}
\argin{type}{The type of the data to be copied --- must be one of the PMIx defined data types. (\refstruct{pmix_data_type_t})}
\end{arglist}

Returns one of the following:
\begin{constantdesc}
\item \refconst{PMIX_SUCCESS} The data has been copied as requested
\item \refconst{PMIX_ERR_NOT_SUPPORTED} The \ac{PMIx} implementation does not support this function.
\item \refconst{PMIX_ERR_BAD_PARAM} The provided src or dest is \code{NULL}
\item \refconst{PMIX_ERR_UNKNOWN_DATA_TYPE} The specified data type is not known to this implementation
\item \refconst{PMIX_ERR_OUT_OF_RESOURCE} Not enough memory to support the operation
\item \refconst{PMIX_ERROR} General error
\end{constantdesc}

%%%%
\descr

Since registered data types can be complex structures, the system needs some way to know how to copy the data from one location to another (e.g., for storage in the registry). This function, which can call other copy functions to build up complex data types, defines the method for making a copy of the specified data type.


%%%%%%%%%%%
\subsection{\code{PMIx_Data_print}}
\declareapi{PMIx_Data_print}

%%%%
\summary

Pretty-print a data value.

%%%%
\format

\versionMarker{2.0}
\cspecificstart
\begin{codepar}
pmix_status_t
PMIx_Data_print(char **output, char *prefix,
                void *src, pmix_data_type_t type);
\end{codepar}
\cspecificend

\begin{arglist}
\argin{output}{The address of a pointer into which the address of the resulting output is to be stored. (\code{char**})}
\argin{prefix}{String to be prepended to the resulting output (\code{char*})}
\argin{src}{A pointer to the memory location of the data value to be printed (handle)}
\argin{type}{The type of the data value to be printed --- must be one of the PMIx defined data types. (\refstruct{pmix_data_type_t})}
\end{arglist}

Returns one of the following:
\begin{constantdesc}
\item \refconst{PMIX_SUCCESS} The data has been printed as requested
\item \refconst{PMIX_ERR_BAD_PARAM} The provided data type is not recognized.
\item \refconst{PMIX_ERR_NOT_SUPPORTED} The \ac{PMIx} implementation does not support this function.
\end{constantdesc}

%%%%
\descr

Since registered data types can be complex structures, the system needs some way to know how to print them (i.e., convert them to a string representation). Primarily for debug purposes.


%%%%%%%%%%%
\subsection{\code{PMIx_Data_copy_payload}}
\declareapi{PMIx_Data_copy_payload}

%%%%
\summary

Copy a payload from one buffer to another

%%%%
\format

\versionMarker{2.0}
\cspecificstart
\begin{codepar}
pmix_status_t
PMIx_Data_copy_payload(pmix_data_buffer_t *dest,
                       pmix_data_buffer_t *src);
\end{codepar}
\cspecificend

\begin{arglist}
\argin{dest}{Pointer to the destination \refstruct{pmix_data_buffer_t} (handle)}
\argin{src}{Pointer to the source \refstruct{pmix_data_buffer_t} (handle)}
\end{arglist}

Returns one of the following:
\begin{constantdesc}
\item \refconst{PMIX_SUCCESS} The data has been copied as requested
\item \refconst{PMIX_ERR_BAD_PARAM} The src and dest \refstruct{pmix_data_buffer_t} types do not match
\item \refconst{PMIX_ERR_NOT_SUPPORTED} The \ac{PMIx} implementation does not support this function.
\end{constantdesc}

%%%%
\descr

This function will append a copy of the payload in one buffer into another buffer. Note that this is \textit{not} a destructive procedure --- the source buffer's payload will remain intact, as will any pre-existing payload in the destination's buffer. Only the unpacked portion of the source payload will be copied.


%%%%%%%%%%%%%%%%%%%%%%%%%%%%%%%%%%%%%%%%%%%%%%%%%


    % Security credentials
    \input{Chap_API_Security.tex}

    % PMIx Server Specific Interfaces
    %  - setup_fork, (de)register_nspace, pmix_server_module_t
    %%%%%%%%%%%%%%%%%%%%%%%%%%%%%%%%%%%%%%%%%%%%%%%%%
% Chapter: API Server
%%%%%%%%%%%%%%%%%%%%%%%%%%%%%%%%%%%%%%%%%%%%%%%%%
\chapter{Server-Specific Interfaces}
\label{chap:api_server}

The \ac{RM} daemon that hosts the \ac{PMIx} server library interacts with that library in two distinct manners. First, \ac{PMIx} provides a set of \acp{API} by which the host can request specific services from its library. This includes generating regular expressions, registering information to be passed to client processes, and requesting information on behalf of a remote process. Note that the host always has access to all \ac{PMIx} client \acp{API} - the functions listed below are in addition to those available to a \ac{PMIx} client.

Second, the host can provide a set of callback functions by which the \ac{PMIx} server library can pass requests upward for servicing by the host. These include notifications of client connection and finalize, as well as requests by clients for information and/or services that the \ac{PMIx} server library does not itself provide.

%%%%%%%%%%%
\section{Server Support Functions}

The following \acp{API} allow the \ac{RM} daemon that hosts the \ac{PMIx} server library to request specific services from the \ac{PMIx} library.

%%%%%%%%%%%
\subsection{\code{PMIx_generate_regex}}
\declareapi{PMIx_generate_regex}

%%%%
\summary

Generate a compressed representation of the input string.

%%%%
\format

\versionMarker{1.0}
\cspecificstart
\begin{codepar}
pmix_status_t
PMIx_generate_regex(const char *input, char **output)
\end{codepar}
\cspecificend

\begin{arglist}
\argin{input}{String to process (string)}
\argout{output}{Compressed representation of \refarg{input} (array of bytes)}
\end{arglist}

Returns \refconst{PMIX_SUCCESS} or a negative value corresponding to a PMIx error constant.

%%%%
\descr

Given a comma-separated list of \refarg{input} values, generate a reduced size representation of the input that can be passed down to the \ac{PMIx} server library's \refapi{PMIx_server_register_nspace} \ac{API} for parsing. The order of the individual values in the \refarg{input} string is preserved across the operation. The caller is responsible for releasing the returned data.

\adviceuserstart
The returned representation may be an arbitrary array of bytes as opposed to a valid NULL-terminated string. However, the method used to generate the representation shall be identified with a colon-delimited string at the beginning of the output. For example, an output starting with \code{"pmix:"} indicates that the representation is a \ac{PMIx}-defined regular expression. In contrast, an output starting with \code{"blob:"} is a compressed binary array.
\adviceuserend


%%%%%%%%%%%
\subsection{\code{PMIx_generate_ppn}}
\declareapi{PMIx_generate_ppn}

%%%%
\summary

Generate a compressed representation of the input identifying the processes on each node.

%%%%
\format

\versionMarker{1.0}
\cspecificstart
\begin{codepar}
pmix_status_t PMIx_generate_ppn(const char *input, char **ppn)
\end{codepar}
\cspecificend

\begin{arglist}
\argin{input}{String to process (string)}
\argout{ppn}{Compressed representation of \refarg{input} (array of bytes)}
\end{arglist}

Returns \refconst{PMIX_SUCCESS} or a negative value corresponding to a PMIx error constant.

%%%%
\descr

The input shall consist of a semicolon-separated list of ranges representing the ranks of processes on each node of the job - e.g.,  "1-4;2-5;8,10,11,12;6,7,9". Each field of the input must correspond to the node name provided at that position in the input to \refapi{PMIx_generate_regex}. Thus, in the example, ranks 1-4 would be located on the first node of the comma-separated list of names provided to \refapi{PMIx_generate_regex}, and ranks 2-5 would be on the second name in the list.

\adviceuserstart
The returned representation may be an arbitrary array of bytes as opposed to a valid NULL-terminated string. However, the method used to generate the representation shall be identified with a colon-delimited string at the beginning of the output. For example, an output starting with \code{"pmix:"} indicates that the representation is a \ac{PMIx}-defined regular expression. In contrast, an output starting with \code{"blob:"} is a compressed binary array.
\adviceuserend

%%%%%%%%%%%
\subsection{\code{PMIx_server_register_nspace}}
\declareapi{PMIx_server_register_nspace}

%%%%
\summary

Setup the data about a particular namespace.

%%%%
\format

\versionMarker{1.0}
\cspecificstart
\begin{codepar}
pmix_status_t
PMIx_server_register_nspace(const pmix_nspace_t nspace,
                        int nlocalprocs,
                        pmix_info_t info[], size_t ninfo,
                        pmix_op_cbfunc_t cbfunc, void *cbdata)
\end{codepar}
\cspecificend

\begin{arglist}
\argin{nspace}{namespace (string)}
\argin{nlocalprocs}{number of local processes (integer)}
\argin{info}{Array of info structures (array of handles)}
\argin{ninfo}{Number of elements in the \refarg{info} array (integer)}
\argin{cbfunc}{Callback function \refapi{pmix_op_cbfunc_t} (function reference)}
\argin{cbdata}{Data to be passed to the callback function (memory reference)}
\end{arglist}

Returns one of the following:

\begin{itemize}
    \item \refconst{PMIX_SUCCESS}, indicating that the request is being processed by the host environment - result will be returned in the provided \refarg{cbfunc}. Note that the library must not invoke the callback function prior to returning from the \ac{API}.
    \item \refconst{PMIX_OPERATION_SUCCEEDED}, indicating that the request was immediately processed and returned \textit{success} - the \refarg{cbfunc} will not be called
    \item a PMIx error constant indicating either an error in the input or that the request was immediately processed and failed - the \refarg{cbfunc} will not be called
\end{itemize}

\reqattrstart
The following attributes are required to be supported by all \ac{PMIx} libraries:

\pastePRIAttributeItem{PMIX_REGISTER_NODATA}

\divider

Host environments are required to provide the following attributes:

\begin{itemize}
    \item for the session containing the given namespace:
        \begin{itemize}
            \item \pastePRRTEAttributeItem{PMIX_UNIV_SIZE}
        \end{itemize}
    \item for the given namespace:
        \begin{itemize}
            \item \pastePRRTEAttributeItem{PMIX_JOBID}
            \item \pastePRRTEAttributeItem{PMIX_JOB_SIZE}
            \item \pastePRRTEAttributeItem{PMIX_MAX_PROCS}
            \item \pastePRRTEAttributeItem{PMIX_NODE_MAP}
            \item \pastePRRTEAttributeItem{PMIX_PROC_MAP}
        \end{itemize}
    \item for its own node:
        \begin{itemize}
            \item \pastePRRTEAttributeItem{PMIX_LOCAL_SIZE}
            \item \pastePRRTEAttributeItem{PMIX_LOCAL_PEERS}
            \item \pastePRRTEAttributeItem{PMIX_LOCAL_CPUSETS}
        \end{itemize}
    \item for each process in the given namespace:
        \begin{itemize}
            \item \pastePRRTEAttributeItem{PMIX_RANK}
            \item \pastePRRTEAttributeItem{PMIX_LOCAL_RANK}
            \item \pastePRRTEAttributeItem{PMIX_NODE_RANK}
            \item \pastePRRTEAttributeItem{PMIX_NODEID}
        \end{itemize}
\end{itemize}

If more than one application is included in the namespace, then the host environment is also required to provide the following attributes:

\begin{itemize}
    \item for each application:
        \begin{itemize}
            \item \pastePRRTEAttributeItem{PMIX_APPNUM}
            \item \pastePRRTEAttributeItem{PMIX_APPLDR}
            \item \pastePRRTEAttributeItem{PMIX_APP_SIZE}
        \end{itemize}
    \item for each process:
        \begin{itemize}
            \item \pastePRRTEAttributeItem{PMIX_APP_RANK}
            \item \pastePRRTEAttributeItem{PMIX_APPNUM}
        \end{itemize}
\end{itemize}

\reqattrend


\optattrstart

The following attributes may be provided by host environments:

\begin{itemize}
    \item for the session containing the given namespace:
        \begin{itemize}
            \item \pastePRRTEAttributeItem{PMIX_SESSION_ID}
        \end{itemize}
    \item for the given namespace:
        \begin{itemize}
            \item \pastePRRTEAttributeItem{PMIX_SERVER_NSPACE}
            \item \pastePRRTEAttributeItem{PMIX_SERVER_RANK}
            \item \pastePRRTEAttributeItem{PMIX_NPROC_OFFSET}
            \item \pastePRRTEAttributeItem{PMIX_ALLOCATED_NODELIST}
            \item \pastePRRTEAttributeItem{PMIX_JOB_NUM_APPS}
            \item \pastePRRTEAttributeItem{PMIX_MAPBY}
            \item \pastePRRTEAttributeItem{PMIX_RANKBY}
            \item \pastePRRTEAttributeItem{PMIX_BINDTO}
        \end{itemize}
    \item for its own node:
        \begin{itemize}
            \item \pastePRRTEAttributeItem{PMIX_AVAIL_PHYS_MEMORY}
            \item \pastePRRTEAttributeItem{PMIX_HWLOC_XML_V1}
            \item \pastePRRTEAttributeItem{PMIX_HWLOC_XML_V2}
            \item \pastePRRTEAttributeItem{PMIX_LOCALLDR}
            \item \pastePRRTEAttributeItem{PMIX_NODE_SIZE}
            \item \pastePRRTEAttributeItem{PMIX_LOCAL_PROCS}
        \end{itemize}
    \item for each process in the given namespace:
        \begin{itemize}
            \item \pastePRRTEAttributeItem{PMIX_PROCID}
            \item \pastePRRTEAttributeItem{PMIX_GLOBAL_RANK}
            \item \pastePRRTEAttributeItem{PMIX_HOSTNAME}
        \end{itemize}
\end{itemize}

Attributes not directly provided by the host environment may be derived by the \ac{PMIx} server library from other required information and included in the data made available to the server library's clients.

\optattrend

%%%%
\descr

Pass job-related information to the \ac{PMIx} server library for distribution to local client processes.

\advicermstart
Host environments are required to execute this operation prior to starting any local application process within the given namespace.

The \ac{PMIx} server must register all namespaces that will participate in collective operations with local processes.
This means that the server must register a namespace even if it will not host any local processes from within that namespace if any local process of another namespace might at some point perform an operation involving one or more processes from the new namespace.
This is necessary so that the collective operation can identify the participants and know when it is locally complete.

The caller must also provide the number of local processes that will be launched within this namespace.
This is required for the \ac{PMIx} server library to correctly handle collectives as a collective operation call can occur before all the local processes have been started.
\advicermend

\adviceuserstart
The number of local processes for any given namespace is generally fixed at the time of application launch. Calls to \refapi{PMIx_Spawn} result in processes launched in their own namespace, not that of their parent. However, it is possible for processes to \textit{migrate} to another node via a call to \refapi{PMIx_Job_control_nb}, thus resulting in a change to the number of local processes on both the initial node and the node to which the process moved. It is therefore critical that applications not migrate processes without first ensuring that \ac{PMIx}-based collective operations are not in progress, and that no such operations be initiated until process migration has completed.
\adviceuserend


%%%%%%%%%%%
\subsubsection{Assembling the registration information}
\label{chap:api_server:assemble}

The following description is not intended to represent the actual layout of information in a given \ac{PMIx} library. Instead, it is describes how information provided in the \refarg{info} parameter of the \refapi{PMIx_server_register_nspace} shall be organized for proper processing by a \ac{PMIx} server library. The ordering of the various information elements is arbitrary - they are presented in a top-down hierarchical form solely for clarity in reading.

\advicermstart
Creating the \refarg{info} array of data requires knowing in advance the number of elements required for the array. This can be difficult to compute and somewhat fragile in practice. One method for resolving the problem is to create a linked list of objects, each containing a single \refstruct{pmix_info_t} structure. Allocation and manipulation of the list can then be accomplished using existing standard methods. Upon completion, the final \refarg{info} array can be allocated based on the number of elements on the list, and then the values in the list object \refstruct{pmix_info_t} structures transferred to the corresponding array element utilizing the \refmacro{PMIX_INFO_XFER} macro.
\advicermend

\label{cptr:api_server:noderegex}A common building block used in several areas is the construction of a regular expression identifying the nodes involved in that area - e.g., the nodes in a \refterm{session} or \refterm{job}. \ac{PMIx} provides several tools to facilitate this operation, beginning by constructing an argv-like array of node names. This array is then passed to the \refapi{PMIx_generate_regex} function to create a regular expression parseable by the \ac{PMIx} server library, as shown below:

\cspecificstart
\begin{codepar}
char **nodes = NULL;
char *nodelist;
char *regex;
size_t n;
pmix_status_t rc;
pmix_info_t info;

/* loop over an array of nodes, adding each
 * name to the array */
for (n=0; n < num_nodes; n++) {
    /* filter the nodes to ignore those not included
     * in the target range (session, job, etc.). In
     * this example, all nodes are accepted */
    PMIX_ARGV_APPEND(&nodes, node[n]->name);
}

/* join into a comma-delimited string */
nodelist = PMIX_ARGV_JOIN(nodes, ',');

/* release the array */
PMIX_ARGV_FREE(nodes);

/* generate regex */
rc = PMIx_generate_regex(nodelist, &regex);

/* release list */
free(nodelist);

/* pass the regex as the value to the PMIX_NODE_MAP key */
PMIX_INFO_LOAD(&info, PMIX_NODE_MAP, regex, PMIX_STRING);
/* release the regex */
free(regex);

\end{codepar}
\cspecificend

Changing the filter criteria allows the construction of node maps for any level of information.

\label{cptr:api_server:ppnregex}A similar method is used to construct the map of processes on each node from the namespace being registered. This may be done for each information level of interest (e.g., to identify the process map for the entire \refterm{job} or for each \refterm{application} in the job) by changing the search criteria. An example is shown below for the case of creating the process map for a \refterm{job}:

\cspecificstart
\begin{codepar}
char **ndppn;
char rank[30];
char **ppnarray = NULL;
char *ppn;
char *localranks;
char *regex;
size_t n, m;
pmix_status_t rc;
pmix_info_t info;

/* loop over an array of nodes */
for (n=0; n < num_nodes; n++) {
    /* for each node, construct an array of ranks on that node */
    ndppn = NULL;
    for (m=0; m < node[n]->num_procs; m++) {
        /* ignore processes that are not part of the target job */
        if (!PMIX_CHECK_NSPACE(targetjob,node[n]->proc[m].nspace)) {
            continue;
        }
        snprintf(rank, 30, "%d", node[n]->proc[m].rank);
        PMIX_ARGV_APPEND(&ndppn, rank);
    }
    /* convert the array into a comma-delimited string of ranks */
    localranks = PMIX_ARGV_JOIN(ndppn, ',');
    /* release the local array */
    PMIX_ARGV_FREE(ndppn);
    /* add this node's contribution to the overall array */
    PMIX_ARGV_APPEND(&ppnarray, localranks);
    /* release the local list */
    free(localranks);
}

/* join into a semicolon-delimited string */
ppn = PMIX_ARGV_JOIN(ppnarray, ';');

/* release the array */
PMIX_ARGV_FREE(ppnarray);

/* generate ppn regex */
rc = PMIx_generate_ppn(ppn, &regex);

/* release list */
free(ppn);

/* pass the regex as the value to the PMIX_PROC_MAP key */
PMIX_INFO_LOAD(&info, PMIX_PROC_MAP, regex, PMIX_STRING);
/* release the regex */
free(regex);

\end{codepar}
\cspecificend

Note that the \refattr{PMIX_NODE_MAP} and \refattr{PMIX_PROC_MAP} attributes are linked in that the order of entries in the process map must match the ordering of nodes in the node map - i.e., there is no provision in the \ac{PMIx} process map regular expression generator/parser pair supporting an out-of-order node or a node that has no corresponding process map entry (e.g., a node with no processes on it). Armed with these tools, the registration \refarg{info} array can be constructed as follows:

\begin{itemize}
\item Session-level information includes all session-specific values. In many cases, only two values (\refattr{PMIX_SESSION_ID} and \refattr{PMIX_UNIV_SIZE}) are included in the registration array. Since both of these values are session-specific, they can be specified independently - i.e., in their own \refstruct{pmix_info_t} elements of the \refarg{info} array. Alternatively, they can be provided as a \refstruct{pmix_data_array_t} array of \refstruct{pmix_info_t} using the \refattr{PMIX_SESSION_INFO_ARRAY} attribute and identifed by including the \refattr{PMIX_SESSION_ID} attribute in the array - this is must in cases where non-specific attributes (e.g., \refattr{PMIX_NUM_NODES} or \refattr{PMIX_NODE_MAP}) are passed to describe aspects of the session. Note that the node map can include nodes not used by the job being registered as no corresponding process map is specified.

The \refarg{info} array at this point might look like (where the labels identify the corresponding attribute - e.g., ``Session ID'' corresponds to the \refattr{PMIX_SESSION_ID} attribute):

\begingroup
\begin{figure*}[ht!]
  \begin{center}
    \includegraphics[clip,width=0.3\textwidth]{figs/sessioninfo.pdf}
  \end{center}
  \caption{Session-level information elements}
  \label{fig:sessioninfo}
\end{figure*}
\endgroup


\item Job-level information includes all job-specific values such as \refattr{PMIX_JOB_SIZE}, \refattr{PMIX_JOB_NUM_APPS}, and \refattr{PMIX_JOBID}. Since each invocation of \refapi{PMIx_server_register_nspace} describes a single \refterm{job}, job-specific values can be specified independently - i.e., in their own \refstruct{pmix_info_t} elements of the \refarg{info} array. Alternatively, they can be provided as a \refstruct{pmix_data_array_t} array of \refstruct{pmix_info_t} identified by the \refattr{PMIX_JOB_INFO_ARRAY} attribute - this is must in cases where non-specific attributes (e.g., \refattr{PMIX_NODE_MAP}) are passed to describe aspects of the job. Note that since the invocation only involves a single namespace, there is no need to include the \refattr{PMIX_NSPACE} attribute in the array.

Upon conclusion of this step, the \refarg{info} array might look like:

\begingroup
\begin{figure*}[ht!]
  \begin{center}
    \includegraphics[clip,width=0.4\textwidth]{figs/jobinfo.pdf}
  \end{center}
  \caption{Job-level information elements}
  \label{fig:jobinfo}
\end{figure*}
\endgroup

Note that in this example, \refattr{PMIX_NUM_NODES} is not required as that information is contained in the \refattr{PMIX_NODE_MAP} attribute. Similarly, \refattr{PMIX_JOB_SIZE} is not technically required as that information is contained in the \refattr{PMIX_PROC_MAP} when combined with the corresponding node map - however, there is no issue with including the job size as a separate entry.

The example also illustrates the hierarchical use of the \refattr{PMIX_NODE_INFO_ARRAY} attribute. In this case, we have chosen to pass several job-related values for each node - since those values are non-unique across the job, they must be passed in a node-info container. Note that the choice of what information to pass into the \ac{PMIx} server library versus what information to derive from other values at time of request is left to the host environment. \ac{PMIx} implementors in turn may, if they choose, pre-parse registration data to create expanded views (thus enabling faster response to requests at the expense of memory footprint) or to compress views into tighter representations (thus trading minimized footprint for longer response times).

\item Application-level information includes all application-specific values such as \refattr{PMIX_APP_SIZE} and \refattr{PMIX_APPLDR}. If the \refterm{job} contains only a single \refterm{application}, then the application-specific values can be specified independently - i.e., in their own \refstruct{pmix_info_t} elements of the \refarg{info} array - or as a \refstruct{pmix_data_array_t} array of \refstruct{pmix_info_t} using the \refattr{PMIX_APP_INFO_ARRAY} attribute and identifed by including the \refattr{PMIX_APPNUM} attribute in the array. Use of the array format is  must in cases where non-specific attributes (e.g., \refattr{PMIX_NODE_MAP}) are passed to describe aspects of the application.

However, in the case of a job consisting of multiple applications, all application-specific values for each application must be provided using the \refattr{PMIX_APP_INFO_ARRAY} format, each identified by its \refattr{PMIX_APPNUM} value.

Upon conclusion of this step, the \refarg{info} array might look like that shown in \ref{fig:appinfo}, assuming there are two applications in the job being registered:

\begingroup
\begin{figure*}[ht!]
  \begin{center}
    \includegraphics[clip,width=0.5\textwidth]{figs/appinfo.pdf}
  \end{center}
  \caption{Application-level information elements}
  \label{fig:appinfo}
\end{figure*}
\endgroup

\item Process-level information includes an entry for each process in the job being registered, each entry marked with the \refattr{PMIX_PROC_DATA} attribute. The \refterm{rank} of the process must be the first entry in the array - this provides efficiency when storing the data. Upon conclusion of this step, the \refarg{info} array might look like the diagram in \ref{fig:procinfo}:

\begingroup
\begin{figure*}[ht!]
  \begin{center}
    \includegraphics[clip,width=0.5\textwidth]{figs/procinfo.pdf}
  \end{center}
  \caption{Process-level information elements}
  \label{fig:procinfo}
\end{figure*}
\endgroup

\item For purposes of this example, node-level information only includes values describing the local node - i.e., it does not include information about other nodes in the job or session. In many cases, the values included in this level are unique to it and can be specified independently - i.e., in their own \refstruct{pmix_info_t} elements of the \refarg{info} array. Alternatively, they can be provided as a \refstruct{pmix_data_array_t} array of \refstruct{pmix_info_t} using the \refattr{PMIX_NODE_INFO_ARRAY} attribute - this is must in cases where non-specific attributes are passed to describe aspects of the node, or where values for multiple nodes are being provided.

The node-level information requires two elements that must be constructed in a manner similar to that used for the node map. The \refattr{PMIX_LOCAL_PEERS} value is computed based on the processes on the local node, filtered to select those from the job being registered, as shown below using the tools provided by \ac{PMIx}:

\cspecificstart
\begin{codepar}
char **ndppn = NULL;
char rank[30];
char *localranks;
size_t m;
pmix_info_t info;

for (m=0; m < mynode->num_procs; m++) {
    /* ignore processes that are not part of the target job */
    if (!PMIX_CHECK_NSPACE(targetjob,mynode->proc[m].nspace)) {
        continue;
    }
    snprintf(rank, 30, "%d", mynode->proc[m].rank);
    PMIX_ARGV_APPEND(&ndppn, rank);
}
/* convert the array into a comma-delimited string of ranks */
localranks = PMIX_ARGV_JOIN(ndppn, ',');
/* release the local array */
PMIX_ARGV_FREE(ndppn);

/* pass the string as the value to the PMIX_LOCAL_PEERS key */
PMIX_INFO_LOAD(&info, PMIX_LOCAL_PEERS, localranks, PMIX_STRING);
/* release the list */
free(localranks);

\end{codepar}
\cspecificend

The \refattr{PMIX_LOCAL_CPUSETS} value is constructed in a similar manner. In the provided example, it is assumed that the \ac{HWLOC} cpuset representation (a comma-delimited string of processor IDs) of the processors assigned to each process has previously been generated and stored on the process description. Thus, the value can be constructed as shown below:

\cspecificstart
\begin{codepar}
char **ndcpus = NULL;
char *localcpus;
size_t m;
pmix_info_t info;

for (m=0; m < mynode->num_procs; m++) {
    /* ignore processes that are not part of the target job */
    if (!PMIX_CHECK_NSPACE(targetjob,mynode->proc[m].nspace)) {
        continue;
    }
    PMIX_ARGV_APPEND(&ndcpus, mynode->proc[m].cpuset);
}
/* convert the array into a colon-delimited string */
localcpus = PMIX_ARGV_JOIN(ndcpus, ':');
/* release the local array */
PMIX_ARGV_FREE(ndcpus);

/* pass the string as the value to the PMIX_LOCAL_CPUSETS key */
PMIX_INFO_LOAD(&info, PMIX_LOCAL_CPUSETS, localcpus, PMIX_STRING);
/* release the list */
free(localcpus);

\end{codepar}
\cspecificend

Note that for efficiency, these two values can be computed at the same time.

\end{itemize}

The final \refarg{info} array might therefore look like the diagram in \ref{fig:nodeinfo}:

\begingroup
\begin{figure*}[ht!]
  \begin{center}
    \includegraphics[clip,width=0.8\textwidth]{figs/nodeinfo.pdf}
  \end{center}
  \caption{Final information array}
  \label{fig:nodeinfo}
\end{figure*}
\endgroup


%%%%%%%%%%%
\subsection{\code{PMIx_server_deregister_nspace}}
\declareapi{PMIx_server_deregister_nspace}

%%%%
\summary

Deregister a namespace.

%%%%
\format

\versionMarker{1.0}
\cspecificstart
\begin{codepar}
void PMIx_server_deregister_nspace(const pmix_nspace_t nspace,
                        pmix_op_cbfunc_t cbfunc, void *cbdata)
\end{codepar}
\cspecificend

\begin{arglist}
\argin{nspace}{Namespace (string)}
\argin{cbfunc}{Callback function \refapi{pmix_op_cbfunc_t} (function reference)}
\argin{cbdata}{Data to be passed to the callback function (memory reference)}
\end{arglist}

%%%%
\descr

Deregister the specified \refarg{nspace} and purge all objects relating to it, including any client information from that namespace.
This is intended to support persistent \ac{PMIx} servers by providing an opportunity for the host \ac{RM} to tell the \ac{PMIx} server library to release all memory for a completed job. Note that the library must not invoke the callback function prior to returning from the \ac{API}.


%%%%%%%%%%%
\subsection{\code{PMIx_server_register_client}}
\declareapi{PMIx_server_register_client}

%%%%
\summary

Register a client process with the PMIx server library.

%%%%
\format

\versionMarker{1.0}
\cspecificstart
\begin{codepar}
pmix_status_t
PMIx_server_register_client(const pmix_proc_t *proc,
                        uid_t uid, gid_t gid,
                        void *server_object,
                        pmix_op_cbfunc_t cbfunc, void *cbdata)
\end{codepar}
\cspecificend

\begin{arglist}
\argin{proc}{\refstruct{pmix_proc_t} structure (handle)}
\argin{uid}{user id (integer)}
\argin{gid}{group id (integer)}
\argin{server_object}{(memory reference)}
\argin{cbfunc}{Callback function \refapi{pmix_op_cbfunc_t} (function reference)}
\argin{cbdata}{Data to be passed to the callback function (memory reference)}
\end{arglist}

Returns one of the following:

\begin{itemize}
    \item \refconst{PMIX_SUCCESS}, indicating that the request is being processed by the host environment - result will be returned in the provided \refarg{cbfunc}. Note that the library must not invoke the callback function prior to returning from the \ac{API}.
    \item \refconst{PMIX_OPERATION_SUCCEEDED}, indicating that the request was immediately processed and returned \textit{success} - the \refarg{cbfunc} will not be called
    \item a PMIx error constant indicating either an error in the input or that the request was immediately processed and failed - the \refarg{cbfunc} will not be called
\end{itemize}


%%%%
\descr

Register a client process with the PMIx server library.

The host server can also, if it desires, provide an object it wishes to be returned when a server function is called that relates to a specific process.
For example, the host server may have an object that tracks the specific client.
Passing the object to the library allows the library to provide that object to the host server during subsequent calls related to that client, such as a \refapi{pmix_server_client_connected_fn_t} function.  This allows the host server to access the object without performing a lookup based on the client's namespace and rank.

\advicermstart
Host environments are required to execute this operation prior to starting the client process.
The expected user ID and group ID of the child process allows the server library to properly authenticate clients as they connect by requiring the two values to match. Accordingly, the detected user and group ID's of the connecting process are not included in the \refapi{pmix_server_client_connected_fn_t} server module function.
\advicermend

\adviceimplstart
For security purposes, the \ac{PMIx} server library should check the user and group ID's of a connecting process against those provided for the declared client process identifier via the \refapi{PMIx_server_register_client} prior to completing the connection.
\adviceimplend

%%%%%%%%%%%
\subsection{\code{PMIx_server_deregister_client}}
\declareapi{PMIx_server_deregister_client}

%%%%
\summary

Deregister a client and purge all data relating to it.

%%%%
\format

\versionMarker{1.0}
\cspecificstart
\begin{codepar}
void
PMIx_server_deregister_client(const pmix_proc_t *proc,
                        pmix_op_cbfunc_t cbfunc, void *cbdata)
\end{codepar}
\cspecificend

\begin{arglist}
\argin{proc}{\refstruct{pmix_proc_t} structure (handle)}
\argin{cbfunc}{Callback function \refapi{pmix_op_cbfunc_t} (function reference)}
\argin{cbdata}{Data to be passed to the callback function (memory reference)}
\end{arglist}


%%%%
\descr

The \refapi{PMIx_server_deregister_nspace} \ac{API} will delete all client information for that namespace. The \ac{PMIx} server library will automatically perform that operation upon disconnect of all local clients.
This \ac{API} is therefore intended primarily for use in exception cases, but can be called in non-exception cases if desired. Note that the library must not invoke the callback function prior to returning from the \ac{API}.


%%%%%%%%%%%
\subsection{\code{PMIx_server_setup_fork}}
\declareapi{PMIx_server_setup_fork}

%%%%
\summary

Setup the environment of a child process to be forked by the host.

%%%%
\format

\versionMarker{1.0}
\cspecificstart
\begin{codepar}
pmix_status_t
PMIx_server_setup_fork(const pmix_proc_t *proc,
                        char ***env)
\end{codepar}
\cspecificend

\begin{arglist}
\argin{proc}{\refstruct{pmix_proc_t} structure (handle)}
\argin{env}{Environment array (array of strings)}
\end{arglist}

Returns \refconst{PMIX_SUCCESS} or a negative value corresponding to a PMIx error constant.

%%%%
\descr

Setup the environment of a child process to be forked by the host so it can correctly interact with the PMIx server.

\advicermstart
Host environments are required to execute this operation prior to starting the client process.
\advicermend

The \ac{PMIx} client needs some setup information so it can properly connect back to the server.
This function will set appropriate environmental variables for this purpose, and will also provide any environmental variables that were specified in the launch command (e.g., via \refapi{PMIx_Spawn}) plus other values (e.g., variables required to properly initialize the client's fabric library).


%%%%%%%%%%%
\subsection{\code{PMIx_server_dmodex_request}}
\declareapi{PMIx_server_dmodex_request}

%%%%
\summary

Define a function by which the host server can request modex data from the local PMIx server.

%%%%
\format

\versionMarker{1.0}
\cspecificstart
\begin{codepar}
pmix_status_t PMIx_server_dmodex_request(const pmix_proc_t *proc,
                        pmix_dmodex_response_fn_t cbfunc,
                        void *cbdata)
\end{codepar}
\cspecificend

\begin{arglist}
\argin{proc}{\refstruct{pmix_proc_t} structure (handle)}
\argin{cbfunc}{Callback function \refapi{pmix_dmodex_response_fn_t} (function reference)}
\argin{cbdata}{Data to be passed to the callback function (memory reference)}
\end{arglist}

Returns one of the following:

\begin{itemize}
    \item \refconst{PMIX_SUCCESS}, indicating that the request is being processed by the host environment - result will be returned in the provided \refarg{cbfunc}. Note that the library must not invoke the callback function prior to returning from the \ac{API}.
    \item a PMIx error constant indicating an error in the input - the \refarg{cbfunc} will not be called
\end{itemize}


%%%%
\descr

Define a function by which the host server can request modex data from the local \ac{PMIx} server. Traditional wireup procedures revolve around the per-process posting of data (e.g., location and endpoint information) via the \refapi{PMIx_Put} and \refapi{PMIx_Commit} functions followed by a \refapi{PMIx_Fence} barrier that globally exchanges the posted information. However, the barrier operation represents a signficant time impact at large scale.

\ac{PMIx} supports an alternative wireup method known as \textit{Direct Modex} that replaces the barrier-based exchange of all process-posted information with on-demand fetch of a peer's data. In place of the barrier operation, data posted by each process is cached on the local \ac{PMIx} server. When a process requests the information posted by a particular peer, it first checks the local cache to see if the data is already available. If not, then the request is passed to the local \ac{PMIx} server, which subsequently requests that its \ac{RM} host request the data from the \ac{RM} daemon on the node where the specified peer process is located. Upon receiving the request, the \ac{RM} daemon passes the request into its \ac{PMIx} server library using the \refapi{PMIx_server_dmodex_request} function, receiving the response in the provided \refarg{cbfunc} once the indicated process has posted its information. The \ac{RM} daemon then returns the data to the requesting daemon, who subsequently passes the data to its \ac{PMIx} server library for transfer to the requesting client.

\adviceuserstart
While direct modex allows for faster launch times by eliminating the barrier operation, per-peer retrieval of posted information is less efficient. Optimizations can be implemented - e.g., by returning posted information from all processes on a node upon first request - but in general direct modex remains best suited for sparsely connected applications.
\adviceuserend

%%%%%%%%%%%
\subsection{\code{PMIx_server_setup_application}}
\declareapi{PMIx_server_setup_application}

%%%%
\summary

Provide a function by which the resource manager can request application-specific setup data prior to launch of a \refterm{job}.

%%%%
\format

\versionMarker{2.0}
\cspecificstart
\begin{codepar}
pmix_status_t
PMIx_server_setup_application(const pmix_nspace_t nspace,
                        pmix_info_t info[], size_t ninfo,
                        pmix_setup_application_cbfunc_t cbfunc,
                        void *cbdata)
\end{codepar}
\cspecificend

\begin{arglist}
\argin{nspace}{namespace (string)}
\argin{info}{Array of info structures (array of handles)}
\argin{ninfo}{Number of elements in the \refarg{info} array (integer)}
\argin{cbfunc}{Callback function \refapi{pmix_setup_application_cbfunc_t} (function reference)}
\argin{cbdata}{Data to be passed to the \refarg{cbfunc} callback function (memory reference)}
\end{arglist}

Returns one of the following:

\begin{itemize}
    \item \refconst{PMIX_SUCCESS}, indicating that the request is being processed by the host environment - result will be returned in the provided \refarg{cbfunc}. Note that the library must not invoke the callback function prior to returning from the \ac{API}.
    \item a PMIx error constant indicating either an error in the input - the \refarg{cbfunc} will not be called
\end{itemize}


\reqattrstart
\ac{PMIx} libraries that support this operation are required to support the following:

\pastePRIAttributeItem{PMIX_SETUP_APP_ENVARS}
\pastePRIAttributeItem{PMIX_SETUP_APP_NONENVARS}
\pastePRIAttributeItem{PMIX_SETUP_APP_ALL}
\pastePRIAttributeItem{PMIX_ALLOC_NETWORK}
\pastePRIAttributeItem{PMIX_ALLOC_NETWORK_ID}
\pastePRIAttributeItem{PMIX_ALLOC_NETWORK_SEC_KEY}
\pastePRIAttributeItem{PMIX_ALLOC_NETWORK_TYPE}
\pastePRIAttributeItem{PMIX_ALLOC_NETWORK_PLANE}
\pastePRIAttributeItem{PMIX_ALLOC_NETWORK_ENDPTS}
\pastePRIAttributeItem{PMIX_ALLOC_NETWORK_ENDPTS_NODE}

\reqattrend

\optattrstart
\ac{PMIx} libraries that support this operation may support the following:

\pastePRIAttributeItem{PMIX_ALLOC_BANDWIDTH}
\pastePRIAttributeItem{PMIX_ALLOC_NETWORK_QOS}
\pastePRIAttributeItem{PMIX_ALLOC_TIME}

\optattrend

%%%%
\descr

Provide a function by which the \ac{RM} can request application-specific setup data (e.g., environmental variables, fabric configuration and security credentials) from supporting \ac{PMIx} server library subsystems prior to initiating launch of a job.

\advicermstart
Host environments are required to execute this operation prior to launching a job. In addition to supported directives, the \refarg{info} array must include a description of the \refterm{job} using the \refattr{PMIX_NODE_MAP} and \refattr{PMIX_PROC_MAP} attributes.
\advicermend

This is defined as a non-blocking operation in case contributing subsystems need to perform some potentially time consuming action (e.g., query a remote service) before responding. The returned data must be distributed by the \ac{RM} and subsequently delivered to the local \ac{PMIx} server on each node where application processes will execute, prior to initiating execution of those processes.

\adviceimplstart
Support for harvesting of environmental variables and providing of local configuration information by the \ac{PMIx} implementation is optional.
\adviceimplend

%%%%%%%%%%%
\subsection{\code{PMIx_server_setup_local_support}}
\declareapi{PMIx_server_setup_local_support}

%%%%
\summary

Provide a function by which the local \ac{PMIx} server can perform any application-specific operations prior to spawning local clients of a given application.

%%%%
\format

\versionMarker{2.0}
\cspecificstart
\begin{codepar}
pmix_status_t
PMIx_server_setup_local_support(const pmix_nspace_t nspace,
                                pmix_info_t info[], size_t ninfo,
                                pmix_op_cbfunc_t cbfunc,
                                void *cbdata);
\end{codepar}
\cspecificend

\begin{arglist}
\argin{nspace}{Namespace (string)}
\argin{info}{Array of info structures (array of handles)}
\argin{ninfo}{Number of elements in the \refarg{info} array (\code{size_t})}
\argin{cbfunc}{Callback function \refapi{pmix_op_cbfunc_t} (function reference)}
\argin{cbdata}{Data to be passed to the callback function (memory reference)}
\end{arglist}

Returns one of the following:

\begin{itemize}
    \item \refconst{PMIX_SUCCESS}, indicating that the request is being processed by the host environment - result will be returned in the provided \refarg{cbfunc}. Note that the library must not invoke the callback function prior to returning from the \ac{API}.
    \item \refconst{PMIX_OPERATION_SUCCEEDED}, indicating that the request was immediately processed and returned \textit{success} - the \refarg{cbfunc} will not be called
    \item a PMIx error constant indicating either an error in the input or that the request was immediately processed and failed - the \refarg{cbfunc} will not be called
\end{itemize}


%%%%
\descr

Provide a function by which the local \ac{PMIx} server can perform any application-specific operations prior to spawning local clients of a given application. For example, a network library might need to setup the local driver for ``instant on'' addressing. The data provided in the \refarg{info} array is the data returned to the host \ac{RM} by the callback function executed as a result of a call to \refapi{PMIx_server_setup_application}.

\advicermstart
Host environments are required to execute this operation prior to starting any local application processes from the specified namespace.
\advicermend

%%%%%%%%%%%
\subsection{\code{PMIx_server_IOF_deliver}}
\declareapi{PMIx_server_IOF_deliver}

%%%%
\summary

Provide a function by which the host environment can pass forwarded \ac{IO} to the \ac{PMIx} server library for distribution to its clients.

%%%%
\format

\versionMarker{3.0}
\cspecificstart
\begin{codepar}
pmix_status_t
PMIx_server_IOF_deliver(const pmix_proc_t *source,
                        pmix_iof_channel_t channel,
                        const pmix_byte_object_t *bo,
                        const pmix_info_t info[], size_t ninfo,
                        pmix_op_cbfunc_t cbfunc, void *cbdata);
\end{codepar}
\cspecificend

\begin{arglist}
\argin{source}{Pointer to \refstruct{pmix_proc_t} identifying source of the \ac{IO} (handle)}
\argin{channel}{\ac{IO} channel of the data (\refstruct{pmix_iof_channel_t})}
\argin{bo}{Pointer to \refstruct{pmix_byte_object_t} containing the payload to be delivered (handle)}
\argin{info}{Array of \refstruct{pmix_info_t} metadata describing the data (array of handles)}
\argin{ninfo}{Number of elements in the \refarg{info} array (\code{size_t})}
\argin{cbfunc}{Callback function \refapi{pmix_op_cbfunc_t} (function reference)}
\argin{cbdata}{Data to be passed to the callback function (memory reference)}
\end{arglist}

Returns one of the following:

\begin{itemize}
    \item \refconst{PMIX_SUCCESS}, indicating that the request is being processed by the host environment - result will be returned in the provided \refarg{cbfunc}. Note that the library must not invoke the callback function prior to returning from the \ac{API}.
    \item \refconst{PMIX_OPERATION_SUCCEEDED}, indicating that the request was immediately processed and returned \textit{success} - the \refarg{cbfunc} will not be called
    \item a PMIx error constant indicating either an error in the input or that the request was immediately processed and failed - the \refarg{cbfunc} will not be called
\end{itemize}

%%%%
\descr

Provide a function by which the host environment can pass forwarded \ac{IO} to the \ac{PMIx} server library for distribution to its clients. The \ac{PMIx} server library is responsible for determining which of its clients have actually registered for the provided data and delivering it. The \refarg{cbfunc} callback function will be called once the \ac{PMIx} server library no longer requires access to the provided data.

%%%%%%%%%%%
\subsection{\code{PMIx_server_collect_inventory}}
\declareapi{PMIx_server_collect_inventory}

%%%%
\summary

Collect inventory of resources on a node

%%%%
\format

\versionMarker{3.0}
\cspecificstart
\begin{codepar}
pmix_status_t
PMIx_server_collect_inventory(const pmix_info_t directives[],
                              size_t ndirs,
                              pmix_info_cbfunc_t cbfunc,
                              void *cbdata);
\end{codepar}
\cspecificend

\begin{arglist}
\argin{directives}{Array of \refstruct{pmix_info_t} directing the request (array of handles)}
\argin{ndirs}{Number of elements in the \refarg{directives} array (\code{size_t})}
\argin{cbfunc}{Callback function to return collected data (\refapi{pmix_info_cbfunc_t} function reference)}
\argin{cbdata}{Data to be passed to the callback function (memory reference)}
\end{arglist}

Returns \refconst{PMIX_SUCCESS} or a negative value corresponding to a PMIx error constant. In the event the function returns an error, the \refarg{cbfunc} will not be called.

%%%%
\descr

Provide a function by which the host environment can request its \ac{PMIx} server library collect an inventory of local resources. Supported resources depends upon the \ac{PMIx} implementation, but may include the local node topology and network interfaces.

\advicermstart
This is a non-blocking \ac{API} as it may involve somewhat lengthy operations to obtain the requested information. Inventory collection is expected to be a rare event – at system startup and upon command from a system administrator. Inventory updates are expected to initiate a smaller operation involving only the changed information. For example, replacement of a node would generate an event to notify the scheduler with an inventory update without invoking a global inventory operation.
\advicermend

%%%%%%%%%%%
\subsection{\code{PMIx_server_deliver_inventory}}
\declareapi{PMIx_server_deliver_inventory}

%%%%
\summary

Pass collected inventory to the \ac{PMIx} server library for storage

%%%%
\format

\versionMarker{3.0}
\cspecificstart
\begin{codepar}
pmix_status_t
PMIx_server_deliver_inventory(const pmix_info_t info[],
                              size_t ninfo,
                              const pmix_info_t directives[],
                              size_t ndirs,
                              pmix_op_cbfunc_t cbfunc,
                              void *cbdata);
\end{codepar}
\cspecificend

\begin{arglist}
\argin{info}{Array of \refstruct{pmix_info_t} containing the inventory (array of handles)}
\argin{ninfo}{Number of elements in the \refarg{info} array (\code{size_t})}
\argin{directives}{Array of \refstruct{pmix_info_t} directing the request (array of handles)}
\argin{ndirs}{Number of elements in the \refarg{directives} array (\code{size_t})}
\argin{cbfunc}{Callback function \refapi{pmix_op_cbfunc_t} (function reference)}
\argin{cbdata}{Data to be passed to the callback function (memory reference)}
\end{arglist}

Returns one of the following:

\begin{itemize}
    \item \refconst{PMIX_SUCCESS}, indicating that the request is being processed by the host environment - result will be returned in the provided \refarg{cbfunc}. Note that the library must not invoke the callback function prior to returning from the \ac{API}.
    \item \refconst{PMIX_OPERATION_SUCCEEDED}, indicating that the request was immediately processed and returned \textit{success} - the \refarg{cbfunc} will not be called
    \item a PMIx error constant indicating either an error in the input or that the request was immediately processed and failed - the \refarg{cbfunc} will not be called
\end{itemize}


%%%%
\descr

Provide a function by which the host environment can pass inventory information obtained from a node to the \ac{PMIx} server library for storage. Inventory data is subsequently used by the \ac{PMIx} server library for allocations in response to \refapi{PMIx_server_setup_application}, and may be available to the library's host via the \refapi{PMIx_Get} \ac{API} (depending upon \ac{PMIx} implementation). The \refarg{cbfunc} callback function will be called once the \ac{PMIx} server library no longer requires access to the provided data.

%%%%%%%%%%%
\section{Server Function Pointers}

\ac{PMIx} utilizes a "function-shipping" approach to support for implementing the server-side of the protocol. This method allows \acp{RM} to implement the server without being burdened with \ac{PMIx} internal details. When a request is received from the client, the corresponding server function will be called with the information.

Any functions not supported by the \ac{RM} can be indicated by a \code{NULL} for the function pointer. Client calls to such functions will return a \refconst{PMIX_ERR_NOT_SUPPORTED} status.

The host \ac{RM} will provide the function pointers in a \refapi{pmix_server_module_t} structure passed to \refapi{PMIx_server_init}.
That module structure and associated function references are defined in this section.

\advicermstart
For performance purposes, the host server is required to return as quickly as possible from all functions. Execution of
the function is thus to be done asynchronously so as to allow the \ac{PMIx} server support library to handle multiple client requests
as quickly and scalably as possible.

All data passed to the host server functions is ``owned'' by the
PMIX server support library and must not be free'd. Data returned
by the host server via callback function is owned by the host
server, which is free to release it upon return from the callback
\advicermend

%%%%%%%%%%%
\subsection{\code{pmix_server_module_t} Module}
\declareapi{pmix_server_module_t}

%%%%
\summary

List of function pointers that a PMIx server passes to \refapi{PMIx_server_init} during startup.

%%%%
\format

\cspecificstart
\begin{codepar}
typedef struct pmix_server_module_3_0_0_t {
    /* v1x interfaces */
    pmix_server_client_connected_fn_t   client_connected;
    pmix_server_client_finalized_fn_t   client_finalized;
    pmix_server_abort_fn_t              abort;
    pmix_server_fencenb_fn_t            fence_nb;
    pmix_server_dmodex_req_fn_t         direct_modex;
    pmix_server_publish_fn_t            publish;
    pmix_server_lookup_fn_t             lookup;
    pmix_server_unpublish_fn_t          unpublish;
    pmix_server_spawn_fn_t              spawn;
    pmix_server_connect_fn_t            connect;
    pmix_server_disconnect_fn_t         disconnect;
    pmix_server_register_events_fn_t    register_events;
    pmix_server_deregister_events_fn_t  deregister_events;
    pmix_server_listener_fn_t           listener;
    /* v2x interfaces */
    pmix_server_notify_event_fn_t       notify_event;
    pmix_server_query_fn_t              query;
    pmix_server_tool_connection_fn_t    tool_connected;
    pmix_server_log_fn_t                log;
    pmix_server_alloc_fn_t              allocate;
    pmix_server_job_control_fn_t        job_control;
    pmix_server_monitor_fn_t            monitor;
    /* v3x interfaces */
    pmix_server_get_cred_fn_t           get_credential;
    pmix_server_validate_cred_fn_t      validate_credential;
    pmix_server_iof_fn_t                iof_pull;
    pmix_server_stdin_fn_t              push_stdin;
} pmix_server_module_t;
\end{codepar}
\cspecificend


%%%%%%%%%%%
\subsection{\code{pmix_server_client_connected_fn_t}}
\declareapi{pmix_server_client_connected_fn_t}

%%%%
\summary

Notify the host server that a client connected to this server.

%%%%
\format

\versionMarker{1.0}
\cspecificstart
\begin{codepar}
typedef pmix_status_t (*pmix_server_client_connected_fn_t)(
                             const pmix_proc_t *proc,
                             void* server_object,
                             pmix_op_cbfunc_t cbfunc,
                             void *cbdata)
\end{codepar}
\cspecificend

\begin{arglist}
\argin{proc}{\refstruct{pmix_proc_t} structure (handle)}
\argin{server_object}{object reference (memory reference)}
\argin{cbfunc}{Callback function \refapi{pmix_op_cbfunc_t} (function reference)}
\argin{cbdata}{Data to be passed to the callback function (memory reference)}
\end{arglist}

Returns one of the following:

\begin{itemize}
    \item \refconst{PMIX_SUCCESS}, indicating that the request is being processed by the host environment - result will be returned in the provided \refarg{cbfunc}. Note that the host must not invoke the callback function prior to returning from the \ac{API}.
    \item \refconst{PMIX_OPERATION_SUCCEEDED}, indicating that the request was immediately processed and returned \textit{success} - the \refarg{cbfunc} will not be called
    \item a PMIx error constant indicating either an error in the input or that the request was immediately processed and failed - the \refarg{cbfunc} will not be called
\end{itemize}

%%%%
\descr

Notify the host environment that a client has called \refapi{PMIx_Init}.
Note that the client will be in a blocked state until the host server executes the callback function, thus allowing the \ac{PMIx} server support library to release
the client.
The server_object parameter will be the value of the server_object parameter passed to
\refapi{PMIx_server_register_client} by the host server when registering the connecting client.  If provided, an implementation of \refapi{pmix_server_client_connected_fn_t}
is only required to
call the callback function designated.  A host server can choose to not be notified when clients connect by setting \refapi{pmix_server_client_connected_fn_t} to \code{NULL}.

It is possible that only a subset of the clients in a namespace call \refapi{PMIx_Init}.   The server's \refapi{pmix_server_client_connected_fn_t} implementation
should not depend on being called once per rank in a namespace or delay calling the callback function until all ranks have connected.
However, if a rank makes any \ac{PMIx} calls, it must first call \refapi{PMIx_Init} and
therefore the server's \refapi{pmix_server_client_connected_fn_t} will be called before any other server functions specific to the rank.

\advicermstart
 This operation is an opportunity for a host environment
 to update the status of the ranks it manages.  It is also a convenient and well defined time to perform initialization necessary to
 support further calls into the server related to that rank.
 \advicermend

%%%%%%%%%%%
\subsection{\code{pmix_server_client_finalized_fn_t}}
\declareapi{pmix_server_client_finalized_fn_t}

%%%%
\summary

Notify the host environment that a client called \refapi{PMIx_Finalize}.

%%%%
\format

\versionMarker{1.0}
\cspecificstart
\begin{codepar}
typedef pmix_status_t (*pmix_server_client_finalized_fn_t)(
                             const pmix_proc_t *proc,
                             void* server_object,
                             pmix_op_cbfunc_t cbfunc,
                             void *cbdata)
\end{codepar}
\cspecificend

\begin{arglist}
\argin{proc}{\refstruct{pmix_proc_t} structure (handle)}
\argin{server_object}{object reference (memory reference)}
\argin{cbfunc}{Callback function \refapi{pmix_op_cbfunc_t} (function reference)}
\argin{cbdata}{Data to be passed to the callback function (memory reference)}
\end{arglist}

Returns one of the following:

\begin{itemize}
    \item \refconst{PMIX_SUCCESS}, indicating that the request is being processed by the host environment - result will be returned in the provided \refarg{cbfunc}. Note that the host must not invoke the callback function prior to returning from the \ac{API}.
    \item \refconst{PMIX_OPERATION_SUCCEEDED}, indicating that the request was immediately processed and returned \textit{success} - the \refarg{cbfunc} will not be called
    \item a PMIx error constant indicating either an error in the input or that the request was immediately processed and failed - the \refarg{cbfunc} will not be called
\end{itemize}

%%%%
\descr

Notify the host environment that a client called \refapi{PMIx_Finalize}.
Note that the client will be in a blocked state until the host server executes the callback function, thus allowing the PMIx server support library to release the client.
The server_object parameter will be the value of the server_object parameter passed to
\refapi{PMIx_server_register_client} by the host server when registering the connecting client.  If provided, an implementation of \refapi{pmix_server_client_finalized_fn_t}
is only required to
call the callback function designated.  A host server can choose to not be notified when clients finalize by setting \refapi{pmix_server_client_finalized_fn_t} to \code{NULL}.

Note that the host server is only being informed that the client has called \refapi{PMIx_Finalize}.  The client might not have exited.  If a client
exits without calling \refapi{PMIx_Finalize}, the server support library will not call the \refapi{pmix_server_client_finalized_fn_t} implementation.

\advicermstart
This operation is an opportunity for a host server
to update the status of the tasks it manages.  It is also a convenient and well defined time to release resources used to support that client.
\advicermend


%%%%%%%%%%%
\subsection{\code{pmix_server_abort_fn_t}}
\declareapi{pmix_server_abort_fn_t}

%%%%
\summary

Notify the host environment that a local client called \refapi{PMIx_Abort}.

%%%%
\format

\versionMarker{1.0}
\cspecificstart
\begin{codepar}
typedef pmix_status_t (*pmix_server_abort_fn_t)(
                             const pmix_proc_t *proc,
                             void *server_object,
                             int status,
                             const char msg[],
                             pmix_proc_t procs[],
                             size_t nprocs,
                             pmix_op_cbfunc_t cbfunc,
                             void *cbdata)
\end{codepar}
\cspecificend


\begin{arglist}
\argin{proc}{\refstruct{pmix_proc_t} structure identifying the process requesting the abort (handle)}
\argin{server_object}{object reference (memory reference)}
\argin{status}{exit status (integer)}
\argin{msg}{exit status message (string)}
\argin{procs}{Array of \refstruct{pmix_proc_t} structures identifying the processes to be terminated (array of handles)}
\argin{nprocs}{Number of elements in the \refarg{procs} array (integer)}
\argin{cbfunc}{Callback function \refapi{pmix_op_cbfunc_t} (function reference)}
\argin{cbdata}{Data to be passed to the callback function (memory reference)}
\end{arglist}

Returns one of the following:

\begin{itemize}
    \item \refconst{PMIX_SUCCESS}, indicating that the request is being processed by the host environment - result will be returned in the provided \refarg{cbfunc}. Note that the host must not invoke the callback function prior to returning from the \ac{API}.
    \item \refconst{PMIX_OPERATION_SUCCEEDED}, indicating that the request was immediately processed and returned \textit{success} - the \refarg{cbfunc} will not be called
    \item a PMIx error constant indicating either an error in the input or that the request was immediately processed and failed - the \refarg{cbfunc} will not be called
\end{itemize}

%%%%
\descr

A local client called \refapi{PMIx_Abort}.
Note that the client will be in a blocked state until the host server executes the callback function, thus allowing the \ac{PMIx} server library to release the client.
The array of \refarg{procs} indicates which processes are to be terminated.
A \code{NULL} indicates that all processes in the client's namespace are to be terminated.


%%%%%%%%%%%
\subsection{\code{pmix_server_fencenb_fn_t}}
\declareapi{pmix_server_fencenb_fn_t}

%%%%
\summary

At least one client called either \refapi{PMIx_Fence} or \refapi{PMIx_Fence_nb}.

%%%%
\format

\versionMarker{1.0}
\cspecificstart
\begin{codepar}
typedef pmix_status_t (*pmix_server_fencenb_fn_t)(
                             const pmix_proc_t procs[],
                             size_t nprocs,
                             const pmix_info_t info[],
                             size_t ninfo,
                             char *data, size_t ndata,
                             pmix_modex_cbfunc_t cbfunc,
                             void *cbdata)
\end{codepar}
\cspecificend

\begin{arglist}
\argin{procs}{Array of \refstruct{pmix_proc_t} structures identifying operation participants(array of handles)}
\argin{nprocs}{Number of elements in the \refarg{procs} array (integer)}
\argin{info}{Array of info structures (array of handles)}
\argin{ninfo}{Number of elements in the \refarg{info} array (integer)}
\argin{data}{(string)}
\argin{ndata}{(integer)}
\argin{cbfunc}{Callback function \refapi{pmix_modex_cbfunc_t} (function reference)}
\argin{cbdata}{Data to be passed to the callback function (memory reference)}
\end{arglist}

Returns one of the following:

\begin{itemize}
    \item \refconst{PMIX_SUCCESS}, indicating that the request is being processed by the host environment - result will be returned in the provided \refarg{cbfunc}. Note that the host must not invoke the callback function prior to returning from the \ac{API}.
    \item \refconst{PMIX_OPERATION_SUCCEEDED}, indicating that the request was immediately processed and returned \textit{success} - the \refarg{cbfunc} will not be called
    \item a PMIx error constant indicating either an error in the input or that the request was immediately processed and failed - the \refarg{cbfunc} will not be called
\end{itemize}

\reqattrstart
\ac{PMIx} libraries are required to pass any provided attributes to the host environment for processing.

\divider

The following attributes are required to be supported by all host environments:

\pastePRRTEAttributeItem{PMIX_COLLECT_DATA}

\reqattrend

\optattrstart
The following attributes are optional for host environments:

\pastePRRTEAttributeItem{PMIX_TIMEOUT}
\pasteAttributeItem{PMIX_COLLECTIVE_ALGO}
\pasteAttributeItem{PMIX_COLLECTIVE_ALGO_REQD}

\optattrend

\advicermstart
Host environment are required to return \refconst{PMIX_ERR_NOT_SUPPORTED} if passed an attributed marked as \refconst{PMIX_INFO_REQD} that they do not support, even if support for that attribute is optional.
\advicermend

%%%%
\descr

All local clients in the provided array of \refarg{procs} called either \refapi{PMIx_Fence} or \refapi{PMIx_Fence_nb}.
In either case, the host server will be called via a non-blocking function to execute the specified operation once all participating local processes have contributed.
All processes in the specified \refarg{procs} array are required to participate in the \refapi{PMIx_Fence}/\refapi{PMIx_Fence_nb} operation.
The callback is to be executed once every daemon hosting at least one participant has called the host server's \refapi{pmix_server_fencenb_fn_t} function.

\adviceimplstart
The \ac{PMIx} server library is required to aggregate participation by local clients, passing the request to the host environment once all local participants have executed the \ac{API}.
\adviceimplend

\advicermstart
The host will receive a single call for each collective operation. It is the responsibility of the host to identify the nodes containing participating processes, execute the collective across all participating nodes, and notify the local \ac{PMIx} server library upon completion of the global collective. Data received from each node must be simply concatenated to form an aggregated unit, as shown in the following example:

\cspecificstart
\begin{codepar}
uint8_t *blob1, *blob2, *total;
size_t sz_blob1, sz_blob2, sz_total;

sz_total = sz_blob1 + sz_blob2;
total = (uint8_t*)malloc(sz_total);
memcpy(total, blob1, sz_blob1);
memcpy(\&total[sz_blob1], blob2, sz_blob2);
\end{codepar}
\cspecificend

Note that the ordering of the data blobs does not matter.
\advicermend

The provided data is to be collectively shared with all \ac{PMIx} servers involved in the fence operation, and returned in the modex \refarg{cbfunc}.
A \code{NULL} data value indicates that the local processes had no data to contribute.

The array of \refarg{info} structs is used to pass user-requested options to the server.
This can include directives as to the algorithm to be used to execute the fence operation.
The directives are optional unless the \refconst{PMIX_INFO_REQD} flag has been set - in such cases, the host \ac{RM} is required to return an error if the directive cannot be met.


%%%%%%%%%%%
\subsection{\code{pmix_server_dmodex_req_fn_t}}
\declareapi{pmix_server_dmodex_req_fn_t}

%%%%
\summary

Used by the PMIx server to request its local host contact the \ac{PMIx} server on the remote node that hosts the specified proc to obtain and return a direct modex blob for that proc.

%%%%
\format

\versionMarker{1.0}
\cspecificstart
\begin{codepar}
typedef pmix_status_t (*pmix_server_dmodex_req_fn_t)(
                             const pmix_proc_t *proc,
                             const pmix_info_t info[],
                             size_t ninfo,
                             pmix_modex_cbfunc_t cbfunc,
                             void *cbdata)
\end{codepar}
\cspecificend

\begin{arglist}
\argin{proc}{\refstruct{pmix_proc_t} structure identifying the process whose data is being requested (handle)}
\argin{info}{Array of info structures (array of handles)}
\argin{ninfo}{Number of elements in the \refarg{info} array (integer)}
\argin{cbfunc}{Callback function \refapi{pmix_modex_cbfunc_t} (function reference)}
\argin{cbdata}{Data to be passed to the callback function (memory reference)}
\end{arglist}

Returns one of the following:

\begin{itemize}
    \item \refconst{PMIX_SUCCESS}, indicating that the request is being processed by the host environment - result will be returned in the provided \refarg{cbfunc}. Note that the host must not invoke the callback function prior to returning from the \ac{API}.
    \item a PMIx error constant indicating either an error in the input or that the request was immediately processed and failed - the \refarg{cbfunc} will not be called
\end{itemize}

\reqattrstart
\ac{PMIx} libraries are required to pass any provided attributes to the host environment for processing.
\reqattrend

\optattrstart
The following attributes are optional for host environments that support this operation:

\pastePRRTEAttributeItem{PMIX_TIMEOUT}

\optattrend

%%%%
\descr

Used by the \ac{PMIx} server to request its local host contact the \ac{PMIx} server on the remote node that hosts the specified proc to obtain and return any information that process posted via calls to \refapi{PMIx_Put} and \refapi{PMIx_Commit}.

The array of \refarg{info} structs is used to pass user-requested options to the server.
This can include a timeout to preclude an indefinite wait for data that may never become available.
The directives are optional unless the \emph{mandatory} flag has been set - in such cases, the host \ac{RM} is required to return an error if the directive cannot be met.


%%%%%%%%%%%
\subsection{\code{pmix_server_publish_fn_t}}
\declareapi{pmix_server_publish_fn_t}

%%%%
\summary

Publish data per the PMIx API specification.

%%%%
\format

\versionMarker{1.0}
\cspecificstart
\begin{codepar}
typedef pmix_status_t (*pmix_server_publish_fn_t)(
                             const pmix_proc_t *proc,
                             const pmix_info_t info[],
                             size_t ninfo,
                             pmix_op_cbfunc_t cbfunc,
                             void *cbdata)
\end{codepar}
\cspecificend

\begin{arglist}
\argin{proc}{\refstruct{pmix_proc_t} structure of the process publishing the data (handle)}
\argin{info}{Array of info structures (array of handles)}
\argin{ninfo}{Number of elements in the \refarg{info} array (integer)}
\argin{cbfunc}{Callback function \refapi{pmix_op_cbfunc_t} (function reference)}
\argin{cbdata}{Data to be passed to the callback function (memory reference)}
\end{arglist}

Returns one of the following:

\begin{itemize}
    \item \refconst{PMIX_SUCCESS}, indicating that the request is being processed by the host environment - result will be returned in the provided \refarg{cbfunc}. Note that the host must not invoke the callback function prior to returning from the \ac{API}.
    \item \refconst{PMIX_OPERATION_SUCCEEDED}, indicating that the request was immediately processed and returned \textit{success} - the \refarg{cbfunc} will not be called
    \item a PMIx error constant indicating either an error in the input or that the request was immediately processed and failed - the \refarg{cbfunc} will not be called
\end{itemize}

\reqattrstart
\ac{PMIx} libraries are required to pass any provided attributes to the host environment for processing. In addition, the following attributes are required to be included in the passed \refarg{info} array:

\pastePRIAttributeItem{PMIX_USERID}
\pastePRIAttributeItem{PMIX_GRPID}

\divider

Host environments that implement this entry point are required to support the following attributes:

\pastePRRTEAttributeItem{PMIX_RANGE}
\pastePRRTEAttributeItem{PMIX_PERSISTENCE}

\reqattrend

\optattrstart
The following attributes are optional for host environments that support this operation:

\pasteAttributeItem{PMIX_TIMEOUT}

\optattrend

%%%%
\descr

Publish data per the \refapi{PMIx_Publish} specification.
The callback is to be executed upon completion of the operation.
The default data range is left to the host environment, but expected to be \refconst{PMIX_SESSION}, and the default persistence \refconst{PMIX_PERSIST_SESSION} or their equivalent.
These values can be specified by including the respective attributed in the \refarg{info} array.

The persistence indicates how long the server should retain the data.

\advicermstart
The host environment is not required to guarantee support for any specific range - i.e., the environment does not need to return an error if the data store doesn't support a specified range so long as it is covered by some internally defined range.
However, the server must return an error (a) if the key is duplicative within the storage range, and (b) if the server does not allow overwriting of published info by the original publisher - it is left to the discretion of the host environment to allow info-key-based flags to modify this behavior.

The \refattr{PMIX_USERID} and \refattr{PMIX_GRPID} of the publishing process will be provided to support authorization-based access to published information and must be returned on any subsequent lookup request.
\advicermend

%%%%%%%%%%%
\subsection{\code{pmix_server_lookup_fn_t}}
\declareapi{pmix_server_lookup_fn_t}

%%%%
\summary

Lookup published data.

%%%%
\format

\versionMarker{1.0}
\cspecificstart
\begin{codepar}
typedef pmix_status_t (*pmix_server_lookup_fn_t)(
                             const pmix_proc_t *proc,
                             char **keys,
                             const pmix_info_t info[],
                             size_t ninfo,
                             pmix_lookup_cbfunc_t cbfunc,
                             void *cbdata)
\end{codepar}
\cspecificend

\begin{arglist}
\argin{proc}{\refstruct{pmix_proc_t} structure of the process seeking the data (handle)}
\argin{keys}{(array of strings)}
\argin{info}{Array of info structures (array of handles)}
\argin{ninfo}{Number of elements in the \refarg{info} array (integer)}
\argin{cbfunc}{Callback function \refapi{pmix_lookup_cbfunc_t} (function reference)}
\argin{cbdata}{Data to be passed to the callback function (memory reference)}
\end{arglist}

Returns one of the following:

\begin{itemize}
    \item \refconst{PMIX_SUCCESS}, indicating that the request is being processed by the host environment - result will be returned in the provided \refarg{cbfunc}. Note that the host must not invoke the callback function prior to returning from the \ac{API}.
    \item \refconst{PMIX_OPERATION_SUCCEEDED}, indicating that the request was immediately processed and returned \textit{success} - the \refarg{cbfunc} will not be called
    \item a PMIx error constant indicating either an error in the input or that the request was immediately processed and failed - the \refarg{cbfunc} will not be called
\end{itemize}

\reqattrstart
\ac{PMIx} libraries are required to pass any provided attributes to the host environment for processing. In addition, the following attributes are required to be included in the passed \refarg{info} array:

\pastePRIAttributeItem{PMIX_USERID}
\pastePRIAttributeItem{PMIX_GRPID}

\divider

Host environments that implement this entry point are required to support the following attributes:

\pastePRRTEAttributeItem{PMIX_RANGE}
\pastePRRTEAttributeItem{PMIX_WAIT}

\reqattrend

\optattrstart
The following attributes are optional for host environments that support this operation:

\pasteAttributeItem{PMIX_TIMEOUT}

\optattrend


%%%%
\descr

Lookup published data.
The host server will be passed a \code{NULL}-terminated array of string keys identifying the data being requested.

The array of \refarg{info} structs is used to pass user-requested options to the server. The default data range is left to the host environment, but expected to be \refconst{PMIX_SESSION}.
This can include a wait flag to indicate that the server should wait for all data to become available before executing the callback function, or should immediately callback with whatever data is available.
In addition, a timeout can be specified on the wait to preclude an indefinite wait for data that may never be published.

\advicermstart
The \refattr{PMIX_USERID} and \refattr{PMIX_GRPID} of the requesting process will be provided to support authorization-based access to published information. The host environment is not required to guarantee support for any specific range - i.e., the environment does not need to return an error if the data store doesn't support a specified range so long as it is covered by some internally defined range.
\advicermend

%%%%%%%%%%%
\subsection{\code{pmix_server_unpublish_fn_t}}
\declareapi{pmix_server_unpublish_fn_t}

%%%%
\summary

Delete data from the data store.

%%%%
\format

\versionMarker{1.0}
\cspecificstart
\begin{codepar}
typedef pmix_status_t (*pmix_server_unpublish_fn_t)(
                             const pmix_proc_t *proc,
                             char **keys,
                             const pmix_info_t info[],
                             size_t ninfo,
                             pmix_op_cbfunc_t cbfunc,
                             void *cbdata)
\end{codepar}
\cspecificend

\begin{arglist}
\argin{proc}{\refstruct{pmix_proc_t} structure identifying the process making the request (handle)}
\argin{keys}{(array of strings)}
\argin{info}{Array of info structures (array of handles)}
\argin{ninfo}{Number of elements in the \refarg{info} array (integer)}
\argin{cbfunc}{Callback function \refapi{pmix_op_cbfunc_t} (function reference)}
\argin{cbdata}{Data to be passed to the callback function (memory reference)}
\end{arglist}

Returns one of the following:

\begin{itemize}
    \item \refconst{PMIX_SUCCESS}, indicating that the request is being processed by the host environment - result will be returned in the provided \refarg{cbfunc}. Note that the host must not invoke the callback function prior to returning from the \ac{API}.
    \item \refconst{PMIX_OPERATION_SUCCEEDED}, indicating that the request was immediately processed and returned \textit{success} - the \refarg{cbfunc} will not be called
    \item a PMIx error constant indicating either an error in the input or that the request was immediately processed and failed - the \refarg{cbfunc} will not be called
\end{itemize}

\reqattrstart
\ac{PMIx} libraries are required to pass any provided attributes to the host environment for processing. In addition, the following attributes are required to be included in the passed \refarg{info} array:

\pastePRIAttributeItem{PMIX_USERID}
\pastePRIAttributeItem{PMIX_GRPID}

\divider

Host environments that implement this entry point are required to support the following attributes:

\pastePRRTEAttributeItem{PMIX_RANGE}

\reqattrend

\optattrstart
The following attributes are optional for host environments that support this operation:

\pasteAttributeItem{PMIX_TIMEOUT}

\optattrend

%%%%
\descr

Delete data from the data store.
The host server will be passed a \code{NULL}-terminated array of string keys, plus potential directives such as the data range within which the keys should be deleted. The default data range is left to the host environment, but expected to be \refconst{PMIX_SESSION}.
The callback is to be executed upon completion of the delete procedure.

\advicermstart
The \refattr{PMIX_USERID} and \refattr{PMIX_GRPID} of the requesting process will be provided to support authorization-based access to published information. The host environment is not required to guarantee support for any specific range - i.e., the environment does not need to return an error if the data store doesn't support a specified range so long as it is covered by some internally defined range.
\advicermend


%%%%%%%%%%%
\subsection{\code{pmix_server_spawn_fn_t}}
\declareapi{pmix_server_spawn_fn_t}

%%%%
\summary

Spawn a set of applications/processes as per the \refapi{PMIx_Spawn} API.

%%%%
\format

\versionMarker{1.0}
\cspecificstart
\begin{codepar}
typedef pmix_status_t (*pmix_server_spawn_fn_t)(
                             const pmix_proc_t *proc,
                             const pmix_info_t job_info[],
                             size_t ninfo,
                             const pmix_app_t apps[],
                             size_t napps,
                             pmix_spawn_cbfunc_t cbfunc,
                             void *cbdata)
\end{codepar}
\cspecificend

\begin{arglist}
\argin{proc}{\refstruct{pmix_proc_t} structure of the process making the request (handle)}
\argin{job_info}{Array of info structures (array of handles)}
\argin{ninfo}{Number of elements in the \refarg{jobinfo} array (integer)}
\argin{apps}{Array of \refstruct{pmix_app_t} structures (array of handles)}
\argin{napps}{Number of elements in the \refarg{apps} array (integer)}
\argin{cbfunc}{Callback function \refapi{pmix_spawn_cbfunc_t} (function reference)}
\argin{cbdata}{Data to be passed to the callback function (memory reference)}
\end{arglist}

Returns one of the following:

\begin{itemize}
    \item \refconst{PMIX_SUCCESS}, indicating that the request is being processed by the host environment - result will be returned in the provided \refarg{cbfunc}. Note that the host must not invoke the callback function prior to returning from the \ac{API}.
    \item \refconst{PMIX_OPERATION_SUCCEEDED}, indicating that the request was immediately processed and returned \textit{success} - the \refarg{cbfunc} will not be called
    \item a PMIx error constant indicating either an error in the input or that the request was immediately processed and failed - the \refarg{cbfunc} will not be called
\end{itemize}

\reqattrstart
\ac{PMIx} libraries are required to pass any provided attributes to the host environment for processing. In addition, the following attributes are required to be included in the passed \refarg{info} array:

\pastePRIAttributeItem{PMIX_USERID}
\pastePRIAttributeItem{PMIX_GRPID}

\divider

Host environments that provide this module entry point are required to pass the \refattr{PMIX_SPAWNED} and \refattr{PMIX_PARENT_ID} attributes to all \ac{PMIx} servers launching new child processes so those values can be returned to clients upon connection to the \ac{PMIx} server. In addition, they are required to support the following attributes when present in either the \refarg{job_info} or the \textit{info} array of an element of the \refarg{apps} array:

\pastePRRTEAttributeItem{PMIX_WDIR}
\pastePRRTEAttributeItem{PMIX_SET_SESSION_CWD}
\pastePRRTEAttributeItem{PMIX_PREFIX}
\pastePRRTEAttributeItem{PMIX_HOST}
\pastePRRTEAttributeItem{PMIX_HOSTFILE}

\reqattrend

\optattrstart
The following attributes are optional for host environments that support this operation:

\pastePRRTEAttributeItem{PMIX_ADD_HOSTFILE}
\pastePRRTEAttributeItem{PMIX_ADD_HOST}
\pastePRRTEAttributeItem{PMIX_PRELOAD_BIN}
\pastePRRTEAttributeItem{PMIX_PRELOAD_FILES}
\pastePRRTEAttributeItem{PMIX_PERSONALITY}
\pastePRRTEAttributeItem{PMIX_MAPPER}
\pastePRRTEAttributeItem{PMIX_DISPLAY_MAP}
\pastePRRTEAttributeItem{PMIX_PPR}
\pastePRRTEAttributeItem{PMIX_MAPBY}
\pastePRRTEAttributeItem{PMIX_RANKBY}
\pastePRRTEAttributeItem{PMIX_BINDTO}
\pastePRRTEAttributeItem{PMIX_NON_PMI}
\pastePRRTEAttributeItem{PMIX_STDIN_TGT}
\pastePRRTEAttributeItem{PMIX_FWD_STDIN}
\pastePRRTEAttributeItem{PMIX_FWD_STDOUT}
\pastePRRTEAttributeItem{PMIX_FWD_STDERR}
\pastePRRTEAttributeItem{PMIX_DEBUGGER_DAEMONS}
\pastePRRTEAttributeItem{PMIX_TAG_OUTPUT}
\pastePRRTEAttributeItem{PMIX_TIMESTAMP_OUTPUT}
\pastePRRTEAttributeItem{PMIX_MERGE_STDERR_STDOUT}
\pastePRRTEAttributeItem{PMIX_OUTPUT_TO_FILE}
\pastePRRTEAttributeItem{PMIX_INDEX_ARGV}
\pastePRRTEAttributeItem{PMIX_CPUS_PER_PROC}
\pastePRRTEAttributeItem{PMIX_NO_PROCS_ON_HEAD}
\pastePRRTEAttributeItem{PMIX_NO_OVERSUBSCRIBE}
\pastePRRTEAttributeItem{PMIX_REPORT_BINDINGS}
\pastePRRTEAttributeItem{PMIX_CPU_LIST}
\pastePRRTEAttributeItem{PMIX_JOB_RECOVERABLE}
\pastePRRTEAttributeItem{PMIX_JOB_CONTINUOUS}
\pastePRRTEAttributeItem{PMIX_MAX_RESTARTS}
\pasteAttributeItem{PMIX_TIMEOUT}

\optattrend

%%%%
\descr

Spawn a set of applications/processes as per the \refapi{PMIx_Spawn} API.
Note that applications are not required to be \ac{MPI} or any other programming model.
Thus, the host server cannot make any assumptions as to their required support.
The callback function is to be executed once all processes have been started.
An error in starting any application or process in this request shall cause all applications and processes in the request to be terminated, and an error returned to the originating caller.

Note that a timeout can be specified in the job_info array to indicate that failure to start the requested job within the given time should result in termination to avoid hangs.


%%%%%%%%%%%
\subsection{\code{pmix_server_connect_fn_t}}
\declareapi{pmix_server_connect_fn_t}

%%%%
\summary

Record the specified processes as \textit{connected}.

%%%%
\format

\versionMarker{1.0}
\cspecificstart
\begin{codepar}
typedef pmix_status_t (*pmix_server_connect_fn_t)(
                             const pmix_proc_t procs[],
                             size_t nprocs,
                             const pmix_info_t info[],
                             size_t ninfo,
                             pmix_op_cbfunc_t cbfunc,
                             void *cbdata)
\end{codepar}
\cspecificend

\begin{arglist}
\argin{procs}{Array of \refstruct{pmix_proc_t} structures identifying participants (array of handles)}
\argin{nprocs}{Number of elements in the \refarg{procs} array (integer)}
\argin{info}{Array of info structures (array of handles)}
\argin{ninfo}{Number of elements in the \refarg{info} array (integer)}
\argin{cbfunc}{Callback function \refapi{pmix_op_cbfunc_t} (function reference)}
\argin{cbdata}{Data to be passed to the callback function (memory reference)}
\end{arglist}

Returns one of the following:

\begin{itemize}
    \item \refconst{PMIX_SUCCESS}, indicating that the request is being processed by the host environment - result will be returned in the provided \refarg{cbfunc}. Note that the host must not invoke the callback function prior to returning from the \ac{API}.
    \item \refconst{PMIX_OPERATION_SUCCEEDED}, indicating that the request was immediately processed and returned \textit{success} - the \refarg{cbfunc} will not be called
    \item a PMIx error constant indicating either an error in the input or that the request was immediately processed and failed - the \refarg{cbfunc} will not be called
\end{itemize}

\reqattrstart
\ac{PMIx} libraries are required to pass any provided attributes to the host environment for processing.
\reqattrend

\optattrstart
The following attributes are optional for host environments that support this operation:

\pastePRRTEAttributeItem{PMIX_TIMEOUT}
\pasteAttributeItem{PMIX_COLLECTIVE_ALGO}
\pasteAttributeItem{PMIX_COLLECTIVE_ALGO_REQD}

\optattrend

%%%%
\descr

Record the processes specified by the \refarg{procs} array as \textit{connected} as per the \ac{PMIx} definition\refsection{chap:api_proc_mgmt:connect}. The callback is to be executed once every daemon hosting at least one participant has called the host server's \refapi{pmix_server_connect_fn_t} function, and the host environment has completed any supporting operations required to meet the terms of the \ac{PMIx} definition of \textit{connected} processes.

\adviceimplstart
The \ac{PMIx} server library is required to aggregate participation by local clients, passing the request to the host environment once all local participants have executed the \ac{API}.
\adviceimplend

\advicermstart
The host will receive a single call for each collective operation. It is the responsibility of the host to identify the nodes containing participating processes, execute the collective across all participating nodes, and notify the local \ac{PMIx} server library upon completion of the global collective.
\advicermend


%%%%%%%%%%%
\subsection{\code{pmix_server_disconnect_fn_t}}
\declareapi{pmix_server_disconnect_fn_t}

%%%%
\summary

Disconnect a previously connected set of processes.

%%%%
\format

\versionMarker{1.0}
\cspecificstart
\begin{codepar}
typedef pmix_status_t (*pmix_server_disconnect_fn_t)(
                             const pmix_proc_t procs[],
                             size_t nprocs,
                             const pmix_info_t info[],
                             size_t ninfo,
                             pmix_op_cbfunc_t cbfunc,
                             void *cbdata)
\end{codepar}
\cspecificend

\begin{arglist}
\argin{procs}{Array of \refstruct{pmix_proc_t} structures identifying participants (array of handles)}
\argin{nprocs}{Number of elements in the \refarg{procs} array (integer)}
\argin{info}{Array of info structures (array of handles)}
\argin{ninfo}{Number of elements in the \refarg{info} array (integer)}
\argin{cbfunc}{Callback function \refapi{pmix_op_cbfunc_t} (function reference)}
\argin{cbdata}{Data to be passed to the callback function (memory reference)}
\end{arglist}

Returns one of the following:

\begin{itemize}
    \item \refconst{PMIX_SUCCESS}, indicating that the request is being processed by the host environment - result will be returned in the provided \refarg{cbfunc}. Note that the host must not invoke the callback function prior to returning from the \ac{API}.
    \item \refconst{PMIX_OPERATION_SUCCEEDED}, indicating that the request was immediately processed and returned \textit{success} - the \refarg{cbfunc} will not be called
    \item a PMIx error constant indicating either an error in the input or that the request was immediately processed and failed - the \refarg{cbfunc} will not be called
\end{itemize}

\reqattrstart
\ac{PMIx} libraries are required to pass any provided attributes to the host environment for processing.
\reqattrend

\optattrstart
The following attributes are optional for host environments that support this operation:

\pasteAttributeItem{PMIX_TIMEOUT}

\optattrend

%%%%
\descr

Disconnect a previously connected set of processes. The callback is to be executed once every daemon hosting at least one participant has called the host server's has called the \refapi{pmix_server_disconnect_fn_t} function, and the host environment has completed any required supporting operations.

\adviceimplstart
The \ac{PMIx} server library is required to aggregate participation by local clients, passing the request to the host environment once all local participants have executed the \ac{API}.
\adviceimplend

\advicermstart
The host will receive a single call for each collective operation. It is the responsibility of the host to identify the nodes containing participating processes, execute the collective across all participating nodes, and notify the local \ac{PMIx} server library upon completion of the global collective.

A \refconst{PMIX_ERR_INVALID_OPERATION} error must be returned if the specified set of \refarg{procs} was not previously \textit{connected} via a call to the \refapi{pmix_server_connect_fn_t} function.
\advicermend


%%%%%%%%%%%
\subsection{\code{pmix_server_register_events_fn_t}}
\declareapi{pmix_server_register_events_fn_t}

%%%%
\summary

Register to receive notifications for the specified events.

%%%%
\format

\versionMarker{1.0}
\cspecificstart
\begin{codepar}
 typedef pmix_status_t (*pmix_server_register_events_fn_t)(
                              pmix_status_t *codes,
                              size_t ncodes,
                              const pmix_info_t info[],
                              size_t ninfo,
                              pmix_op_cbfunc_t cbfunc,
                              void *cbdata)
\end{codepar}
\cspecificend

\begin{arglist}
\argin{codes}{Array of \refstruct{pmix_status_t} values (array of handles)}
\argin{ncodes}{Number of elements in the \refarg{codes} array (integer)}
\argin{info}{Array of info structures (array of handles)}
\argin{ninfo}{Number of elements in the \refarg{info} array (integer)}
\argin{cbfunc}{Callback function \refapi{pmix_op_cbfunc_t} (function reference)}
\argin{cbdata}{Data to be passed to the callback function (memory reference)}
\end{arglist}

Returns one of the following:

\begin{itemize}
    \item \refconst{PMIX_SUCCESS}, indicating that the request is being processed by the host environment - result will be returned in the provided \refarg{cbfunc}. Note that the host must not invoke the callback function prior to returning from the \ac{API}.
    \item \refconst{PMIX_OPERATION_SUCCEEDED}, indicating that the request was immediately processed and returned \textit{success} - the \refarg{cbfunc} will not be called
    \item a PMIx error constant indicating either an error in the input or that the request was immediately processed and failed - the \refarg{cbfunc} will not be called
\end{itemize}

\reqattrstart
\ac{PMIx} libraries are required to pass any provided attributes to the host environment for processing. In addition, the following attributes are required to be included in the passed \refarg{info} array:

\pastePRIAttributeItem{PMIX_USERID}
\pastePRIAttributeItem{PMIX_GRPID}

\reqattrend

%%%%
\descr

Register to receive notifications for the specified status codes. The \refarg{info} array included in this API is reserved for possible future directives to further steer notification.


\adviceimplstart
The \ac{PMIx} server library must track all client registrations for subsequent notification. This module function shall only be called when:

\begin{itemize}
    \item the client has requested notification of an environmental code (i.e., a \ac{PMIx} code in the range between \refconst{PMIX_ERR_SYS_BASE} and \refconst{PMIX_ERR_SYS_OTHER}, inclusive) or a code that lies outside the defined \ac{PMIx} range of constants; and
    \item the \ac{PMIx} server library has not previously requested notification of that code - i.e., the host environment is to be contacted only once a given unique code value
\end{itemize}

\adviceimplend

\advicermstart
The host environment is must to pass to its \ac{PMIx} server library all non-environmental events that directly relate to a registered namespace without the \ac{PMIx} server library explicitly requesting them. Environmental events are to be translated to their nearest \ac{PMIx} equivalent code as defined in the range between \refconst{PMIX_ERR_SYS_BASE} and \refconst{PMIX_ERR_SYS_OTHER} (inclusive).
\advicermend


%%%%%%%%%%%
\subsection{\code{pmix_server_deregister_events_fn_t}}
\declareapi{pmix_server_deregister_events_fn_t}

%%%%
\summary

Deregister to receive notifications for the specified events.

%%%%
\format

\versionMarker{1.0}
\cspecificstart
\begin{codepar}
 typedef pmix_status_t (*pmix_server_deregister_events_fn_t)(
                              pmix_status_t *codes,
                              size_t ncodes,
                              pmix_op_cbfunc_t cbfunc,
                              void *cbdata)
\end{codepar}
\cspecificend

\begin{arglist}
\argin{codes}{Array of \refstruct{pmix_status_t} values (array of handles)}
\argin{ncodes}{Number of elements in the \refarg{codes} array (integer)}
\argin{cbfunc}{Callback function \refapi{pmix_op_cbfunc_t} (function reference)}
\argin{cbdata}{Data to be passed to the callback function (memory reference)}
\end{arglist}

Returns one of the following:

\begin{itemize}
    \item \refconst{PMIX_SUCCESS}, indicating that the request is being processed by the host environment - result will be returned in the provided \refarg{cbfunc}. Note that the host must not invoke the callback function prior to returning from the \ac{API}.
    \item \refconst{PMIX_OPERATION_SUCCEEDED}, indicating that the request was immediately processed and returned \textit{success} - the \refarg{cbfunc} will not be called
    \item a PMIx error constant indicating either an error in the input or that the request was immediately processed and failed - the \refarg{cbfunc} will not be called
\end{itemize}

%%%%
\descr

Deregister to receive notifications for the specified events to which the \ac{PMIx} server has previously registered.

\adviceimplstart
The \ac{PMIx} server library must track all client registrations. This module function shall only be called when:

\begin{itemize}
    \item the library is deregistering environmental codes (i.e., a \ac{PMIx} codes in the range between \refconst{PMIX_ERR_SYS_BASE} and \refconst{PMIX_ERR_SYS_OTHER}, inclusive) or codes that lies outside the defined \ac{PMIx} range of constants; and
    \item no client (including the server library itself) remains registered for notifications on any included code - i.e., a code should be included in this call only when no registered notifications against it remain.
\end{itemize}

\adviceimplend


%%%%%%%%%%%
\subsection{\code{pmix_server_notify_event_fn_t}}
\declareapi{pmix_server_notify_event_fn_t}

%%%%
\summary

Notify the specified processes of an event.

%%%%
\format

\versionMarker{2.0}
\cspecificstart
\begin{codepar}
typedef pmix_status_t (*pmix_server_notify_event_fn_t)(pmix_status_t code,
                             const pmix_proc_t *source,
                             pmix_data_range_t range,
                             pmix_info_t info[],
                             size_t ninfo,
                             pmix_op_cbfunc_t cbfunc,
                             void *cbdata);
\end{codepar}
\cspecificend

\begin{arglist}
\argin{code}{The \refstruct{pmix_status_t} event code being referenced structure (handle)}
\argin{source}{\refstruct{pmix_proc_t} of process that generated the event (handle)}
\argin{range}{\refstruct{pmix_data_range_t} range over which the event is to be distributed (handle)}
\argin{info}{Optional array of \refstruct{pmix_info_t} structures containing additional information on the event (array of handles)}
\argin{ninfo}{Number of elements in the \refarg{info} array (integer)}
\argin{cbfunc}{Callback function \refapi{pmix_op_cbfunc_t} (function reference)}
\argin{cbdata}{Data to be passed to the callback function (memory reference)}
\end{arglist}

Returns one of the following:

\begin{itemize}
    \item \refconst{PMIX_SUCCESS}, indicating that the request is being processed by the host environment - result will be returned in the provided \refarg{cbfunc}. Note that the host must not invoke the callback function prior to returning from the \ac{API}.
    \item \refconst{PMIX_OPERATION_SUCCEEDED}, indicating that the request was immediately processed and returned \textit{success} - the \refarg{cbfunc} will not be called
    \item a PMIx error constant indicating either an error in the input or that the request was immediately processed and failed - the \refarg{cbfunc} will not be called
\end{itemize}

\reqattrstart
\ac{PMIx} libraries are required to pass any provided attributes to the host environment for processing.

\divider

Host environments that provide this module entry point are required to support the following attributes:

\pastePRRTEAttributeItem{PMIX_RANGE}

\reqattrend

%%%%
\descr

Notify the specified processes (described through a combination of \refarg{range} and attributes provided in the \refarg{info} array) of an event generated either by the \ac{PMIx} server itself or by one of its local clients.
The process generating the event is provided in the \refarg{source} parameter, and any further descriptive information is
included in the \refarg{info} array.

\advicermstart
The callback function is to be executed once the host environment no longer requires that the \ac{PMIx} server library maintain the provided data structures. It does not necessarily indicate that the event has been delivered to any process, nor that the event has been distributed for delivery
\advicermend


%%%%%%%%%%%
\subsection{\code{pmix_server_listener_fn_t}}
\declareapi{pmix_server_listener_fn_t}

%%%%
\summary

Register a socket the host server can monitor for connection requests.

%%%%
\format

\versionMarker{1.0}
\cspecificstart
\begin{codepar}
typedef pmix_status_t (*pmix_server_listener_fn_t)(
                             int listening_sd,
                             pmix_connection_cbfunc_t cbfunc,
                             void *cbdata)
\end{codepar}
\cspecificend

\begin{arglist}
\argin{incoming_sd}{(integer)}
\argin{cbfunc}{Callback function \refapi{pmix_connection_cbfunc_t} (function reference)}
\argin{cbdata}{ (memory reference)}
\end{arglist}

Returns \refconst{PMIX_SUCCESS} indicating that the request is accepted, or a negative value corresponding to a PMIx error constant indicating that the request has been rejected.

%%%%
\descr

Register a socket the host environment can monitor for connection requests, harvest them, and then call the \ac{PMIx} server library's internal callback function for further processing.
A listener thread is essential to efficiently harvesting connection requests from large numbers of local clients such as occur when running on large SMPs.
The host server listener is required to call accept on the incoming connection request, and then pass the resulting socket to the provided cbfunc.
A \code{NULL} for this function will cause the internal \ac{PMIx} server to spawn its own listener thread.


%%%%%%%%%%%
\subsection{\code{pmix_server_query_fn_t}}
\declareapi{pmix_server_query_fn_t}

%%%%
\summary

Query information from the resource manager.

%%%%
\format

\versionMarker{2.0}
\cspecificstart
\begin{codepar}
typedef pmix_status_t (*pmix_server_query_fn_t)(
                             pmix_proc_t *proct,
                             pmix_query_t *queries, size_t nqueries,
                             pmix_info_cbfunc_t cbfunc,
                             void *cbdata)
\end{codepar}
\cspecificend

\begin{arglist}
\argin{proct}{\refstruct{pmix_proc_t} structure of the requesting process (handle)}
\argin{queries}{Array of \refstruct{pmix_query_t} structures (array of handles)}
\argin{nqueries}{Number of elements in the \refarg{queries} array (integer)}
\argin{cbfunc}{Callback function \refapi{pmix_info_cbfunc_t} (function reference)}
\argin{cbdata}{Data to be passed to the callback function (memory reference)}
\end{arglist}

Returns one of the following:

\begin{itemize}
    \item \refconst{PMIX_SUCCESS}, indicating that the request is being processed by the host environment - result will be returned in the provided \refarg{cbfunc}. Note that the host must not invoke the callback function prior to returning from the \ac{API}.
    \item \refconst{PMIX_OPERATION_SUCCEEDED}, indicating that the request was immediately processed and returned \textit{success} - the \refarg{cbfunc} will not be called
    \item a PMIx error constant indicating either an error in the input or that the request was immediately processed and failed - the \refarg{cbfunc} will not be called
\end{itemize}

\reqattrstart
\ac{PMIx} libraries are required to pass any provided attributes to the host environment for processing. In addition, the following attributes are required to be included in the passed \refarg{info} array:

\pastePRIAttributeItem{PMIX_USERID}
\pastePRIAttributeItem{PMIX_GRPID}

\reqattrend


\optattrstart
The following attributes are optional for host environments that support this operation:

\pastePRRTEAttributeItem{PMIX_QUERY_NAMESPACES}
\pastePRRTEAttributeItem{PMIX_QUERY_JOB_STATUS}
\pasteAttributeItem{PMIX_QUERY_QUEUE_LIST}
\pasteAttributeItem{PMIX_QUERY_QUEUE_STATUS}
\pastePRRTEAttributeItem{PMIX_QUERY_PROC_TABLE}
\pastePRRTEAttributeItem{PMIX_QUERY_LOCAL_PROC_TABLE}
\pasteAttributeItem{PMIX_QUERY_SPAWN_SUPPORT}
\pasteAttributeItem{PMIX_QUERY_DEBUG_SUPPORT}
\pastePRRTEAttributeItem{PMIX_QUERY_MEMORY_USAGE}
\pastePRRTEAttributeItem{PMIX_QUERY_LOCAL_ONLY}
\pastePRRTEAttributeItem{PMIX_QUERY_REPORT_AVG}
\pastePRRTEAttributeItem{PMIX_QUERY_REPORT_MINMAX}
\pasteAttributeItem{PMIX_QUERY_ALLOC_STATUS}
\pasteAttributeItem{PMIX_TIME_REMAINING}

\optattrend
%%%%
\descr

Query information from the host environment.
The query will include the namespace/rank of the process that is requesting the info, an array of \refstruct{pmix_query_t} describing the request, and a callback function/data for the return.

\adviceimplstart
The \ac{PMIx} server library should not block in this function as the host environment may, depending upon the information being requested, require significant time to respond.
\adviceimplend



%%%%%%%%%%%
\subsection{\code{pmix_server_tool_connection_fn_t}}
\declareapi{pmix_server_tool_connection_fn_t}

%%%%
\summary

Register that a tool has connected to the server.

%%%%
\format

\versionMarker{2.0}
\cspecificstart
\begin{codepar}
typedef void (*pmix_server_tool_connection_fn_t)(
                             pmix_info_t info[], size_t ninfo,
                             pmix_tool_connection_cbfunc_t cbfunc,
                             void *cbdata)
\end{codepar}
\cspecificend

\begin{arglist}
\argin{info}{Array of \refstruct{pmix_info_t} structures (array of handles)}
\argin{ninfo}{Number of elements in the \refarg{info} array (integer)}
\argin{cbfunc}{Callback function \refapi{pmix_tool_connection_cbfunc_t} (function reference)}
\argin{cbdata}{Data to be passed to the callback function (memory reference)}
\end{arglist}

\reqattrstart
\ac{PMIx} libraries are required to pass the following attributes in the \refarg{info} array:

\pastePRIAttributeItem{PMIX_USERID}
\pastePRIAttributeItem{PMIX_GRPID}

\reqattrend


\optattrstart
The following attributes are optional for host environments that support this operation:

\pastePRRTEAttributeItem{PMIX_FWD_STDOUT}
\pastePRRTEAttributeItem{PMIX_FWD_STDERR}
\pastePRRTEAttributeItem{PMIX_FWD_STDIN}

\optattrend

%%%%
\descr

Register that a tool has connected to the server, and request that the tool be assigned a namespace/rank identifier for further interactions.
The \refstruct{pmix_info_t} array is used to pass qualifiers for the connection request, including the effective uid and gid of the calling tool for authentication purposes.

\advicermstart
The host environment is solely responsible for authenticating and authorizing the connection, and for authorizing all subsequent tool requests. The host must not execute the callback function prior to returning from the \ac{API}.

\advicermend


%%%%%%%%%%%
\subsection{\code{pmix_server_log_fn_t}}
\declareapi{pmix_server_log_fn_t}

%%%%
\summary

Log data on behalf of a client.

%%%%
\format

\versionMarker{2.0}
\cspecificstart
\begin{codepar}
typedef void (*pmix_server_log_fn_t)(
                             const pmix_proc_t *client,
                             const pmix_info_t data[], size_t ndata,
                             const pmix_info_t directives[], size_t ndirs,
                             pmix_op_cbfunc_t cbfunc, void *cbdata)
\end{codepar}
\cspecificend

\begin{arglist}
\argin{client}{\refstruct{pmix_proc_t} structure (handle)}
\argin{data}{Array of info structures (array of handles)}
\argin{ndata}{Number of elements in the \refarg{data} array (integer)}
\argin{directives}{Array of info structures (array of handles)}
\argin{ndirs}{Number of elements in the \refarg{directives} array (integer)}
\argin{cbfunc}{Callback function \refapi{pmix_op_cbfunc_t} (function reference)}
\argin{cbdata}{Data to be passed to the callback function (memory reference)}
\end{arglist}


\reqattrstart
\ac{PMIx} libraries are required to pass any provided attributes to the host environment for processing. In addition, the following attributes are required to be included in the passed \refarg{info} array:

\pastePRIAttributeItem{PMIX_USERID}
\pastePRIAttributeItem{PMIX_GRPID}

\divider

Host environments that provide this module entry point are required to support the following attributes:

\pastePRRTEAttributeItem{PMIX_LOG_STDERR}
\pastePRRTEAttributeItem{PMIX_LOG_STDOUT}
\pastePRRTEAttributeItem{PMIX_LOG_SYSLOG}

\reqattrend

\optattrstart
The following attributes are optional for host environments that support this operation:

\pasteAttributeItem{PMIX_LOG_MSG}
\pasteAttributeItem{PMIX_LOG_EMAIL}
\pasteAttributeItem{PMIX_LOG_EMAIL_ADDR}
\pasteAttributeItem{PMIX_LOG_EMAIL_SUBJECT}
\pasteAttributeItem{PMIX_LOG_EMAIL_MSG}

\optattrend

%%%%
\descr

Log data on behalf of a client. This function is not intended for output of computational results, but rather for reporting status and error messages. The host must not execute the callback function prior to returning from the \ac{API}.


%%%%%%%%%%%
\subsection{\code{pmix_server_alloc_fn_t}}
\declareapi{pmix_server_alloc_fn_t}

%%%%
\summary

Request allocation operations on behalf of a client.

%%%%
\format

\versionMarker{2.0}
\cspecificstart
\begin{codepar}
typedef pmix_status_t (*pmix_server_alloc_fn_t)(
                             const pmix_proc_t *client,
                             pmix_alloc_directive_t directive,
                             const pmix_info_t data[], size_t ndata,
                             pmix_info_cbfunc_t cbfunc, void *cbdata)
\end{codepar}
\cspecificend

\begin{arglist}
\argin{client}{\refstruct{pmix_proc_t} structure of process making request (handle)}
\argin{directive}{Specific action being requested (\refstruct{pmix_alloc_directive_t})}
\argin{data}{Array of info structures (array of handles)}
\argin{ndata}{Number of elements in the \refarg{data} array (integer)}
\argin{cbfunc}{Callback function \refapi{pmix_info_cbfunc_t} (function reference)}
\argin{cbdata}{Data to be passed to the callback function (memory reference)}
\end{arglist}

Returns one of the following:

\begin{itemize}
    \item \refconst{PMIX_SUCCESS}, indicating that the request is being processed by the host environment - result will be returned in the provided \refarg{cbfunc}. Note that the host must not invoke the callback function prior to returning from the \ac{API}.
    \item \refconst{PMIX_OPERATION_SUCCEEDED}, indicating that the request was immediately processed and returned \textit{success} - the \refarg{cbfunc} will not be called
    \item a PMIx error constant indicating either an error in the input or that the request was immediately processed and failed - the \refarg{cbfunc} will not be called
\end{itemize}

\reqattrstart
\ac{PMIx} libraries are required to pass any provided attributes to the host environment for processing. In addition, the following attributes are required to be included in the passed \refarg{info} array:

\pastePRIAttributeItem{PMIX_USERID}
\pastePRIAttributeItem{PMIX_GRPID}

\divider

Host environments that provide this module entry point are required to support the following attributes:

\pasteAttributeItem{PMIX_ALLOC_ID}
\pasteAttributeItem{PMIX_ALLOC_NUM_NODES}
\pasteAttributeItem{PMIX_ALLOC_NUM_CPUS}
\pasteAttributeItem{PMIX_ALLOC_TIME}

\reqattrend

\optattrstart
The following attributes are optional for host environments that support this operation:

\pasteAttributeItem{PMIX_ALLOC_NODE_LIST}
\pasteAttributeItem{PMIX_ALLOC_NUM_CPU_LIST}
\pasteAttributeItem{PMIX_ALLOC_CPU_LIST}
\pasteAttributeItem{PMIX_ALLOC_MEM_SIZE}
\pasteAttributeItem{PMIX_ALLOC_NETWORK}
\pasteAttributeItem{PMIX_ALLOC_NETWORK_ID}
\pasteAttributeItem{PMIX_ALLOC_BANDWIDTH}
\pasteAttributeItem{PMIX_ALLOC_NETWORK_QOS}

\optattrend

%%%%
\descr

Request new allocation or modifications to an existing allocation on behalf of a client. Several broad categories are envisioned, including the ability to:

\begin{compactitem}
%
\item Request allocation of additional resources, including memory, bandwidth, and compute for an existing allocation. Any additional allocated resources will be considered as part of the current allocation, and thus will be released at the same time.
%
\item Request a new allocation of resources. Note that the new allocation will be disjoint from (i.e., not affiliated with) the allocation of the requestor - thus the termination of one allocation will not impact the other.
%
\item Extend the reservation on currently allocated resources, subject to scheduling availability and priorities.
%
\item Return no-longer-required resources to the scheduler.
This includes the \textit{loan} of resources back to the scheduler with a promise to return them upon subsequent request.
\end{compactitem}

The callback function provides a \refarg{status} to indicate whether or not the request was granted, and to provide some information as to the reason for any denial in the \refapi{pmix_info_cbfunc_t} array of \refstruct{pmix_info_t} structures.


%%%%%%%%%%%
\subsection{\code{pmix_server_job_control_fn_t}}
\declareapi{pmix_server_job_control_fn_t}

%%%%
\summary

Execute a job control action on behalf of a client.

%%%%
\format

\versionMarker{2.0}
\cspecificstart
\begin{codepar}
typedef pmix_status_t (*pmix_server_job_control_fn_t)(
                             const pmix_proc_t *requestor,
                             const pmix_proc_t targets[], size_t ntargets,
                             const pmix_info_t directives[], size_t ndirs,
                             pmix_info_cbfunc_t cbfunc, void *cbdata)
\end{codepar}
\cspecificend

\begin{arglist}
\argin{requestor}{\refstruct{pmix_proc_t} structure of requesting process (handle)}
\argin{targets}{Array of proc structures (array of handles)}
\argin{ntargets}{Number of elements in the \refarg{targets} array (integer)}
\argin{directives}{Array of info structures (array of handles)}
\argin{ndirs}{Number of elements in the \refarg{info} array (integer)}
\argin{cbfunc}{Callback function \refapi{pmix_op_cbfunc_t} (function reference)}
\argin{cbdata}{Data to be passed to the callback function (memory reference)}
\end{arglist}

Returns one of the following:

\begin{itemize}
    \item \refconst{PMIX_SUCCESS}, indicating that the request is being processed by the host environment - result will be returned in the provided \refarg{cbfunc}. Note that the host must not invoke the callback function prior to returning from the \ac{API}.
    \item \refconst{PMIX_OPERATION_SUCCEEDED}, indicating that the request was immediately processed and returned \textit{success} - the \refarg{cbfunc} will not be called
    \item a PMIx error constant indicating either an error in the input or that the request was immediately processed and failed - the \refarg{cbfunc} will not be called
\end{itemize}

\reqattrstart
\ac{PMIx} libraries are required to pass any attributes provided by the client to the host environment for processing. In addition, the following attributes are required to be included in the passed \refarg{info} array:

\pastePRIAttributeItem{PMIX_USERID}
\pastePRIAttributeItem{PMIX_GRPID}

\divider

Host environments that provide this module entry point are required to support the following attributes:

\pastePRRTEAttributeItem{PMIX_JOB_CTRL_ID}
\pastePRRTEAttributeItem{PMIX_JOB_CTRL_PAUSE}
\pastePRRTEAttributeItem{PMIX_JOB_CTRL_RESUME}
\pastePRRTEAttributeItem{PMIX_JOB_CTRL_KILL}
\pastePRRTEAttributeItem{PMIX_JOB_CTRL_SIGNAL}
\pastePRRTEAttributeItem{PMIX_JOB_CTRL_TERMINATE}

\reqattrend

\optattrstart
The following attributes are optional for host environments that support this operation:

\pasteAttributeItem{PMIX_JOB_CTRL_CANCEL}
\pasteAttributeItem{PMIX_JOB_CTRL_RESTART}
\pasteAttributeItem{PMIX_JOB_CTRL_CHECKPOINT}
\pasteAttributeItem{PMIX_JOB_CTRL_CHECKPOINT_EVENT}
\pasteAttributeItem{PMIX_JOB_CTRL_CHECKPOINT_SIGNAL}
\pasteAttributeItem{PMIX_JOB_CTRL_CHECKPOINT_TIMEOUT}
\pasteAttributeItem{PMIX_JOB_CTRL_CHECKPOINT_METHOD}
\pasteAttributeItem{PMIX_JOB_CTRL_PROVISION}
\pasteAttributeItem{PMIX_JOB_CTRL_PROVISION_IMAGE}
\pasteAttributeItem{PMIX_JOB_CTRL_PREEMPTIBLE}

\optattrend

%%%%
\descr

Execute a job control action on behalf of a client. The \refarg{targets} array identifies the processes to which the requested job control action is to be applied.
A \code{NULL} value can be used to indicate all processes in the caller's namespace.
The use of \refconst{PMIX_RANK_WILDARD} can also be used to indicate that all processes in the given namespace are to be included.

The directives are provided as \refstruct{pmix_info_t} structures in the \refarg{directives} array.
The callback function provides a \refarg{status} to indicate whether or not the request was granted, and to provide some information as to the reason for any denial in the \refapi{pmix_info_cbfunc_t} array of \refstruct{pmix_info_t} structures.


%%%%%%%%%%%
\subsection{\code{pmix_server_monitor_fn_t}}
\declareapi{pmix_server_monitor_fn_t}

%%%%
\summary

Request that a client be monitored for activity.

%%%%
\format

\versionMarker{2.0}
\cspecificstart
\begin{codepar}
typedef pmix_status_t (*pmix_server_monitor_fn_t)(
                             const pmix_proc_t *requestor,
                             const pmix_info_t *monitor, pmix_status_t error,
                             const pmix_info_t directives[], size_t ndirs,
                             pmix_info_cbfunc_t cbfunc, void *cbdata);
\end{codepar}
\cspecificend

\begin{arglist}
\argin{requestor}{\refstruct{pmix_proc_t} structure of requesting process (handle)}
\argin{monitor}{\refstruct{pmix_info_t} identifying the type of monitor being requested (handle)}
\argin{error}{Status code to use in generating event if alarm triggers (integer)}
\argin{directives}{Array of info structures (array of handles)}
\argin{ndirs}{Number of elements in the \refarg{info} array (integer)}
\argin{cbfunc}{Callback function \refapi{pmix_op_cbfunc_t} (function reference)}
\argin{cbdata}{Data to be passed to the callback function (memory reference)}
\end{arglist}

Returns one of the following:

\begin{itemize}
    \item \refconst{PMIX_SUCCESS}, indicating that the request is being processed by the host environment - result will be returned in the provided \refarg{cbfunc}. Note that the host must not invoke the callback function prior to returning from the \ac{API}.
    \item \refconst{PMIX_OPERATION_SUCCEEDED}, indicating that the request was immediately processed and returned \textit{success} - the \refarg{cbfunc} will not be called
    \item a PMIx error constant indicating either an error in the input or that the request was immediately processed and failed - the \refarg{cbfunc} will not be called
\end{itemize}

This entry point is only called for monitoring requests that are not directly supported by the \ac{PMIx} server library itself.

\reqattrstart
If supported by the \ac{PMIx} server library, then the library must not pass any supported attributes to the host environment. Any attributes provided by the client that are not directly supported by the server library must be passed to the host environment if it provides this module entry. In addition, the following attributes are required to be included in the passed \refarg{info} array:

\pastePRIAttributeItem{PMIX_USERID}
\pastePRIAttributeItem{PMIX_GRPID}

\divider

Host environments are not required to support any specific monitoring attributes.

\reqattrend

\optattrstart
The following attributes may be implemented by a host environment.

\pastePRIAttributeItem{PMIX_MONITOR_ID}
\pastePRIAttributeItem{PMIX_MONITOR_CANCEL}
\pastePRIAttributeItem{PMIX_MONITOR_APP_CONTROL}
\pastePRIAttributeItem{PMIX_MONITOR_HEARTBEAT}
\pastePRIAttributeItem{PMIX_MONITOR_HEARTBEAT_TIME}
\pastePRIAttributeItem{PMIX_MONITOR_HEARTBEAT_DROPS}
\pastePRIAttributeItem{PMIX_MONITOR_FILE}
\pastePRIAttributeItem{PMIX_MONITOR_FILE_SIZE}
\pastePRIAttributeItem{PMIX_MONITOR_FILE_ACCESS}
\pastePRIAttributeItem{PMIX_MONITOR_FILE_MODIFY}
\pastePRIAttributeItem{PMIX_MONITOR_FILE_CHECK_TIME}
\pastePRIAttributeItem{PMIX_MONITOR_FILE_DROPS}

\optattrend

%%%%
\descr

Request that a client be monitored for activity.

\advicermstart
If this module entry is provided and called by the \ac{PMIx} server library, then the host environment must either provide the requested services or return \refconst{PMIX_ERR_NOT_SUPPORTED} to the provided \refarg{cbfunc}.
\advicermend

%%%%%%%%%%%%%%%%%%%%%%%%%%%%%%%%%%%%%%%%%%%%%%%%%
%%%%%  v3 Module Functions
%%%%%%%%%%%
\subsection{\code{pmix_server_get_cred_fn_t}}
\declareapi{pmix_server_get_cred_fn_t}

%%%%
\summary

Request a credential from the host environment

%%%%
\format

\versionMarker{3.0}
\cspecificstart
\begin{codepar}
typedef pmix_status_t (*pmix_server_get_cred_fn_t)(
                             const pmix_proc_t *proc,
                             const pmix_info_t directives[],
                             size_t ndirs,
                             pmix_credential_cbfunc_t cbfunc,
                             void *cbdata);
\end{codepar}
\cspecificend

\begin{arglist}
\argin{proc}{\refstruct{pmix_proc_t} structure of requesting process (handle)}
\argin{directives}{Array of info structures (array of handles)}
\argin{ndirs}{Number of elements in the \refarg{info} array (integer)}
\argin{cbfunc}{Callback function to return the credential (\refapi{pmix_credential_cbfunc_t} function reference)}
\argin{cbdata}{Data to be passed to the callback function (memory reference)}
\end{arglist}

Returns \refconst{PMIX_SUCCESS} or a negative value corresponding to a PMIx error constant. In the event the function returns an error, the \refarg{cbfunc} will not be called.

\reqattrstart
If the \ac{PMIx} library does not itself provide the requested credential, then it is required to pass any attributes provided by the client to the host environment for processing. In addition, it must include the following attributes in the passed \refarg{info} array:

\pastePRIAttributeItem{PMIX_USERID}
\pastePRIAttributeItem{PMIX_GRPID}

\reqattrend

\optattrstart
The following attributes are optional for host environments that support this operation:

\pasteAttributeItem{PMIX_CRED_TYPE}
\pasteAttributeItem{PMIX_TIMEOUT}

\optattrend

\adviceimplstart
We recommend that implementation of the \refattr{PMIX_TIMEOUT} attribute be left to the host environment due to race condition considerations between completion of the operation versus internal timeout in the \ac{PMIx} server library. Implementers that choose to support \refattr{PMIX_TIMEOUT} directly in the \ac{PMIx} server library must take care to resolve the race condition and should avoid passing \refattr{PMIX_TIMEOUT} to the host environment so that multiple competing timeouts are not created.
\adviceimplend


%%%%
\descr

Request a credential from the host environment

\advicermstart
If this module entry is provided and called by the \ac{PMIx} server library, then the host environment must either provide the requested credential in the callback function or immediately return an error to the caller.
\advicermend

%%%%%%%%%%%
\subsection{\code{pmix_server_validate_cred_fn_t}}
\declareapi{pmix_server_validate_cred_fn_t}

%%%%
\summary

Request validation of a credential

%%%%
\format

\versionMarker{3.0}
\cspecificstart
\begin{codepar}
typedef pmix_status_t (*pmix_server_validate_cred_fn_t)(
                             const pmix_proc_t *proc,
                             const pmix_byte_object_t *cred,
                             const pmix_info_t directives[],
                             size_t ndirs,
                             pmix_validation_cbfunc_t cbfunc,
                             void *cbdata);
\end{codepar}
\cspecificend

\begin{arglist}
\argin{proc}{\refstruct{pmix_proc_t} structure of requesting process (handle)}
\argin{cred}{Pointer to \refstruct{pmix_byte_object_t} containing the credential (handle)}
\argin{directives}{Array of info structures (array of handles)}
\argin{ndirs}{Number of elements in the \refarg{info} array (integer)}
\argin{cbfunc}{Callback function to return the result (\refapi{pmix_validation_cbfunc_t} function reference)}
\argin{cbdata}{Data to be passed to the callback function (memory reference)}
\end{arglist}

Returns one of the following:

\begin{itemize}
    \item \refconst{PMIX_SUCCESS}, indicating that the request is being processed by the host environment - result will be returned in the provided \refarg{cbfunc}
    \item \refconst{PMIX_OPERATION_SUCCEEDED}, indicating that the request was immediately processed and returned \textit{success} - the \refarg{cbfunc} will not be called
    \item a PMIx error constant indicating either an error in the input or that the request was immediately processed and failed - the \refarg{cbfunc} will not be called
\end{itemize}

\reqattrstart
If the \ac{PMIx} library does not itself validate the credential, then it is required to pass any attributes provided by the client to the host environment for processing. In addition, it must include the following attributes in the passed \refarg{info} array:

\pastePRIAttributeItem{PMIX_USERID}
\pastePRIAttributeItem{PMIX_GRPID}

\divider

Host environments are not required to support any specific attributes.

\reqattrend

\optattrstart
The following attributes are optional for host environments that support this operation:

\pasteAttributeItem{PMIX_TIMEOUT}

\optattrend

\adviceimplstart
We recommend that implementation of the \refattr{PMIX_TIMEOUT} attribute be left to the host environment due to race condition considerations between completion of the operation versus internal timeout in the \ac{PMIx} server library. Implementers that choose to support \refattr{PMIX_TIMEOUT} directly in the \ac{PMIx} server library must take care to resolve the race condition and should avoid passing \refattr{PMIX_TIMEOUT} to the host environment so that multiple competing timeouts are not created.
\adviceimplend


%%%%
\descr

Request validation of a credential obtained from the host environment via a prior call to the \refapi{pmix_server_get_cred_fn_t} module entry.

%%%%%%%%%%%
\subsection{\code{pmix_server_iof_fn_t}}
\declareapi{pmix_server_iof_fn_t}

%%%%
\summary

Request the specified IO channels be forwarded from the given array of processes.

%%%%
\format

\versionMarker{3.0}
\cspecificstart
\begin{codepar}
typedef pmix_status_t (*pmix_server_iof_fn_t)(
                        const pmix_proc_t procs[], size_t nprocs,
                        const pmix_info_t directives[], size_t ndirs,
                        pmix_iof_channel_t channels,
                        pmix_op_cbfunc_t cbfunc, void *cbdata);
\end{codepar}
\cspecificend

\begin{arglist}
\argin{procs}{Array \refstruct{pmix_proc_t} identifiers whose \ac{IO} is being requested (handle)}
\argin{nprocs}{Number of elements in \refarg{procs} (\code{size_t})}
\argin{directives}{Array of \refstruct{pmix_info_t} structures further defining the request (array of handles)}
\argin{ndirs}{Number of elements in the \refarg{info} array (integer)}
\argin{channels}{Bitmask identifying the channels to be forwarded (\refstruct{pmix_iof_channel_t})}
\argin{cbfunc}{Callback function \refapi{pmix_op_cbfunc_t} (function reference)}
\argin{cbdata}{Data to be passed to the callback function (memory reference)}
\end{arglist}

Returns one of the following:

\begin{itemize}
    \item \refconst{PMIX_SUCCESS}, indicating that the request is being processed by the host environment - result will be returned in the provided \refarg{cbfunc}. Note that the library must not invoke the callback function prior to returning from the \ac{API}.
    \item \refconst{PMIX_OPERATION_SUCCEEDED}, indicating that the request was immediately processed and returned \textit{success} - the \refarg{cbfunc} will not be called
    \item a PMIx error constant indicating either an error in the input or that the request was immediately processed and failed - the \refarg{cbfunc} will not be called
\end{itemize}


\reqattrstart
The following attributes are required to be included in the passed \refarg{info} array:

\pastePRIAttributeItem{PMIX_USERID}
\pastePRIAttributeItem{PMIX_GRPID}

\divider

Host environments that provide this module entry point are required to support the following attributes:

\pastePRRTEAttributeItem{PMIX_IOF_CACHE_SIZE}
\pastePRRTEAttributeItem{PMIX_IOF_DROP_OLDEST}
\pastePRRTEAttributeItem{PMIX_IOF_DROP_NEWEST}

\reqattrend

\optattrstart
The following attributes may be supported by a host environment.

\pasteAttributeItem{PMIX_IOF_BUFFERING_SIZE}
\pasteAttributeItem{PMIX_IOF_BUFFERING_TIME}

\optattrend

%%%%
\descr

Request the specified IO channels be forwarded from the given array of processes. An error shall be returned in the callback function if the requested service from any of the requested processes cannot be provided.

\adviceimplstart
The forwarding of stdin is a \textit{push} process - processes cannot request that it be \textit{pulled} from some other source. Requests including the \refconst{PMIX_FWD_STDIN_CHANNEL} channel will return a \refconst{PMIX_ERR_NOT_SUPPORTED} error.
\adviceimplend


%%%%%%%%%%%
\subsection{\code{pmix_server_stdin_fn_t}}
\declareapi{pmix_server_stdin_fn_t}

%%%%
\summary

Pass standard input data to the host environment for transmission to specified recipients.

%%%%
\format

\versionMarker{3.0}
\cspecificstart
\begin{codepar}
typedef pmix_status_t (*pmix_server_stdin_fn_t)(
                           const pmix_proc_t *source,
                           const pmix_proc_t targets[],
                           size_t ntargets,
                           const pmix_info_t directives[],
                           size_t ndirs,
                           const pmix_byte_object_t *bo,
                           pmix_op_cbfunc_t cbfunc, void *cbdata);
\end{codepar}
\cspecificend

\begin{arglist}
\argin{source}{\refstruct{pmix_proc_t} structure of source process (handle)}
\argin{targets}{Array of \refstruct{pmix_proc_t} target identifiers (handle)}
\argin{ntargets}{Number of elements in the \refarg{targets} array (integer)}
\argin{directives}{Array of info structures (array of handles)}
\argin{ndirs}{Number of elements in the \refarg{info} array (integer)}
\argin{bo}{Pointer to \refstruct{pmix_byte_object_t} containing the payload (handle)}
\argin{cbfunc}{Callback function \refapi{pmix_op_cbfunc_t} (function reference)}
\argin{cbdata}{Data to be passed to the callback function (memory reference)}
\end{arglist}

Returns one of the following:

\begin{itemize}
    \item \refconst{PMIX_SUCCESS}, indicating that the request is being processed by the host environment - result will be returned in the provided \refarg{cbfunc}. Note that the library must not invoke the callback function prior to returning from the \ac{API}.
    \item \refconst{PMIX_OPERATION_SUCCEEDED}, indicating that the request was immediately processed and returned \textit{success} - the \refarg{cbfunc} will not be called
    \item a PMIx error constant indicating either an error in the input or that the request was immediately processed and failed - the \refarg{cbfunc} will not be called
\end{itemize}

\reqattrstart
The following attributes are required to be included in the passed \refarg{info} array:

\pastePRIAttributeItem{PMIX_USERID}
\pastePRIAttributeItem{PMIX_GRPID}

\reqattrend

%%%%
\descr

Passes stdin to the host environment for transmission to specified recipients. The host environment is responsible for forwarding the data to all locations that host the specified \refarg{targets} and delivering the payload to the \ac{PMIx} server library connected to those clients.

\advicermstart
If this module entry is provided and called by the \ac{PMIx} server library, then the host environment must either provide the requested services or return \refconst{PMIX_ERR_NOT_SUPPORTED} to the provided \refarg{cbfunc}.
\advicermend

%%%%%%%%%%%%%%%%%%%%%%%%%%%%%%%%%%%%%%%%%%%%%%%%%



%
% Appendix
%
    \setcounter{chapter}{0}  % restart chapter numbering with "letter A"
    \renewcommand{\thechapter}{\Alph{chapter}}%
    \appendix

    % Support funcitons outside of the standard
%    \input{App_Support}

    % Revisions, Acknowledgements
    %%%%%%%%%%%%%%%%%%%%%%%%%%%%%%%%%%%%%%%%%%%%%%%%%
% Chapter: Acknowledgements
%%%%%%%%%%%%%%%%%%%%%%%%%%%%%%%%%%%%%%%%%%%%%%%%%
\chapter{Acknowledgements}
\label{chap:acknowledgements}

This document represents the work of many people who have contributed to the PMIx community.
Without the hard work and dedication of these people this document would not have been possible.
The sections below list some of the active participants and organizations in the various PMIx standard iterations.

%%%%%%%%%% Version 4.0
\section{Version 4.0}

The following list includes some of the active participants in the PMIx v4 standardization process.

\begin{itemize}
\item Ralph H. Castain and Danielle Sikich
\item Joshua Hursey abd David Solt
\item Dirk Schubert
\item John DelSignore
\item Aurelien Bouteiller
\item Michael A Raymond
\item Howard Pritchard and Nathan Hjelm
\item Brice Goglin
\item Kathryn Mohror and Stephen Herbein
\item Thomas Naughton and Swaroop Pophale
\item William E. Allcock and Paul Rich
\item Michael Karo
\item Artem Polyakov
\end{itemize}

The following institutions supported this effort through time and travel support for the people listed above.

\begin{itemize}
\item Intel Corporation
\item IBM, Inc.
\item Allinea (ARM)
\item Perforce
\item University of Tennessee, Knoxville
\item The Exascale Computing Project, an initiative of the US Department of Energy
\item National Science Foundation
\item HPE Co.
\item Los Alamos National Laboratory
\item INRIA
\item Lawrence Livermore National Laboratory
\item Oak Ridge National Laboratory
\item Argonne National Laboratory
\item Altair
\item NVIDIA
\end{itemize}

%%%%%%%%%% Version 3.0
\section{Version 3.0}

The following list includes some of the active participants in the PMIx v3 standardization process.

\begin{itemize}
\item Ralph H. Castain, Andrew Friedley, Brandon Yates
\item Joshua Hursey
\item Aurelien Bouteiller and George Bosilca
\item Dirk Schubert
\item Kevin Harms
\end{itemize}

The following institutions supported this effort through time and travel support for the people listed above.

\begin{itemize}
\item Intel Corporation
\item IBM, Inc.
\item University of Tennessee, Knoxville
\item The Exascale Computing Project, an initiative of the US Department of Energy
\item National Science Foundation
\item Argonne National Laboratory
\item Allinea (ARM)
\end{itemize}

%%%%%%%%%% Version 2.0
\section{Version 2.0}

The following list includes some of the active participants in the PMIx v2 standardization process.

\begin{itemize}
\item Ralph H. Castain, Annapurna Dasari, Christopher A. Holguin, Andrew Friedley, Michael Klemm and Terry Wilmarth
\item Joshua Hursey, David Solt, Alexander Eichenberger, Geoff Paulsen, and Sameh Sharkawi
\item Aurelien Bouteiller and George Bosilca
\item Artem Polyakov, Igor Ivanov and Boris Karasev
\item Gilles Gouaillardet
\item Michael A Raymond and Jim Stoffel
\item Dirk Schubert
\item Moe Jette
\item Takahiro Kawashima and Shinji Sumimoto
\item Howard Pritchard
\item David Beer
\item Brice Goglin
\item Geoffroy Vallee, Swen Boehm, Thomas Naughton and David Bernholdt
\item Adam Moody and Martin Schulz
\item Ryan Grant and Stephen Olivier
\item Michael Karo
\end{itemize}

The following institutions supported this effort through time and travel support for the people listed above.

\begin{itemize}
\item Intel Corporation
\item IBM, Inc.
\item University of Tennessee, Knoxville
\item The Exascale Computing Project, an initiative of the US Department of Energy
\item National Science Foundation
\item Mellanox, Inc.
\item Research Organization for Information Science and Technology
\item HPE Co.
\item Allinea (ARM)
\item SchedMD, Inc.
\item Fujitsu Limited
\item Los Alamos National Laboratory
\item Adaptive Solutions, Inc.
\item INRIA
\item Oak Ridge National Laboratory
\item Lawrence Livermore National Laboratory
\item Sandia National Laboratory
\item Altair
\end{itemize}


%%%%%%%%%% Version 1.0
\section{Version 1.0}

The following list includes some of the active participants in the PMIx v1 standardization process.

\begin{itemize}
\item Ralph H. Castain, Annapurna Dasari and Christopher A. Holguin
\item Joshua Hursey and David Solt
\item Aurelien Bouteiller and George Bosilca
\item Artem Polyakov, Elena Shipunova, Igor Ivanov, and Joshua Ladd
\item Gilles Gouaillardet
\item Gary Brown
\item Moe Jette
\end{itemize}

The following institutions supported this effort through time and travel support for the people listed above.

\begin{itemize}
\item Intel Corporation
\item IBM, Inc.
\item University of Tennessee, Knoxville
\item Mellanox, Inc.
\item Research Organization for Information Science and Technology
\item Adaptive Solutions, Inc.
\item SchedMD, Inc.
\end{itemize}


%
% Bibliography
%
	\nolinenumbers
	\bibliography{pmix-standard}{}
	\addcontentsline{toc}{chapter}{Bibliography}
	\bibliographystyle{plain}

%
% Index
%
	\nolinenumbers
	\printindex

\end{document}

%%%%%%%%%%%%%%%%%%%%%%%%%%%%%%%%%%%%%%%%%%%%%%%%%
